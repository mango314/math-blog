\documentclass[12pt]{article}
%Gummi|065|=)
\usepackage{amsmath, amsfonts, amssymb}
\usepackage[margin=0.5in]{geometry}
\usepackage{xcolor}
\usepackage{graphicx}
\newcommand{\off}[1]{}
\DeclareMathSizes{20}{30}{21}{18}

\usepackage{tikz}

\title{\textbf{ Pell's Equation }}
\author{John D Mangual}
\date{}
\begin{document}

\fontfamily{qag}\selectfont \fontsize{23}{25}\selectfont

\maketitle

\noindent I have decided to move all the string theory papers to a separate blog.  Just a short reading list for now\footnote{The other experiment is posting this on the front-page, seeing if maybe I can write just a little bit neater.}. \\ \\
While there is a procedure for solving equations like $x^2 - 7y^2 = 1$ but there is little understanding beyond running steps of that procedure. \\ \\
Why is that?



\fontfamily{qag}\selectfont \fontsize{12}{10}\selectfont

\begin{thebibliography}{}

\item Adam R. Brown, Leonard Susskind, Ying Zhao.  \\ \textbf{Quantum Complexity and Negative Curvature. } \texttt{arXiv:1608.02612}
	
\item  Michael Freedman, Matthew Headrick. \textbf{Bit threads and holographic entanglement.} \texttt{arXiv:1604.00354}

\item Steven Gubser \textbf{Evolution of Segmented Strings} \texttt{arXiv:1601.08209}

\item Akshay Venkatesh \textbf{ Sparse equidistribution problems, period bounds, and subconvexity.} \texttt{arXiv:math/0506224}

\item Caroline Series \textbf{The Modular Surface and Continued Fractions} J. London Math. Soc. (1985) s2-31 (1): 69-80.

\item David Mumford \textbf{Lectures on Theta I} (on his web-site)

\item Don Zagier \textbf{Elliptic Modular Forms and Their Applications} (on his web-site)

\end{thebibliography}


\end{document}