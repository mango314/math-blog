\documentclass[12pt]{article}
%Gummi|065|=)
\usepackage{amsmath, amsfonts, amssymb}
\usepackage[landscape, margin=0.5in]{geometry}
\usepackage{xcolor}
\newcommand{\off}[1]{}
\DeclareMathSizes{20}{30}{21}{18}

\title{\textbf{ Some Interesting Formulas Involving the GCD }}
\author{John D Mangual}
\date{}
\begin{document}

\fontfamily{qag}\selectfont \fontsize{25}{30}\selectfont

\maketitle

Sometimes, when I read a String Theory paper, I try to find a verifiable statement.  Here is one I found in a paper:

$$ I^{\mathcal{N}=1^\ast}(N,1,0) = N \sum_{d| N} 1 = N \sigma_0(N) $$
This number is called a \textbf{superconformal index} and it also equals: 
$$ I^{\mathcal{N}=1^\ast}(N,N,n) =  \sum_{d| N} \sum_{l=1}^N \mathrm{gcd}(d,l ) $$
I was heckled on MathOverflow for posting such an elementary formula.  It's not mine, it's his.
\newpage

\fontfamily{qag}\selectfont \fontsize{12}{10}\selectfont

\begin{thebibliography}{}

\item arXiv:1606.01022 The Arithmetic of Supersymmetric Vacua. Antoine Bourget, Jan Troost. physics.hep-th.



\item arXiv:1511.03116 On the N=1* Gauge Theory on a Circle and Elliptic Integrable Systems. Antoine Bourget, Jan Troost. physics.hep-th.


\item arXiv:1506.03222 Counting the Massive Vacua of N=1* Super Yang-Mills Theory. Antoine Bourget, Jan Troost. physics.hep-th.

\item arXiv:1305.0318 Reading between the lines of four-dimensional gauge theories. Ofer Aharony, Nathan Seiberg, Yuji Tachikawa. WIS/03/13-APR-DPPA, UT-13-15, IPMU13-0081. physics.hep-th.



\end{thebibliography}


\end{document}
