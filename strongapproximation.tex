\documentclass[12pt]{article}
%Gummi|065|=)
\usepackage{amsmath, amsfonts, tabto, amssymb}
\title{Strong Approximation}
\usepackage{xcolor}
\usepackage[a4paper, total={6.5in, 10in}]{geometry}
\usepackage{framed}
\usepackage{tgadventor}
\colorlet{shadecolor}{red!10}
\author{John Mangual}
\date{}

\definecolor{green}{HTML}{BED46D}
\definecolor{blue}{HTML}{7A6BED}
\definecolor{red}{HTML}{7A6BED}
\usepackage{hyperref}

\begin{document}
{\fontfamily{lmss}\selectfont

\maketitle

\section{ The Pigeonhole Principle }


\hfill\begin{minipage}{\dimexpr\textwidth-2cm}

The chinese remainder theorem for $SL(n, \mathbb{Z})$ asserts, 
among other things that for $q > 1$ the reduction $SL(n, \mathbb{Z}) \to SL(n, \mathbb{Z}/q\mathbb{Z})$ is onto. Far less elementary is the extension of this feature to $G(\mathbb{Z})$ where $G$ is a suitable matrix algebraic group defined over $\mathbb{Q}$.  The general form of this phenomenon is known as Strong Approximation... \newline \newline
There is a quantification of the above that is not as well known as it should be, as it turns out to be very powerful in many contexts.  We call this ``superstrong"  approximation and it asserts that if we choose a finite symmetric generating set $S$ (such that $s, s^{-1} \in S$) then the congruence Cayley graphs $X_q(S)$ form an expander family as $q \to \infty$.

\xdef\tpd{\the\prevdepth}
\end{minipage} \newline

\noindent In this striking paragraph, Peter Sarnak sketches expansion from the Chinese Remainder Theorem to Strong Approxmation and even to Super-Strong Appproximation.  I can't hold back my surprise. \newline

% later FIX FORMATTING - do something fun and creative!

\noindent \textbf{Ex} Chinese Remainder Theorem is a way to solve simultaneous congruences.  Let $(p,q) = 1$ then:

$$ \mathbb{Z}_p \times \mathbb{Z}_q \simeq \mathbb{Z}_{pq}  $$

\noindent This isomorphism looks obvious to me since $p \times q = pq$.  In order to \texttt{prove} this isomorphism we need either the continued fraction of $\frac{p}{q}$ or the pigeonhole principle in order to ensure ourselves a solution to the congruence:
\begin{eqnarray*}
x &\equiv & 1 \mod p \\
x &\equiv & 0 \mod q 
\end{eqnarray*}

\noindent It is quite reasonable this factorization should extend to matrix groups.  Withouth commutativity there is more work.

$$ SL(2,\mathbb{Z}/pq\mathbb{Z}) \simeq SL(2,\mathbb{Z}/p\mathbb{Z}) \times SL(2,\mathbb{Z}/q\mathbb{Z})$$

\noindent There also seems to be some kind of connection to Hasse principle. \newline 

\noindent \textbf{Ex} It's hard to think of how strong Strong Approximation actually is.  The Adeles are the product of infinitely many completions of the rational numbers\footnote{ Categorical description of the restricted product (Adeles) \url{http://mathoverflow.net/a/96138/1358}} (not too much permitted after the decimal point):

$$ \mathbb{A} = \mathbb{R} \times \hat{\prod_{p \in \mathbb{Z}}} \mathbb{Q}_p $$

\noindent However, the maximally compact subgroup $\hat{\mathbb{Z}}$ is the product of all possible p-adic integers (everything before the decimal is permissible).

$$ \widehat{\mathbb{Z}} =  \prod_{p \in \mathbb{Z}} \mathbb{Z}_p $$

\noindent \textbf{Note} it is really surprising that $\mathbb{R}$ just appears out of nowhere in the completion.  The adeles seem to be a very formal way of thinking about the Chinese Remainder Theorem.  The beat goes on:  

$$ \mathrm{SO}_3(\mathbb{Z}) \backslash \mathrm{SO}_3(\mathbb{R}) \simeq 
\mathrm{SO}_3(\mathbb{Q}) \backslash \mathrm{SO}_3(\mathbb{A})/\mathrm{SO}_3(\widehat{\mathbb{Z}})$$

\noindent This is an ``Adelic" way of looking at the rotation group in 3 dimensions.  They even get the 3-sphere. Let:

$$ K_f[q] := \mathrm{ker} \bigg( SO_3(\widehat{\mathbb{Z}}) \to SO_3(\mathbb{Z}/q\mathbb{Z})\bigg)$$

\noindent The transitive action of $SO_3(\mathbb{R})$ is transitive, they use the stabilizer $K_\infty \simeq  SO_2(\mathbb{R})$ the sphere is

$$ S^2 \simeq  \mathrm{SO}_3(\mathbb{Q}) \backslash \mathrm{SO}_3(\mathbb{A})/K_\infty.K_f[3] $$

\noindent Here what are the rotation over the adeles?  Do they preserve $x^2 + y^2 + z^2 = 1$ in $\mathbb{A}$?

$$ \mathrm{SO}_3(\mathbb{A}) = \hat{\prod_{p \in \mathbb{Z}} }\mathrm{SO}_3(\mathbb{Q}_p)$$

\noindent \textbf{Ex} Next and more difficult to articulate are the extension to Cayley graphs and super-strong approximation. How does the expansion property of Ramanujan graphs generalize the Pigeonhole Principle or the Chinese Remainder Theorem??

$$ \mathcal{H}_d(q) \simeq \Gamma_{(3,q)} \backslash \mathrm{SO}_3(\mathbb{Q}_5) / \mathrm{SO}_3(\mathbb{Z}_5)$$

\noindent \textbf{Ex} Sarnak's article start with the case of ``thin" subgroups of algebraic groups $G(\mathbb{Z})$ of infinite index and checking that strong/superstrong approximation still applies there.  \cite{BGS, H}

\begin{thebibliography}{9}
\bibitem{S} 
Peter Sarnak.
\textit{Notes on Thin Matrix Groups}.  \newline
\texttt{http://web.math.princeton.edu/sarnak/NotesOnThinGroups.pdf}
 
\bibitem{EMV}
Jordan S. Ellenberg, Philippe Michel, Akshay Venkatesh. \textit{Linnik's ergodic method and the distribution of integer points on spheres}   \texttt{arXiv:1001.0897v1}

\bibitem{BGS}
Jean Bourgain, Alex Gamburd, Peter Sarnak. \textit{Generalization of Selberg's 3/16 Theorem and Affine Sieve }
\texttt{arXiv:0912.5021}

\bibitem{H} Harald Helfgott
\textit{Growth and generation in $SL_2(\mathbb{Z}/p\mathbb{Z})$}
\texttt{arXiv:math/0509024}

\end{thebibliography}
}
\end{document}
