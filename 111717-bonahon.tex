\documentclass[12pt]{article}
%Gummi|065|=)
\usepackage{amsmath, amsfonts, amssymb}
\usepackage[margin=0.5in]{geometry}
\usepackage{xcolor}
\usepackage{graphicx}

%\usepackage{pifont}
\usepackage{amsmath}

\newcommand{\off}[1]{}
\DeclareMathSizes{20}{30}{20}{18}

\newcommand{\two }{\sqrt[3]{2}}
\newcommand{\four}{\sqrt[3]{4}}
\newcommand{\red}{\begin{tikz}[scale=0.25]
\draw[fill=red, color=red] (0,0)--(1,0)--(1,1)--(0,1)--cycle;\end{tikz}}
\newcommand{\blue}{\begin{tikz}[scale=0.25]
\draw[fill=blue, color=blue] (0,0)--(1,0)--(1,1)--(0,1)--cycle;\end{tikz}}
\newcommand{\green}{\begin{tikz}[scale=0.25]
\draw[fill=green, color=green] (0,0)--(1,0)--(1,1)--(0,1)--cycle;\end{tikz}}

\newcommand{\sq}[3]{\draw[#3] (#1,#2)--(#1+1,#2)--(#1+1,#2+1)--(#1,#2+1)--cycle;}

\usepackage{tikz}

\newcommand{\susy}{{\bf Q}}
\newcommand{\RV}{{\text{R}_\text{V}}}

\title{Scratchwork: Miraculous Cancellations}
\date{}
\begin{document}

%\fontfamily{qag}\selectfont \fontsize{12.5}{15}\selectfont

\sffamily

\maketitle

\noindent Let's explore the ``miraculous cancellation" identites, proven by Francois Bonahon and Helen Wong.  These authors know a lot about 2D surfaces.  One the one hand, 2D surfaces are classfied by their genus, so why is a surface to these people a more complicated object?  Bonahon's latest argument shows these cancellation type arguments in a purely algebraic framework, withouth any discussion -- separate from the geometry of surfaces. \\ \\
\textbf{Proposition} Let $XY = q YX$ with $q^n = 1$.  Then $(X+Y)^n = X^n + Y^n$. \\ \\
\textbf{Lemma} Let $q^n = 1$ be ``primitive" root of unity, with $q^k \neq 1$.  The ``quantum binomial coefficent" $\binom{n}{k}_q =0$, for $1 < k < n$. \\ \\
Since we are researchers and not historians, I have to -- without any prior knowledge of the subject -- identify the unknown problems / conjecture to ``work on". 

\vfill

\begin{thebibliography}{}

\item Francis Bonahon \textbf{Miraculous cancellations for quantum $SL_2$} \texttt{arXiv:1708.07617}

\item Francis Bonahon, Helen Wong \\ 
\textbf{Representations of the Kauffman bracket skein algebra I: invariants and miraculous cancellations} \texttt{arXiv:1206.1638} \\
\textbf{Representations of the Kauffman bracket skein algebra II: punctured surfaces
} \texttt{arXiv:1206.1639} \\
\textbf{Representations of the Kauffman bracket skein algebra III: closed surfaces and naturality} \texttt{arXiv:1505.01522} 

\end{thebibliography}

\end{document}