\documentclass[12pt]{article}
%Gummi|065|=)
\usepackage{amsmath, amsfonts, tabto, amssymb}
\title{Numbers and Entropy}
\usepackage{xcolor}
\usepackage[a4paper, total={6.5in, 10in}]{geometry}
\usepackage{framed}
\usepackage{tgadventor}
\colorlet{shadecolor}{red!10}
\author{John Mangual}
\date{}

\definecolor{green}{HTML}{BED46D}
\definecolor{blue}{HTML}{7A6BED}
\usepackage{hyperref}

\begin{document}
{\fontfamily{lmss}\selectfont

\maketitle

\section{Random Sorting Networks}

We argue that random sorting networks are a discretization of the geodesic flow on the $n$-sphere.

\subsection{The Great Circle Conjecture}

A sorting network is a shortest past from $(123\dots n)$ to $(n\dots 21)$ in the $S_n$ Cayley graph generated by transpositions $(i,i+1)$.  

$$ d(\Delta_n) = \frac{\binom{n}{2}!}{1^{n-1}3^{n-2}5^{n-3} \dots (2n-3)^1}$$

\noindent This is also the number of staircase Tableaux, \cite{St}. \newline

\noindent \textbf{Ex} Let $\omega_n$ be an $n$-element uniform soritng network.  For each $n$ there exists a random great circle $C_n \in \mathbb{S}_n$ such that $$ || \omega_n - C_n ||_\infty < \alpha \sqrt{n} $$
The sphere is the intersection $\mathbb{S}_n = \big\{\mathbb{E}[x] = \frac{n (n+1)}{2} \big\} \cap \big \{ \mathbb{E}[x^2] = \frac{(n+1)(2n+1)}{6} \big\} $ maybe center and rescale for symmetry. 

\subsection{Grassmanian}
How can we parameterize great circles on the $n$-sphere $S^n$?  The circle has to start somewhere $x \in S^n$ and travel in some direction $T_x^1(S^n)\simeq S^{n-1}$.   So I have that great circles in the sphere are indexed by $$T^1(S^n)\approx S^n \times S^{n-1} $$

\noindent Here I use $\color{green}{\approx}$ to mean the tangent bundle is like the product, but the gluing maps are different.  I can't think of the correct vector bundle term or notation. \newline

\noindent Then rotation group $SO(n)$ acts on the space of great circles.  Certainly stabilizer of this group action is the copy of $SO(2)$ fixing the circle itself.  This this leaves $SO(n)/SO(2)$; hopefully I have not missed any additional symmetries.  \newline

\noindent The Grassmanian of 2-planes in $n$ dimenesional space is $SO(n)/SO(2)\times SO(n-2)$ since not only the great circle can be fixed (and the plane it spans) but also the complimentary subspace orthogonal to the plane.

\subsection{Back to $S_n$}

\noindent \textbf{Ex} All the points of $S_n$ can embed into a sphere of center and radius:

$$ \frac{1}{n}\sum_{i=0}^n i \times (1,1,\dots, 1) = \frac{1}{n}\binom{n+1}{2}(1,1,\dots, 1) $$ 

\noindent I am stumbling a bit on the radius computation:

\begin{eqnarray*} \sum_{i=0}^n (i - \mathbb{E}[x] )^2 &=& 
\sum_{i=0}^n i^2 -
2 \,\mathbb{E}[x]\sum_{i=0}^n i + n\,\mathbb{E}[x]^2 \\
&=& 
\sum_{i=0}^n i^2 -  n\mathbb{E}[x]^2 \\
&=&
\frac{n(n+1)(2n+1)}{6} - n  \bigg[\frac{n+1}{2}\bigg]^2 \\
&=& \frac{n(n^2 - 1)}{12} \end{eqnarray*}

\noindent Not sure it matters 100\% but mainly none of the authors of \cite{AHRV} take advantage of the symmetry of the Cayley graph problem they have. \newline

\noindent 

\noindent \textbf{Ex} In fact the sorting network should map to a half a great arc around the sphere.  How do we complete the other half?  I think we just do it twice.  Let $\sigma $ be a word in $\langle (12), (23), \dots (n-1,n) \rangle $

$$ (12\dots n) \stackrel{\sigma}{\to} (n\dots 21) \stackrel{\sigma}{\to} (12\dots n)$$

\noindent We can do a bit more and only get partial rotations using the conjugation map.

$$ \prod_{i=0}^k \prod_{i=k+1}^{\binom{n+1}{2}} \approx \prod_{i=k+1}^{\binom{n+1}{2}} \prod_{i=0}^k  $$

\noindent Starting from any Random Sorting Network we can get $\binom{n+1}{2}$ ``rotated" versions of the same network.  \newline

\noindent Is there a way to ``parallel translate" the action of $(\sigma, (ij)) \in S_n \times \langle (12), (23), \dots, (n-1,n)\rangle $ ?  There is some literature by Diaconis on ``random walks on groups" in the case when you allow all transpositions.  There some literature on nearest neighbor transposition random walks as well \cite{W}.  Here we have condition on the beginning and end of our process as well as how much time it should take and studying the stochastic process that emerges there. \newline

\noindent The easiest way to discretize the sphere is to sort the elements.  If $\sigma^{-1}(x) $ is increasing then $x \in \sigma$.  I am nervous we are reproducing the ``geometric" sorting networks also discovered by Omer Angel, but those live in $\mathbb{R}^2$, so I don't think he exploits the geometry of Lie groups in any way. \newline

\noindent So there are $S^{n-1}$ tangent directions to go half of which increase the inversion number and half do not.  These two sets are convex - closed under averages $(x,y)\mapsto xt + y\sqrt{1-t^2}$, $t \in [0,1]$. These are separated by a hyperplane. \newline

\noindent \textbf{Algorithm} reasonably, any great circle determines a sequence of permutations by watching the order of the elements as $R_\theta \vec{x}$ evolves $\theta:0 \to \pi$.  
\begin{itemize}
\item Do we get all sorting networks this way?
\item Do we get things that are not sorting networks?
\item Are all sorting networks from $(12 \dots n)  $ to $(n \dots 2 1 )$ equally likely?
\end{itemize}

\noindent If we choose Haar measure on the Grassmanian $\mathrm{Gr}(2,n) = SO(n) / (SO(2) \times SO(n-2))$ this uniformity should pass down to the sorting networks themselves. \newline

\noindent Consider $i(\theta) = \mathrm{inv}(R_\theta \vec{x})$ as a function of $\theta \in [0,2\pi]$.  The inversion number foliates the sphere into level sets along hyperplanes.  Any other hyperplane is transverse to this foluation, and so the inversion number must have a unique maximum and minimum.  Since the hyperplane cuts through both critical points $(12 \dots n)$ and $(n \dots 21)$ of this foliation, the inversion number increases monotically from $0$ to $\binom{n}{2}$, so these are always sorting networks. This rules out question \#2. \newline 

\noindent The answer to the first question \#1 is certainly \textbf{no}.  Consider the sorting network:

$$ \bigg[(12)(13)(14)\dots (1n)\bigg]\bigg[(23)\dots (2n)\bigg]\dots \bigg[ (n-1,n) \bigg]$$ 

\noindent This curve makes successive right angles and so is definite \textbf{not} a great circle.  This curve is also not typical.  We conjuecture that \textit{typical} random sorting networks will have approximately great arc shapes.  

\section{Calculus of Variations}

We now know that typical random sorting networks (but certainly not all of them) will be near great circles.  How do we prove this? 

$$  \mathbb{P}(f) = \mathrm{exp} \bigg[- \int_0^\pi |f(t)|^2 dt \bigg]$$

\noindent Probably they need to minimize a functional such as the distance functional on the sphere to get a geodesic.  Since this is a probability measure we probably just need to take the exponent. \newline


\begin{thebibliography}{9}

\bibitem{AHRV}
Omer Angel, Alexander E. Holroyd, Dan Romik, Balint Virag. \newline \textit{Random Sorting Networks}   \texttt{arXiv:math/0609538} 

\bibitem{W}
David Bruce Wilson.
 \textit{Mixing times of lozenge tiling and card shuffling Markov chains } \texttt{ arXiv:math/0102193}
 
\bibitem{WPW}
David Wilson, Yuval Peres, Elizabeth Winter. \textbf{Markov Chains and Mixing Times} \newline American Mathematical Society, 2008. 

\bibitem{St}
R. P. Stanley. On the number of reduced decompositions of elements of
Coxeter groups. European J. Combin., 5(4):359–372, 1984.
 
\bibitem{AS}
Menny Aka, Uri Shapira. \textit{On the evolution of continued fractions in a fixed quadratic field}  \texttt{arXiv:1201.1280}

\end{thebibliography}
}
\end{document}
