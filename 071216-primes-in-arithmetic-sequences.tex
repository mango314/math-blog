\documentclass[12pt]{article}
%Gummi|065|=)
\usepackage{amsmath, amsfonts, amssymb}
\usepackage[landscape, margin=0.5in]{geometry}
\usepackage{xcolor}
\usepackage{graphicx}
\newcommand{\off}[1]{}
\DeclareMathSizes{20}{30}{21}{18}

\title{\textbf{ Primes in Arithmetic Sequences, Entropy, Hyperbolic 3-Manifolds }}
\author{John D Mangual}
\date{}
\begin{document}

\fontfamily{qag}\selectfont \fontsize{25}{30}\selectfont

\maketitle

\noindent One way to show there are infinitely many primes is to show certain series diverges\footnote{I get bored just looking at such a series. ``Music of the primes" my foot!}.
$$ \sum_{p \in \mathcal{P} }\frac{1}{p}
= \frac{1}{2} + \frac{1}{3} 
+ \frac{1}{5} + \frac{1}{7}+ \frac{1}{11} + \dots = \infty $$
By \textbf {\color{red}fundamental theorem of arithmetic} $n = p_1^{a_1}\dots p_n^{a_n}$ and then
$$ \sum_{n=1}^\infty \frac{1}{n}= \prod_{p \in \mathcal{P}} ( 1 + \frac{1}{p} + \frac{1}{p^2} + \dots ) \asymp \mathrm{exp} \left[\sum_{p \in \mathcal{P}} \frac{1}{p} \right] = \infty$$
Both of these sequences diverge. I left out a tiny bit of work.
\newpage

\noindent In representation theory we learn about character theory over finite groups.  In our case $G = (\mathbb{Z}/q\mathbb{Z})^\times$ the multiplicative group of mod $q$ arithmetic.  And $q$ is prime (so there are no zero-divisors\footnote{In Haskell these could be \textbf{Maybe} types}). \newline

\noindent The group ring we are dealing with in this representation theory is: $\mathbb{C}[(\mathbb{Z}/q\mathbb{Z})^\times](s) $ and instead of doing Pigeonhole Principle\footnote{this is \textbf{Dirichlet's} Pigeonhole Principle} over $\mathbb{Z}/q\mathbb{Z}$ or even $(\mathbb{Z}/q\mathbb{Z})^\times$ we are doing it over the dual space of functions $f:(\mathbb{Z}/q\mathbb{Z})^\times \to \mathbb{C}^\times$. \newline

\noindent One of the character $\chi \equiv 1$ has infinitely many primes and the rest $\chi \not \equiv 1$ have finitely many at best\footnote{In quantum mechanics one might have $\langle p | x\rangle = e^{ipx}$ - with the position-space and momentum space.  Then $(x,p)\in \mathbb{C}^2$ is ``symplectic".}. \newline

\noindent I don't know if Schur's Lemma exists on a space like $(\mathbb{Z}/q\mathbb{Z})^\times $ but Dirichlet didn't worry about that.

\newpage

\noindent Let $L(1,\chi) = \sum \chi(n) n^{-s}$, then using finite fourier analysis:
$$ \frac{1}{q-1}\sum_\chi \log L(1, \chi)
= \sum_{p \equiv a(q)} \sum_{m=1}^\infty m^{-1} p^{-ms} 
\approx \sum_{p \equiv a(q)}  p^{-s}  \geq 0$$
I left out $p^m \equiv 1 \mod q$ since I am just copying the formula\footnote{Furstenberg might be pleased the solution set to $a^m \equiv 1 \mod q$ is the union of arithmetic sequences.}\footnote{OK.  The only terms that matter are $m = 1$.  This is also pigeonhole since we have subtracted away only finitely many.}. \newline

\noindent The pigeonhole principle reads:
$$ \prod_\chi L(1, \chi) \geq 1 $$  
and the problem is not all the characters can ``fit" inside a certain space\footnote{I am willing to bet this is a volume of some kind in the space of adeles $\mathbb{A} = \prod' \mathbb{Q}_p$.}.  The volume probably \textbf{is} an L-function.  
$$ L_1(s) < (1 - q^{-2})\zeta(s) < \frac{A}{s-1}$$
Plausible.
\newpage 

\noindent If $\omega \neq 1$, by the \textbf{Mean Value Theorem}

$$ L_\omega(s) - L_\omega(1) = (s-1)L'_\omega (s_1)$$

I would be challenged to find the value of $s_1$... this is a far cry from moving towards the tangent line of the parabola.

$$ L'_\omega(s) = \sum_{n \not \equiv 0 (q)} \chi(n) \, \log n \, n^{-s} \ll 1 $$

We have approximated $L(1, \chi)$ with a line.  The contradiction is something like:

$$ L(s, \mathbf{1})\times \dots \times L(s, \chi)\overline{L(s, \chi)} < A(s-1) \times 
\left[\frac{B}{s-1}\right]^2 $$
and if we let $s \to 1$ the product is zero, but we know this product is $\geq 1$. \newline

\noindent I can't decipher how this isabout prime number in arithmetic sequence.  We have yet to show $L(1, \chi) \neq 0$ for $\chi \in \{ 1, -1\}$  

\newpage

\fontfamily{qag}\selectfont \fontsize{12}{10}\selectfont

\begin{thebibliography}{}

\item JP Serre \textbf{Course on Arithmetic} Springer-Verlag

\item Davenport \textbf{Multiplicative Number Theory} Springer-Verlag



\end{thebibliography}


\end{document}
