\documentclass[12pt]{article}
%Gummi|065|=)
\usepackage{amsmath, amsfonts, amssymb}
\usepackage[margin=0.5in]{geometry}
\usepackage{xcolor}
\usepackage{graphicx}
\newcommand{\off}[1]{}
\DeclareMathSizes{20}{30}{21}{18}

\newcommand{\myhrule}{}

\usepackage{tikz}

\title{\textbf{ Examples:  Theta Functions }}
\author{John D Mangual}
\date{}
\begin{document}

\fontfamily{qag}\selectfont \fontsize{25}{30}\selectfont

\maketitle

\noindent I found some theta-functions that I like\footnote{I don't know anything about:
\begin{itemize}
\item black hole microstate counting
\item K3 surfaces
\item $\frac{1}{4}$-BPS state counting
\item Motivic Donaldson-Thomas invariants
\item elliptic genera
\end{itemize}
So I won't say much about them here.}:
$$ 8 \times \Bigg[ \left(\frac{\theta_2(\tau, z)}{\theta_2(\tau, 0)} \right)^2 +
\left(\frac{\theta_3(\tau, z)}{\theta_3(\tau, 0)} \right)^2 +
\left(\frac{\theta_4(\tau, z)}{\theta_4(\tau, 0)} \right)^2 \Bigg]
$$
I found it very interesting to read this formula is an ``entropy" -- and we might ask what it is counting. \\ \\
Unfortunately the other formulas were hard to read.  Theta functions are known to exhibit a very rich collection of symmetries.  

\newpage

\fontfamily{qag}\selectfont \fontsize{12}{10}\selectfont

\begin{thebibliography}{}



\item Shamit Kachru, Arnav Tripathy \textbf{The Hodge-elliptic genus, spinning BPS states, and black holes} \texttt{arXiv:1609.02158}
\item Miranda C.N. Cheng, Erik P. Verlinde. \textbf{Wall crossing, discrete attractor flow and Borcherds algebra.} \texttt{arXiv:0806.2337}
\item S. Kharchev, A. Zabrodin \textbf{Theta vocabulary I} \texttt{ arXiv:1502.04603}

\end{thebibliography}


\end{document}