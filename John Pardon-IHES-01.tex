\documentclass[12pt]{article}
%Gummi|065|=)
\usepackage{amsmath, amsfonts, tabto, amssymb}
\title{Contact homology and virtual fundamental cycles}
\usepackage{xcolor}
\usepackage[a4paper, total={6.5in, 10in}]{geometry}
\usepackage{framed}
\usepackage{tgadventor}
\colorlet{shadecolor}{red!10}
\author{John Pardon - Lectures @ IHES}
\date{}

\definecolor{green}{HTML}{BED46D}
\definecolor{blue}{HTML}{7A6BED}
\usepackage{hyperref}

\begin{document}
{\fontfamily{lmss}\selectfont

\maketitle

\begin{abstract}
These notes are related to the IHES course "Contact homology and virtual fundamental cycles" given by John Pardon in Fall of 2015.  Basically I copy the blackboards and add my own comments.  These are not meant to be authoritative, or even very accurate.  In fact, they are quite advanced and test the limits of my understanding. Experts should look at \cite{P}, instead. The hope is over the course of time, to give my own twist on the discussion.
\end{abstract}

\section{Lecture \#1}
 

\begin{itemize}
\item $Y^{2n-1}$ be odd-dimension manifold.
\item $\xi^{2n-2} \subseteq TY$ be a \textbf{hyperplane field} $\longleftrightarrow$ $\lambda$ with $\xi = \mathrm{ker} \lambda$
\begin{itemize}
\item $\xi$ defines a foliation (is integrable) iff $d\lambda|_\xi = 0$
\item $\xi$ defines a contact structure (is maximally non-integrable) 
iff $d\lambda|_\xi$ is non-degenerate, i.e. $\lambda \wedge (d\lambda)^{n-1} \neq 0$.
\end{itemize}
\end{itemize}

\noindent \textbf{Example}  $(\mathbb{R}^{2n+1} = \mathbb{R}_{r,\theta}^n \times \mathbb{R}_z, \xi_{\text{std}} = \mathrm{ker}(dz + \sum r_i^2 d\theta_i)) $ or $(\mathbb{R}^3, \xi_{\text{twist}} = \mathrm{ker}( \cos r \; dz + \sin r \; r d\theta) $  Bannequin's theorem says says in 3-dimensions, the standard contact structure is \textbf{not} equivalent to the overtwisted one. \newline

\noindent \textbf{Darboux Theorem} $(Y, \xi)$ is locally isomorphic to $(\mathbb{R}^{2n-1}, \xi_\text{std})$.  
\begin{itemize}
	\item All contact structures look locally the same. 
	\item Riemanninan metric have many local (diffeomorphism) invariants e.g. curvature. 
\end{itemize}

\noindent \textbf{Gray's Theorem} For any $(Y, \xi_t)$ of contact structures, $\exists$ isotopy $\phi_t: Y \to Y$ so that $\xi_t = \phi_t^\ast \circ \xi_0$
\begin{itemize}
	\item The moduli space of contact structures is \textbf{discrete}.  We can \textit{count}  them.
	\item $Y$ should be compact or we contradict Bannequin theorem. 
	\item Gray's theorem is false for foliations: no isotopy can achieve a deformation of foliations.  E.g. look at the mapping cylinder of $f:t \mapsto t + \alpha$ for $t \in S^1$ and $\alpha \in \mathbb{R}^1$.
\end{itemize}

\noindent \textbf{Gromov} Let $Y^{2n-1}$ be open, then we have a homotopy equivalence.
$$ \{ \xi \subseteq TY | \text{contact} \}  \leftrightarrow \{ \xi \subseteq TY | \text{almost complex structure} \} $$
\begin{itemize}
	\item .
	\item .
\end{itemize}


\begin{thebibliography}{9}


\bibitem{KS}
John Pardon. \textit{Contact homology and virtual fundamental cycles}  \texttt{arXiv:1508.03873}


\end{thebibliography}
}
\end{document}
