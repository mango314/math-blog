\documentclass[12pt]{article}
%Gummi|065|=)
\usepackage{amsmath, amsfonts, amssymb}
\usepackage[landscape, margin=0.5in]{geometry}
\usepackage{xcolor}
\usepackage{graphicx}
\newcommand{\off}[1]{}
\DeclareMathSizes{20}{30}{21}{18}

\title{\textbf{ Primes in Arithmetic Sequences, Entropy, Hyperbolic 3-Manifolds }}
\author{John D Mangual}
\date{}
\begin{document}

\fontfamily{qag}\selectfont \fontsize{25}{30}\selectfont

\maketitle

\noindent One way to show there are infinitely many primes is to show certain series diverges\footnote{I get bored just looking at such a series. ``Music of the primes" my foot!}.
$$ \sum_{p \in \mathcal{P} }\frac{1}{p}
= \frac{1}{2} + \frac{1}{3} 
+ \frac{1}{5} + \frac{1}{7}+ \frac{1}{11} + \dots = \infty $$
By \textbf {\color{red}fundamental theorem of arithmetic} $n = p_1^{a_1}\dots p_n^{a_n}$ and then
$$ \sum_{n=1}^\infty \frac{1}{n}= \prod_{p \in \mathcal{P}} ( 1 + \frac{1}{p} + \frac{1}{p^2} + \dots ) \asymp \mathrm{exp} \left[\sum_{p \in \mathcal{P}} \frac{1}{p} \right] = \infty$$
Both of these sequences diverge. I left out a tiny bit of work.
\newpage

\noindent Another way is to consider the set of numbers not divisible by any prime\footnote{Always watch this words ``consider" and ``let" in mth papers.}, call it $A$.
$$ \frac{1}{N}|A \cap \{ 1, \dots, N\}| = \frac{1}{N}|A_k \cap \{ 1, \dots, N\} | - O\left(\sum_{p > p_k} \frac{1}{p} \right)$$
The density\footnote{By \textbf{\color{red}{fundamental theorem of arithmetic}}, $A = \{ 1\}$, which has density $0$.} of $A$ is $0$.  Here $A_k$ is the integers after crossing out multiples of the prime number $p_k$.
$$ 0 = \prod_{p \in \mathcal{P}} \left( 1 - \frac{1}{p} \right) $$
As we run over all primes, we cross out all natural numbers\footnote{I do not like arguing by contradiction.  In the future, it could matter which one I pick.  Here it seems to be a matter of style.  It is best we continue...}. \newline \newline
\noindent There's not a whole lot more to say, but I thought this was a good test case.  Remember:
$$ \sum \frac{1}{n} = 1 + \frac{1}{2} + \left( \frac{1}{3} + \frac{1}{4}\right)
+   \left( \frac{1}{5} + \frac{1}{6}
+ \frac{1}{7} + \frac{1}{8}\right) + \dots  = \infty $$
\newpage

\noindent I wonder if there are other starting points for the infinitude of primes:
$$ 0 <  \frac{1}{5}< \frac{1}{4}<\frac{1}{3}< \frac{2}{5}<\frac{1}{2}< \frac{3}{5}<\frac{2}{3}< \frac{3}{4}< \frac{4}{5}< 1 $$
No obvious contradiction here.  Maybe there are enough prime powers to fill up the number line\footnote{This is how I waste my day... with dead-end questions like these.}.
\newline

\noindent In order to prove Legendre's 3-squares theorem, we need to show every arithmetic sequence has infinitely many primes
$$ \sum_{p \in \mathcal{P}_{a\mathbb{Z}+b}} \frac{1}{p}$$
Even though the proof uses L-series, we can get away with just $L(1)$.  In fact we have written it.

\newpage

\noindent Even before we do that let's meditate a little bit
\begin{itemize}
\item $ n \neq 4^a(8k+7)$
\item Can be solved with ``geometry of numbers" but still quite a mess\footnote{spoiler: we need Adelic geometry of numbers in order to unify all the places}
\end{itemize}
I thought I found a counterexample:
$ 4 = 2^2 + 0^2 + 0^2 $
\hrule \vspace{6pt}
$3 = 1^2 + 1^2 + 1^2 $ and $7 \neq $ but $21 = 3 \cdot 7 = 4^2 + 2^2 + 1^2$ \newline 

Some degenerate examples: $49 = 7^2$ and $25 = 5^2$ \newline

$2 \times 7 = 3^2 + 2^2 + 1^2 $ so we can get $8k+7$ sometimes. \newline

$3 \times 5 = 2 \cdot 8 + 7$ and we can't have $15 = \square + \square + \square$ \newline

$ 5 \times 7 = 35 = 5^2 + 3^2 + 1^2 $

\newpage


\fontfamily{qag}\selectfont \fontsize{12}{10}\selectfont

\begin{thebibliography}{}

\item JP Serre \textbf{Course on Arithmetic} Springer-Verlag

\item Davenport \textbf{Multiplicative Number Theory} Springer-Verlag



\end{thebibliography}


\end{document}
