\documentclass[12pt]{article}
%Gummi|065|=)
\usepackage{amsmath, amsfonts, amssymb}
\usepackage[margin=0.5in]{geometry}
\usepackage{xcolor}
\usepackage{graphicx}
\newcommand{\off}[1]{}
\DeclareMathSizes{20}{30}{21}{18}

\usepackage{tikz}

\title{\textbf{ Theta Functions }}
\author{John D Mangual}
\date{}
\begin{document}

\fontfamily{qag}\selectfont \fontsize{25}{30}\selectfont

\maketitle

$$\theta(x; p) = (x;p)_\infty (px^{-1}; p)_\infty = \exp \left( - \sum_{ m \neq 0} \frac{x^m}{m(1-p^m)} \right) $$
another one
$$\theta(z; q) := (z;q)_\infty (q/z ;q)_\infty = \frac{1}{(q;q)_\infty} \sum_{k \in \mathbb{Z}} z^k q^{\binom{k}{2}}  $$
the shifted factorials are defined by:
$$ (z;q)_\infty = \prod_{i \geq 0} (1 - z q^i)$$
Let's see if
$$ \binom{k}{2} = \frac{k(k-1)}{2} = \frac{k^2}{2} - \frac{k}{2}$$

\newpage

\noindent Then it could be:

$$ \theta( q^2 ; q) = \frac{1}{(q; q^2) }
\sum_{k \in \mathbb{Z}} q^k q^{2\binom{k}{2}}
 = \frac{1}{(q; q^2) }
\sum_{n \in \mathbb{Z}}  q^{n^2}
$$
Wikipedia has
$$ \sum_{n \in \mathbb{Z}}
a^{\frac{n(n+1)}{2}}
b^{\frac{n(n-1)}{2}}
= (-a; ab)_\infty (-b; ab)_\infty (ab; ab)_\infty
$$
and we can set $a = b = q$:
$$
\sum_{n \in \mathbb{Z}}
q^{n^2}
= (-q; q^2)_\infty (-q; q^2)_\infty (q^2; q^2)_\infty
 $$
 This also seems odd we can try
$$ \theta(q;q^2)
= (q;q^2)_\infty (q;q)_\infty (q^2; q^2)_\infty = \sum_{n \in \mathbb{Z}}q^{n^2}$$

\newpage 

\noindent It might parameterized in terms of two angles:
$$ \theta(z; \tau) = \sum_{n \in \mathbb{Z}} e^{\pi i n^2 \tau + 2\pi i n z}$$
which has another triple product
$$\prod_{m=1}^\infty
(1 - e^{2\i m i \tau})
\left[ 1 + e^{(2m-1)\pi i \tau + 2\pi i z} \right]
\left[ 1 + e^{(2m-1)\pi i \tau - 2\pi i z} \right]$$
Then $q = e^{2\pi i \tau}$ and $x = e^{2\pi i z}$:
$$\theta( 0; q) =  \prod (1 - q^2 )(1 + q^{2m-1})(1 - q^{2m+1}) $$
This is a beautiful triple product but we have to write in terms of rising and falling factorials.
$$
\sum_{n \in \mathbb{Z}}
q^{n^2}
= (-q; q^2)_\infty (-q; q^2)_\infty (q^2; q^2)_\infty
 $$
 \hrule
\vspace{12pt}
\noindent The exponent formula looks like
$$ \log(1-x) = \sum \frac{x^m}{m} $$
and the geometric series formula:
$$ \sum p^{km} = \frac{1}{1-p^m}$$
\newpage

\noindent If we put two of them together it says:
$$ \sum_m \sum_k \frac{1}{m}x^m p^{km} = \sum_m \frac{1}{m} \frac{x^m }{1 - p^m} $$
This is very much the logarithm in the beginning of this article. \newpage

\noindent \textbf{Part II}

\fontfamily{qag}\selectfont \fontsize{12}{10}\selectfont

\begin{thebibliography}{}

\item Taro Kimura, Vasily Pestun \textbf{Quiver elliptic W-algebras} \texttt{arXiv:1608.04651}
\item Wikipedia ``Jacobi Triple Product", ``Ramanujan Theta Function"
\item Eric M. Rains, S. Ole Warnaar \textbf{Bounded Littlewood identities} \texttt{arXiv:1506.02755}

\end{thebibliography}


\end{document}