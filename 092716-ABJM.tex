\documentclass[12pt]{article}
%Gummi|065|=)
\usepackage{amsmath, amsfonts, amssymb}
\usepackage[margin=0.5in]{geometry}
\usepackage{xcolor}
%\usepackage{graphicx}
\usepackage{graphics}
\newcommand{\off}[1]{}
\DeclareMathSizes{20}{30}{21}{18}

\newcommand{\myhrule}{}

\usepackage{tikz}

\title{\textbf{ Examples:  ABJM Theory }}
\author{John D Mangual}
\date{}
\begin{document}

\fontfamily{qag}\selectfont \fontsize{25}{30}\selectfont

\maketitle

\noindent There's not a whole lot to say until we write the formula:

$$ \int \left( \prod_i e^{ik\pi (\theta_i^2 - \phi^2)} \right) \times \frac{\prod_{i \neq j}\big(2 \sinh \pi (\theta_i - \theta_j) 2 \sinh \pi (\phi_i - \phi_j)\big)}{\prod_{i,j}(2 \cosh \pi (\theta_i - \phi_j))^2} $$
This integral could possibly be over $[0, 2\pi]^2$ but I am expecting it's over the real numbers:
$$ (\theta, \phi) \in \mathbb{R}^N \times \mathbb{R}^N$$
since this the domain of integration of the reals.  Here is the simplest one:
$$ \int_{-\infty}^\infty d\theta \; e^{i k \pi \theta^2} = \sqrt{\frac{\pi}{8}} (1+i)$$
I have not done that yet.  If we set $N=1$ or $N=2$ and this integral is totally feasible.

\newpage

\noindent ABJM theory could refer to two things:
\begin{itemize}
\item The $U(N) \times U(N)$ Chern-Simons-matter theory with $\mathcal{N}=6$ superconformal symmetry.  The first $U(N)$ is at level $k$ and the second $U(N)$ is at level $-k$.
\item The $U(N) \times U(N)$ Gaussian matrix integral the CS-theory localizes to.  
\end{itemize}
Both of these are called \textbf{ABJM theory}.  This is rather confusing. \\ \\
Here is another Chern-Simons formula:
$$ Z_\text{CS} (N, k) = \frac{1}{\sqrt{N+k}} \prod_{\alpha > 0} 2 \sin \frac{\pi \,\alpha \cdot \rho}{k+N} $$
where $\alpha_{ij} = e_i - e_j \in \mathbb{C}^N$ are vectors\footnote{This is called the \textbf{Cartan subalgebra} a term which should mean nothing right now.} and 
$$ \rho = \frac{1}{2} \sum_{\alpha > 0} \alpha = \sum_{i=1}^N \left(\frac{N+1}{2} - i \right) e_i $$
and also $k \in \mathbb{Z}$ is a positive integer, and so is $n \in \mathbb{Z}$. \\ \\
Therefore our Chern-Simons partition function $Z_\text{CS}$ should be a number.

\newpage

\noindent My understanding is that any localization result of a supersymmetric gauge theory that is not \textbf{flat} is indebted to Vasily Pestun\footnote{This is just looking at the citations.  The 4-sphere is a curved four-dimensional space, I guess.}.   \\ \\
Flat spaces are things like $\mathbb{R}^4$ and even $\mathbb{R}^4_{\epsilon_1, \epsilon_2}$ which are \textbf{distorted} versions of flat 4-dimensional space.  \\ \\
The curved spaces being considered are remarkably simple $\textbf{S}^3$ the 3-sphere and $\textbf{S}^4$ the 4-sphere\footnote{  In either case the procedure is the same:
\begin{itemize}
\item Do a very complicated algebra
\item Claim a localization result
\item Solve the integral
\end{itemize}
I will have a little bit to say about each of these steps.  Here are some old ideas that may help:
\begin{itemize}
\item Invariant Theory, Spherical Harmonics
\item Symlectic Geometry, ODE 
\item Integral Geometry
\end{itemize}}
\\

\noindent We get hints from Pestun's original paper:
\begin{quotation}
\noindent {\color{blue}{equivariant Euler class of the infinite-dimensional normal
bundle to the localization locus}}
\end{quotation}
I don't know what this means.   Maybe some complicated infinite dimensional object has been erected over our 4-sphere, $\text{S}^4$\,?
\newpage

\noindent The Chern-Simon's matter action is involved:
$$ \frac{k}{4\pi} 
\int \text{Tr} \big( A \wedge dA + \frac{2}{3} A^3 - \overline{\chi}\chi + 2D \sigma \big) $$
called a \textbf{multiplet} and a \textbf{superpotential} 
$$W = \frac{2\pi}{k} \epsilon^{ab} \epsilon_{\dot{a}\dot{b}}
\text{Tr}
(A_a B_{\dot{a}}
A_b B_{\dot{b}}) $$
and these are somehow going to localize. \\ \\
\textbf{Why do Path Integrals Localize to Gaussians}\footnote{
As soon as you hear the logic two things should happen:
\begin{itemize}
\item You should remember this has appeared in every single quantum mechanics textbook.
\item Realize this is entirely bull-shit.
\end{itemize}}\\ \\
An action depends on a function.  By calculus of variations:
$$ S(x + \delta x) = S(x) + \delta x \, S'(x) + \frac{(\delta x)^2}{2} S''(x)$$
Our job should be to set $\boxed{S'(x) = 0}$ this is our critical point.  Feynman writes the path integral:
$$ Z = \sum e^{S(x)} $$
However our logic is the same, this sum should concentrate \textit{near} the classical trajectory:
$$ Z \approx e^{x_\text{cl}} \sum_{x = x_\text{cl} + t \delta x}  e^{ \frac{1}{2}\, t^2 \,S''(x)}$$
\newpage

\fontfamily{qag}\selectfont \fontsize{12}{10}\selectfont

\begin{thebibliography}{}


\item Anton Kapustin, Brian Willett, Itamar Yaakov.  \textbf{Exact Results for Wilson Loops in Superconformal Chern-Simons Theories with Matter } \\ \texttt{https://terrytao.wordpress.com/2009/08/23/determinantal-processes/}

\item Ofer Aharony, Oren Bergman, Daniel Louis Jafferis, Juan Maldacena \textbf{N=6 superconformal Chern-Simons-matter theories, M2-branes and their gravity duals}
\texttt{arXiv:0806.1218v4}

\item Richard Feynman \textbf{Path Integrals and Quantum Mechanics} Dover, 2010.


\end{thebibliography}


\end{document}