\documentclass[12pt]{article}
%Gummi|065|=)

\usepackage{lmodern}
\usepackage{amsmath, amsfonts, amssymb}
\usepackage{url}
\usepackage[margin=1in]{geometry}
\usepackage{xcolor}
\newcommand{\ket}[1]{| #1 \rangle }
\newcommand{\braket}[2]{ \langle #1 | #2 \rangle}
\newcommand{\bra}[1]{ \langle #1 | }
\newcommand{\off}[1]{}


\title{\textbf{ Group Theory of Lagrange's 3-Squares Theorem}}
\author{John D Mangual}
\date{}
\begin{document}

{ \fontfamily{qag}\selectfont

\maketitle

\section{ \fontfamily{qag} \selectfont Group Theory}

``Number theory problems often relate to orbits of subgroups (periods) and so can be 
attacked by dynamical methods`` \newline

\noindent As $\mathbb{Z}$ is a group, I would like to believe that all number theory problems come from group theory.  Let's try a specific one:  Lagrange's sum of 3 squares theorem. \newline

\noindent Let $n = 4^j (8k+7)$ then $n = a^2 + b^2 + c^2$.  Like time's arrow one direction is easy and the other is hard.  This problem is obviously related to the sphere:
$$ S^2 = SO_3(\mathbb{Z}) \backslash SO_3(\mathbb{R}) = 
SO_3(\mathbb{A}) \backslash SO_3(\mathbb{A}) / SO_3(\hat{\mathbb{Z}})$$ 
but the sphere is not a group.  And this seems like a really complicated way of drawing the sphere.  Ellenberg-Venkatesh also quotient out ``obvious" $SO_3(\mathbb{Z})$ symmetries:
$$ \left(\begin{array}{ccc} \pm 1 & & \\ & \pm  & \\ & & \pm 1 \end{array}\right) 
\hspace{0.5in}\left(\begin{array}{ccc} 0 & 1 & \\ -1 & 0  & \\ & &  1 \end{array}\right)$$
The rotation group on the integers can't have that many symmetries.  It has to be smaller than the symmetries of the unit cube $\{ 0,1 \}^3$.
\newline

\noindent So what could be the group-theoretic content of the sum of three squares?
$$ \overline{\mathcal{H}_d} = SO_3(\mathbb{Z}) \backslash \mathcal{H}_d = 
\{ (\pm a,\pm b,\pm c): a^2 + b^2 + c^2 = d \} $$ 
This space has an adelic formulation.  Here ar ethe bijections:
$$ SO_3(\mathbb{Z})\backslash \mathcal{H}_d \rightarrow SO_3(\mathbb{Q})\backslash \mathcal{P} \leftarrow SO_{x_0}(\mathbb{Q}) \backslash SO_{x_0}(\mathbb{Q}) / 
SO_{x_0}(\mathbb{\hat{Z}}) $$ 
There is some notation pecular to that paper, so let's try a few:
\begin{itemize}
	\item $G = SO_{x_0}(\cdot)$ is the stabilizer of $x_0$ in $SO_3(\cdot)$.  The dot $\cdot$ could be $\mathbb{A}$ or $\mathbb{Z}$ or $\mathbb{Q}_p$ or $\mathbb{R}$.
	\item Therefore $SO_3(\cdot)$ is a kind of functor from rings (since in matrix multiplication we multiply and add things) to groups (the various groups of rotations on different spaces)
	\item $\{ (L, x): L \in \mathrm{genus}_{SO_3}, x \in L, x.x = d \} $
\end{itemize}
This last line looks like a Grassmanian or something we might see when studying Integral Geometry and the \textbf{crofton formula}.  This term \textbf{genus} might be the same genus, that appears in the theory of quadratic forms (e.g. in books of John Conway).
$$ \mathrm{genus}_{SO_3}(L_0) \simeq SO_3(\mathbb{A}_f) / SO_3(\hat{\mathbb{Z}}) = SO_3(\mathbb{Q})$$ 
All of these equations look very comlicated, but one important issue... if we are going to try to solve $n = a^2  + b^2 + c^2$ then we might try to start by finding rational rotations.  Here are two:
$$A = \left( \begin{array}{cr|c}
\frac{3}{5} & -\frac{4}{5} & 0\\  
\frac{4}{5} &  \frac{3}{5} & 0\\ \hline
0  & 0  & 1\\
 \end{array} \right) \hspace{0.5in} 
 B = \left( \begin{array}{c|rr}
 1 & 0 & 0 \\ \hline
0 & \frac{3}{5} & -\frac{4}{5} \\ 
0 & \frac{4}{5} &  \frac{3}{5} \\
 \end{array} \right)
 $$
 Showing these two matrices ``almost" generate $SO(3)$ we can say the \textbf{closure} of
 the matrices is all rotations:
 $$ \overline{\langle A, B \rangle }= SO_3(\mathbb{R})$$  This notion of closure is not exotic, I think people on the street just say ``almost" but our term closure is more correct.
 \newline

\noindent \dots \newline

\noindent \dots \newline

\noindent \dots \newline

\noindent We do not discuss matrix integrals at this time.

\begin{thebibliography}{9}

\bibitem{EMMV} arXiv:1503.05884 Effective equidistribution and property tau. Manfred Einsiedler, Grigory Margulis, Amir Mohammadi, Akshay Venkatesh.

\bibitem{LT} S. Lang and H. Trotter, \textbf{Continued fractions for some algebraic numbers}, J. Reine Angew. Math. 255 (1972), 112-134. https://eudml.org/doc/151239

\end{thebibliography}



}

\end{document}
