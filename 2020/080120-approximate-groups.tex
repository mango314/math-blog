\documentclass[12pt]{article}
%Gumm{\color{blue}i}|065|=)
\usepackage{amsmath, amsfonts, amssymb}
\usepackage[margin=0.5in]{geometry}
\usepackage{xcolor}
\usepackage{graphicx}
\usepackage{amsmath}
\usepackage{hyperref}

\usepackage{fontspec}
\usepackage{xcolor}

\newcommand{\off}[1]{}
\DeclareMathSizes{20}{30}{20}{18}
\usepackage{tikz}

%\setmainfont[Color=brown]{Linux Libertine}


\title{Reading: Approximate Groups}
\date{}
\begin{document}

\sffamily

\maketitle

{\fontsize{16pt}{16pt}\selectfont 

\noindent   \\ \\
 \textbf{Lemma 2.5.1} Let $G$ be a {\color{red!75!black}arbitrary} group and let $A \subset G$ b a finite subset with $|A^2| \leq K | A|$.  Then $|A^{-1}A|\leq K^2 |A|$ and $|A A^{-1}| \leq K^2|A|$. \\ \\
This is generalization to non-abelian groups, only for $m=1$ and $n=1$ \\ \\
\textbf{Theorem 2.3.1}  Let $G$ be an {\color{yellow!50!black}abelian} group and let $A, B$ be finite subsets of $G$.  
\begin{itemize}
	\item suppose that $|A + B| \leq K |A|$ then $|mA - nA| \leq K^{m+n}|A|$.
	\item if $|A+A| \leq K|A|$ then $|mA - nA| \leq K^{m+n}|A|$.
\end{itemize}
for all non-negatve integers $m,n$. \\ \\
These looking into the axioms of group theory.  There are several instances of group theory that we encounter in other branches of mathemamtics:
\begin{itemize}
\item permutation groups \\ $\text{ABCDE} \to \text{BCDEA} \to \text{CDEAB} \to \text{DEABC} \to \text{EABCD} \to [\dots]$
\item groups of substutions, e.g. 
$x \mapsto 3x + 2y \text{ and }y \mapsto 4x + 3y $
\item groups of transformations of physical objects (e.g. symmetries of square)
\end{itemize}
These symmetries in general were approximate since there was an enormous amount of work move and old the objects in a perfect evenly spaced circle. \\ \\
\textbf{Triangle inequality} Let $U,V,W$ be subsets of a group. There exists an injection:
$$ \phi: U \times V^{-1}W \to UV \times UW $$
In particular if $U,V,W$ are finite then $|U| \times |V^{-1}W| \leq |UV||UW|$. \\ \\ 
We're left wondering why the theorem is formatted in this particular way.  The proof use basic notions of algebra like ``function" an  ``inverse" and ``injection".
\begin{itemize}
\item $v : V^{-1} W \to V$ and $w : V^{-1} W \to W$ under then constraint 
that $x \in V^{-1}W$ leads to $x = v(x)^{-1} w(x)$.  
\item set $\phi(u,x) = (uv(x), uw(x))$.
\item check that $\phi$ is injective
\begin{itemize}
\item $(uv(x))^{-1} (uw(x)) = v(x)^{-1} w(x) = x $ so that $x$ is {\color{red}\textbf{uniquely}} determined by $\phi(u,x)$.  
\item $(uv(x)) v(x)^{-1} = u $ so that $u$ is uniquely determined by $\phi(u,x)$ and $x$.
\end{itemize}
The triangle ineqality has a logarithmic form:
$$ \log \frac{|V^{-1}W|}{|V|^{1/2}|W|^{1/2}} 
\leq \log \frac{|U^{-1}V|}{|U|^{1/2}|V|^{1/2}} + \log \frac{|U^{-1}W|}{|U|^{1/2}|W|^{1/2}}$$
\end{itemize}
Rusza's triangle inequality was appliled with all three set's being identical $U = V = W = A$.  In fact, that's the entire argument. \\ \\
So are have not made any spacifications like $G = \mathbb{Z}^2$ or $G = SU(2)$ or $G = \text{SL}_2(\mathbb{Z}[i])$ or $G = (\text{SL}_2(\mathbb{Z}) \backslash \text{SL}_2(\mathbb{R})[5]$ or anything else.  These were deduced abstractly from our guess on how the notion of multipicaton $\times$ or of how mirrors actually work. \\ \\
\textbf{Counterexample} Let $H$ be a finite group and let $G = H \ast \langle x \rangle$ (which is called the \textbf{free product}) of $H$ with the infinite cyclic group (basically a copy of $\mathbb{Z}$, e.g. how many times we do something).  Let $A = H \cup \langle x \rangle$ (this is called the ``union").  Then 
\begin{itemize}
\item $|A^2| \leq 3 |A|$ and yet
\item $HxH \subseteq A^3$ and $|HxH| = |H|^2 \asymp |A|^2$ these sets have the same number of elements without being the same set.  Their definitions look similar too.
\end{itemize}

\vfill

\begin{thebibliography}{}

\item \dots 

\end{thebibliography}
}
\end{document}