\documentclass[12pt]{article}
%Gumm{\color{blue}i}|065|=)
\usepackage{amsmath, amsfonts, amssymb}
\usepackage[margin=0.5in]{geometry}
\usepackage{xcolor}
\usepackage{graphicx}
\usepackage{amsmath}
\usepackage{hyperref}

\usepackage{fontspec}
\usepackage{xcolor}

\newcommand{\off}[1]{}
\DeclareMathSizes{20}{30}{20}{18}
\usepackage{tikz}

%\setmainfont[Color=brown]{Linux Libertine}


\title{Tune-Up: Decimals}
\date{}
\begin{document}

\sffamily

\maketitle

{\fontsize{16pt}{16pt}\selectfont 

\noindent 

\noindent Why are such advanced professors writing about decimals\dots?  Let's see if we can put these results further within reach. \\ \\
\textbf{Lemma} Let $\mu$ be a probablity measure on a finite set $S$, with $|S| = N$ and $H_\mu = \rho \log N$.  Let $\frac{\log 2}{\log N} < \delta <  \frac{1}{2}\rho$. There is a probablity measure $\mu \geq (\rho - \delta) \nu $ and $||\nu||_2^2 \leq 4 \rho^{-1} N^{-\delta} $. \\ \\
We are guessing based on the reading the text that ``finite subsets" $S \subseteq \mathbb{R}/\mathbb{Z}$ yet the statement at the moment says probability measure on \textit{any} finite set.  In fact this section is called 	``notations and preliminaries". \\ \\
Reading comprehension: how do we turn these into statements that actually make sense to us? Also their idea of ``finite" is worth remarking, that we could eventually break everything into \textbf{individual}  or \textbf{atomic} or \textbf{prime} parts.  Also finite here could mean $N = 10^{100}$ that's technically finite. \\ \\
Let $\mu(\{ s\}) = w_s$. \\ \\
Let $S_1 = \{ s \in S : w_s < 2 N^{-\delta} \}$ and $S_2 = S \backslash S_1$ and for $i = 1, 2$ let $\nu_i = (1/\mu(S_i)) \mu|_{S_i}$. \\ \\
Write $\cdot$ as the partition of $S$ into singletons. 
\begin{eqnarray*}
\rho \log N &=& H_\mu = H_\mu ( \{ S_1, S_2\} ) + H_\mu( \cdot | \{ S_1, S_2\} ) \\ 
&=& H_\mu( \{ S_1, S_2 \}) + \mu(S_1) H_{v_1} + \mu(S_2) H_{\nu_2} \\ 
&\leq & \mu(S_1) \log N + \delta \mu(S_2) \log N + \mu(S_1) \log 2  
\end{eqnarray*}
Here $ \nu_1$ is supported on at most $N$ elements, so $H_{\nu_1}\leq \log N $. \\
Here $ \nu_2$ is supported on at most $N^\delta/2$ elements, $ H_{\nu_2} \leq \delta \log N - \log 2$ \\ 
Observe that $H_\mu(\{S_1, S_2\}) \leq \log 2 $. \\ \\
Using these three entropy estimates, they measure  
$$ \nu(S_1) \geq \frac{ \rho - \delta }{ 1 - \delta + \log 2 / \log N } > \rho - \delta $$
The authors claim this is sufficient to prove the theorem by setting $\nu = \nu_1$ and observing
$$ ||\nu||_2^2 \leq ||\nu||_\infty \leq 2 \mu (S_1)^{-1} N^{-\delta} \leq 4\rho^{-1} N^{-\delta} $$
[exercise] \hfill $\square$ \\ \\
$H_\mu$ denotes an \textbf{entropy} measure of some kind.  It's not in our best interst to copy this proof at least without thinking about what all the parts mean.  Most of the numbere theory doesn't happen until section 3. \\ \\
\textbf{Effective Rudolph Theorem} Let $a, b$ be relatively prime integers.  Let $\mu$ be an arbitrary probability measure on $\mathbb{R}/\mathbb{Z}$ with entropy condition $H_\mu(\mathcal{P}_N) \geq \rho \log N $. There exist 
\begin{itemize}
	\item $\dots < \delta < \dots$ and $\dots < T < \dots$ and 
	\item $f \in C^1(\mathbb{R}/\mathbb{Z})$ non-negative. 
	\item $ 0 \leq s \leq \dots$ and $0 \leq t \leq \dots $  
\end{itemize}  
solving 
$$ [a^s b^t .\mu](f) \geq (\rho - 3 \delta) \lambda(f) - \kappa  \, T^{-\delta/2} ||f'||_2 $$
Here $||f'||_2^2 = \int_0^1 |f'(x)|^2 \, dx $. How does this build on the Chinese Remainder Theorem?  Why is intropy useful there? \\ \\
Our use of equations if rather interesting here.  We are absolutely sure the left thing is the same as the right thing of the equation (that the two sides of the scale balance).  Or in ``real" calculations we are shuffling around approximations, we just get another example of this kind of theorem.  What if we are  using computers and the decimal points run out?  Then we still get another case of these.  

\newpage

\noindent \textbf{03/27} What does {\color{blue} \textbf{effective}} mean? The other question is on mathematical writing and \textit{statements}.  Here is the experimental observation:
$$ \overline{\{ a^k \times b^\ell \times x : k, \ell \geq 0 \}} = \mathbb{R}/\mathbb{Z} $$ 
These are specialists and have presented this observation as number theory, a statement about the decimal system.  Here $\overline{X}$ is the \textit{closure} of $X$ with respect to a certain topology (or notion of ``closeness").\\ \\
\textbf{Rudolph-Johnson Theorem} 
\begin{itemize}
\item Let $\mu$ be a probablity measure on $\mathbb{R}/\mathbb{Z}$ invariant under $t_a$ and $t_b$
\item Let $a,b$ be multiplicatively independent.
\item \texttt{Suppose} $h_\mu( \times a) = \mu \log a $
\end{itemize}
Then $\mu \geq \eta \lambda$ for any $A \subset \mathbb{R}/\mathbb{Z}$ and $\mu(A) \geq \eta \lambda(A)$.  Here $\mu$ is \textit{any} probability measure, $\lambda = dx$ is Lebesgue measure (which may be an enormous amount of work to compute accurately, $\eta$ is a \textit{constant} -- what's that? \\ \\
\textbf{Effective Rudolph-Johnson Theorem}
\begin{itemize}
\item Let $a,b$ be multiplicatively indpendent.
\item $\mu$ be a probability measure on $\mathbb{R}/\mathbb{Z}$.  
\item Entropy condition $H_\mu(\mathcal{P}_N) \geq \rho \log N $ for some $\rho > 0 $ $N > N_0(a,b)$
\item $ 100/\log_a N \leq \delta \leq \rho/20 $ and $f \in C^1(\mathbb{R}/\mathbb{Z})$. 
\end{itemize}
There is an integer $m = a^sb^t < N$ so that
$$ [m.\mu](f) \geq (\rho - 3\delta) \lambda(f) - \kappa_1 \log (N)^{-\kappa_2 \delta} ||f'||_2 $$
where $\kappa_1, \kappa_2$ depend only on $a,b$.  We can't accept this.  We need to see what these constants and measures look like.  \\ \\
\textbf{Effective Rudolph Theorem}
\begin{itemize}
\item Let $a,b$ be relatively prime integers.
\item Let $\mu$ be an arbitrary probability measure on $\mathbb{R}/\mathbb{Z}$
\item Suppose an \textit{entropy condition} $H_\mu(\mathcal{P}_N \geq \rho \log N$ for some $\rho > 0$ and $N > N_0(a,b)$
\item Let $ \frac{10}{\log_a N} \leq \delta \leq \frac{\rho}{20}$ and $a^{20/\delta} \leq T \leq \frac{\delta}{4} \log_b(N)  $ and $f \in C^1(\mathbb{R}/\mathbb{Z})$.
\end{itemize}
Then there exist integers $s,t$ with 
\begin{itemize}
\item $ 0 \leq s \leq (1 - \delta) \log_a(N) $
\item $ 0 \leq t \leq T$
\item $[a^sb^t.\mu](f) \geq (\rho - 3\delta) \lambda(f) - \kappa T^{-\delta/2} ||f'||_2 $ 
\end{itemize}
Here $\kappa$ depends only on $a,b$ and $||f'||_2^2  = \int_0^1 |f'|^2 \, dt$  \\ \\
Out of these difficult statements, we get more elementary statements. \\ \\
\textbf{Thm} Let $a,b$ be multiplicatively independent with $(ab,N) = 1$ Then for any $m \in (\mathbb{Z}/N\mathbb{Z})^\times$ the set 
$$ \{ a^k \times b^\ell \times \frac{m}{N} : 0 < k, l < 3 \log N\} $$
This set is $\kappa (\log \log \log N)^{-\kappa/100} $ with depending only on $a,b$.  So we would spend the rest of the time looking for these scaling constants.  \\ \\
These subsets of the integers seem to be highly chaotic, so complicated that we can barely estimate the entropy to describe them. \\ \\
There are two different hypotheses:
\begin{itemize}
\item Let $a,b$ be \textbf{relatively prime}.
\item Let $a,b$ be \textbf{multiplicatively independent}.
\end{itemize}
What are the inputs in such a theorem?  Why are the constaints so specific? What would we like to see instead?  

\vfill

\begin{thebibliography}{}

\item Jean Bourgain, Elon Lindenstrauss, Philippe Michel, Akshay Venkatesh. \textbf{Some Effective Results of $\times a \times b$} Volume 29, Issue 6. December 2009 , pp. 1705-1722

\item 

\item

\end{thebibliography}
\end{document}