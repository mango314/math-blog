\documentclass[12pt]{article}
%Gumm{\color{blue}i}|065|=)
\usepackage{amsmath, amsfonts, amssymb}
\usepackage[margin=0.5in]{geometry}
\usepackage{xcolor}
\usepackage{graphicx}
\usepackage{amsmath}
\usepackage{hyperref}

\usepackage{fontspec}
\usepackage{xcolor}

\newcommand{\off}[1]{}
\DeclareMathSizes{20}{30}{20}{18}
\usepackage{tikz}

%\setmainfont[Color=brown]{Linux Libertine}


\title{Tune-Up: Weak Mixing}
\date{}
\begin{document}

\sffamily

\maketitle

{\fontsize{16pt}{16pt}\selectfont 

\noindent 

\noindent So if you have an average to compute, you might try to find a dynmical system and function (an ``observable") decide that it's ergodic, measure-preserving etc and show that the averages line up. \\ \\
Often our limiting measure could be something very simple, like $(\mathbb{R}, dx)$ or $(S^1, d\theta)$ (the angles of a circle). Or we infer our answer from these simple answer (this is why we do integrals). \\ \\
This is an example of an \textbf{interval exchange transformation}. 
$$ T(x) = \left\{ 
\begin{array}{c|c} 
x + (\sqrt{3} + \sqrt{5}) & 0 < x < \sqrt{2}\quad \; \; \, \\
x + (\sqrt{5} - \sqrt{3}) & \sqrt{2} < x < \sqrt{2} + \sqrt{3}\\
x - (\sqrt{2} + \sqrt{3}) & \sqrt{2} + \sqrt{3} < x < \sqrt{2} + \sqrt{3} + \sqrt{5} \end{array} \right. $$
Here is the graph of the function $x \mapsto T^{1000}(x)$. In fact, we need to be more specific we only sampled on a discrete set of points, $10^{-5}\mathbb{Z} \subseteq \mathbb{R}$ and we .  After a million points, we are testing the limits of my laptop computer or at least I have to make an explicit request to my computer, here using {\color{yellow!50!black}\texttt{Python 3}}.{\color{yellow!50!black}\texttt{6}}.{\color{yellow!50!black}\texttt{9}} \\ 
\includegraphics[width=0.5\textwidth]{measure-05.png} \\ \\
We can guess from the picture what the limiting measure $\mu$ might be.

\newpage 

\noindent The statements of the mean ergodic theorem (there's also a point-wise ergodic theorem) could be important or helpful. A measure-preserving system $(X, \mathcal{B}, \mu, T) $ is ergodic if and only iff 
$$ \frac{1}{N} \sum_{n=0}^\infty f \circ T^n \longrightarrow_{L^2_\mu} \int f \, d\mu $$
Here's another systement $(X, \mathcal{B}, \mu, T)$ is ergodic if and only if 
$$ \frac{1}{N} \sum_{n=0}^\infty \langle f \circ T^n, g \rangle \to \int f \, d\mu \int g \, d\mu $$
for any $f, g \in L^2_\mu$.  The book says we can optionally use set-theory that $(X, \mathcal{B}, \mu , T)$ is ergodic if 
$$ \frac{1}{N} \sum_{n=0}^\infty \mu (A \cap T^{-n} B) \to \mu(A) \mu(B) $$
these aveages are multiplicative \textit{in the limit}.  The definition of {\color{green!80!black}\textbf{mixing}} looks similar.
$$ \mu (A \cap T^{-n} B) \to \mu(A) \mu(B) $$
The \textbf{pre-images} are getting more and more shuffled until they are perfectly mixed \dots as mixed as they're ever going to get.  In fact, a (measure-preserving) dynamical system \\ \\
Two very innocent-looking culprits:
\begin{itemize}
\item \textbf{shape}
\item \textbf{area} (or ``volume" or ``measure")
\end{itemize}
If we try to balance things on a ``scale" what did we mean by that.  What were we measuring or counting? \\ \\
\textbf{Def} A measure-preserving system $(X, \mathcal{B}, \mu, T)$ is \textit{weak-mixing} if 
$$ \frac{1}{N} \sum |\mu(A \cap T^{-n} B)  - \mu(A) \mu(B)| \to 0 $$
as $N \to \infty$ for all measurable sets $A, B \in \mathcal{B}$. \\ \\
Ergodic theory is a topic developed in the 20th century and definitions of ``weak-mixing" and ``strong-mixing" were devised with something in mind that we'll hopefully figure out later.

\newpage

\noindent \textbf{Def} The following properties of the $(X, \mathcal{B}, \mu, T)$ are equivalent:
\begin{itemize}
\item $T$ is weakly mixing
\item $T \times T$ is ergodic with respect to $\mu \times \mu$
\item $T \times T$ is weakly mixing with respect to $\mu \times \mu$
\item For any ergodic measure preserving system $(Y, \mathcal{B}_Y, \nu, S)$ the system
$$ (X \times Y, \mathcal{B} \times \mathcal{B}_Y, \mu \times \nu, T \times S)$$
is ergodic.
\item The associated operator $U_T$ has no no-constant measurable eigenfunctions (that is, $T$ has continuous spectrum).  So if $f \in L^2_\mu$ then $(Tf)(x) = (f \circ T)(x)$ mapping functions to functions.
\item For every $A, B \in \mathcal{B}$ there is a set $J_{A,B} \subseteq \mathbb{N}$ with density zero for which 
$$ \mu(A \cap T^{-n} B) \to \mu(A) \mu(B) $$
as $n \to \infty $ with $ n \notin J_{A,B}$ (for increasing values of $n$ avoiding a zero-density set).
\item For every measurable set $A, B \in \mathcal{B}$:
$$ \frac{1}{N} \sum_{n=0}^N | \mu(A \cap T^{-n}B) - \mu(A)\mu(B)|^2 \to 0 $$
as $N \to \infty$.
\end{itemize}
So I don't know how the textbook proceeds without \textit{examples} of weak-mixing systems.  Maybe that's why it's called Ergodic Theory.  These definitions and the fact that they're equivalent will take a while to digest and to argue that these definitions are equivalent in some way. \\ \\
\textbf{Example} what if we aren't sure about the location about the integers $\mathbb{Z} \subseteq \mathbb{R}$ or possibly or we only have an approximate value of $T$, let's say the first $10^5$ terms of the Fourier series.  So that ergodic, mixing weak-mixing has to be thought about considerably.  Maybe an easy literature search will provide an answer there. \\ \\
The goal of the textbook is Szemeredi theorem and the mixing of the geodesic and horocycle flows on $\text{SL}_2(\mathbb{Z})\backslash \mathbb{H}$ which have to do with fractions.  There's a similar question there, since $\mathbb{Q}$ is dense in $\mathbb{R}$ and since our measurements are always prone to error, how did know which fracton was best? \\ \\
It clearly says ``measure theory" and "Borel $\sigma$-algebra" and all we're doing is this tiny bit.


\vfill

\begin{thebibliography}{}

\item Manfred Einsiedler, Thomas Ward \textbf{Erodic Theory with a View Towards Number Theory} (GTM \#259) Springer, 2011.

\end{thebibliography}

\end{document}