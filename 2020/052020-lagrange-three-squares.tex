\documentclass[12pt]{article}
%Gumm{\color{blue}i}|065|=)
\usepackage{amsmath, amsfonts, amssymb}
\usepackage[margin=0.5in]{geometry}
\usepackage{xcolor}
\usepackage{graphicx}
\usepackage{amsmath}
\usepackage{hyperref}

\usepackage{fontspec}
\usepackage{xcolor}

\newcommand{\off}[1]{}
\DeclareMathSizes{20}{30}{20}{18}
\usepackage{tikz}

%\setmainfont[Color=brown]{Linux Libertine}


\title{Tune-Up: Lagrange 3-Squares Theorem}
\date{}
\begin{document}

\sffamily

\maketitle

{\fontsize{16pt}{16pt}\selectfont 

\noindent  \textbf{Problem Statement} For a rational two-dimensional subspace $L$ of $\mathbb{R}^4$ the discriminant of $L$ is the square of the co-volume of $L \cap \mathbb{Z}^2$ in $L$:
$$ \text{disc}(L) = \text{vol}(L / (L \cap \mathbb{Z}^4)^2 \in \mathbb{N} $$
For any $D \in \mathbb{N}$ let $\mathcal{R}_D$ be the finite set of rational planes of discriminant $D$ (which is a subset of the real Grassmanian 
$\text{Gr}_{2 \times 4}(\mathbb{R})$.\\ \\
\textbf{Theorem}(Folklore, 1930s ?) $\mathcal{R}_D$ is non-empty iff
$$ D \in \mathcal{D} := \big\{ D \in \mathbb{N} : D \not \equiv 0,7,12,15, \pmod {16} \big\} $$
This is analogous to Legendre's theorem of sums of three squares, and works in the early 20th century on the sums of four squares.  
\\ \\
Let's read their conjecture.   \\ \\
\textbf{Conjecture}(2019) The normalized counting measure on the finite set
$$ \mathcal{J}_D = 
\big\{ (L, [L(\mathbb{Z})], [L^+(\mathbb{Z})], [\Lambda_{a_1(L)},, [\Lambda_{a_2(L)}] : L \in \mathcal{R}_D \big\} \subseteq \text{Gr}(\mathbb{R}) \times \mathcal{X}_2^4 $$
equidistributes ot the uniform probability measure on $\text{Gr}_{2 \times 4} (\mathbb{R}) \times \mathcal{X}_2^4$ as $D \to \infty$ with $D \in \mathbb{D}$.  That is:
$$ 
\frac{1}{\mathcal{J}_D}
\sum_{x \in \mathcal{J}_D}
\delta_x \to m_{\text{Gr}_{2 \times 4} \times \mathcal{X}_2^4}  
$$ 
in the weak*-topology where $m_{\text{Gr}_{2 \times 4}}$ is the probability measure obtained from an $SO(4)$ invariant maesure on $\text{Gr}_{2 \times 4}(\mathbb{R})$ and an 
$\text{SL}_2(\mathbb{R})$-invariant measure on $\mathbb{H}^2$ \\ \\
There's an infinite amount of geometric background here.
\begin{itemize}
	\item E.g. what does weak* convergence of measures look like?
	\item $SO(4)$-invariant measureis usually given careful discussion in Lie algebra textbooks for some reason.
\end{itemize}
Here's what they actually prove (with congruence conditions).  Resting on this conjecture (like a table) we could try to do a lot of numerical experimentation. \\ \\
\textbf{Theorem} (Equidistribution) Let $p,q$ be any two odd primes.  The normalized counting measure on the (finite set) $\mathcal{J}_D$ equidistributes to uniform probability measure on 
$ \text{Gr}_{2 \times 4} (\mathbb{R}) \times \mathcal{X}_2^4$ 
as $D \in \mathbb{D}$ goes off to infiity while $D$ has the dditional conditions $-D \in (\mathbb{F}_P^\times)^2$ and $-D \in (\mathbb{F}_q^\times)^2$.  \\ \\
These innocent counting-measure problems have been argued using modular forms (in particular Maass forms).  Results of this type have only been around since the 1950's.   \\ \\
Results due to Maass himself show that $(L, L(\mathbb{Z})) $ are equidistributed for rational planes with equidistribution up to $D$. 
\begin{itemize}
\item $L(\mathbb{Z}):= L \cap \mathbb{Z}^4$
\end{itemize}
\textbf{Exercise} Let $L = \{ \frac{3}{5}x + \frac{4}{7} + \frac{5}{11}z + \frac{6}{13}w = 0  \}\cap \{ \frac{1}{7}x + \frac{2}{11} + \frac{6}{5}z + \frac{7}{13}w = 0  \}\subseteq \mathbb{Q}^4$. 
\begin{itemize}
\item Solve the equation in rational numbers $\mathbb{Q}$.
\item  Is $L(\mathbb{Z}) = \varnothing$ ? 
\item Find the co-volue $[L: (L \cap \mathbb{Z}^4)]$.
\end{itemize}
This one exercise should already illustrate the complexity of the equation involved, before we ask about a limiting statement.
\vfill

\begin{thebibliography}{}

\item Manfred Einsiedler, Elon Lindenstrauss \textbf{Joinings of Higher Rank Torus Actions on Homogeneous Spacs} \texttt{arXiv:1502.05133}
\item Manfred Einsiedler, Manuel Luethi, Nimish Shah \textbf{Priitive Rational Points on Expanding Horocycles in Products of the Modular Surface with the Torus} \texttt{arXiv:1901.03078} 
\item Manny Aka, Manfred Einsiedler, Andreas Wieser \textbf{Planes in Four-Space and Four Associated CM Points} \texttt{arXiv:1901.05833}
\end{thebibliography}
}
\end{document}