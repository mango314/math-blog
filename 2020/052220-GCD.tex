\documentclass[12pt]{article}
%Gumm{\color{blue}i}|065|=)
\usepackage{amsmath, amsfonts, amssymb}
\usepackage[margin=0.5in]{geometry}
\usepackage{xcolor}
\usepackage{graphicx}
\usepackage{amsmath}
\usepackage{hyperref}

\usepackage{fontspec}
\usepackage{xcolor}

\newcommand{\off}[1]{}
\DeclareMathSizes{20}{30}{20}{18}
\usepackage{tikz}

%\setmainfont[Color=brown]{Linux Libertine}


\title{Tune-Up: GCD}
\date{}
\begin{document}

\sffamily

\maketitle

{\fontsize{16pt}{16pt}\selectfont 

\noindent  How does the GCD of Number theory map to Chaos Theory ? \\ \\
First of all number theory handles numbers to arbitrary accuracy.  That is itself an ideal.  When we said $\sqrt{2}$ we means the number $a$ such that $a  \times a = 2$ and we realize that no fraction $a = \frac{p}{q}$ is every going to solve this. \\ \\
If we say $\sqrt{2} \pm 10^{-3}$ here are now an infinte number of fractions and possiby even some very good estmatimes.  $|\sqrt{2} - \frac{99}{70}| < 10^{-4}$ approximations like these have been around (on some place of the globe) for for the past 4000 years.  So lastly we are looking for very ``efficient" estimates of this kind, estimates that are unusually good compared to the job they were trying to do. \\ \\
So even the rational numbers $\mathbb{Q}$ are build from 
\begin{itemize}
\item Ratios of numbers of $\mathbb{Z}$.  When did we care about ``ratio" or  ``proportion"?  
\item $\mu(\mathbb{Q}) = 0$ subset of measure zero in $\mathbb{R}$
\item rational numbers emerging out of piles of observations of the ``real world".  (Refer to previous two steps). 
\end{itemize}
We're stuck because we're losing motiation and our result heavily depends on our choice of motive.  Then maybe later on, the characters again look like $\mathbb{Q}$ and $\mathbb{R}$ and $\mathbb{Z}$.  Our approimation of a data stream as $\mathbb{Z}$ lost a lot of information which we may try to recover at some point.  \\ \\
So there's lot of philosophy here.  Without further ado, this result goes back to only 1926 with improvements in the 1940's.
$$ \frac{1}{\phi(n)} \sum_{(k,n) = 1} f(\frac{k}{n}, \frac{\overline{k}}{n} \to \int_{\mathbb{T}^2} f(x,y) dm(x,y)$$
Look at that.  The choice of Euler phi function $\phi$ is regardless of how we chose to extract our ideal number line $\mathbb{Z}$ from the messy ``real world" of obversations that aren't even numbers.  They are experimental averages that change with the weather. \\ \\
In 2016, we get interpretation of this result as geometric problem involving sphere packings, ``horospheres" with chaotic orbits.  It's unlikely we've had enough time as a civilization to absorb what all of this meant.  For example, the 2016 paper asks if certain horocycle flow obits are \textbf{closed} and argues in the affirmative.  This object must be pretty complicated.\\ \\
Here the lattice is $\Gamma = \textbf{SL}_2 (\mathbb{Z})$ 
and the group is $G = \text{SL}_2(\mathbb{R})$ as we've said many times.  Now I have to try to map our number theory problem into this group action $\Gamma \backslash G$ framework.  There doesn't seem to be a universal way, this is just reflective of how we have choices in the matter. \\ \\
\textbf{05/26} In this particular paper there are the notations:
\begin{itemize}
\item $G = \text{SL}_2(\mathbb{R})$ 
\item $\Gamma = \text{SL}_2(\mathbb{Z})$
\end{itemize}
Ostensibly these are the groups of $2 \times 2$ matrices with real and integer entries.  So we can look at upper-triangular matrices:
$$  U = 
\left\{ \left( \begin{array}{cc} 1 & t \\ 0 & 1\end{array}\right) : t \in \mathbb{R} \right\}$$
and the lower triangular matrices:
$$  V = 
\left\{ \left(\begin{array}{cc} 1 & 0 \\ t & 1\end{array}\right) : t \in \mathbb{R} \right\}$$
They also list the diagonal matrices:
$$  A = 
\left\{ \left(\begin{array}{cc} n & t \\ 0 & \frac{1}{n} \end{array}\right) : n  \in \mathbb{Z} \right\} \text{ or }
\left\{ \left(\begin{array}{cc} y & t \\ 0 & \frac{1}{y} \end{array}\right) : n  \in \mathbb{R} \right\}$$
These groups describe linear transformations in the Euclidean plane such as translations and rotations and perspective transforms (e.g. from Projective Geometry).  So These geometric objects have rich historical and cultural context. \\ \\
These results are part of a larger program to connect number theory and geometry and dynamical systems and this seems to be fairly modern idea even though we can backtrack and find lots of historical examples.  This doesn't seem to be the prevailing way of thinking.  We have to think about what these objects are the \textbf{horospheres}. \\ \\
\textbf{Exercise} $\text{SL}_2(\mathbb{R} \times \mathbb{Q}_p) \simeq \text{SL}_2(\mathbb{R}) \times \text{SL}_2(\mathbb{Q}_p) $ (this is ``functorial" property of special linear group, it also places number theory and Eucliden geometry on equal footing). \\ \\
\textbf{Exercise} $\text{SL}_2(\mathbb{Z}[\frac{1}{p}]) $ is a lattice in $\text{SL}_2(\mathbb{R} \times \mathbb{Q}_p)$.  Example:
$$ a_p = \left( \left(\begin{array}{cc} p & 0 \\ 0 & p^{-1} \end{array} \right), \left(\begin{array}{cc} p & 0 \\ 0 & p^{-1} \end{array} \right)\right) $$
What does ``lattice" mean in this setting?
\newpage
\noindent \textbf{Example} $\mathbb{Z}[\frac{1}{p}] \backslash \mathbb{Q}_p$  is an arithmetic ``extension" of the torus $\mathbb{T} = \mathbb{Z} \backslash \mathbb{R} $. \\ \\
\textbf{Q} Why do we need $p$-adic Sobolev norms to discuss averages of the GCD function? \\ \\
These objects are stated in generality yet judging from the writing, nobody is sure what these arithmetic objects are.0522
}
\vfill

\begin{thebibliography}{}

\item Manfred Einsiedler, Manuel Luethi, Nimish Shah. \textbf{Primitive Rational Points on Expanding Horocycles in Products of the Modular Surface with the Torus} \texttt{arXiv:1901.03078}

\item 

\end{thebibliography}
\end{document}