\documentclass[12pt]{article}
%Gumm{\color{blue}i}|065|=)
\usepackage{amsmath, amsfonts, amssymb}
\usepackage[margin=0.5in]{geometry}
\usepackage{xcolor}
\usepackage{graphicx}
\usepackage{amsmath}
\usepackage{hyperref}

\usepackage{fontspec}
\usepackage{xcolor}

\newcommand{\off}[1]{}
\DeclareMathSizes{20}{30}{20}{18}
\usepackage{tikz}

%\setmainfont[Color=brown]{Linux Libertine}


\title{Tune-Up: Eisenstein Series}
\date{}
\begin{document}

\sffamily

\maketitle

{\fontsize{16pt}{16pt}\selectfont 

\noindent Maybe the reason that recitation is such a important teaching technique \dots is we have no idea what anything is yet.  Hilbert space here is a theoretical object containing all the possible types of things we could imagine.  The other problem is build a ``new" one.   \\ \\
\textbf{Thm} The Hilbert space $L^2(\Gamma \backslash \mathbb{H})$ is the orthogonal sum 
$$ L^2(\Gamma \backslash \mathbb{H}) = \overline{\mathcal{C}(\Gamma \backslash \mathbb{H})} \oplus \overline{\mathcal{E}(\Gamma \backslash \mathbb{H})} $$
of the closures of $\mathcal{C}(\Gamma \backslash \mathbb{H})$ and $\mathcal{E}(\Gamma \backslash \mathbb{H})$ (the space of incomplete Eisenstein series).  \\ \\
There's an infinite amount of content here.  Why did we choose these particular abstractions in the first place?  Maybe we like Number Theory or Hyperbolic Geometry. 
\begin{itemize}
\item What does ``incomplete" or ``closure" mean here?
\item In many parts of the chapter $\Gamma = \text{SL}_2(\mathbb{Z})$ can also be congruence group $\Gamma_0(N)$.
\end{itemize}
\textbf{Argument} \\ \\
* This proof (as usual) rests on a previous lemma and a bunch of definitions.   \\ \\
* The strip $\{ z \in \mathbb{H}: |x| < \frac{1}{2} \} $ is fundamental domain for $\Gamma_\infty$. \\ \\
* \textbf{Ex} Let $f \in L^2(\Gamma \backslash \mathbb{H}) $.  Use the Cauchy-Schwartz inequality and the finiteness of the area of $D$ (the fundamental domain of $\text{SL}_2(\mathbb{Z}) \backslash \mathbb{H} = \{ |x| < \frac{1}{2} \} \cap \{ |z| < 1  \}$) to show that $|f|$ is integrable in $D$.
\begin{eqnarray*}
\langle f , E(\cdot | \psi) \rangle  &=& \frac{1}{2} \int_D f(z) \sum_{\gamma \in \Gamma_\infty \backslash \Gamma} \overline{\psi}(\text{im}(\gamma z)) \, d\mu(z)  \\ 
&=& \frac{1}{2} \sum_{\gamma \in E} \int_{\gamma D} f(z) \overline{\psi}(y) \, d\mu(z) = \int_S f(z) \overline{\psi}(y) \, d\mu(z) \\ 
&=& \int_0^{+\infty} \left( \int^{+\frac{1}{2}}_{-\frac{1}{2}} f(z) \, dx  \right) \overline{\psi}(y) \, y^{-2} dy
\end{eqnarray*}
The inner term $(\cdot)$ is the  Fourier series (notice integral with respect to $dx$. 
$$ \langle f , E (\cdot | \psi) \rangle = \int_0^{+\infty}  f_0(y) \overline{\psi} (y) y^{-2} dy  $$
Assume\footnote{[Totally unrealistic thing.]} that $f \perp \mathcal{E}(\Gamma \backslash \mathbb{H})$ -- that $f$ is ``orthogonal" to $\mathcal{E}(\Gamma \backslash \mathbb{H})$.
\textbf{Ex} Prove the two statements are equivalent.
\begin{itemize}
\item * is 0 for all $\psi$
\item $f \in \overline{ \mathcal{C} (\Gamma \backslash \mathbb{H}})$
\end{itemize}
Thus concludes the proof \hfill $\square$ \\ \\
Group theory offers us nice succinct arguments for why something can or cannot (eventually) occur. That doesn't mean we know what any of the details look like.
\begin{itemize}
\item $d\mu(z)$ is the Lebesgue measure in $\text{SL}_2(\mathbb{Z}) \backslash \mathbb{H}$.  So there is both group theory and measure theory even when we were just starting out with fractions.
\item What are we using \textbf{Eisenstein} series for?  And why are they ``incomplete"?
\item The use of ``closures" and functional anaysis for what are basically number theory and geometry of circles problems. What are $\overline{\mathcal{C}(\Gamma \backslash \mathbb{H}})$ and $\overline{\mathcal{E}(\Gamma \backslash \mathbb{H}})$?
\end{itemize}
Intersetingly we retain the Dirac bra-ket notion $\langle a | b \rangle $ or the inner product notation $\langle , \rangle$.

\vfill

\begin{thebibliography}{}

\item \dots

\end{thebibliography}
\end{document}