\documentclass[12pt]{article}
%Gumm{\color{blue}i}|065|=)
\usepackage{amsmath, amsfonts, amssymb}
\usepackage[margin=0.5in]{geometry}
\usepackage{xcolor}
\usepackage{graphicx}
\usepackage{amsmath}
\usepackage{hyperref}

\usepackage{fontspec}
\usepackage{xcolor}

\newcommand{\off}[1]{}
\DeclareMathSizes{20}{30}{20}{18}
\usepackage{tikz}

%\setmainfont[Color=brown]{Linux Libertine}


\title{Tune-Up: Topological Groups}
\date{}
\begin{document}

\sffamily

\maketitle

{\fontsize{16pt}{16pt}\selectfont 

\noindent Number theory offers general results about numbers as true, how do we turn these into more actionable statements?
$$  \mathbb{Z}_p = \lim_{\leftarrow} \mathbb{Z}/p^n \mathbb{Z}$$
This result doesn't say anything about \textbf{decimal representation} of numbers or what \textbf{additive structur}  is.  Since when do we care about 20 digit numbers to exact precision and still have addition?  \\ \\
Let $K$ be complete with respect to a discrete valuation.  There are homomorphism(s):
\begin{itemize}
\item $ \mathcal{O} \to \mathcal{O}/\mathfrak{p}^n $
\item $ \mathcal{O}/\mathfrak{p} \leftarrow \mathcal{O}/\mathfrak{p}^2 \leftarrow \mathcal{O}/\mathfrak{p}^3 \leftarrow \dots $
\item $\displaystyle \mathcal{O} \to \lim_{\longleftarrow} \mathcal{O}/p^n $
\end{itemize}
This gives us a whole slew of exotic \textbf{number systems} as \textbf{projective limits} of rings of various kinds.
$$ \lim_{\longleftarrow} \mathcal{O}/\mathfrak{p}^n  = \big\{ (x_n) \in \prod_{n=1}^\infty \mathcal{O}/\mathfrak{p}^n : \lambda_n( x_{n+1}) = x_n  \big\}  $$
\textbf{Proposition} The canonical mappings are isomorphisms and homeomorphisms:
\begin{itemize}
\item $\displaystyle \mathcal{O} \;\;\to \lim_{\longleftarrow} \mathcal{O}/\mathfrak{p}^n $
\item $\displaystyle \mathcal{O}^\times \to \lim_{\longleftarrow} \mathcal{O}^\times/U^{(n)} $
\end{itemize}
The difference between $\mathbb{Z}_{10}$ possibly as $\displaystyle \lim_{\longleftarrow} \mathbb{Z}/{10}^n \mathbb{Z} \simeq \lim_{\longleftarrow} \mathbb{Z}/2^n \mathbb{Z} \times \lim_{\longleftarrow} \mathbb{Z}/5^n \mathbb{Z}$ and $\mathbb{R}$ is \dots  They have very different shapes, they are connected very different.

\vfill

\begin{thebibliography}{}

\item Terence Tao, Van Vu.  \textbf{Additive Combinatorics} (Cambridge Advanced Studies in Mathematics \#105) Cambridge University Press, 2006.

\item Atiyah McDonald's 

\end{thebibliography}

\end{document}