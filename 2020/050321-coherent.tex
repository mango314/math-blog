\documentclass[12pt]{article}
%Gumm{\color{blue}i}|065|=)
\usepackage{amsmath, amsfonts, amssymb}
\usepackage[margin=0.5in]{geometry}
\usepackage{xcolor}
\usepackage{graphicx}
\usepackage{amsmath}
\usepackage{hyperref}

\usepackage{fontspec}
\usepackage{xcolor}

\usepackage{tikz-cd}

\newcommand{\off}[1]{}
\DeclareMathSizes{20}{30}{20}{18}
\usepackage{tikz}

%\setmainfont[Color=brown]{Linux Libertine}


\title{Notes: Quasi-Coherent Sheaves}
\date{}
\begin{document}

\sffamily

\maketitle

{\fontsize{16pt}{16pt}\selectfont 


\noindent A quasi-coherent sheaf on a ringed space $(X, \mathcal{O}_X)$ is a sheaf $\mathcal{F}$ of $\mathcal{O}_X$-modules where every point in $X$ has an open neighborhood with an exact sequence:
$$ \mathcal{O}_X^{\oplus I}|_U \to \mathcal{O}_X^{\oplus J}|_U \to \mathcal{F}|U \to 0 $$
In order to be called a coherent sheaf there are two more restrictions:
\begin{itemize}
\item $\mathcal{F}$ is of finite type on $\mathcal{O}_X$; every point in $X$ has an open neighborhood $U$ in $X$ such there is a surjective morphism $\mathcal{O}^n_X|U \to \mathcal{F}|_U$ for some natural number $n \in \mathbb{N}$. 
\item for any open set $U \subseteq X$ and any morhism $\phi: \mathcal{O}^n_X|_U \to \mathcal{F}_U$ of $\mathcal{O}_X$ modules, the kernel is of finite type.
\end{itemize} 
Notice we haven't consulted a textbook yet.  We could rewind and look at the definition of a vector bundle. \\ \\ 
A real vector bundle consist of:
\begin{itemize}
	\item topological space $X$ (base space) and $E$ (total space)
	\item continous surjection $\pi: E \to X$ (bundle projection)
	\item for every $x \in X$ there is structure of finite dimenaional vector space on the fiber $\pi^{-1}(x)$.
\end{itemize}
Example, the M\"{o}bius band is a bundle over the circle $S^1$. \\ \\
When we move to quasi-coherent sheaves, perhaps we are sacrificing the third item.  
\begin{itemize}
\item $\varphi_U: U \times \mathbb{R}^k \to \pi^{-1}(U) $
\item $\varphi_V: V \times \mathbb{R}^k \to \pi^{-1}(V) $
\item $\varphi^{-1}_U \circ \varphi_V: (U \cap V) \times \mathbb{R}^k \to (U \cap V) \times \mathbb{R}^k $ (composite function)
\item The transition functions form a \v{C}ech cocyle:
\begin{itemize}
\item $g_{UV}: U \cap V \to \text{GL}(\mathbb{R}^k)$
\item $g_{UU} \equiv I$ (the identity function)
\item $g_{UV}g_{VW}g_{WU} = I$ (cocycle condition) whenver $U \cap V \cap W = \varnothing$.
\end{itemize}
\end{itemize}
So there's a vague but not perfect way of moving information around.  At least we get the two Wikipedia pages to match.

\newpage


%\begin{tikzpicture}\end{tikzpicture}

\includegraphics[width=0.5\textwidth]{wastebasket-01.png}



\vfill

\begin{thebibliography}{}

\item Wikipedia ``Coherent Sheaf" \texttt{https://en.wikipedia.org/wiki/Coherent\_sheaf}
\item Wikipedia ``Vector Bundle" \texttt{https://en.wikipedia.org/wiki/Vector\_bundle}
\item . . .

\end{thebibliography}

\end{document}