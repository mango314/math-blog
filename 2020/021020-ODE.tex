\documentclass[12pt]{article}
%Gumm{\color{blue}i}|065|=)
\usepackage{amsmath, amsfonts, amssymb}
\usepackage[margin=0.5in]{geometry}
\usepackage{xcolor}
\usepackage{graphicx}
\usepackage{amsmath}
\usepackage{hyperref}

\usepackage{fontspec}
\usepackage{xcolor}

\newcommand{\off}[1]{}
\DeclareMathSizes{20}{30}{20}{18}
\usepackage{tikz}

%\setmainfont[Color=brown]{Linux Libertine}


\title{Tune-Up: Differential Equations}
\date{}
\begin{document}

\sffamily

\maketitle

{\fontsize{16pt}{16pt}\selectfont 

\noindent Here's an easy differential equation to solve.  If we move at constant speed, the thing moves in a straight line.
$$ \left[ \frac{dx}{dt} = 1 \right] \to \bigg[ x(t) = t + x_0 \bigg] $$
Therefore, by \textbf{continuity} if the speed is ``close to" $1$ then the past should be ``close to" a straight line.
$$ \left[ \frac{dx}{dt} \approx 1 \right] \to \bigg[ x(t) \approx t + x_0 \bigg] $$
Let's test this logic.  Merely integrate both sides:
\begin{eqnarray*} 
 \frac{dx}{dt} &=& 1 + ``noise" \\ \\
 \int \frac{dx}{dt} \, dt &=& \int \big( 1 + ``noise" \big) \, dt \\ \\
 x(t) - x(0) &=& t + \int ``noise" \, dt
\end{eqnarray*}
What could this \textbf{noise} or \textbf{error} term look like ?  Maybe this is a highly deliberate feature, let's stick with the term ``noise".  
$$ \frac{dx}{dt}  = 1 + 0.01 \times \sin (2\pi t) $$
Integrate both sides\dots \\ \\
Q: How did we know that integration was the reverse derivative?  
$$ x(T) - x(0) = \int_0^T \frac{dx}{dt} \, dt = 
\int_0^T \left[ 1 + 0.01 \times \sin (2\pi t)\right]\,dt = T + \frac{0.01}{2\pi}\cos (2\pi T) $$
This actually looks correct.  The integral is actually pretty close to the original. 

\newpage

\noindent Let's try another example.  The books gives us the example of the exponential function describing compound growth and decay:
$$ \frac{dx}{dt} = 1 \cdot x(t)  $$
The novelty here was to ``cross-multiply" which - it's up for grabs why that's acceptable - 
$$ \frac{dx}{x} = 1 \, dt $$
Then we can take integral ``of both sides" -- why should we get the same result?  is our notion of equality that strong?
$$ \log x(T) - \log x(0) =  \int_0^T \frac{ dx}{x} = \int_0^T 1 \, dt = T$$
and if we take the exponent of both sides we can obtain the solution to the differential equation:
$$ x(T) = x(0) e^T $$
Let's remind ourselves if we are sure we accept this conclusion?
$$  \frac{dx}{dt} = \frac{x(0)}{x(0)}  \lim_{\Delta t \to 0} \frac{e^{t+\Delta t} - e^t}{\Delta t} = x(t) = e^t $$
Our acceptance of this solution to differential equation had other consequences.  Also, all our calculations relied on several abstractions.
\begin{itemize}
	\item the interal as an ``antiderivative"
	\item the exponent function as the limit of other operations such as $(1 + \frac{t}{n})^n \to e^t$
	\item integral as a limit of Riemann sums
	\item the differential equation as a dynamical system -- as a model of physical reality
	\item 
\end{itemize}
The idea of checking these facts took a few hundred years to develop, so there's no rush. \\ \\
Let's try reading the book which is done in it's baroque style\dots

\newpage

\noindent \textbf{Def 3.1.4}
Let $\mathcal{V} = \mathcal{V}(S,T)$ be the class of functions
$$f(t,\omega): [0, \infty) \times \Omega \to \mathbb{R} $$
such that 
\begin{itemize}
\item $(t,\omega)\to f(t, \omega) $ is $\mathcal{B} \times \mathcal{F}$ measurable - $\mathcal{B}$ denotes the Borel $\sigma$-algebra on $[0, \infty)$
\item $f(t, \omega)$ is $\mathcal{F}_t$-adapted; \quad\quad $\omega \to f(t,\omega)$ is $\mathcal{F}_t$-measurable.
\item $E[\int_S^T f(t,\omega)^2 dt] < \infty $ (the integral is \textit{expected} to be finite)
\end{itemize}
The function $f$ depends on [time] and on any [random information] we have up to now. \\ \\
This is a book on Financial Mathematics, so a great example could be just a {\color{blue}card game} (e.g. if the next card is a 5 or an Ace), watching the river and testing our expected payoffs for specific hands.  Unlike traders or bankers, I think we \textit{are} interested in the technicalities of measure theory over $\mathbb{R}$ and over $\mathbb{R}^2$ which is our simplistic view of 2D Geometry. \\ \\
Let's see how this definition is used and turn into something we might actually care about \dots \\ \\
\textbf{Thm} For [an elementary function] $\phi(t,\omega)$:
$$ E \left[ \left( \int_S^T \phi(t,\omega) dB_t(\omega) \right)^2 \right] = E \left[ \int_S^T \phi(t, \omega)^2\, dt \right] $$
The idea here is to approximte our adapted function $f \in \mathcal{V}(S,T)$ with elementary [random] functions $\phi$. \\ \\
{\color{blue}\textbf{Randomness}} was just an abstraction. $dB_t$ is just an idealization of whatever limiting random walk proccess we actually choose.  Brownianian motion could be read as the integral of ``white noise" or the solution to the differential equation:
$$ \frac{dB_t}{dt} = W_t $$
These are wanting of a mathematical definition. \\ \\
What might we use this result for? An example of what we might do is ``arbitrage" exploiting differences in market value of whatever is in our portfolio.  \\ \\
The textbook give us one tempting example.  Since we know calculus, we can find quite a few others using the It\^{o} lemma:
$$ \int_0^1 B_t \, dB_t = \frac{1}{2}B_t^2  - \frac{1}{2} t $$
Or a rearrangement we have that Brownian motion can be re-constructed from the random integral:
$$ \frac{1}{2}B_t^2 = \frac{1}{2} t  + \int_0^t B_t \; dB_t  $$
To compare with the classical integral the rule now looks slightly different;
$$ \int t \, dt = \frac{1}{2} t^2 + const. $$
Here we need to validate that random walks behave the way we expect (e.g. ``quadratic variation"), the result is a profound use of integration-by-parts.  What does graduate-level integration by parts look like?
$$ d( \frac{1}{2}B_t^2) = \frac{1}{2} dt +  $$
\textbf{Exercise} Prove directly from the definition of It\^{o} integrals:
$$ \int_0^t B_t^2 \, dB_t = \frac{1}{3} B_t^3  - \int_0^t B_s \, ds $$
\textbf{Exercise} $ B_t^3  - 3t \, B_t $ is a martingale. \\ \\
I think just by completing the square we get some convincing arguments for the quadratic case:
$$ \sum B \, \Delta B = \frac{1}{2}B^2  - \frac{1}{2} \sum (\Delta B)^2  = \frac{1}{2}B^2 - \frac{1}{2}t $$
We have to argue that the quadratic variation of Brownian motion is growing linearly 
$$\sum (\Delta B)^2 \to t$$
It\^{o} integrals are martingales. \\ \\
\textbf{Def} A \textit{filtration} $\mathcal{F}_t$ is\dots

\vfill

\begin{thebibliography}{}

\item Bernt \O{}ksendal \textbf{Stochastic Differential Equations} (Universitex) Springer, 2003.

\end{thebibliography}

\end{document}