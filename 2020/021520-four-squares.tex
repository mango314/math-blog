\documentclass[12pt]{article}
%Gumm{\color{blue}i}|065|=)
\usepackage{amsmath, amsfonts, amssymb}
\usepackage[margin=0.5in]{geometry}
\usepackage{xcolor}
\usepackage{graphicx}
\usepackage{amsmath}
\usepackage{hyperref}

\usepackage{fontspec}
\usepackage{xcolor}

\newcommand{\off}[1]{}
\DeclareMathSizes{20}{30}{20}{18}
\usepackage{tikz}

%\setmainfont[Color=brown]{Linux Libertine}


\title{Tune-Up: Four Squares}
\date{}
\begin{document}

\sffamily

\maketitle

{\fontsize{16pt}{16pt}\selectfont 

\noindent Let $g$ be elements of $SU(2)$ let's define an averaging operator $z : L^2(SU(2)) \to L^2(SU(2)) $ with 
$$ z f(x) = \sum \big( f(gx) + f(g^{-1}x) \big)  $$
These operators might have a \textbf{spectral gap} $\lambda (z_g) < 2k$.  Somewhat fancy, question asked in 1921 is whether Lebesgue measure is the unique finitely additive measure on $S^2$ (we could construct all sorts of measures -- notions of area or volume or mass -- on what is geometrically, a sphere). \\ \\
\textbf{noncommutative diophantine property} we can find a universal constant $D$ such taht for any word $W \in \langle g \rangle$  of length $m$, the norms are greater then a certain size, 
$$ || W_m \pm e || \geq D^{-m}  $$
Using the a norm on $2 \times 2$ matrices:
$$ \left| \left[ \begin{array}{cc}  a & b \\ c & d\end{array} \right] \right| = a^2 + b^2 + c^2 + d^2 $$
\textbf{Exercise}  for the case in question we have an easy 
$$ || g \pm e||^2 = 2 |\text{trace}(g) \mp 2| $$
What were the main results:
\begin{itemize}
\item if $g \in SU(2) \cap M_{2 \times 2}(\overline{\mathbb{Q}})$ ($2 \times 2$ matrices with algebraic entries) then $z_g$ satisfies non-commutative diophantine property
\item Let $\{ g_1, \dots, g_m\}$ be a set of elements in $SU(2)$ generating a free griph and satisfying [noncommuutative diophantine property] then $z$ has a spectral gap.
\end{itemize}
Just a reminder, irreducible representations $G = SU(2)$ are given by $\pi_n = \text{sym}^N(V) $ it's a linear map on the space of homogeneious polynomials:
$$ (x,y) \mapsto (ax+by cx+dy) ,
 \left[ \begin{array}{cc} a & b \\ c & d \end{array} \right] \in SU(2) $$ 
abc
$$ \pi_N (z) = \pi_N(g) + \pi_N(g^{-1}) + \dots +  \pi_N(g_m) + \pi_N(g_m^{-1}) $$ 
In all of these theorems, something complex and chaotic occurs.  The first moves look pretty systematic, we are going to define point-measures on the compact lie group:
$$ \nu = \frac{1}{2k} \sum \delta_g + \delta_{g^{-1}} $$
in order to show that averages tend to a limit, we can use an \textbf{approximate identity}
$$ P_\delta = 
\frac { \chi_{B_{1, \delta}}} {|B(1, \delta)|} $$
where $\delta \ll 1$ and $l \sim \log \frac{1}{\delta}$ and $\mu = \nu^{(l)} * P_\delta $ (density is bounded by $||P_\delta||_\infty \sim \delta^{-3} $).\\ \\
\textbf{Lemma} ``$L^2$-flattening Lemma"
\begin{itemize}
\item $ \delta < ||\mu||_2 < \delta$
\item $ || \mu * \mu ||< \delta^{\epsilon} ||\mu||_2 $
\end{itemize}
These number-theoretic averages -- this should be concrete and super-tangible -- is compared to \textbf{random walk} so we have to measure cost of comparing these number patterns to random.  We now have an infinite number of examples of free groups with $\overline{\langle g \rangle} \simeq SU(2)$ the sphere.  Before we take the limit, these objects are very close to perfectly symmetric so we an call them \textbf{approximate group}.
\vfill

\begin{thebibliography}{}

\item Terence Tao, Van Vu.  \textbf{Additive Combinatorics} (Cambridge Advanced Studies in Mathematics \#105) Cambridge University Press, 2006.

\end{thebibliography}

\end{document}