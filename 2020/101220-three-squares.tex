\documentclass[12pt]{article}
%Gumm{\color{blue}i}|065|=)
\usepackage{amsmath, amsfonts, amssymb}
\usepackage[margin=0.5in]{geometry}
\usepackage{xcolor}
\usepackage{graphicx}
\usepackage{amsmath}
\usepackage{hyperref}

\usepackage{fontspec}
\usepackage{xcolor}

\newcommand{\off}[1]{}
\DeclareMathSizes{20}{30}{20}{18}
\usepackage{tikz}

%\setmainfont[Color=brown]{Linux Libertine}


\title{Reading: \dots}
\date{}
\begin{document}

\sffamily

\maketitle

{\fontsize{16pt}{16pt}\selectfont 

\noindent  The statement is a bit technical that the ``full [counting] measure" is close to the ``joint measure".  These terms don't mean anything until we at least provide a definition.  That doesn't mean we've looked at them \dots ever. \\ \\
\textbf{Thm 1.1} [AES 16a] Let $d \geq 3$ and for any integer $D$ define
$$ Q_D = 
\left\{ 
\left( \frac{v}{||v||} , [\Lambda_v]   \right) :
v \in \hat{\mathbb{Z}}^d , ||v||^2 = D \right\} 
\subset S^{d-1} \times \mathcal{X}_{d-1} $$
Then the normalized counting measure $m_D$ on $\mathcal{Q}_D$ converges to $m_{S^{d-1}} \times m_{\chi_{d-1}} $ in the weak* topology as $D \to \infty$, provided that the set $\mathcal{Q}_D$ is non-empty and that in addition one has the following conditions on the number $D$ 
\begin{itemize}
\item $d=3$ \dots
\item $d=4$ \dots
\end{itemize}
For $d \geq 4$ a stronger claim is given.  They remove a ``congruence condition" and instead of the ``shape of the lattice"  they look at a ``grid".  These terms might deserve enunciation. \\ \\ 
\textbf{Thm 1.2} Let $d = 4$ or $d = 5$ for any positive integer $D$, let $m_{Q_D}$ denote the normalized counting measure  
$$ |m_{Q_D}(f)  - m_S^{d-1} \times m_{\mathcal{X}}| < D^{-\kappa} S_\infty(f)$$
(provided that $Q_D$ is non-empty). \\ \\
How we can read this as an elementary problem?  Since $f \in C^\infty_c(S^{d-1} \times \mathcal{X}_{d-1})$ could be any \textbf{smooth} and \textbf{compactly supported} function on the space of $[\text{sphere}] \times [\text{space of lattices}]$.  \\ \\
How do we decide about elements of $Q_D$ ? Maybe this set is empty?  Counting measure suggests we locate the object exactly at the point.  That's unlikely we might try to find equidistribution, we \\ \\
The theorem clearly says that something ``nearly factors" and our solutions is ``nearly symmetric":
$$[\text{counting measure]} \approx [\text{sphere measure}] \times [\text{lattice measure}] $$
We are unlikely to solve these problem explicitly as $D \to \infty$.  How can we write $v^2 = v_1^2 + v_2^2 + v_3^2 + v_4^2 = D$ with $D \gg 10^{100}$?  Maybe there are appoximate algorithms for budgeting space and time resources? \\ \\
Finally smooth functions can be really arbitrary, we could try to construct a smooth function that vanishes on a dense set of the sphere (as $D \to \infty$) or with other properties, and our equidistribution result should old.  Indeed, these are left as exercises. \\ \\
Around 1959, there's interest in writing as an integer as sums of four squares.  First of all:
\begin{itemize}
\item $\widehat{\mathbb{Z}}^d$ is the set of ``primitive" vectors in $\mathbb{Z}^d$
\item If we have any four numbers $\vec{v} = (v_1, v_2, v_3, v_4) \in \mathbb{Z}^4 $ the vector has a radius $v^2 = v_1^2 + v_2^2 + v_3^2 + v_4^2 = D = \text{radius}^2$.
\item One more computation we can do is the orhogonal lattice: $$\Lambda_v =v^\perp \cap \mathbb{Z}^4 $$
These exercises in 4-dimesional geomtry might be done formally using Linear Algebra (in fact module theory it's over $\mathbb{Z}$) yet \dots 
\end{itemize}
So we can study the joint equidistribution problem:
$$ \frac{1}{\sqrt{D}} v \in  [\text{diretion}] \times [\text{shape of the lattice}] $$
The very notion of \textbf{vector} took centuries to develop to describe quantities in geometry and physics.\footnote{Euclid's elements had a notion of ``line" or of ``rectangle" not a notion of ``physics".  Wikipedia credits the term ``vector" to dissions clarifying Electricity and Magnetism in 1860's.}
\vfill

\begin{thebibliography}{}

\item \dots
\item \dots 

\end{thebibliography}

\newpage

\noindent abc

}
\end{document}