\documentclass[12pt]{article}
%Gummi|065|=)
\usepackage{amsmath, amsfonts, amssymb}
\usepackage[margin=0.5in]{geometry}
\usepackage{xcolor}
%\usepackage{graphicx}
\usepackage{graphics}
\newcommand{\off}[1]{}
\DeclareMathSizes{20}{30}{21}{18}

\newcommand{\myhrule}{}

\usepackage{tikz}

\title{\textbf{ Examples:  WKB Approximation }}
\author{John D Mangual}
\date{}
\begin{document}

\fontfamily{qag}\selectfont \fontsize{25}{30}\selectfont

\maketitle

\noindent We hit the ground running a bit here\footnote{I have been encouraged to look at this topic by various sources and have had the privilege to meet Davide Gaiotto during my one-time visit to Perimeter and Andrew Neitzke at various conferences.  Since the papers were written about 7 years ago.  These read somewhat like Harry Potter, and they are quite lengthy.  The reward, potentially is a new look at holomorphic functions and complex analysis (at least for me.} - why are we drawing all these curves in the first place.  I'll draw a few in a moment.  \\ \\
The \textbf{WKB approximation} says if the guage where:
$$ \phi = \left( \begin{array}{cc} 
\lambda & 0 \\ 0 & -\lambda \end{array} \right) $$
there are two independent $\mathcal{A}$-flat sections of the form:
$$\psi^1 \sim \left( \begin{array}{c} 
e^{- \frac{R}{\zeta} \int^z \lambda } \\ 0 \end{array} \right), 
\psi^2 \sim \left( \begin{array}{c} 0 \\ 
e^{- \frac{R}{\zeta} \int^z \lambda }\end{array} \right) $$
therefore in the limit as $\zeta \to 0$ we obtain \textbf{essential singularities}, which behave like $e^{1/z}$ near $z = 0$. 

\newpage

\noindent Locally the essential singularity looks like a spiral.  Let's solve:
$$ e^{1/z} \in \mathbb{R} $$
then $z = R e^{i\theta}$, 
$$ z(t) = z_0 e^{t\, e^{i\theta}} $$
so the generic WKB curve is a \textbf{logarithmic spiral}.
$$ x^K + \sum_{k=2}^K u_k(z) x^{K-k} = 0$$
the coefficients polynomials of suitable degree\footnote{There are lecture notes of Nigel Hitchen or Simon Donaldson -- one of them -- for undergraduates and he gives you the Riemann existence theorem.  I didn't think to much of this but it's \textbf{rare} for any good explanation of this result.} \\ \\
This is a polynomial in $x$ of degree $K$, but let's use some earth-shattering language and say that $\Sigma$ it is $K$-fold cover of $C$, and that coefficients are sections of the sheaf $u_k \in H^0(C, K^{\otimes k})$.  Which in turn which define a curve $\Sigma \subset T^\ast C$. \\ \\ 
And we will set $K = 2$.  \newpage

\noindent I can say a tiny bit more the ``Coulomb branch" of this theory is:
$$ \mathcal{B} = \bigoplus_{k=1}^r H^0(C, K^{\otimes d_k}) $$
so an element of the Coulomb branch is the as choosing these polynomial cofficients.  \\ \\
I feel these singularities are interesting in their own right, but we have indicated other motives.  \\ \\
Our choice of lie group $G = SU(2)$ or quiver $A = A_2$.  The central charge is the integral of this differential:
$$ Z = \frac{1}{\pi} \otimes_\gamma \lambda $$
yet this theory is also described by a case of Hitchin's equations:
\begin{eqnarray}
F + R^2 [ \phi, \overline{\phi}] &=& 0 \\
\overline{\partial}_A \phi := \big( \partial_{\overline{z}} \phi_z + [A_{\overline{z}}, \phi_z)] \big) d\overline{z} \wedge dz &=& 0 \\
\partial_A \phi := \big( \partial_z \phi_{\overline{z}} + [A_z, \phi_{\overline{z}})] \big) dz \wedge d\overline{z}&=& 0 
\end{eqnarray}
and the solutions to this equation can be combined into a single connection:
$$ \mathcal{A} = \frac{R}{\zeta} \phi + A + R \zeta \overline{\phi} $$
this is much analogous to how the \textbf{real} and \textbf{imaginary} parts combine to form a \textbf{\color{blue!50!white}{complex number}}.  These equations involve {\color{red!50!white}{twistor}} theory which for now is just a parameter $\zeta \in \hat{\mathbb{C}}$ and especially $\zeta \in {0, \infty}$ but also $\zeta \in \hat{\mathbb{C}}$. \\ \\
Then near $\zeta = 0$ we know the singular behavior of $\mathcal{A}$ is determined by the zeros and polesof $\lambda$, but putting all this physics aside our singularities are of the type:
$$ e^{ \frac{R}{\zeta} \int^z \sqrt{\frac{p(x)}{q(x)}} \, dz} $$
and which obviously has a bunch of essential singuarlties the tool we shall use to evaluate these are \textbf{resurgence analysis} otherwise known as ``steepest descent".

\newpage

\noindent -- TODO -- 

\begin{itemize}
\item what is a 2D wall-crossing formula?
\item what is a 4D wall-crossing formula?
\item how the WKB curves predict such formulas?

\item how to obtain \textbf{flat-surfaces} and the \textbf{siegel-veech} constants.
\end{itemize}

\newpage

\fontfamily{qag}\selectfont \fontsize{12}{10}\selectfont


\begin{thebibliography}{}

\item Davide Gaiotto, Gregory W. Moore, Andrew Neitzke. \textbf{Wall-Crossing in Coupled 2d-4d Systems.} \texttt{ arXiv:1103.2598v1}


\item Davide Gaiotto, Gregory W. Moore, Andrew Neitzke. \textbf{Wall-crossing, Hitchin Systems, and the WKB Approximation.} \texttt{ arXiv:0907.3987}

\end{thebibliography}

\vspace{0.25in}

\begin{thebibliography}{}

\item Alex Eskin, Howard Masur, Anton Zorich \textbf{Moduli Spaces of Abelian Differentials: The Principal Boundary, Counting Problems and the Siegel--Veech Constants} \texttt{math/0202134}

\item David Sauzin.	\textbf{ Introduction to 1-summability and resurgence. } \texttt{arXiv:1405.0356}

\item Tom Bridgeland, Ivan Smith.  \textbf{Quadratic differentials as stability conditions.} \texttt{1302.7030}

\item Kohei Iwaki, Tomoki Nakanishi \textbf{Exact WKB analysis and cluster algebras} \texttt{1401.7094}

\item David Aulicino \textbf{The Cantor-Bendixson Rank of Certain Bridgeland-Smith Stability Conditions} \texttt{1512.02336}

\end{thebibliography}


\end{document}