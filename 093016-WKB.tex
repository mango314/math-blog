\documentclass[12pt]{article}
%Gummi|065|=)
\usepackage{amsmath, amsfonts, amssymb}
\usepackage[margin=0.5in]{geometry}
\usepackage{xcolor}
%\usepackage{graphicx}
\usepackage{graphics}
\newcommand{\off}[1]{}
\DeclareMathSizes{20}{30}{21}{18}

\newcommand{\myhrule}{}

\usepackage{tikz}

\title{\textbf{ Examples:  WKB Approximation }}
\author{John D Mangual}
\date{}
\begin{document}

\fontfamily{qag}\selectfont \fontsize{25}{30}\selectfont

\maketitle

\noindent We hit the ground running a bit here\footnote{I have been encouraged to look at this topic by various sources and have had the privilege to meet Davide Gaiotto during my one-time visit to Perimeter and Andrew Neitzke at various conferences.  Since the papers were written about 7 years ago.  These read somewhat like Harry Potter, and they are quite lengthy.  The reward, potentially is a new look at holomorphic functions and complex analysis (at least for me.} - why are we drawing all these curves in the first place.  I'll draw a few in a moment.  \\ \\
The \textbf{WKB approximation} says if the guage where:
$$ \phi = \left( \begin{array}{cc} 
\lambda & 0 \\ 0 & -\lambda \end{array} \right) $$
there are two independent $\mathcal{A}$-flat sections of the form:
$$\psi^1 \sim \left( \begin{array}{c} 
e^{- \frac{R}{\xi} \int^z \lambda } \\ 0 \end{array} \right), 
\psi^2 \sim \left( \begin{array}{c} 0 \\ 
e^{- \frac{R}{\xi} \int^z \lambda }\end{array} \right) $$
therefore in the limit as $\xi \to 0$ we obtain \textbf{essential singularities}, which behave like $e^{1/z}$ near $z = 0$. 

\newpage

\noindent Locally the essential singularity looks like a spiral.  Let's solve:
$$ e^{1/z} \in \mathbb{R} $$
then $z = R e^{i\theta}$, 
$$ z(t) = z_0 e^{t\, e^{i\theta}} $$
so the generic WKB curve is a \textbf{logarithmic spiral}.
$$ x^K + \sum_{k=2}^K u_k(z) x^{K-k} = 0$$
the coefficients polynomials of suitable degree\footnote{There are lecture notes of Nigel Hitchen or Simon Donaldson -- one of them -- for undergraduates and he gives you the Riemann existence theorem.  I didn't think to much of this but it's \textbf{rare} for any good explanation of this result.} \\ \\
This is a polynomial in $x$ of degree $K$, but let's use some earth-shattering language and say that $\Sigma$ it is $K$-fold cover of $C$, and that coefficients are sections of the sheaf $u_k \in H^0(C, K^{\otimes k})$.  Which in turn which define a curve $\Sigma \subset T^\ast C$. \\ \\ 
And we will set $K = 2$.
\newpage

\fontfamily{qag}\selectfont \fontsize{12}{10}\selectfont

\begin{thebibliography}{}

\item Davide Gaiotto, Gregory W. Moore, Andrew Neitzke. \textbf{Wall-Crossing in Coupled 2d-4d Systems.} \texttt{ arXiv:1103.2598v1}


\item Davide Gaiotto, Gregory W. Moore, Andrew Neitzke. \textbf{Wall-crossing, Hitchin Systems, and the WKB Approximation.} \texttt{ arXiv:0907.3987}


\end{thebibliography}


\end{document}