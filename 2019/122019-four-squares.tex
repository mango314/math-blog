\documentclass[12pt]{article}
%Gumm{\color{blue}i}|065|=)
\usepackage{amsmath, amsfonts, amssymb}
\usepackage[margin=0.5in]{geometry}
\usepackage{xcolor}
\usepackage{graphicx}
\usepackage{amsmath}
\usepackage{hyperref}

\newcommand{\off}[1]{}
\DeclareMathSizes{20}{30}{20}{18}
\usepackage{tikz}


\title{Tune-Up: Four Squares Theorem}
\date{}
\begin{document}

\sffamily

\maketitle

\noindent List of Theorems  

\begin{itemize}
\item Def 1
\item Thm 1
\item Prop 1.1, 2.2, 2.4, 3.1, 3.2, 4.1, 4.2, 
\item Cor 1.1, 5.1, 5.2, 
\item Lemma 2.1, 2.3, 2.5, 5.1, 5.2, 5.3, 5.4, 5.5, 6.1, 
\end{itemize}
List of Chapters 
\begin{itemize}
\item Introduction
\item Proof of Theorem 1
\item Preliminaries on the [Discretied Ring Conjecture]
\item A [Scalar amplification Property]
\item Preliminary Lemmas on subsets of $SU(2)$
\item Product Theorem of $SU(2)$
\item Convolution of Property Measures of $SU(2)$
\end{itemize}
Look like we can think of this paper as discussion of ``coherent" collections of solutions to the Lagrange 4-squares theorem (which is easier than the Legendre 3-squares theorem). 
$$ n = a^2 + b^2 + c^2 + d^2 = \det \left[\begin{array}{cr} a + bi & -c + di \\ c + di & a - bi\end{array} \right] $$
The literature indicates that generealizations of statements exist, and quite a few have been written in articles, many others left unproven or even unstated.  One objection is the use of harmonic analysis or class field theory has to itself be translated back into elementary terms. \\ \\
Prop 1.1 + Theorem 1 = Corollary 1.1 \\ \\
Theorem 1 $\to$ ( Lemma 2.1 + [Proof] ) +  [Proof] \\ \\
Proposition 2.2 ( ``$\ell^2$ flattening lemma" ) \\ \\
Lemma 2.3 $\to$ Proposition 2.4 + [Proof] \\ \\
Lemma 2.5 $\to$ [Proof] \\ \\
Proposition 3.1 $\leq$ Proposition 3.2 $\to$ Proof (``sum-product theorem" what are $+$ or $\times$ anyway?) \\ \\
Proposition 4.1 $||$ 3.1, 3.3 $\leq$ 4.2 (random theorems about angles $\measuredangle$)\\ \\
Lemma 5.1 $\to$ Proof (random theorems about $SU(2)$ quaternions, parallel with $\text{SL}_2(\mathbb{F}_p)$)\\
Lemma 5.2 $\to$ Proof \\
Lemma 5.3 $\to$ Proof \\
Lemma 5.4 $\to$ Proof \\ \\
( 5.1 + 5.5 ) + (5.4 + 5.2) $\to$ 6.1 (Sumsets) \\ \\
7: Prove $\ell^2$-flattening lemma.  6.1 $\to$ 2.2 ( approximate groups of $G = SU(2) \simeq SO(3)$ ) \\ \\
\textbf{A small amount of content}\\ \\
Short of obvious coincidences, the Pythagorean triples lead to copies of the free group on two generators $\mathbb{Z}*\mathbb{Z}$ :
$$ \overline{\left\langle \left[ 
\begin{array}{crc} \frac{3}{5} & -\frac{4}{5} & 0 \\ \\ 
\frac{4}{5} & \frac{3}{5} & 0 \\ \\
0 & 0 & 1 \end{array} \right], 
\left[ 
\begin{array}{ccr} 1 & 0 & 0 \\ \\ 0 & \frac{3}{5} & -\frac{4}{5}  \\ \\ 
0 & \frac{4}{5} & \frac{3}{5}  \end{array} \right] \right\rangle } \simeq SO(3, \mathbb{R})$$
Here's another Pythagorean triple and another copy of the free group on two generators $F = \langle x, y \rangle$ that still fits in the rotation group.  Just remind myself $5 \times 5 + 12 \times 12 = 25 + 144 = 169 = 13 \times 13$ (yes I resorted to base-10 arithmetic a.k.a. ``decimals").
$$ 
\overline{\left\langle \left[ 
\begin{array}{crc} \frac{5}{13} & -\frac{12}{13} & 0 \\ \\ 
\frac{12}{13} & \frac{5}{13} & 0 \\ \\
0 & 0 & 1 \end{array} \right], 
\left[ 
\begin{array}{ccr} 1 & 0 & 0 \\ \\ 0 & \frac{5}{13} & -\frac{12}{13}  \\ \\ 
0 & \frac{12}{13} & \frac{5}{13}  \end{array} \right] \right\rangle } \simeq SO(3, \mathbb{R})
$$
Establishing density, the more copies of matrices we use we just have higher powers of $\frac{1}{5^n} \to 0$ and $\frac{1}{13^n} \to 0$.  So these are two groups of matrices with totally different elements, yet by \textbf{compactness} of $SO(3)$ we know these things get arbitrarily close.  After a dozen iterations (maybe $2^{10} \approx 1000 = 10^3$ elements) we quickly run out of decimal places.
\begin{verbatim}
>>> 0.2 + 0.1
0.30000000000000004
\end{verbatim}
So, like we said, the use of Harmonic analysis leads to very good arithemtic results, enough to fool your calculator.  So there's an endless source of patterns which are \textit{implied} yet not explicitly enough stated in this advanced article.

\end{document}