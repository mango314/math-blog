\documentclass[12pt]{article}
%Gumm{\color{blue}i}|065|=)
\usepackage{amsmath, amsfonts, amssymb}
\usepackage[margin=0.5in]{geometry}
\usepackage{xcolor}
\usepackage{graphicx}
\usepackage{amsmath}
\usepackage{hyperref}

\newcommand{\off}[1]{}
\DeclareMathSizes{20}{30}{20}{18}
\usepackage{tikz}


\title{Tune-Up: Rational Functions}
\date{}
\begin{document}

\sffamily

\maketitle

\noindent There is so much semantics in this one, it's hard to decide where to begin. \\ \\
\textbf{Proposition IX.2.7} {\color{black!25!white}If two curves are birationally equivalent, they have the same geometric genus.  The converse is true if $g = 0$:} a curve is rational if and only if its geometric genus is zero. \\ \\
This theorem sounds great if we can decide on the meaning of a few terms:
\begin{itemize}
\item ``curve"
\item birational equivalence
\item ``isomorphic"
\item (geometric) genus
\item ``rational"
\end{itemize} 
Are these close to the same terms we learned in high school?  Let $x = \cos 5\theta$ and $y = 1 + \sin 3 \theta$ find the equation $f(x,y) = 0$ or the ring $\mathbb{C}[x,y]$.  This math-speak has gone too far.
\vfill



\begin{thebibliography}{}
\item Daniel Perrin \textbf{Algebraic Geometry} (Universitext) Springer, 2008.
\end{thebibliography}


\end{document}