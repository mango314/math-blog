\documentclass[12pt]{article}
%Gumm{\color{blue}i}|065|=)
\usepackage{amsmath, amsfonts, amssymb}
\usepackage[margin=0.5in]{geometry}
%\usepackage{xcolor}
\usepackage{graphicx}
\usepackage{caption}
\usepackage[x11names, table]{xcolor}
\usepackage{amsmath}
\usepackage{hyperref}
\usepackage{multicol}
\newcommand{\off}[1]{}
\DeclareMathSizes{20}{30}{20}{18}
%\usepackage{tikz}


\title{Scratchwork: Multiplication}
\date{}


\begin{document}

\sffamily

\maketitle

\noindent 
\textbf{Example} Furstenburg shows in 1967, that the only closed infinite subset of $\mathbb{R}/\mathbb{Z}$ invariant under  $S:(\cdot) \mapsto \cdot \times a$ and $T:(\cdot) \mapsto \cdot \times b$ is all of $\mathbb{R}/\mathbb{Z}$ itself.  For any irrational $\theta \notin \mathbb{Q}$, 
$$ \overline{ \{ a^k \times b^\ell \times \theta : k, \ell \geq 0 \} } = \mathbb{R}/\mathbb{Z} $$
I don't know how such a basic point could be under argument in this first place.  No non-expert would even question such a thing.  It takes a small bit of anaysis and topology to \textit{state} that the closure of an infinite set is something.  And then we need to consult a resource on ``commuting automorphisms", here $\theta \mapsto \theta \times a$ and $\theta \mapsto \theta \times b$. \\ \\ 
For example, Let's find $k$ and $\ell$ such that $\big|2^k \times 3^\ell \times \sqrt{2} - \sqrt{3} \big| < 10^{-2}$.  These questions (even to me) seem like novelties, and each single case can be solved with a computer, given enough time and resources.\\ \\
\textbf{Example} Weyl's Law (say for $H = \partial^2_x + \sqrt{n} \, \partial^2_y$) gives an estimate for the distribution of eigenvalues (in $\mathbb{R}$):
$$ \# \{ j : \lambda_j < X \} = \# \{ (a,b) : a^2 + \sqrt{n} \, b^2 < X \} \sim
\frac{\pi}{4 \sqrt{n}} X  $$ 
We notice the distribution is approximately a line.  Perhaps we could find the ``doubling constant": 
$$ A + A \approx k\, A $$ 
The paper introduces all sorts of interesting measurable sets.  Starting with $\mu(\mathbb{Q}) = 0$ and yet $\overline{\mathbb{Q}} = \mathbb{R}$. The paper also computes the gaps:
$$ \delta_{min}(N) = \min (\{\lambda_{i+1} - \lambda_i : 1 \leq i \leq N \})  $$
Here the $\lambda_i$ are the sorted vales of $a^2 + \sqrt{n} \, b^2 \in \mathbb{Z}[\sqrt{n}]$.  Notice we do not even need all of $\mathbb{R}$ to define this thing. \\ \\
Let's even take a step further and remind ourselves that the numbers $\overline{\{a^2 + b^2 - \sqrt{2} c^2 : a,b, c \geq 0 \}} = \mathbb{R}$ and we could have an exercise:
$$ \big| a^2 + b^2 - \sqrt{2} c^2 \big| < 10^{-6} $$
Finding such integers one time, is certainly tractable.  I think the difficulty is showing this can \textit{always} happen to arbitrary accuracy that requires the Ratner theory.  In other words, that there is something complicated about these numbers. \\ \\ 
We have a dilemma, no matter how many times we use a computer, no matter how much data we have, we are no closer to a proof.  This problem and many others, originated in this use of computers, where $\mathbb{R}\approx 2^{-N}\,\mathbb{Z}$ where we only get $N = 20$ or $30$ decimal places and the addition is truncated towards the end. Maybe we can find other measurable subsets $X \subseteq \mathbb{R}$ that show why this elementary result might offer such difficulty.   \\ \\
If we don't have enough decimal places, the approximate relation $a^2 + b^2 \approx \sqrt{2}c^2$ becomes an equality $a^2 + b^2 \mathbf{=} \sqrt{2}c^2$ or $\sqrt{2} = \frac{a^2+b^2}{c^2} \in \mathbb{Q}$.  This is a standard geometric exercise from Euclid that $\sqrt{2} \notin \mathbb{Q}$.

\newpage \noindent
%Let's settle for a timeline.  
%\begin{multicols}{2}
%\noindent abc \vfill \null \columnbreak
%\noindent cdf
%\end{multicols}
abcd \\ \\
\textbf{Example} For a generic irrational point, $x \notin \mathbb{Q}$, the orbit under $S: \theta \mapsto \{a \times \theta\}$ and $T: \theta \mapsto \{b \times \theta\}$ as elements of $\mathbb{R}/\mathbb{Z}$ is still a set of measure zero, yet it is dense in $\mathbb{R}/\mathbb{Z}$.
$$ \overline{\{ a^k \times b^\ell \times x : k, \ell \geq 0 \}} = \mathbb{R}/\mathbb{Z} $$
This is just like when $\mu(\mathbb{Q}) = 0$ and yet $\overline{\mathbb{Q}} = \mathbb{R}$, in its entirety.  Let's see a quantitative statement of this result:
\begin{quotation}
Let $a,b$ be multiplicative independent.  Suppose $\alpha \in \mathbb{R}/\mathbb{Z}$ be diophantine-generic; there exists $k$ such that
$$ \Big|\alpha - \frac{p}{q}\Big| \geq q^{-k} \text{ with }q \geq 2,\;\;\; p, q \in \mathbb{Z} $$
Then $\{ a^k \times b^\ell \times \alpha : 0 < k, \ell < N \}$ is $(\log \log N)^{-\kappa}$ dense in $\mathbb{R}/\mathbb{Z}$ for some constant $\kappa > 0$.
\end{quotation}
Even if we restrict to a single map, our entire decimal system is based on the map $T: \theta \mapsto 10 \times \theta$.  Here is a set of measure zero, where every other decimal is zero and the other numbers are in $\{1,2,\dots, 9\}$.
$$ A = \{ 0.x_1\,0\,x_2\,0\,x_3\,0\dots : 1 \leq x_i \leq 9\} \text{ can we show that }A+A = \mathbb{R}/\mathbb{Z} $$
We have that $\mu(A) = 0$ can we show that $\mu(A+A) > 0$?  Or let's construct a measurable function.  What happens if we stop after $N$ digits?
$$ A_n = \{ 0.x_1\,0\,x_2\,0\,x_3\,0\,\dots\,0\,x_n: 1 \leq x_i \leq 9\} \text{ with }\mu(A_n) = \Big(1 - \frac{1}{10}\Big)^n \times \frac{1}{10^n} $$
Then we could write down a measurable function by adding functions supported on these subsets in various ways:
$$ f(x) = \sum \mathbf{1}_{A_n}(x) \quad\text{and}\quad g(x)= \sum_{n \text{ squarefree}} 2^n \mathbf{1}_{A_n}(x) $$ 
I have not checked for convergence.  The next question would be $\mu( f^{-1}([0,\frac{1}{2}])$ this set is measurable. \\ \\
What happens if we try to take derivatives of these functions?  We could try to find a limit $\frac{1}{\epsilon}(f(x+\epsilon) - f(x))$ for these functions which are roughly straight lines and have that $f'(x) \approx 1$.  \\ \\
Given a dynamical system $T: X \to X$ we could define a smaller subset of the orbit: $\{ T^n x : n \text{ squarefree}\}$ and since the density of square-free numbers in $\mathbb{Z}$ is roughly $\frac{6}{\pi^2} \approx \frac{2}{3}$, we could have a shift-map related to these partial orbits.  This might already have a name, like an ``factor" or ``return-map". \\ \\ 
The measurable sets in this paper of $\times a \times b$ are 
$$ A_{M,N} = \Big\{ \{ 5^k \times 7^\ell  \times \sqrt{2} \} : 0 < k < M, 0 < \ell < N \Big\} + [-\epsilon, \epsilon] $$
for small enough $0 < \epsilon \ll 1$. The paper seems to be concerned with deviations from uniformity (since that's what we expect) and attempting to quantify that. Here's a sample.  What can we say about this series?
$$ h(x) = \sum_{M, N \geq 0} (-1)^{M+N} 1_{\mathbf{A}_{M,N}}(x) $$
and any other imaginable statistic. What is the correct averaging factor?
\vfill

\begin{thebibliography}{} 
\item Proofs that $\sqrt{2}\notin\mathbb{Q}$ \\
\texttt{https://math.stackexchange.com/q/2382318}\\
\texttt{https://math.stackexchange.com/q/451700} \\ \\ 
Proof that $\sqrt{2}+\sqrt{3} \notin \mathbb{Q}$ \\
\texttt{https://math.stackexchange.com/q/452078}\\ 
\texttt{https://math.stackexchange.com/q/457382}
\item Commuting Automorphisms \\
Jean Bourgain, Philippe Michel, Elon Lindenstrauss, Akshay Venkatesh \textbf{Some Effective Results in $\times a \times b$} Ergodic Theory and Dynamical Systems, Volume 29, Issue 6, December 2009, pp 1705-1722. \\
Daniel J. Rudolph \textbf{$\times 2 \times 3$ invariant measures and entropy} Ergodic Theory and Dynamical Systems, Volume 10, Issue 2, June 1990, pp. 395-406. \\ 
Harry Furstenburg \textbf{Disjointness in Ergodic Theory, Minimal Sets, and a Problem in Diophantine Approximation} Mathematical Systems Theory, March 1967, Volume 1, Issue 1, pp. 1-49.
\item \dots
\end{thebibliography}
\end{document}