\documentclass[12pt]{article}
%Gumm{\color{blue}i}|065|=)
\usepackage{amsmath, amsfonts, amssymb}
\usepackage[margin=0.5in]{geometry}
\usepackage{xcolor}
\usepackage{graphicx}
\usepackage{amsmath}
\usepackage{hyperref}
\usepackage{multicol}

\newcommand{\off}[1]{}
\DeclareMathSizes{20}{30}{20}{18}
\usepackage{tikz}


\title{Tune-Up: Pythagoras Triples}
\date{}
\begin{document}

\sffamily

\maketitle

I seem to confuse two slightly different problems in Number Theory: pythagoras triples and primes as the sum of two squares.  Without these, there's not much hope for anything else.   The more we solve it, the more we can ask questions about the various methods, which seem a bit arbitrary.  These advanced methods, seem to lump together entire classes of problems into a single bucket, without looking at any individual problem too carefully. \\ \\
Ferma's Theorem says $p = a^2 + b^2$ if $p = 4k+1$.  This is true for $p \in \mathbb{Z}$.  In order to do such a thing, we have actually called on \textit{complex numbers} since we could say:
$$ p = a^2 + b^2 = (a+bi)(a-bi) \in \mathbb{Z}[i] $$ 
In that case, we could define this as a statement of \textbf{ideals} in $\mathbb{Z}[i]$.  We are trying to find the prime ideals $\mathfrak{p} \subset \mathbb{Z}[i]$.  
$$ [ p \mathbb{Z}[i]: \mathbb{Z}[i]]  = [ p \mathbb{Z}[i] : (a + bi)\mathbb{Z}[i]] [(a + bi)\mathbb{Z}[i] : \mathbb{Z}[i]]$$
So that $p = \mathfrak{p}\overline{\mathfrak{p}}$, with $\mathfrak{p} = a + bi \in \mathbb{Z}[i]$. We are counting the sizes of the ideals\\ \\
A numerial example would to find a large prime number.   
$$ 271 \times 271 + 476 \times 476 = 300017 $$ 
and therefore we get a factorization of prime ideals.
$$ \big[ 300017 \, \mathbb{Z}[i] : \mathbb{Z}[i]\big]
 = \big[ 300017 \, \mathbb{Z}[i] : (271 + 476i)\mathbb{Z}[i]\big] \;
 \big[ (271 + 476i) \mathbb{Z}[i] : \mathbb{Z}[i] \big]$$
How does this neat and tidy world translate to the messy world of empirical data and statistics where nothing is certain?  Where $\mathbb{R}$ is no longer adequate, since these are just limits of sequences of other measurements. \\ \\
Computationally, how would we try to find numbers $a, b \in \mathbb{Z}$ such that $a^2 + b^2 = p = 300017$ and how did we know it was a prime number? 
\begin{itemize}
\item $\sqrt{300017 - 178^2} \approx 518 $ in that it's off by a small amount  $\sqrt{p - a_1^2} = [\dots]   + 0.0087$
\end{itemize}
This is an \textit{approximation} type problem which gives us one more close solution in addition to the exact answer. 

\newpage

\textbf{Examples}
% https://www.overleaf.com/learn/latex/Multiple_columns
\begin{multicols}{2}
\begin{itemize}
\item $6\times6 + 100\times100 = 10036$
\item $\sqrt{ {\color{green!75!black}{10037}} - 6\times 6 } \approx [...] + 0.005$
\item $36\times36 + 94\times94 = 10132$
\item $\sqrt{ {\color{green!75!black}{10133}} - 36\times 36 } \approx [...] + 0.00532$
\item $24\times24 + 98\times98 = 10180$
\item $\sqrt{ {\color{green!75!black}{10181}} - 24\times 24 } \approx [...] + 0.0051$
\item $56\times56 + 84\times84 = 10192$
\item $\sqrt{ {\color{green!75!black}{10193}} - 56\times 56 } \approx [...] + 0.00595$
\item $54\times54 + 86\times86 = 10312$
\item $\sqrt{ {\color{green!75!black}{10313}} - 54\times 54 } \approx [...] + 0.00581$
\item $72\times72 + 72\times72 = 10368$
\item $\sqrt{ {\color{green!75!black}{10369}} - 72\times 72 } \approx [...] + 0.00694$
\item $36\times36 + 96\times96 = 10512$
\item $\sqrt{ {\color{green!75!black}{10513}} - 36\times 36 } \approx [...] + 0.00521$
\item $14\times14 + 102\times102 = 10600$
\item $\sqrt{ {\color{green!75!black}{10601}} - 14\times 14 } \approx [...] + 0.0049$
\item $60\times60 + 84\times84 = 10656$
\item $\sqrt{ {\color{green!75!black}{10657}} - 60\times 60 } \approx [...] + 0.00595$
\item $68\times68 + 78\times78 = 10708$
\item $\sqrt{ {\color{green!75!black}{10709}} - 68\times 68 } \approx [...] + 0.00641$
\item $18\times18 + 102\times102 = 10728$
\item $\sqrt{ {\color{green!75!black}{10729}} - 18\times 18 } \approx [...] + 0.0049$
\item $6\times6 + 104\times104 = 10852$
\item $\sqrt{ {\color{green!75!black}{10853}} - 6\times 6 } \approx [...] + 0.00481$
\item $22\times22 + 102\times102 = 10888$
\item $\sqrt{ {\color{green!75!black}{10889}} - 22\times 22 } \approx [...] + 0.0049$
\item $6\times6 + 106\times106 = 11272$
\item $\sqrt{ {\color{green!75!black}{11273}} - 6\times 6 } \approx [...] + 0.00472$
\item $42\times42 + 98\times98 = 11368$
\item $\sqrt{ {\color{green!75!black}{11369}} - 42\times 42 } \approx [...] + 0.0051$
\item $24\times24 + 104\times104 = 11392$
\item $\sqrt{ {\color{green!75!black}{11393}} - 24\times 24 } \approx [...] + 0.00481$
\item $4\times4 + 108\times108 = 11680$
\item $\sqrt{ {\color{green!75!black}{11681}} - 4\times 4 } \approx [...] + 0.00463$
\item $6\times6 + 108\times108 = 11700$
\item $\sqrt{ {\color{green!75!black}{11701}} - 6\times 6 } \approx [...] + 0.00463$
\item $30\times30 + 104\times104 = 11716$
\item $\sqrt{ {\color{green!75!black}{11717}} - 30\times 30 } \approx [...] + 0.00481$
\item $24\times24 + 106\times106 = 11812$
\item $\sqrt{ {\color{green!75!black}{11813}} - 24\times 24 } \approx [...] + 0.00472$
\item $48\times48 + 98\times98 = 11908$
\item $\sqrt{ {\color{green!75!black}{11909}} - 48\times 48 } \approx [...] + 0.0051$
\item $0\times0 + 110\times110 = 12100$
\item $\sqrt{ {\color{green!75!black}{12101}} - 0\times 0 } \approx [...] + 0.00455$
\item $36\times36 + 104\times104 = 12112$
\item $\sqrt{ {\color{green!75!black}{12113}} - 36\times 36 } \approx [...] + 0.00481$
\item $22\times22 + 108\times108 = 12148$
\item $\sqrt{ {\color{green!75!black}{12149}} - 22\times 22 } \approx [...] + 0.00463$
\item $64\times64 + 90\times90 = 12196$
\item $\sqrt{ {\color{green!75!black}{12197}} - 64\times 64 } \approx [...] + 0.00556$
\item $24\times24 + 108\times108 = 12240$
\item $\sqrt{ {\color{green!75!black}{12241}} - 24\times 24 } \approx [...] + 0.00463$
\item $60\times60 + 94\times94 = 12436$
\item $\sqrt{ {\color{green!75!black}{12437}} - 60\times 60 } \approx [...] + 0.00532$
\item $66\times66 + 90\times90 = 12456$
\item $\sqrt{ {\color{green!75!black}{12457}} - 66\times 66 } \approx [...] + 0.00556$
\item $12\times12 + 112\times112 = 12688$
\item $\sqrt{ {\color{green!75!black}{12689}} - 12\times 12 } \approx [...] + 0.00446$
\item $78\times78 + 82\times82 = 12808$
\item $\sqrt{ {\color{green!75!black}{12809}} - 78\times 78 } \approx [...] + 0.0061$
\item $34\times34 + 108\times108 = 12820$
\item $\sqrt{ {\color{green!75!black}{12821}} - 34\times 34 } \approx [...] + 0.00463$
\item $54\times54 + 100\times100 = 12916$
\item $\sqrt{ {\color{green!75!black}{12917}} - 54\times 54 } \approx [...] + 0.005$
\item $2\times2 + 114\times114 = 13000$
\item $\sqrt{ {\color{green!75!black}{13001}} - 2\times 2 } \approx [...] + 0.00439$
\item $6\times6 + 114\times114 = 13032$
\item $\sqrt{ {\color{green!75!black}{13033}} - 6\times 6 } \approx [...] + 0.00439$
\item $38\times38 + 108\times108 = 13108$
\item $\sqrt{ {\color{green!75!black}{13109}} - 38\times 38 } \approx [...] + 0.00463$
\item $24\times24 + 112\times112 = 13120$
\item $\sqrt{ {\color{green!75!black}{13121}} - 24\times 24 } \approx [...] + 0.00446$
\item $64\times64 + 96\times96 = 13312$
\item $\sqrt{ {\color{green!75!black}{13313}} - 64\times 64 } \approx [...] + 0.00521$
\item $20\times20 + 114\times114 = 13396$
\item $\sqrt{ {\color{green!75!black}{13397}} - 20\times 20 } \approx [...] + 0.00439$
\item $0\times0 + 116\times116 = 13456$
\item $\sqrt{ {\color{green!75!black}{13457}} - 0\times 0 } \approx [...] + 0.00431$
\item $74\times74 + 90\times90 = 13576$
\item $\sqrt{ {\color{green!75!black}{13577}} - 74\times 74 } \approx [...] + 0.00556$
\item $72\times72 + 92\times92 = 13648$
\item $\sqrt{ {\color{green!75!black}{13649}} - 72\times 72 } \approx [...] + 0.00543$
\item $18\times18 + 116\times116 = 13780$
\item $\sqrt{ {\color{green!75!black}{13781}} - 18\times 18 } \approx [...] + 0.00431$
\item $78\times78 + 88\times88 = 13828$
\item $\sqrt{ {\color{green!75!black}{13829}} - 78\times 78 } \approx [...] + 0.00568$
\item $36\times36 + 112\times112 = 13840$
\item $\sqrt{ {\color{green!75!black}{13841}} - 36\times 36 } \approx [...] + 0.00446$
\item $76\times76 + 90\times90 = 13876$
\item $\sqrt{ {\color{green!75!black}{13877}} - 76\times 76 } \approx [...] + 0.00556$
\item $24\times24 + 116\times116 = 14032$
\item $\sqrt{ {\color{green!75!black}{14033}} - 24\times 24 } \approx [...] + 0.00431$
\item $34\times34 + 114\times114 = 14152$
\item $\sqrt{ {\color{green!75!black}{14153}} - 34\times 34 } \approx [...] + 0.00439$
\item $18\times18 + 118\times118 = 14248$
\item $\sqrt{ {\color{green!75!black}{14249}} - 18\times 18 } \approx [...] + 0.00424$
\item $36\times36 + 114\times114 = 14292$
\item $\sqrt{ {\color{green!75!black}{14293}} - 36\times 36 } \approx [...] + 0.00439$
\item $52\times52 + 108\times108 = 14368$
\item $\sqrt{ {\color{green!75!black}{14369}} - 52\times 52 } \approx [...] + 0.00463$
\item $0\times0 + 120\times120 = 14400$
\item $\sqrt{ {\color{green!75!black}{14401}} - 0\times 0 } \approx [...] + 0.00417$
\item $6\times6 + 120\times120 = 14436$
\item $\sqrt{ {\color{green!75!black}{14437}} - 6\times 6 } \approx [...] + 0.00417$
\item $78\times78 + 92\times92 = 14548$
\item $\sqrt{ {\color{green!75!black}{14549}} - 78\times 78 } \approx [...] + 0.00543$
\item $16\times16 + 120\times120 = 14656$
\item $\sqrt{ {\color{green!75!black}{14657}} - 16\times 16 } \approx [...] + 0.00417$
\item $36\times36 + 116\times116 = 14752$
\item $\sqrt{ {\color{green!75!black}{14753}} - 36\times 36 } \approx [...] + 0.00431$
\item $54\times54 + 110\times110 = 15016$
\item $\sqrt{ {\color{green!75!black}{15017}} - 54\times 54 } \approx [...] + 0.00455$
\item $26\times26 + 120\times120 = 15076$
\item $\sqrt{ {\color{green!75!black}{15077}} - 26\times 26 } \approx [...] + 0.00417$
\item $66\times66 + 104\times104 = 15172$
\item $\sqrt{ {\color{green!75!black}{15173}} - 66\times 66 } \approx [...] + 0.00481$
\item $0\times0 + 124\times124 = 15376$
\item $\sqrt{ {\color{green!75!black}{15377}} - 0\times 0 } \approx [...] + 0.00403$
\item $6\times6 + 124\times124 = 15412$
\item $\sqrt{ {\color{green!75!black}{15413}} - 6\times 6 } \approx [...] + 0.00403$
\item $24\times24 + 122\times122 = 15460$
\item $\sqrt{ {\color{green!75!black}{15461}} - 24\times 24 } \approx [...] + 0.0041$
\item $50\times50 + 114\times114 = 15496$
\item $\sqrt{ {\color{green!75!black}{15497}} - 50\times 50 } \approx [...] + 0.00439$
\item $48\times48 + 116\times116 = 15760$
\item $\sqrt{ {\color{green!75!black}{15761}} - 48\times 48 } \approx [...] + 0.00431$
\item $0\times0 + 126\times126 = 15876$
\item $\sqrt{ {\color{green!75!black}{15877}} - 0\times 0 } \approx [...] + 0.00397$
\item $2\times2 + 126\times126 = 15880$
\item $\sqrt{ {\color{green!75!black}{15881}} - 2\times 2 } \approx [...] + 0.00397$
\item $6\times6 + 126\times126 = 15912$
\item $\sqrt{ {\color{green!75!black}{15913}} - 6\times 6 } \approx [...] + 0.00397$
\item $40\times40 + 120\times120 = 16000$
\item $\sqrt{ {\color{green!75!black}{16001}} - 40\times 40 } \approx [...] + 0.00417$
\item $14\times14 + 126\times126 = 16072$
\item $\sqrt{ {\color{green!75!black}{16073}} - 14\times 14 } \approx [...] + 0.00397$
\item $48\times48 + 118\times118 = 16228$
\item $\sqrt{ {\color{green!75!black}{16229}} - 48\times 48 } \approx [...] + 0.00424$
\item $84\times84 + 96\times96 = 16272$
\item $\sqrt{ {\color{green!75!black}{16273}} - 84\times 84 } \approx [...] + 0.00521$
\item $22\times22 + 126\times126 = 16360$
\item $\sqrt{ {\color{green!75!black}{16361}} - 22\times 22 } \approx [...] + 0.00397$
\item $6\times6 + 128\times128 = 16420$
\item $\sqrt{ {\color{green!75!black}{16421}} - 6\times 6 } \approx [...] + 0.00391$
\item $24\times24 + 126\times126 = 16452$
\item $\sqrt{ {\color{green!75!black}{16453}} - 24\times 24 } \approx [...] + 0.00397$
\item $12\times12 + 128\times128 = 16528$
\item $\sqrt{ {\color{green!75!black}{16529}} - 12\times 12 } \approx [...] + 0.00391$
\item $26\times26 + 126\times126 = 16552$
\item $\sqrt{ {\color{green!75!black}{16553}} - 26\times 26 } \approx [...] + 0.00397$
\item $42\times42 + 122\times122 = 16648$
\item $\sqrt{ {\color{green!75!black}{16649}} - 42\times 42 } \approx [...] + 0.0041$
\item $28\times28 + 126\times126 = 16660$
\item $\sqrt{ {\color{green!75!black}{16661}} - 28\times 28 } \approx [...] + 0.00397$
\item $36\times36 + 124\times124 = 16672$
\item $\sqrt{ {\color{green!75!black}{16673}} - 36\times 36 } \approx [...] + 0.00403$
\item $0\times0 + 130\times130 = 16900$
\item $\sqrt{ {\color{green!75!black}{16901}} - 0\times 0 } \approx [...] + 0.00385$
\item $6\times6 + 130\times130 = 16936$
\item $\sqrt{ {\color{green!75!black}{16937}} - 6\times 6 } \approx [...] + 0.00385$
\item $34\times34 + 126\times126 = 17032$
\item $\sqrt{ {\color{green!75!black}{17033}} - 34\times 34 } \approx [...] + 0.00397$
\item $64\times64 + 114\times114 = 17092$
\item $\sqrt{ {\color{green!75!black}{17093}} - 64\times 64 } \approx [...] + 0.00439$
\item $48\times48 + 122\times122 = 17188$
\item $\sqrt{ {\color{green!75!black}{17189}} - 48\times 48 } \approx [...] + 0.0041$
\item $54\times54 + 120\times120 = 17316$
\item $\sqrt{ {\color{green!75!black}{17317}} - 54\times 54 } \approx [...] + 0.00417$
\item $38\times38 + 126\times126 = 17320$
\item $\sqrt{ {\color{green!75!black}{17321}} - 38\times 38 } \approx [...] + 0.00397$
\item $24\times24 + 130\times130 = 17476$
\item $\sqrt{ {\color{green!75!black}{17477}} - 24\times 24 } \approx [...] + 0.00385$
\item $8\times8 + 132\times132 = 17488$
\item $\sqrt{ {\color{green!75!black}{17489}} - 8\times 8 } \approx [...] + 0.00379$
\item $12\times12 + 132\times132 = 17568$
\item $\sqrt{ {\color{green!75!black}{17569}} - 12\times 12 } \approx [...] + 0.00379$
\item $16\times16 + 132\times132 = 17680$
\item $\sqrt{ {\color{green!75!black}{17681}} - 16\times 16 } \approx [...] + 0.00379$
\item $72\times72 + 112\times112 = 17728$
\item $\sqrt{ {\color{green!75!black}{17729}} - 72\times 72 } \approx [...] + 0.00446$
\item $18\times18 + 132\times132 = 17748$
\item $\sqrt{ {\color{green!75!black}{17749}} - 18\times 18 } \approx [...] + 0.00379$
\item $22\times22 + 132\times132 = 17908$
\item $\sqrt{ {\color{green!75!black}{17909}} - 22\times 22 } \approx [...] + 0.00379$
\item $0\times0 + 134\times134 = 17956$
\item $\sqrt{ {\color{green!75!black}{17957}} - 0\times 0 } \approx [...] + 0.00373$
\item $42\times42 + 128\times128 = 18148$
\item $\sqrt{ {\color{green!75!black}{18149}} - 42\times 42 } \approx [...] + 0.00391$
\item $48\times48 + 126\times126 = 18180$
\item $\sqrt{ {\color{green!75!black}{18181}} - 48\times 48 } \approx [...] + 0.00397$
\item $96\times96 + 96\times96 = 18432$
\item $\sqrt{ {\color{green!75!black}{18433}} - 96\times 96 } \approx [...] + 0.00521$
\item $66\times66 + 120\times120 = 18756$
\item $\sqrt{ {\color{green!75!black}{18757}} - 66\times 66 } \approx [...] + 0.00417$
\item $76\times76 + 114\times114 = 18772$
\item $\sqrt{ {\color{green!75!black}{18773}} - 76\times 76 } \approx [...] + 0.00439$
\item $54\times54 + 126\times126 = 18792$
\item $\sqrt{ {\color{green!75!black}{18793}} - 54\times 54 } \approx [...] + 0.00397$
\item $38\times38 + 132\times132 = 18868$
\item $\sqrt{ {\color{green!75!black}{18869}} - 38\times 38 } \approx [...] + 0.00379$
\item $90\times90 + 104\times104 = 18916$
\item $\sqrt{ {\color{green!75!black}{18917}} - 90\times 90 } \approx [...] + 0.00481$
\item $56\times56 + 126\times126 = 19012$
\item $\sqrt{ {\color{green!75!black}{19013}} - 56\times 56 } \approx [...] + 0.00397$
\item $24\times24 + 136\times136 = 19072$
\item $\sqrt{ {\color{green!75!black}{19073}} - 24\times 24 } \approx [...] + 0.00368$
\item $6\times6 + 138\times138 = 19080$
\item $\sqrt{ {\color{green!75!black}{19081}} - 6\times 6 } \approx [...] + 0.00362$
\item $84\times84 + 110\times110 = 19156$
\item $\sqrt{ {\color{green!75!black}{19157}} - 84\times 84 } \approx [...] + 0.00455$
\item $16\times16 + 138\times138 = 19300$
\item $\sqrt{ {\color{green!75!black}{19301}} - 16\times 16 } \approx [...] + 0.00362$
\item $60\times60 + 126\times126 = 19476$
\item $\sqrt{ {\color{green!75!black}{19477}} - 60\times 60 } \approx [...] + 0.00397$
\item $46\times46 + 132\times132 = 19540$
\item $\sqrt{ {\color{green!75!black}{19541}} - 46\times 46 } \approx [...] + 0.00379$
\item $36\times36 + 136\times136 = 19792$
\item $\sqrt{ {\color{green!75!black}{19793}} - 36\times 36 } \approx [...] + 0.00368$
\item $64\times64 + 126\times126 = 19972$
\item $\sqrt{ {\color{green!75!black}{19973}} - 64\times 64 } \approx [...] + 0.00397$
\end{itemize}
\end{multicols}
\noindent These examples are \textit{abundant} yet they cost us time and resources.  Should we look for more patterns?  How do these compare to what we already have?  What do we do about them?  Just a quick-look one of them should go to a runoff:
$$   19013 - 56^2  =  19973 - 64^2  \approx 0.0039682 $$
This one was equality.  We are looking for $\approx$ or $\asymp$ and not really $=$.  Maybe $\equiv$. And there is some typos.
\newpage
Even more patterns.
\begin{multicols}{2}
\begin{itemize}
\item $6×6 + 1000×1000 = 1000036$
\item $\sqrt{ {\color{green!50!black}{1000037}} - 6\times 6 } \approx [...] + 0.0005$
\item $90×90 + 996×996 = 1000116$
\item $\sqrt{ {\color{green!50!black}{1000117}} - 90\times 90 } \approx [...] + 0.000502$
\item $24×24 + 1000×1000 = 1000576$
\item $\sqrt{ {\color{green!50!black}{1000577}} - 24\times 24 } \approx [...] + 0.0005$
\item $302×302 + 954×954 = 1001320$
\item $\sqrt{ {\color{green!50!black}{1001321}} - 302\times 302 } \approx [...] + 0.000524$
\item $182×182 + 984×984 = 1001380$
\item $\sqrt{ {\color{green!50!black}{1001381}} - 182\times 182 } \approx [...] + 0.000508$
\item $98×98 + 996×996 = 1001620$
\item $\sqrt{ {\color{green!50!black}{1001621}} - 98\times 98 } \approx [...] + 0.000502$
\item $262×262 + 966×966 = 1001800$
\item $\sqrt{ {\color{green!50!black}{1001801}} - 262\times 262 } \approx [...] + 0.000518$
\item $612×612 + 792×792 = 1001808$
\item $\sqrt{ {\color{green!50!black}{1001809}} - 612\times 612 } \approx [...] + 0.000631$
\item $462×462 + 888×888 = 1001988$
\item $\sqrt{ {\color{green!50!black}{1001989}} - 462\times 462 } \approx [...] + 0.000563$
\item $100×100 + 996×996 = 1002016$
\item $\sqrt{ {\color{green!50!black}{1002017}} - 100\times 100 } \approx [...] + 0.000502$
\item $672×672 + 742×742 = 1002148$
\item $\sqrt{ {\color{green!50!black}{1002149}} - 672\times 672 } \approx [...] + 0.000674$
\item $284×284 + 960×960 = 1002256$
\item $\sqrt{ {\color{green!50!black}{1002257}} - 284\times 284 } \approx [...] + 0.000521$
\item $386×386 + 924×924 = 1002772$
\item $\sqrt{ {\color{green!50!black}{1002773}} - 386\times 386 } \approx [...] + 0.000541$
\item $696×696 + 720×720 = 1002816$
\item $\sqrt{ {\color{green!50!black}{1002817}} - 696\times 696 } \approx [...] + 0.000694$
\item $186×186 + 984×984 = 1002852$
\item $\sqrt{ {\color{green!50!black}{1002853}} - 186\times 186 } \approx [...] + 0.000508$
\item $54×54 + 1000×1000 = 1002916$
\item $\sqrt{ {\color{green!50!black}{1002917}} - 54\times 54 } \approx [...] + 0.0005$
\item $138×138 + 992×992 = 1003108$
\item $\sqrt{ {\color{green!50!black}{1003109}} - 138\times 138 } \approx [...] + 0.000504$
\item $216×216 + 978×978 = 1003140$
\item $\sqrt{ {\color{green!50!black}{1003141}} - 216\times 216 } \approx [...] + 0.000511$
\item $242×242 + 972×972 = 1003348$
\item $\sqrt{ {\color{green!50!black}{1003349}} - 242\times 242 } \approx [...] + 0.000514$
\item $286×286 + 960×960 = 1003396$
\item $\sqrt{ {\color{green!50!black}{1003397}} - 286\times 286 } \approx [...] + 0.000521$
\item $234×234 + 974×974 = 1003432$
\item $\sqrt{ {\color{green!50!black}{1003433}} - 234\times 234 } \approx [...] + 0.000513$
\item $258×258 + 968×968 = 1003588$
\item $\sqrt{ {\color{green!50!black}{1003589}} - 258\times 258 } \approx [...] + 0.000517$
\item $60×60 + 1000×1000 = 1003600$
\item $\sqrt{ {\color{green!50!black}{1003601}} - 60\times 60 } \approx [...] + 0.0005$
\item $306×306 + 954×954 = 1003752$
\item $\sqrt{ {\color{green!50!black}{1003753}} - 306\times 306 } \approx [...] + 0.000524$
\item $154×154 + 990×990 = 1003816$
\item $\sqrt{ {\color{green!50!black}{1003817}} - 154\times 154 } \approx [...] + 0.000505$
\item $432×432 + 904×904 = 1003840$
\item $\sqrt{ {\color{green!50!black}{1003841}} - 432\times 432 } \approx [...] + 0.000553$
\item $126×126 + 994×994 = 1003912$
\item $\sqrt{ {\color{green!50!black}{1003913}} - 126\times 126 } \approx [...] + 0.000503$
\item $336×336 + 944×944 = 1004032$
\item $\sqrt{ {\color{green!50!black}{1004033}} - 336\times 336 } \approx [...] + 0.00053$
\item $110×110 + 996×996 = 1004116$
\item $\sqrt{ {\color{green!50!black}{1004117}} - 110\times 110 } \approx [...] + 0.000502$
\item $210×210 + 980×980 = 1004500$
\item $\sqrt{ {\color{green!50!black}{1004501}} - 210\times 210 } \approx [...] + 0.00051$
\item $112×112 + 996×996 = 1004560$
\item $\sqrt{ {\color{green!50!black}{1004561}} - 112\times 112 } \approx [...] + 0.000502$
\item $268×268 + 966×966 = 1004980$
\item $\sqrt{ {\color{green!50!black}{1004981}} - 268\times 268 } \approx [...] + 0.000518$
\item $114×114 + 996×996 = 1005012$
\item $\sqrt{ {\color{green!50!black}{1005013}} - 114\times 114 } \approx [...] + 0.000502$
\item $32×32 + 1002×1002 = 1005028$
\item $\sqrt{ {\color{green!50!black}{1005029}} - 32\times 32 } \approx [...] + 0.000499$
\item $34×34 + 1002×1002 = 1005160$
\item $\sqrt{ {\color{green!50!black}{1005161}} - 34\times 34 } \approx [...] + 0.000499$
\item $450×450 + 896×896 = 1005316$
\item $\sqrt{ {\color{green!50!black}{1005317}} - 450\times 450 } \approx [...] + 0.000558$
\item $704×704 + 714×714 = 1005412$
\item $\sqrt{ {\color{green!50!black}{1005413}} - 704\times 704 } \approx [...] + 0.0007$
\item $160×160 + 990×990 = 1005700$
\item $\sqrt{ {\color{green!50!black}{1005701}} - 160\times 160 } \approx [...] + 0.000505$
\item $474×474 + 884×884 = 1006132$
\item $\sqrt{ {\color{green!50!black}{1006133}} - 474\times 474 } \approx [...] + 0.000566$
\item $310×310 + 954×954 = 1006216$
\item $\sqrt{ {\color{green!50!black}{1006217}} - 310\times 310 } \approx [...] + 0.000524$
\item $48×48 + 1002×1002 = 1006308$
\item $\sqrt{ {\color{green!50!black}{1006309}} - 48\times 48 } \approx [...] + 0.000499$
\item $418×418 + 912×912 = 1006468$
\item $\sqrt{ {\color{green!50!black}{1006469}} - 418\times 418 } \approx [...] + 0.000548$
\item $684×684 + 734×734 = 1006612$
\item $\sqrt{ {\color{green!50!black}{1006613}} - 684\times 684 } \approx [...] + 0.000681$
\item $264×264 + 968×968 = 1006720$
\item $\sqrt{ {\color{green!50!black}{1006721}} - 264\times 264 } \approx [...] + 0.000517$
\item $552×552 + 838×838 = 1006948$
\item $\sqrt{ {\color{green!50!black}{1006949}} - 552\times 552 } \approx [...] + 0.000597$
\item $138×138 + 994×994 = 1007080$
\item $\sqrt{ {\color{green!50!black}{1007081}} - 138\times 138 } \approx [...] + 0.000503$
\item $444×444 + 900×900 = 1007136$
\item $\sqrt{ {\color{green!50!black}{1007137}} - 444\times 444 } \approx [...] + 0.000556$
\item $456×456 + 894×894 = 1007172$
\item $\sqrt{ {\color{green!50!black}{1007173}} - 456\times 456 } \approx [...] + 0.000559$
\item $540×540 + 846×846 = 1007316$
\item $\sqrt{ {\color{green!50!black}{1007317}} - 540\times 540 } \approx [...] + 0.000591$
\item $318×318 + 952×952 = 1007428$
\item $\sqrt{ {\color{green!50!black}{1007429}} - 318\times 318 } \approx [...] + 0.000525$
\item $392×392 + 924×924 = 1007440$
\item $\sqrt{ {\color{green!50!black}{1007441}} - 392\times 392 } \approx [...] + 0.000541$
\item $336×336 + 946×946 = 1007812$
\item $\sqrt{ {\color{green!50!black}{1007813}} - 336\times 336 } \approx [...] + 0.000529$
\item $0×0 + 1004×1004 = 1008016$
\item $\sqrt{ {\color{green!50!black}{1008017}} - 0\times 0 } \approx [...] + 0.000498$
\item $294×294 + 960×960 = 1008036$
\item $\sqrt{ {\color{green!50!black}{1008037}} - 294\times 294 } \approx [...] + 0.000521$
\item $64×64 + 1002×1002 = 1008100$
\item $\sqrt{ {\color{green!50!black}{1008101}} - 64\times 64 } \approx [...] + 0.000499$
\item $568×568 + 828×828 = 1008208$
\item $\sqrt{ {\color{green!50!black}{1008209}} - 568\times 568 } \approx [...] + 0.000604$
\item $274×274 + 966×966 = 1008232$
\item $\sqrt{ {\color{green!50!black}{1008233}} - 274\times 274 } \approx [...] + 0.000518$
\item $200×200 + 984×984 = 1008256$
\item $\sqrt{ {\color{green!50!black}{1008257}} - 200\times 200 } \approx [...] + 0.000508$
\item $128×128 + 996×996 = 1008400$
\item $\sqrt{ {\color{green!50!black}{1008401}} - 128\times 128 } \approx [...] + 0.000502$
\item $144×144 + 994×994 = 1008772$
\item $\sqrt{ {\color{green!50!black}{1008773}} - 144\times 144 } \approx [...] + 0.000503$
\item $594×594 + 810×810 = 1008936$
\item $\sqrt{ {\color{green!50!black}{1008937}} - 594\times 594 } \approx [...] + 0.000617$
\item $192×192 + 986×986 = 1009060$
\item $\sqrt{ {\color{green!50!black}{1009061}} - 192\times 192 } \approx [...] + 0.000507$
\item $560×560 + 834×834 = 1009156$
\item $\sqrt{ {\color{green!50!black}{1009157}} - 560\times 560 } \approx [...] + 0.0006$
\item $72×72 + 1002×1002 = 1009188$
\item $\sqrt{ {\color{green!50!black}{1009189}} - 72\times 72 } \approx [...] + 0.000499$
\item $246×246 + 974×974 = 1009192$
\item $\sqrt{ {\color{green!50!black}{1009193}} - 246\times 246 } \approx [...] + 0.000513$
\item $254×254 + 972×972 = 1009300$
\item $\sqrt{ {\color{green!50!black}{1009301}} - 254\times 254 } \approx [...] + 0.000514$
\item $600×600 + 806×806 = 1009636$
\item $\sqrt{ {\color{green!50!black}{1009637}} - 600\times 600 } \approx [...] + 0.00062$
\item $42×42 + 1004×1004 = 1009780$
\item $\sqrt{ {\color{green!50!black}{1009781}} - 42\times 42 } \approx [...] + 0.000498$
\item $204×204 + 984×984 = 1009872$
\item $\sqrt{ {\color{green!50!black}{1009873}} - 204\times 204 } \approx [...] + 0.000508$
\end{itemize}
\end{multicols}
\noindent Even more skepitical than ever.  These coincidences look really lame.  Wait till you see the statement.  We need prime numbers $p, q \in 4\mathbb{Z}+1$.  And clearly $p = a_1^2 + b_1^2$ and $q = a_2^2 + b_2^2$.  We also have:
\begin{eqnarray*}
 \sqrt{ p - (c_1^2 + d_1^2)}&=& M_0  + 10^{-3}M_1 + O(10^{-6})\\
 \sqrt{ q - (c_2^2 + d_2^2)}&=& M_0' + 10^{-3}M_1 + O(10^{-6})
\end{eqnarray*}
we ask that $M_0, M_0', M_1 \in \mathbb{Z}$ with $M_0 \neq M_0'$ be different and $M_1 = M_1'$ are the same. Is this possible?  \\ \\
All these coincidences and others we haven't thought of can be put into a single framework about the arithmetic of the circle $X = \{ x^2 + y^2 - 1 = 0\}$. We are looking for solutions over $\mathbb{Q}$ or solutions where $(x + iy)^2 \in \mathbb{Q}(i)$. We could state, more specifically what we mean by ``approximate" solutions -  $x^2 + y^2 - 1 \approx 0$ in fact these coincidences were found in order to check a time-saving measure.  Every time we figure something out, we consume time and space resources which may be greater than the original problem itself. \\ \\
First of all we're asking that $p \approx c^2 + d^2$ (twice) then we're asking that the error term that's left over $0 < \sqrt{ \, p - (c^2 + d^2)\,} < 10^{-3}$ is small and amongst those we can find that $10^3 [ \sqrt{ p - (c^2 + d^2)}] \in \mathbb{Z}$ are the same.  \\ \\
A fancy way could be to define \textbf{torus} (in this case, just a circle) $\mathbf{T} = \{ x^2 + y^2 = 1\} = \{ |z| = 1\} \subseteq \mathbb{Q}(i)^\times $.  This circle or torus becomes a ``functor" that chomps up a number system such as $\mathbb{Q}$ and turns into a group
$\mathbf{T}(\mathbb{Q}) \subseteq \mathbb{Q}(i)^\times $ or $\mathbf{T}(\mathbb{R}) \subseteq \mathbb{C}^\times$.  The square roots of rational numbers form a number system as well $p =  z \overline{z} \in \mathbb{Q}^\times$.  All this information could be rescaled towards the unit circle over $\mathbb{R}$.  There's a tiny bit of measure theory.  Let $f(x) = \sqrt{x}$ and $g(x) = 10^3[x]$, then $ \{ x : (g \circ f)(x) = n   \} \cap \{ x : 0 < f(x) < 10^{-3} \} $ is measurable.  The behavior of primes should be ``random" enough to find infinintely many to fit into this basket. \\ \\
A growing amount of information is in the statement ``$x < p < 2x $ is prime" which adds to the case for measure theory.
\vfill



\begin{thebibliography}{}

\item Yann Bugeaud {\textbf{Effective simultaneous rational approximation to pairs of real quadratic numbers}} \texttt{1907.10253}
\item Ivan Nourdin, Giovanni Peccatti, Maurizia Rossi \textbf{Nodal Statistics of Planar Random Waves} \texttt{arXiv:1708.02281}
\item Zeev Rudnick, Ezra Waxman \textbf{Angles of Gaussian Primes} \texttt{arXiv:1705.07498}
\item Alexander Logunov, Eugenia Malinnikova \textbf{Nodal Sets of Laplace Eigenfunctions: Estimates of the Hausdorff Measure in Dimension Two and Three} \texttt{arXiv:1605.02595}
\end{thebibliography} 

\newpage

\noindent \textbf{08.14} If we have a solution to $x^2 + y^2 = z^2$ in integers, then we have a solution to $x^2 + y^2 = 1$ in rationals.  In contract, if we try Fermat's problem $x^2 + y^2 = p$ and divide both sides by $p$ we obtain $(x/\sqrt{p})^2 + (y/\sqrt{p})^2 = 1$ and $x/\sqrt{p} \notin \mathbb{Q}$.  So we can measure that Fermat's theorem is a little it outside of the ``scope" of Pythagoras theorem.  The equation $x^2 + y^2 = 1$ over $\mathbb{R}$ is so broad, it contains both of them, without any further information about how they fit together with the same space. \\ \\
\textbf{Q} Let's approximate the solution $(\frac{1}{\sqrt{2}})^2 + (\frac{1}{\sqrt{2}})^2 = 1$ using rational points in $x^2 + y^2 = 1$.  We could measure the distance over $\mathbb{R}$ ($p = \infty$ or $p = -1$) or possibly combined with a non-archimedian place $\mathbb{R} \times \mathbb{Q}_5$  ($p = \infty$ and $p = 5$). 
\begin{itemize}
\item Solve $x^2 + y^2 = 1$ with $x,y \in \mathbb{Q}$ and minimize $f(x,y) =  (x - \frac{1}{\sqrt{2}})^2 + (y - \frac{1}{\sqrt{2}})^2  $. 
\item Since $\sqrt{2}$ is not an element of $\mathbb{Q}$, it might be easier to measure $x^2 - 2 \in \mathbb{Q}$ and $g(x,y) = (x^2 - \frac{1}{2})$. This doesn't seem quite right since $x^2 + y^2 - 1 \equiv 0$ identically. Let's try a multiplicative function:
$$ g(x,y) = \text{max} \big( |x^2 - \tfrac{1}{2}| , |y^2 - \tfrac{1}{2}| \big) \in \mathbb{Q} $$
This is not ideal.  Another possibility, we are looking at all rational maps from $X(\mathbb{Q}) \to \mathbb{Q}$ that maybe behave like distances. 
\item Next we try to incorporate formation over the other place, such as $\mathbb{Q}_5$.
$$ \text{max} \big( |x^2 - \tfrac{1}{2}| , |y^2 - \tfrac{1}{2}| \big) \times \text{max} \big( |x^2 - \tfrac{1}{2}|_5 , |y^2 - \tfrac{1}{2}|_5 \big)  \ll 1 $$
This is clearly an element of $\mathbb{Q}$ and finding the highest power of $5$ dividing a rational number is an ``easy" problem.
\item Finally we need some way of iterating over Pythagorean triples, or at least some of them.  Let's just try powers of a single solution $z \mapsto (\frac{3}{5} + \frac{4}{5} ) \times z$.  This map preserves the unit circle and it is \textit{ergodic}.
$$ \left[ \begin{array}{cr} \frac{3}{5} & -\frac{4}{5} \\ \\
\frac{4}{5} & \frac{3}{5}\end{array} \right]^n $$   
It's just a circle, it's rotating the circle.  We could also try $ \times \frac{5}{13} +  \frac{12}{13}i$.
\item Here is a matching function over the integers: $ |\frac{m^2 - n^2}{m^2 + n^2} - \frac{1}{2}| \times |\frac{2mn}{m^2 + n^2} - \frac{1}{2}| \times 
|\frac{m^2 - n^2}{m^2 + n^2} - \frac{1}{2}|_5 \times |\frac{2mn}{m^2 + n^2} - \frac{1}{2}|_5$ with $m, n \in \mathbb{Z}$ and we could try to iterate over some or all of the Pythagorean triples, in a natural way. \\ \\
Think about it, we are solving $a^2 + b^2 = c^2$ with $a \approx b$, or $|a - b| \ll 1$.  Example:
$$ 11753^2 + 10296^2 = (5^6)^2 = (15625)^2 $$
This is a little too miraculous.  That's why we would like approximation theory, for an even bigger miracle.
\end{itemize}
This problem could be very broad.  Every time we find a dense subset (or counting measure) on the unit circle, we could try these approximation problems. \\ \\
\textbf{Q} Let $a^2 + b^2 = p$ be a solution over integers, so we are trying to approximate $(\frac{a}{\sqrt{p}}, \frac{b}{\sqrt{p}})$ which is a point in the unit circle over $\mathbb{R}$ with rational solutions to $x^2 + y^2 = 1$.  \\ \\
Rudnick and Waxman suggest looking at all primes, e.g. $10^5 < p < 2 \times 10^5$, solving $a^2 + b^2 = p$ and measuring the gaps between the angles $\theta = \tan^{-1} \frac{a}{b}$.  So\dots depending on how we look at it, there are infinitely many problems or just one or a few. 

\end{document}