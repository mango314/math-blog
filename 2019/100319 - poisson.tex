\documentclass[12pt]{article}
%Gumm{\color{blue}i}|065|=)
\usepackage{amsmath, amsfonts, amssymb}
\usepackage[margin=0.5in]{geometry}
\usepackage{xcolor}
\usepackage{graphicx}
\usepackage{amsmath}
\usepackage{hyperref}

\newcommand{\off}[1]{}
\DeclareMathSizes{20}{30}{20}{18}
\usepackage{tikz}


\title{Tune-Up: Holomorphic Functions}
\date{}
\begin{document}

\sffamily

\maketitle

\noindent From the mathematician's point of view, Number Theory is so far along many of our questions should have thorough answers and still be quite accessible.  From the academic point of view, these cases were complete decades ago.  Nature might feel a bit differently, it's up to us to try them out. \\ \\
\textbf{Thm} If $f \in \mathcal{S}(\mathbb{R})$, then
$\displaystyle \sum_{n \in \mathbb{Z}} f(x+n) = \sum_{n \in \mathbb{Z}} \hat{f}(n) \, e^{- 2\pi i n x} $. \\ \\
In particular, setting $x = 0$ we have: $\displaystyle \sum_{n \in \mathbb{Z}} f(n) = \sum_{n \in \mathbb{Z}}\hat{f}(n) $. \\ \\
Stein's book is somewhat stringent.  The Schwartz space on $\mathbb{R}$ is the space of infinitely differentiable functions so that $f$ and all it's derivatives are rapidly decreasing.
$$ \sup_{x \in \mathbb{R}} |x|^k \, |f^{\ell}(x)| < \infty \text{ for every } k, \ell \geq 0 $$
Example is the Gaussian function $f(x) = e^{-x^2}$.  There should be a Fourier transform:
$$ f(x) = \int_{\mathbb{R}} \hat{f}(\xi) \, e^{2\pi i x \xi} \, d\xi $$
\textbf{Foundation} Hopefully we have some aggreement on the symbols $\frac{d}{dx}$ and $\int_{\mathbb{R}}$ or a choice of $f(x)$.   How do we handle uncertainty in the choice of $f$? \\ \\
\textbf{Exercise (i)} Both sides are continuous.  We have that if $f \in \mathcal{S}(\mathbb{R})$ that 
\begin{itemize}
\item show that $\displaystyle \sum_{n \in \mathbb{Z}} f(x+n)$ is continuous. 
\item show that $\displaystyle \sum_{n \in \mathbb{Z}} \hat{f}(n) \, e^{2\pi i \, n x}$ is continuous.
\end{itemize}
\textbf{Exercise (i')} Both sides are continuous.  We have that if $f \in \mathcal{S}(\mathbb{R})$ that 
\begin{itemize}
\item show that $\displaystyle \sum_{n \in \mathbb{Z}} e^{-\pi(x+n)^2}$ is continuous. 
\item (optional) verify that $ f(x) = e^{- \pi s x^2} $ and 
$\hat{f}(x) = \frac{1}{\sqrt{x}} e^{ - \pi \xi^2 / x } $.  The Gaussian is it's own Fourier transform. \\ 
In fact $\mathcal{F}^4 = I$.
\item show that $\displaystyle \sum_{n \in \mathbb{Z}} \frac{1}{\sqrt{x}} \, e^{- \pi n^2/x} \, e^{2\pi i \, n x}$ is continuous.
\item Observe that $\displaystyle \sum_{n \in \mathbb{Z}} e^{-\pi(x+n)^2} = \sum_{n \in \mathbb{Z}} \frac{1}{\sqrt{x}} \, e^{- \pi n^2/x} \, e^{2\pi i \, n x}$ at least numerically.
\end{itemize}
Stein's book reviews the definition of ``integral" only the definition of $\int_{[0,1]}$ and $\int_\mathbb{R}$ and why they are numbers.  Even\dots what is ``continuity"?  What is $\mathbb{R}$?  These are not discussed in calculus class.  Contruction in $\mathbb{R}$ is done in the Analysis course. \\ \\
The example in the book of Lighthill is:
$$ 1 + 2 e^{-\pi} + 2 e^{-4\pi} + 2 e^{-9\pi} + \dots = 1 + 0.0864278 + 0.0000070 + (10^{-12}) + \dots $$
This is done at a time when there very few computers, and the estimates are to very high accuracy.\footnote{When do we need measurements of high accuracy?} and everything else is left as ``an exercise for the interested reader".   \\ \\ 
\textbf{10/30} Let's read the proof in Lighthill's book, which uses generalized functions, so this will look like the physics textbook.  We'd like to admit equations like this:
$$ \lim_{t \to \infty}  \frac{1}{\sqrt{2\pi t}} \, e^{- x^2 / t} = \delta(x) $$
The ``=" symbol breaks down, because it no longer behaves nicely if we change the value of $x$.  We only show one example, it can really break down. \\ \\
\textbf{11/01} Let's try this again.  $f \in \mathcal{S}(\mathbb{R})$ e.g. $f(x) = e^{-x^2}$.  Then
$$ \sum_{n \in \mathbb{Z}} f(x+n) = \sum_{n \in \mathbb{Z}} \hat{f}(n) e^{2\pi i \, n x} $$
. \\
Let's try using the Dirac bra-ket mnemonic that is useful to physicists:
$$\sum_{n \in \mathbb{Z}} \langle f | x+n \rangle = \sum_{n \in \mathbb{Z}} \hat{f}(n) e^{2\pi i \, n x}  $$
I can't think of a next step.  Let's try another. Looks like we shuffle information around the number line.  In fact, there's a subject called ergodic theory.  
$$ \sum_{n \in \mathbb{Z}} f(x+n) = \sum_{n \in \mathbb{Z}} \left[ \int_0^1 \left( \sum_{m \in \mathbb{Z}} f(x+m)  \right) e^{-2\pi i n x } \; dx \right] e^{2\pi i n x} $$
This logic is tautology and we have lost information about $\hat{f}(m)$. \\ \\
If $f$ and $g$ are continuous ``on the circle", then we have $f(\theta) = g(\theta)$ if $\hat{f}(m) = \hat{g}(m)$ for all $m \in \mathbb{Z}$.  The are going to say, $f(\theta) - g(\theta) = 0$ if $\hat{f}(m) - \hat{g}(m)\equiv 0$. \\ \\
The left side is using the ``method of images" and the right side the reconstructed signal using Fourier modes.  What happens if we use the Dirichlet Kernel, some sequence of functions $f_n \to f$ in some way.
\begin{eqnarray*}
S_N(f)(x) &=& \sum_{n = - N}^N \hat{f}(n) e^{in x}  \\
&=& \sum_{n = -N}^N \left( \frac{1}{2\pi} \int_{-\pi}^\pi f(y) e^{-iny} dy \right) e^{in x} \\
&=& \frac{1}{2\pi} \int_{-\pi}^\pi f(y) \left( \sum_{n = - N}^N e^{in(x-y)}\right) dy \\
&=& (f * D_N)(x)
\end{eqnarray*}
This is a reminder of how we get Dirichlet kernel.  Convolution with Dirichlet kernel leads to partial sums from $-N$ to $N$:
$$ D_N(x) = \sum_{n = -N}^N e^{inx} $$
These definitions of Fourier transform are no longer for free.  The fact that we could attach a number to the integral sign for all integers $n$ is not trivial.  No matter how we partition the real line, $\mathbb{R}$.
$$ \hat{f}(n) = \int_{-\pi}^\pi f(x) e^{inx} \, dx $$
Why does this even converge to number? \\ \\
\textbf{Thm} Let $\{ K_n\}_{n=1}^\infty$ be a fmaily of good kernels and $f$ be an integrable function on the circle.  Then 
$$ \lim_{n \to \infty} (f * K_n)(x) = f(x) $$
Interestingly it says that for finite $n$ we have that evaluating $f$  ``at" $x$ is the same as integrating $f*K_n$.
$$ (f * K_n) (x) = \frac{1}{2\pi} \int_{-\pi}^\pi f(x-y) K_n(y) dy $$
instead of evaluating $f$ ``at" $x = x_0$ we are evluating at nearbypoints $f(x \pm \epsilon)$ as well.  The proof of the good-kernels theorem is
\begin{eqnarray*}
(f * K_n)(x) - f(x) &=& \frac{1}{2\pi} \int_{-\pi}^\pi K_n (y) f(x-y) dy - f(x) \\ 
&=& \frac{1}{2\pi} \int_{-\pi}^\pi K_n(y) \big[ f(x-y) - f(x)\big] dy 
\end{eqnarray*}
This is a very funny thing.  Imagine the convergence did not occur.  $f(x)$ is continuous so that $f(x\pm \epsilon)$ is always close to $f(x)$ itself.  $|f(x-y) - f(x) |< \epsilon$. \\ \\ 
\textbf{Problem} Let $f(x) = 0$ when $x < 0$ and $f(x) > 1$ if $x \geq 0$.  One can construct Riemann integrable functions on $[0,1]$ that have a dense set of discontinuities.  Choose a countable dense sequence $\{ r_n\}$ on $[0,1]$.  Then show 
$$ F(x) = \sum_{n = 1}^\infty \frac{1}{n^2} f(x - r_n) $$
is ``integrable" (on $[0,1]$) and has discontinities at all the points $r_n$.  The hint is $f$ is monotonic and bounded, it's the sum of step functions, $0 \leq f \leq 1$, so that $0 \leq \sum f \leq \sum \frac{1}{n^2} < 2$.  This function could be difficult to draw and this was an easy way to create many of these.  Conversely, this thing is continuous at any point that's not in $[0,1] \backslash \{ r_n : n \in \mathbb{N}\}$.
\begin{eqnarray*}
|(f*K_n)(x) - f(x)| & \approx & \frac{1}{2\pi} \int_{|y|< \delta } |K_n(y)| |f(x-y) - f(x)| < \epsilon \; \frac{M}{2\pi} \\
\end{eqnarray*}  
Just like physicist we could ignore the error term we have created by using this approximation.  We have separated the convolution into a multiplication
\begin{itemize}
\item $\epsilon$ is the smallness of the error due to continuity
\item $M$ is the error of the good kernel
\item $2\pi$ is the circumference of the circle (which is implied). 
\end{itemize}
\vfill



\begin{thebibliography}{}

\item Stein \dots
\item Lighthill \textbf{Fourier Analysis and Generalized Functions} \dots
\item K\"{o}rner 
\end{thebibliography}

\end{document}