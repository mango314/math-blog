\documentclass[12pt]{article}
%Gumm{\color{blue}i}|065|=)
\usepackage{amsmath, amsfonts, amssymb}
\usepackage[margin=0.5in]{geometry}
\usepackage{xcolor}
\usepackage{graphicx}
\usepackage{amsmath}
\usepackage{hyperref}

\newcommand{\off}[1]{}
\DeclareMathSizes{20}{30}{20}{18}
\usepackage{tikz}


\title{Tune-Up: Fubini Theorem}
\date{}
\begin{document}

\sffamily

\maketitle

\noindent What is additive combinatorics and can we extract more common-sense statements from them?  For the time being, these are stated in a somewhat sophisticated manner.  \\ \\
These two exercises, require Lebesgue integrals (that's how they are stated).  \\ \\
\textbf{Ex \#1} Let $f \in L^1(\mathbb{R}) \cap L^2(\mathbb{R})$.  Show that:
$$ \int_\mathbb{R} \int_\mathbb{R} f(x) f(x+t) dx dt \leq || f||^2_{L^1(\mathbb{R})} $$
Hint: Fubini Theorem.  Here, ``$\int$" is Lebesgue integral. \\ \\
\textbf{Ex \#2} For any $t \in \mathbb{R}$, 
$$ \int_\mathbb{R} f(x) f(x+t) dx \leq || f||^2_{L^2(\mathbb{R})} $$
Hint: Cauchy-Schwartz inequality. \\ \\
The author is at Yale (this is the same department as Grigori Margulis and Richard Kenyon, for example.) Here's the toy example have about the autocorrelation of functions.  \\ \\
\textbf{Proposition} Let $f \in L^1(\mathbb{R}) \cap L^2(\mathbb{R}) $ Then
$$ \left|\left| \int_\mathbb{R} f(x) f(x+t) dt - \chi_{[-1,1]}(t) \right|\right|_{L^2(\mathbb{R})} \geq \frac{3}{10} $$
Hint: The Fourier transform is unitary. \\ \\
\textbf{Analogy} A set $A \subset \mathbb{Z}$ is a difference basis with respect to $n$ if 
$$ \{ 1, 2, \dots, n \} \subseteq A - A $$
What is the minimum size of $A$? This is an example due to Hungarian mathematicians Redei and Renyi.  Here is the trivial estimate:
$$ n = \# \{ 1, 2, \dots, n \} \leq \# ( (A-A) \cap \mathbb{N}  ) \leq \binom{|A|}{2} \leq \frac{|A|^2}{2} $$
therfore the size of the mimimum set is $|A| \geq \sqrt{2n} $. \\ \\
\textbf{Exercise} (Half-of-Fubini) Let $f \in L^1(\mathbb{R})$, then 
$$ \int_0^1 \int_\mathbb{R} f(x) f(x+t) dx dt \leq 0.5 ||f||^2_{L^1(\mathbb{R})} $$
What are the reasonable choices of $f$ ? This type of number theory is describing patterns that we find spread out in physical space, and frequency space and interactions.

\newpage 

\noindent If we look at the other result\dots \\ \\
\textbf{Exercise} For a sequence $\{ x_n\}$ with $0 \leq x_n \leq 1$ of independent, uniformly distributed random variables then
$$\lim_{N \to \infty} \frac{1}{N} \left\{ 1 \leq m \neq n \leq N : |x_m - x_n| \leq \frac{s}{N} \right\} = 2s $$
This is known as ``Poissonian pair correlation" in the literature. \\ \\
Our sequence of numbers $x_n$ is \textbf{not} random, in fact maybe a deterministic or pseudo-random sequence. \\ \\
Pair correlation is one of the measures being used to count ``coincidences" of a general kind, that could happen in a dynamical system.   So this result explores connection between two concepts:
$$ [\text{uniform distribution}] \stackrel{?}{\longleftrightarrow} [\text{Poisson pair correlation}]$$
There are three separate statements that are considered:
\begin{itemize}
\item Let $\{ x_n\}_{n=1}^\infty$ be a sequence on $[0,1]$.  For all $s > 0$:
$$ \lim_{N \to \infty} \frac{1}{N} \# \left\{ 1 \leq m \neq n \leq N : |x_m - x_n| \leq \frac{s}{N} \right\} = 2s $$
then the sequence is uniformly distributed. 
\item Let $\{ x_n\}_{n=1}^\infty$ be a sequence on $[0,1]$.  For all $s > 0$:
$$ \lim_{N \to \infty} \frac{1}{N} \# \left\{ 1 \leq m \neq n \leq N : ||x_m - x_n||_\infty \leq \frac{s}{N} \right\} = 2s $$
then the sequence is uniformly distributed. 
\item Let $\{ x_n\}_{n=1}^\infty$ be a sequence on $[0,1]$.  For all $s > 0$:
$$ \lim_{N \to \infty} \frac{1}{N} \# \left\{ 1 \leq m \neq n \leq N : ||x_m - x_n||_2 \leq \frac{s}{N} \right\} = 2s $$
\end{itemize}
Then the sequence is uniformly distributed. What does this say about numbers, and number sequences?  And why do we need three different statements?  Where did the classical Mathematics go?
\vfill



\begin{thebibliography}{}
\item Richard C. Barnard, Stefan Steinerberger.  \textbf{Three Convolution Inequalities on the Real Line with Connections to Additive Combinatorics} Journal of Number Theory 207 (2020) 42-55.

\item Stefan Steinerberger \textbf{Poissonian Pair Correlation in Higher Dimensions} \\ Journal of Number Theory 208 (2020) 47-58.
\item Elias Stein \textbf{Real Analysis: Measure Theory, Integration and Hilbert Spaces} (Princeton Lectures in Analysis, III) Princeton University Press, 2005.
\item Yitzak Katznelson \textbf{Harmonic Analysis} Dovr Publications, 1976.
\item Manfred Einsiedler, Thomas Ward \textbf{Ergodic Theory with a View Towards Number Theory} (GTM \#259) Springer, 2011.
\end{thebibliography}


\end{document}