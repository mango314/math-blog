\documentclass[12pt]{article}
%Gumm{\color{blue}i}|065|=)
\usepackage{amsmath, amsfonts, amssymb}
\usepackage[margin=0.5in]{geometry}
\usepackage{xcolor}
\usepackage{graphicx}
\usepackage{amsmath}
\usepackage{hyperref}

\newcommand{\off}[1]{}
\DeclareMathSizes{20}{30}{20}{18}
\usepackage{tikz}


\title{Scratchwork: Jacobi Triple-Product Formula}
\date{}
\begin{document}

\sffamily

\maketitle

\noindent Comparing apples and oranges.  There is lots of discussion of theta functions in the literature.  How do we normalize our definitions?  Here's an examplle from Integrable Systems:
$$ \theta_1(u;p) = \theta_1(u) = 2 p^{1/4} \sin u \prod_{n=1}^\infty 
\big( 1 - 2 p^{2n} \cos 2u + p^{4n} \big)(1 - p^{2n}) $$
These are called \textbf{elliptic theta functions}, and the branch of mathematics is called \textit{integrable systems}.   Here is anothe theta function:
$$ \theta_4(u;p) = \theta_4(u) = \prod_{n=1}^\infty \big( 1 - 2 p^{2n-1} \cos 2u + p^{4n} \big) (1-p^{2n}) $$
Why is this going to be so confusing.  Here's another definition of theta function.
\begin{eqnarray*}
\overline{\theta}_1( x | \tau) &=& 
2 e^{i\pi \frac{\tau}{4}} \sin (x) \prod_{n=1}^\infty \big(1 - e^{2ix} e^{\pi i \tau (2n)}\big)\big(1 - e^{-\pi i x} e^{\pi i \tau (2n)}\big) \\ 
\overline{\theta}_4( x | \tau) &=& \prod_{n=1}^\infty \big(1 - e^{2ix} e^{\pi i \tau (2n -1)}\big)\big(1 - e^{-\pi i x} e^{\pi i \tau (2n -1 )}\big)
\end{eqnarray*} 
and the Jacobi theta function, which is missing a factor of an infinite product.  
$$ \theta_i(x|\tau) = \prod_{n=1}^\infty (1 - e^{2\pi i \tau n }) \theta_i(x|\tau) $$
The same discussion tells is there are some modlarity transformations available to us:
\begin{eqnarray*} 
\theta_1(z|\tau + 1) &\stackrel{S}{=}& \exp \left( \frac{\pi i}{4} \right) \theta_1(z|\tau) \\ 
\theta_1\left(z | - \frac{1}{\tau} \right) &\stackrel{T}{=}& -i \sqrt{- \tau i}\exp\left( - \frac{1}{\pi i} \tau z^2 \right) \theta_1(z\tau | \tau) \\
\theta_1\left(z | \frac{\tau}{1-\tau}\right) &\stackrel{STS}{=}& \exp \left( \frac{\pi i }{4} \right)
\sqrt{1 - \tau}  \exp\left( - \frac{1}{i\pi} (\tau - 1) \right) \theta_1 \big( z(\tau - 1) | \tau \big)  \end{eqnarray*}
These formulas are unfortunately, very complicated have an unclear meaning as stated and we don't know where they come from.   Here's another source of examples:
\begin{eqnarray*}
\Theta(z|\tau) &=& \sum_{n = - \infty}^\infty e^{\pi i n^2 \tau} e^{2\pi i n z} \\
\theta(\tau) &=& \sum_{n=-\infty}^\infty e^{\pi i n^2 \tau}, \tau \in \mathbb{H} \\
\vartheta(t) &=& \sum_{n=-\infty}^\infty e^{-\pi n^2 t}, t > 0
\end{eqnarray*}
Between these three definitions, we get there is a symmetry within the set of perfect squares of integers $\square = \{ n^2 : n \in \mathbb{Z} \}$. 
\newpage \noindent
Here's a fourth, more varied definition of theta function which follows the same idea:
$$ \theta(z; u) = \sum_{m \in \mathbb{Z}^3} u(m) e(|m|z) $$
with $u$ is a spherical harmonic of degree $\ell$.  The Fourier coefficients are given by:
$$ a(n) = n^{\ell/2} r_3(n) W_u(n) \quad\text{with}\quad W_u(n) = \frac{1}{r_3(n)} \sum_{\xi \in V_n} u(\xi) $$ 
and we have that $\theta(z;u)$ is a holomorphic cusp form on $\Gamma_0(4)$.  \\ \\
Let's see a statement of the Jacobi Triple Product formula. Here's the exercise in Apostol's \textbf{Modular Function and Dirichlet Series in Number Theory}.  
$$ \theta(\tau) = 1 + 2\sum_{n=1}^\infty e^{\pi i n^2 \tau} = \sum_{n=-\infty}^\infty e^{\pi i n^2 \tau}  $$
The statement of the Jacobi triple product formula is the sum is equal to the product.
$$ \prod_{n=1}^\infty (1-x^{2n})(1+x^{2n-1}z^2)( 1+x^{2n-1} z^{-2}) =  \sum_{m = - \infty}^\infty x^{m^2} z^{2m} $$
Here's the exercise in \textbf{Conformal Field Theory}.  
$$ \dots $$
Here's the exercise in Elias Stein's textbook on \textbf{Complex Analysis}. 
\begin{eqnarray*}
\Theta(z|\tau) &=& \sum_{n=-\infty}^\infty e^{\i i n^2 \tau} e^{2\pi i n z} \\
\Pi(z|\tau) &=& \prod_{n=1}^\infty (1 - q^{2n}) (1 + q^{2n-1}e^{2\pi i z})(1 + q^{2n-1} e^{-2\pi i z}) \\ \\
\Theta(z|\tau) &=& \Pi (z|\tau) 
\end{eqnarray*}
$\Theta$ is entire in $z \in \mathbb{C}$ and holomorphic in $\tau \in \mathbb{H}$. 
\begin{itemize}
\item $\Theta(z+1|\tau) = \Theta(z|\tau) $
\item $\Theta(z+\tau|\tau) = \Theta(z|\tau) e^{- \pi i \tau} e^{-2\pi i z} $
\item $\Theta(z|\tau) = 0 $ whenever $z \in \frac{1}{2}(1 + \tau) + \mathbb{Z} + \tau \mathbb{Z} $.
\end{itemize} 
$\Pi$ is entire in $z \in \mathbb{C}$ and holomorphic in $\tau \in \mathbb{H}$. 
\begin{itemize}
\item $\Pi(z+1|\tau) = \Pi(z|\tau) $
\item $\Pi(z+\tau|\tau) = \Pi(z|\tau) e^{-\pi i \tau}e^{-2\pi i z } $
\item $\Pi(z|\tau) = 0$ whenever $z \in \frac{1}{2}(1 + \tau) + \mathbb{Z} + \tau \mathbb{Z}$. 
\end{itemize}
Let's see this in action.  If $\text{Im}(\tau) = t \geq t_0 > 0$ and $z = x + iy$, then $|q| \leq e^{- \pi t_0} < 1$ and 
$$ (1 - q^{2n}) ( 1 + q^{2n-1} e^{2\pi i z} )(1 + q^{2n-1} e^{- 2\pi i z} = 1 + O( |q|^{2n-1} e^{2\pi |z|}) $$
In off-handed way, we could conclude that $\Theta(z|\tau) \approx 1 + O(1)$. 
$$ \Pi(z|\tau) = \prod_{n=1}^\infty (1 - q^{2n}) (1 + q^{2n-1}e^{2\pi i z})(1 + q^{2n-1} e^{-2\pi i z}) = 1 + O\left( \sum_{n=1}^\infty  |q|^{2n-1}e^{2\pi |z|}\right)  < \infty$$

\newpage 
\noindent \textbf{Example} The addition formula for Elliptic Theta function
$$ \theta_1(u+x) \theta_1(u-x) \theta_1(v+y)\theta_1(v-y) - 
\theta_1(u+y) \theta_1(u-y) \theta_1(v+x)\theta_1(v-x) = 
\theta_1(u+v) \theta_1(u-v) \theta_1(x+y)\theta_1(x-y)
$$
\textbf{Example} The Elliptic Gamma Function
$$ \Phi_1(z; \mathbf{p}, \mathbf{q}) 
= \prod_{j,k=0}^\infty \frac{1 - e^{+2iz}\mathbf{p}^{2j+1}\mathbf{q}^{2k+1}}{1 - e^{-2iz}\mathbf{p}^{2j+1}\mathbf{q}^{2k+1}} 
= \prod_{j=0}^\infty \frac{( e^{+2iz} \mathbf{p}^{2j-1}\mathbf{q}; \mathbf{q}^2)_\infty}{(e^{-2iz} \mathbf{p}^{2j-1}\mathbf{q}; \mathbf{q}^2)_\infty} $$
\vfill
\begin{thebibliography}{} 
\item Miki Wadati, Tetsuo Deguchi, Yasuhiro Akutsu \textbf{Exactly Solvable Models and Knot Theory} Physics Reports, Vol. 180, Issues 4–5, September 1989, p. 247-332
\item Andrew Kels, Masahito Yamazaki \textbf{Lens elliptic gamma function solution of the
Yang-Baxter equation at roots of unity} \texttt{1709.07148}
\item Tom Apostol \textbf{} (GTM \# ) Springer, 1990.
\item Elias Stein \textbf{Complex Analysis} (Princeton Lectures in Analysis, Book II) Princeton University Press, 2006.
\item Rodney Baxter \textbf{Exactly Solved Models in Statistical Mechanics} 197?.
\end{thebibliography}
\end{document}