\documentclass[12pt]{article}
%Gumm{\color{blue}i}|065|=)
\usepackage{amsmath, amsfonts, amssymb}
\usepackage[margin=0.5in]{geometry}
\usepackage{xcolor}
\usepackage{graphicx}
\usepackage{amsmath}
\usepackage{hyperref}

\newcommand{\off}[1]{}
\DeclareMathSizes{20}{30}{20}{18}
\usepackage{tikz}


\title{Reading: Commutative Algebra}
\date{}
\begin{document}

\sffamily

\maketitle

\noindent Let's take a look at EGA, itself.  It starts with a Chapter 0: the Ring of Fractions. \\ \\ 
Fractions are going to be the leading characters in the story of modern algebraic geometry.  Naively, these techniques are going to move around classes of geometry problems. Let's skim the next 100 pages: \\ \\
\textbf{0.1} Ring of Fractions \\ 
\indent \textbf{1.0} Rings and Algebras \\
\indent \textbf{1.1} The ``root" of an ideal.  Nilradical and radical of a ring. \\
\indent \textbf{1.2} Modules and Rings of Fractions \\ 
\indent \textbf{1.3} Functorial Properties \\
\indent \textbf{1.4} Change of the Multiplicative ``Part" \\
\indent \textbf{1.5} Change of Ring \\
\indent \textbf{1.6} Identification of the module $M$ as an ``inductdive limit" \\
\indent \textbf{1.7} Support of a module \\ \\
\textbf{0.2} Irreducible Spaces; Noetherian Spaces. \\
\indent \textbf{2.1} Irreducible Spaces \\
\indent \textbf{2.2} Noetherian Spaces \\ \\
\textbf{0.3} Notes on Beams (``Sheaves") \footnote{The translation of the French word ``\textit{faisceaux}" as ``sheaves" or ``beams".  Here ``\textit{faisceaux electrique}" is an electric wire.} \\ 
\indent \textbf{3.1} Sheaves with values within a Category. \\ 
\indent \textbf{3.2} pre-Sheaves with a base with values in an open set. \\ 
\indent \textbf{3.3} recollections on sheaves \\ 
\indent \textbf{3.4} Direct images of presheaves \\ 
\indent \textbf{3.5} recipriprocal images of presheaves \\
\indent \textbf{3.6} simple sheaves and locally simple sheaves \\
\indent \textbf{3.7} reciprocal images of presheaves and group rings \\ 
\indent \textbf{3.8} sheaves of pseudo-discrete spaces \\ \\
\textbf{0.4} Ringed Spaces \\
\indent \textbf{4.1} ringed spaces, $\mathcal{A}$-modules, $\mathcal{A}$-algebras \\
\indent \textbf{4.2} direct image of an $\mathcal{A}$-module \\
\indent \textbf{4.3} reciprocal image of a $\mathcal{B}$-module \\
\indent \textbf{4.4} relations between direct images and reciprocal images \\ \\
\textbf{0.5} Quasi-Coherent sheaves and coherent sheaves \\ 
\indent \textbf{5.1} quasi-coherent sheaves \\ 
\indent \textbf{5.2} finite type sheaves \\ 
\indent \textbf{5.3} coherent sheaves \\ 
\indent \textbf{5.4} loally free sheaves \\ 
\indent \textbf{5.5} sheaves on a ring space with local rings \\ \\
\textbf{0.6} Flatness (``Platitude")\\
\indent \textbf{6.1} Flat modules \\ 
\indent \textbf{6.2} change of rings \\ 
\indent \textbf{6.3} localization of flatness \\
\indent \textbf{6.4} Faithfully flat modules \\ 
\indent \textbf{6.5} restriction of scalars \\ 
\indent \textbf{6.6} faithfully flat rings \\ 
\indent \textbf{6.7} morphisms of flat ringed spaces \\ \\
\textbf{0.7} Adic Rings \\
\indent \textbf{7.1} Admissible Rings \\
\indent \textbf{7.2} Adic rings and projective limits \\
\indent \textbf{7.3} pre-radical Noetherian rings \\ 
\indent \textbf{7.4} quasi-finite rings over local rings \\ 
\indent \textbf{7.5} rings of formal series \dots \\ 
\indent \textbf{7.6} complete rings of fractions \\ 
\indent \textbf{7.7} complete tensor products $\otimes$ \\
\indent \textbf{7.8} topologies on modules of homomorphisms \\ 
\indent \textbf{7.9} 
  \\
Algbraic geometry seems to be not only about shapes and properties of shapes, they are about collections of shapes and how they are derived from one another, and the algebraic equations involved in discussing such shapes.  One shape turns into another and shapes form in collections.  \\ \\
This book seems to suggest there is a world of things we can derive from even basic simple shapes. 
\begin{itemize}
\item How do we describe shapes? 
\item How do we point to things?
\item What is the fundamental unit of a shape?
\item How do we say that two shapes are ``similar"?
\item How do we say two things are ``close"?
\item What number system do we use? 
\item How do we describe a point that's ``close" to an integer or whole number.
\item What is ``\textbf{proportion}"?  What is ``shape"?  What is ``commensurable"? 
\end{itemize}
The program of algebraic geometry was to describe some common shapes with equations, and maybe coordinates.  Euclid's geometry did \textit{not} use coordinates.  He describes points using geometric constructions.  At Euclid's time, they were already \textbf{idealizations} of the forms that appear in engineering. \\ \\
incommensurable numbers? \\ \\
In the mean time, these books offer us a good starting point for some nice new ideas for studying shapes.

\newpage

\includegraphics[width=5in]{cardioid-03.png}

\vfill



\begin{thebibliography}{}

\item MF Atiyah, IG MacDonald.  \textbf{Introduction to Commutative Algebra}  Addison-Wesley, 1969.

\item Alexander Grothendieck. \textbf{\'{E}l\'{e}ments de g\'{e}om\'{e}trie alg\'{e}brique : I. Le langage des sch\'{e}mas} \\
Publications Mathématiques de l'IHÉS,  Volume 4  (1960),  p. 5-228

\item Robin Hartshorne \textbf{Algebraic Geometry} (GTM \#52) Springer, 1977.

\item Ravi Vakil \textbf{Foundations of Algebraic Geometry} \texttt{https://math216.wordpress.com/} 

\end{thebibliography}

\end{document}