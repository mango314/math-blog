\documentclass[12pt]{article}
%Gumm{\color{blue}i}|065|=)
\usepackage{amsmath, amsfonts, amssymb}
\usepackage[margin=0.5in]{geometry}
\usepackage{xcolor}
\usepackage{graphicx}
\usepackage{amsmath}

\newcommand{\off}[1]{}
\DeclareMathSizes{20}{30}{20}{18}
\usepackage{tikz}


\title{Scratchwork: Decimals}
\date{}
\begin{document}

\sffamily

\maketitle

\noindent At this point in my mathematical training, I take for granted that $\mathbb{R}$ is the number system that we all use.  That $\mathbb{R}^2$ is the Euclidean plane.  In order to represent numbers in $\mathbb{R}$ we should use decimals.  Yet, when we solve equations we use Taylor series expansions or Fourier expansions or something less common.  And finally, we turn our answer into a decimal representation of a number in $\mathbb{R}$. \\ \\
We write in the decimal system base 10. We spend a few years learning a few idosyncracies of these basic operaions.  Even type-setting decimal addiion and multiplication can be a chore.  
$$ 
\begin{array}{rrrr} 
  & 4 & \hspace{-3pt}{}^12 & 3 \\
+ & 7 & 5 & 8 \\ \hline
 1 & 1 & 8 &1
\end{array} $$
A modern \textit{dynamical systems} view point is that we are studying the dynamics of the map $T: a \to (10 \times a) \% 1$ on the real number line $\mathbb{R}/\mathbb{Z}$. We multiply by 10 and then remove the integer part.   This requires an input the definition of a function: $f(x) = x \% 1 $ or sometimes written $\{ x \} $ with some bit of fastidiousness
$$ f(x) = \{ x \}   \stackrel{?}{=}  \min_{n \in \mathbb{Z}} | x - n|  $$
This definition is wrong since it returns $\{ \frac{5}{4}\} = \frac{1}{4}$ but also $\{\frac{7}{4}\} = \frac{1}{4}$, since $\frac{7}{4} - 2  = - \frac{1}{4} $.  So a careful definition of $f(x)$ is missing. \\ \\
\textbf{Exercise} Find a correct definition of $f(x) = \{ x\}$. \\ \\
At the moment all I have is this annoying defintion: $\text{min}\;A$ with $A= \{ x - n : n \in \mathbb{Z} \text{ and } x > n \}$.  And later we could ask what ``$a > b$" even means? And ``$n \in A$"?  At some point we'll become too lazy to even check. \\ \\
An insepction of the properties of $\mathbb{R}$ like this happens when we get stuck.  In addition to base $b = 10$ we could have binary $b = 2$ with digits $\{ 0, 1\}$, so that $15_{10} = 1111_2$.  There is even base systems for irrational base, so we could have silver ratio base $b = 1 + \sqrt{2}$ or Golden ratio base $b= \frac{1 + \sqrt{5}}{2}$. \\ \\
The map $a \mapsto \big[ (1 + \sqrt{2}) \times a \big] \% 1$ would have two outcomes:
\begin{itemize}
\item $0 < (1 + \sqrt{2}) \times a  < 1 $ so that $0 < a < \sqrt{2}-1$.
\item $1 < (1 + \sqrt{2}) \times a  < 2 $ so that $\sqrt{2}-1 < a < 1 $.
\end{itemize}
Then we have a partition of $[0,1)$ that behaves nicely under the dynamical system $T$ just described.  We have a binary decimal system with digits $\{ 0, 1\}$ just as before but with some unusual properties.  So what exceptional number shall we give?  Let's try $3$:
$$1  + \sqrt{2} < \mathbf{3} < (1 + \sqrt{2})^2 = 1 + 2 + 2 \times \sqrt{2} = 3 + 2 \sqrt{2} $$
So this number would have two digits before the decimal place, $3 = 1\_\,.\_\,\_\dots$
I'm not even sure if the decimal terminates.The next digit would be:
$$ 3 - (1 + \sqrt{2}) = 2 - \sqrt{2} \stackrel{?}{<} 1 + \sqrt{2} $$
and we'd like to do this without peeking\dots without reverting to the decimal system, as any calculator does.\footnote{And at some point we could inspect the how our calculators implement the decimal system.} \\ \\
\textbf{Ex}  Using Pythagoras theorem we can find that $5^2 + 12^2 = 13^2$ what happens if we write them out in decimals.  First as fractions: $(\frac{5}{13})^2 + (\frac{12}{13})^2 = 1$.  Then let's try out the decimals:
\begin{itemize}
\item $\frac{5}{13} = 0.\overline{384615}_{\;10} = \frac{384615}{9999999} $ (this fraction is exact)
\item $\frac{12}{13} = 0.\overline{923076}_{\;10} = \frac{923076}{9999999}$
\end{itemize}
There's a long division problem using $\div$ if we try to find the repeating decimal.  \\ \\
The fraction equality is \textit{exact} $5 \times (10^7 - 1) = 13 \times 384615$. This could motivate us to find result such as Fermat's Little Theorem, that $p \,\big|\, a^p - a $ or now we have a dynamical system $T: b \mapsto a \times b $, and now it says that $T^p$ as a fixed point or that $T^p - T$ has a non-trivial kernel (in Linear Algebra-speak). E.g. 
$$ 
\left[ \begin{array}{cc} 1 & 2 \\ 3 & 4 \end{array} \right]^p - 
\left[ \begin{array}{cc} 1 & 2 \\ 3 & 4 \end{array} \right] \equiv 0 \pmod p $$
Is that correct?  If we diagonalize this matrix we get two algebraic numbers, $x^2 - (1+4)x + (1 \times 4 - 2 \times 3) = 0 $ giving $x = \frac{5 \pm \sqrt{33}}{2}$.  Then we are asking if $7$ ``divides"  $(\frac{5 \pm \sqrt{33}}{2})^7 - (\frac{5 \pm \sqrt{33}}{2})$ and things like that.\\ \\
Our use of the quadratic formula starts to look bad.  Our symbol $\sqrt{x}$ means $f^{-1}(x)$ with $f(x) = x^2$.  The notation $\sqrt{a}$ is the thing that solves $x^2 = a$.  We are looking for the number that solves $x^2 - 5x - 2 = 0$.  
  \\ \\
\textbf{Note} Why do we need to inspect $\mathbb{R}$ so carefully?  Tayĺor's Theorem is going to call for infinitely many steps involvin $+$ and $-$ and $\times$ and $\div$:
$$ f(x + \epsilon) = f(x) + \epsilon \times f'(x) + \frac{\epsilon^2}{2} \times f''(x) + \dots  $$
This is also a dynamical system.  Trivially, $T : x \mapsto x + 1$ so that $T^\epsilon f(x) = f(x+\epsilon)$, so we are moving the thing slightly to the left.
\vfill 

\begin{thebibliography}{}

\item Michael Coornaert. \textbf{Topological Dimension and Dynamical Systems} (Universitext) Springer, 2015.

\item Michael Field. \textbf{Essential Real Analysis} (Springer Undergraduate Texts in Analysis) Springer, 2017.

%\item David Williams \textbf{Probability with Martingales} (Cambridge Matematical Textbooks) Cambridge University Press, 1991 / 2011.

\item Manfred Einsiedler, Thomas Ward. \textbf{Ergodic Theory: with a view towards Number Theory} \\GTM \#259 Springer, 2011.

\item Steve Smale. \textbf{The Fundamental Theorem of Algebra and Complexity Theory} Bulletin of the American Matematical Society, Vol. 4 No. (1) 1-36.


\end{thebibliography}

\end{document}