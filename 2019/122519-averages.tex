\documentclass[12pt]{article}
%Gumm{\color{blue}i}|065|=)
\usepackage{amsmath, amsfonts, amssymb}
\usepackage[margin=0.5in]{geometry}
\usepackage{xcolor}
\usepackage{graphicx}
\usepackage{amsmath}
\usepackage{hyperref}

\newcommand{\off}[1]{}
\DeclareMathSizes{20}{30}{20}{18}
\usepackage{tikz}


\title{Tune-Up: Ergodic Theorem}
\date{}
\begin{document}

\sffamily

\maketitle

\noindent Dynamical systems are difficult to discuss because they are literally describe ``everything".  Anything that ``happens" is an orbit in a dynamical system.  Turning those observations into a meaningful discussion is another matter.  \footnote{Example:  one textbook prove the \textbf{central limit theore} as a consequence of Birkhoff's \textbf{ergodic theorem}.}  A physical system might display randomness yet be quite decisive.  The elements of chance could emerge through specific determined behavior. E.g. a baseball game, water escaping a fire hydrant, cars driving on a street.  Or more theoretical, E.g. gas particles moving in a box, the billiard ball flow.  We might try to ask about \textit{periodic} behavior in these circumstances.  \\ \\
So we need to be able to choose one example or a class of examples as a way to read the text.  Or we would study ``generic" behavior.  Let's see on example:

\begin{quotation}
Ergodic theory in its broadest sense is the study of group actions on measure spaces.
Historically the discipline has tended to concentrate on the framework of integer
actions, in line with its formal origins in the work of John von Neumann and
George David Birkhoff in the early 1930s. To a considerable extent this continues
to hold today, not least due to a variety of deep interactions with number theory and
smooth dynamics. \end{quotation}
Even though the dynamical systems we study are physical, quite general, they have no obvious arithmetical properties.  Spaces like $\mathbb{R}^2$ or $\mathbb{H}$ are what happen \textit{after} uniformation, so this is after we try to force Nature to behave for us.  So there is obviously going to be a lot of noise between the generic models we choose and nature.  \\ \\
It also looks like computer programs could be a source of examples for us.  What is a computer anyway?  There is some kind of memory storage -- a tray that holds things -- and some computing process -- a dynamical system and then a result -- something on that try which we take out and move someplace else.  So, in Von Neumann's time we had a big question \textit{what is computation?} This is already being discussed in babbages time yet all cultures have had to develop bookkeeping systems. 

\begin{thebibliography}{}

\item  \dots 

\end{thebibliography}

\end{document}