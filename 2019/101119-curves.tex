\documentclass[12pt]{article}
%Gumm{\color{blue}i}|065|=)
\usepackage{amsmath, amsfonts, amssymb}
\usepackage[margin=0.5in]{geometry}
\usepackage{xcolor}
\usepackage{graphicx}
\usepackage{amsmath}
\usepackage{hyperref}

\newcommand{\off}[1]{}
\DeclareMathSizes{20}{30}{20}{18}
\usepackage{tikz}


\title{Tune-Up: Interpolation}
\date{}
\begin{document}

\sffamily

\maketitle

% stamp-collecting

\noindent Let's say we want to draw a curve passing through four points.  There are (at least) two ways to deribe a curve in flat space.  $f(x,y) = 0$ or possibly $t \mapsto (f(t), g(t))$.  It could be some work to show that one description is a good as the other.  
\begin{itemize}
	\item a curve passes through $(a_1, b_1), (a_2, b_2), (a_3, b_3), (a_4, b_4)$, so that $f(a_n, b_n) = 0$ for $n = 1,2,3,4$.  
	\item if we do the parameterization, then we have $t_n \mapsto (f(t_n), g(t_n)) = (a_n, b_n)$ where we have specified that the curve passes through $(a,b)$ at time $t$.  
\end{itemize} 
If we also specify tangents to the curve, I think it's easier to specify using patermized version.
\begin{itemize}
	\item the curve passes though point $(a,b)$ at time $t$ with tangent vector $(a', b')$.  Then $(f'(t), g'(t)) = (a', b')$.
	\item If we use equations we have that $(a,b) \in C$ is in our curve and $(a + \epsilon, b + \epsilon) \in C$ is also in our curve so that $f(a,b) = 0$ and $f(a+\epsilon, b + \epsilon) = 0$ with $\epsilon \ll 1$. 
	\begin{eqnarray*} f(a + \epsilon, b + \epsilon) &=& f(a,b) + \epsilon \left(\frac{\partial f}{\partial x}(a,b) ,  \frac{\partial f}{\partial y}(a,b) \right)\cdot (a', b') + O(\epsilon^2) = 0 \\ 
(\nabla f)(a,b)\cdot(a',b')	&=& 0 \end{eqnarray*}
	The \textbf{gradient} $\nabla f$ of the curve at the point $(a,b)$ is parallel to the tangent vector $(a', b')$. \\ \\ 
Since there are four points and four tangent vectors there are $4+4 = 8$ equations.  If we consider the vector space $\mathbb{R}[x,y]$, the subspace spanned by $ x^m y^n$ with $0 \leq m+n \leq 4$ there are $5 \times 5 = 25$ degrees of freedom.    
\end{itemize}
These considerations will make more sense if we draw the pictures.  Let's take as our four points $(a,b) = (1,0), (0,1), (-1,0), (0,-1)$.  Let's try a quadratic curve $f(x,y) =  \cdot \; x^2 + \cdot \; xy + \cdot \; y^2 + \cdot \; x + \cdot \; y + \cdot = 0$. We are only counting six degrees of freedom, because \dots I don't remember combinatorics. $ 1 + 2 + 3  = 6 = \frac{1}{2} \times 4 \times 3 = \binom{4}{2}$. \\ \\
\begin{tikzpicture}
% draw table of the possile value of m, n
\end{tikzpicture}
There's not enough ``room" in our space of conic sections to specify four points and their tangents.  On the right side, if we consider polynomials up to degree $100$ there's certainly more than enough room.  In fact, we could show there's lots of curves in this space that fit this criterion.  Let's try $N = 4$ we want $\langle x^m y^n : 0 \leq m + n \leq 4 \rangle \subseteq \mathbb{R}[x,y] $. \\ \\
We have this is a $1 + 2 + 3 + 4 + 5 = 15$ dimensional space also  $15 = \frac{1}{2} \times 6 \times 5  =  \binom{6}{2}  $.

\newpage

\noindent Let's try solving one.  We have that a circle passes through the points $0 \mapsto (1,0), 1 \mapsto (0,1),2 \mapsto (-1,0), 3 \mapsto (0,-1)$ and the tangent vectors are $(0,1), (-1,0), (0,-1), (1,0)$.  And we could try to switch the first tangent vector from $(0,1)$ to $(0,-1)$ or $(0,1) + \epsilon \, ( \cos \theta, \sin \theta )$.  And we take a guess for our curve:
$$ f(t) = \sum_{n = 0}^7 a_n e^{2\pi i n t} \text{ and } f'(t) = \sum_{n = 0}^7 n \, a_n e^{2\pi i n t}$$
and our time intervals are not $t = 0, \frac{1}{4}, \frac{1}{2}, \frac{3}{4}$.  
\begin{eqnarray*}
f(0) &=& a_0 + a_1 + a_2 + a_3 + a_4 + a_5 + a_6 + a_7 = 1 \\
f(\tfrac{1}{4}) &=& a_0 + i \, a_1 - a_2 - i \, a_3 + a_4 + i\, a_5 - a_6 - i \,a_7 = i \\
f(\tfrac{1}{2}) &=& a_0 + a_1 + a_2 + a_3 + a_4 + a_5 + a_6 + a_7 = - 1 \\
f(\tfrac{3}{4}) &=& a_0 - i a_1 - a_2 + i a_3 + a_4 - i a_5 - a_6 + i a_7 = - i \\ \\
f'(0) &=& 0\cdot a_0 + a_1 + 2 \cdot a_2 + 3 \cdot a_3 + 4 \cdot a_4 + 5 \cdot a_5 + 6 \cdot a_6 + 7 \cdot  a_7 = 1 \\
f'(\tfrac{1}{4}) &=& 0\cdot a_0 + i \, a_1 - 2 \cdot a_2 - 3 \cdot i \, a_3 + 4 \cdot a_4 + 5 \cdot i\, a_5 - 6 \cdot a_6 - 7 \cdot i \,a_7 = i \\
f'(\tfrac{1}{2}) &=& 0\cdot a_0 + a_1 + 2 \cdot a_2 + 3 \cdot a_3 + 4 \cdot a_4 + 5 \cdot a_5 + 6 \cdot a_6 + 7 \cdot  a_7 = - 1 \\
f'(\tfrac{3}{4}) &=& 0\cdot a_0 - i a_1 - 2 \cdot a_2 + 3 \cdot i a_3 + 4 \cdot a_4 - 5 \cdot i a_5 - 6 \cdot a_6 + 7 \cdot  i a_7 = - i 
\end{eqnarray*} 
Linear algebra just turns this into an 8 dimensional space of possible curves.
$$ 
\left[ \begin{array}{rrrrrrrr}
1 & 1 & 1 & 1 & 1 & 1 & 1 & 1 \\
1 & i & -1 & -i & 1 & i & -1 & -i \\
1 & -1 & 1 & -1 & 1 & -1 & 1 & -1 \\
1 & -i & -1 & i & 1 & -i & -1 & i \\ \hline
0 & 1 & 2 & 3 & 4 & 5 & 6 & 7 \\
0 & i & -2 & -3i & 4 & 5i & -6 & -7i \\
0 & -1 & 2 & -3 & 4 & -5 & 6 & -7 \\
0 & -i & -2 & 3i & 4 & -5i & -6 & 7i \\
 \end{array} \right] 
 \left[ \begin{array}{c} a_0 \\
 a_1 \\ 
 a_2 \\ 
 a_3 \\ 
 a_4 \\ 
 a_5 \\ 
 a_6 \\ 
 a_7 \end{array} \right]
 = 
  \left[ \begin{array}{r} 1 \\
 i \\ 
 -1 \\ 
 -i \\ \hline
 i \\ 
 -1 \\ 
 -i \\ 
 1 \end{array} \right] $$
The answer should be a circle.  Let's hope that our matrix math still works here.

\includegraphics[width=4in]{graph-01.png} \\
The circle is a conic section, the outer curve which is also tangent at $(\pm 1,0), (0, \pm 1)$ now has tangents pointing in the opposite direction, has degree $\geq 6$.  Easy examples like these -- exercises in Linear Algebra -- can motivate Algebraic Geometry and the Hilbert Nullstellensatz (theorem on whether curves and equations are the same thing). The Python code here is written in a tedious way and could be more elegant:
\begin{verbatim}
import numpy as np
import matplotlib.pyplot as plt

M = 2
for m in np.arange(M):
    A = np.zeros((8,8))*0j
	// curve passes through four points
    A[:,0] = np.exp(2j*np.pi*(1.0/8)*np.arange(8)*0)
    A[:,1] = np.exp(2j*np.pi*(1.0/8)*np.arange(8)*2)
    A[:,2] = np.exp(2j*np.pi*(1.0/8)*np.arange(8)*4)
    A[:,3] = np.exp(2j*np.pi*(1.0/8)*np.arange(8)*6)

	// tangents at four points
    A[:,4] = np.arange(8)*(2*np.pi*1j)
    A[:,5] = np.arange(8)*(2*np.pi*1j)
    A[:,6] = np.arange(8)*(2*np.pi*1j)
    A[:,7] = np.arange(8)*(2*np.pi*1j)

    A[:,4] *= np.exp(2j*np.pi*(1.0/8)*np.arange(8)*0)
    A[:,5] *= np.exp(2j*np.pi*(1.0/8)*np.arange(8)*2)
    A[:,6] *= np.exp(2j*np.pi*(1.0/8)*np.arange(8)*4)
    A[:,7] *= np.exp(2j*np.pi*(1.0/8)*np.arange(8)*6)

  // specify tangents at four points
    a = np.array([1, 1j, -1, -1j, 
    	1*(2*np.pi*1j)*np.exp(2j*np.pi*(1.0/M)*m), 
    	1j*(2*np.pi*1j)*np.exp(2j*np.pi*(1.0/M)*m), 
    	-1*(2*np.pi*1j)*np.exp(2j*np.pi*(1.0/M)*m), 
    	-1j*(2*np.pi*1j)*np.exp(2j*np.pi*(1.0/M)*m) ])

  // solve for coefficents
    b = np.dot( np.linalg.inv(A).T , a)

  // draw curve
    dt = 10**-3
    t  = np.arange(0,1+dt,10**-3)

    z = np.exp(2j*np.pi*t)

    w = 0
    for n in range(8):
        w += z**n*b[n]

    plt.plot(w.real, w.imag, 'k-', linewidth=0.5)
    plt.plot(w[::250].real, w[::250].imag, 'k.', markersize=20)
plt.grid(True)

\end{verbatim}

\noindent This is good enough for numerical methods.  More interesting examples await.

\vfill



\begin{thebibliography}{}

\item \dots 

\end{thebibliography}

\end{document}