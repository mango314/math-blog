\documentclass[12pt]{article}
%Gumm{\color{blue}i}|065|=)
\usepackage{amsmath, amsfonts, amssymb}
\usepackage[margin=0.5in]{geometry}
\usepackage{xcolor}
\usepackage{graphicx}
\usepackage{amsmath}
\usepackage{hyperref}

\newcommand{\off}[1]{}
\DeclareMathSizes{20}{30}{20}{18}
\usepackage{tikz}


\title{Tune-Up: Curves}
\date{}
\begin{document}

\sffamily

\maketitle

\noindent Let me draw a sequence of points and a path through them. $x[0]$, $x[1]$ and $x[2]$. The curve as an edge at $x[1]$ how can we make the thing ``smoother"?  There are infinintey many curves $\phi:[0,1] \to \mathbb{R}^2$ that start with $\phi(0) = x[0]$ and $\phi(1) = x[1]$.  Here is an example:
$$ \phi_0(t) = x[0] \cdot (1-t)^3 + A \cdot 3(1-t)^2 t + B \cdot 3(1-t)t^2 + x[1] \cdot t^3 $$
Then we have a second curve connecting $x[1]$ and $x[2]$ of the same ``cubic" type.
$$ \phi_1(t) = x[1] \cdot (2-t)^3 + A_1 \cdot 3(2-t)^2(t-1) + B_1 \cdot 3(2-t)(t-1)^2 + x[2] \cdot (t-1)^3 $$
and we observee that $\phi_1(1) = x[1]$ and $\phi_1(2) = x[2]$.  Then let's ``glue" the two functions together:
$$ \phi(t) = \left\{ 
\begin{array}{cc} \phi_0(t) & t \in [0,1] \\
\phi_1(t) & t \in [1,2] \end{array}
\right. $$
We have that $\phi(0^+) = \phi(0^-)$.  This is a great example to draw since it appears both in Calculus class and also in Topology class. \\ \\
\begin{tikzpicture}
\draw (-2,0)--(2,0);
\draw (0,-2)--(0,2);

\draw[line width=2] (-2,-0.5)--(0,-0.5);
\draw[line width=2] (0,0.5)--(2,0.5);
\draw[fill=white] (0,0.5) circle (0.1);
\draw[fill=white] (0,-0.5) circle (0.1);
\end{tikzpicture} vs. 
\begin{tikzpicture}
\draw (-2,0)--(2,0);
\draw (0,-2)--(0,2);

\draw[line width=2] (-2,0)--(0,0);
\draw[line width=2] (0,0)--(2,0);
\draw[fill=white] (0,0) circle (0.1);
\draw[fill=white] (0,0) circle (0.1);
\end{tikzpicture} \\ \\
Here we've drawn the picture suggestively that we can join the two matchsticks together at the origin $(0,0)$ to obtain a copy of the number line (such as $\mathbb{R}$) this is not necessarily true we could have joined the two segments any way we wanted.  The decision to the value of $\phi(0)$ or the ``shape" of our number line depends on the choice of functions we want to use.  \\ \\
In our case, we want $\phi'(0^+) = \phi'(0^-)$ and maybe even $\phi''(0^+) = \phi''(0^-)$.  How can we compute the values of $A$ and $B$ in this case?  Let's do the subtraction problem:
$$ \phi'(0^+) - \phi'(0^-) = 0 \text{ and }\phi''(0^+) - \phi''(0^-) = 0 \text{ and }\phi \text{ is cubic}$$
The only solution I can think of is $\phi(t) = C \cdot t^3$.  If we set the ``jump" at $t = 1$ there is $\phi(t) \approx C\cdot (t-1)^3$ for $t \approx 1$.  Then $\phi(2) = C\cdot (2-1)^3 = C$. 
\newpage \noindent \textbf{Exercise} We have three points $A = (1,2)$, $B = (2,5)$ and $C = (3,-2)$ let's try to find a cubic curve passing through them.  

\begin{thebibliography}{}

\item  \dots 

\end{thebibliography}

\end{document}