\documentclass[12pt]{article}
%Gumm{\color{blue}i}|065|=)
\usepackage{amsmath, amsfonts, amssymb}
\usepackage[margin=0.5in]{geometry}
\usepackage{xcolor}
\usepackage{graphicx}
\usepackage{amsmath}
\usepackage{hyperref}

\newcommand{\off}[1]{}
\DeclareMathSizes{20}{30}{20}{18}
\usepackage{tikz}


\title{Reading: Quantum Coherence}
\date{}
\begin{document}

\sffamily

\maketitle

\noindent \textbf{``Coherence can be regarded as a resource and can be systematically manipulated and quantified."} \\ \\
Let's try just rearranging the abstract:
\begin{itemize}
\item ``coherence quantifies the [ intrinsic randomness ] of the outcome of the projective measurement in the system's computational basis"
\item ``however such a relation is \textit{only} proven when the randomness is characterized by the von Neumann entropy"
\item ``we consider several \textbf{\color{blue!70!black}{recently proposed coherence measures}} and relate them to the general uncertainties of the projective measurement outcome conditioned on all the other systems"
\end{itemize}
The word we're kind of nervious about is the word ``conditional".  Coherence is something that we expect, yet everytime in a genuinely real situation we quickly run out of that.  The word ``measurement" it's already a given. Also we'd like to know more about ``intrinsic randomness". \\ \\
A physicist talks about a Hilbert space $\mathcal{H}$ and yet a mathematician talks about $L^2(\mathbb{R})$ or the somewhat smaller $L^2([0,1])$.  In the end, these are just things in the ``real world".  In Quantum Computation we only consider copies of $\mathbb{C}^2$ spanned by $|\uparrow \rangle$ and $|\downarrow \rangle$.   We have finite dimensional Hilbert spaces and computations relating them and then we have Hilbert spaces in the limit. \\ \\
Here is an easy one, $S( \frac{1}{d} \sum_i | i \rangle \langle i |) = \log d$.  We'd like to remember a bit why \textbf{log} is even there.  What is log in the first place \dots?  Here $\log d$ is the number of bits requires to represent a number $n \leq d$.  This is just plain-old Shannon entropy. \\ \\
We are going to take an ordinary system and disrupt it with entropy, or high the information we need to protect behind a complex computation. Their goal seems to be to describe a TV or a general electronic device.  Quantum computer's are being built in a few small technical cases.

% more writing

\vfill



\begin{thebibliography}{}

\item Yunchao Liu, Qi Zhao, Xiao Yuan. \textbf{Quantum Coherence Via Conditional Entropy} \\
J. Phys. A: Math. Theor. \textbf{51} (2018)
\item Isaac Chuang, Michael Nielsen. \textbf{Quantum Computation and Quantum Information} \\ Cambridge University Press, 2000.
\item Claude Shannon, Warren Weaver. \textbf{A Mathematical Theory of Communication} 
\item \dots 

\end{thebibliography}

\end{document}