\documentclass[12pt]{article}
%Gumm{\color{blue}i}|065|=)
\usepackage{amsmath, amsfonts, amssymb}
\usepackage[margin=0.5in]{geometry}
\usepackage{xcolor}
\usepackage{graphicx}
\usepackage{amsmath}
\usepackage{hyperref}

\newcommand{\off}[1]{}
\DeclareMathSizes{20}{30}{20}{18}
\usepackage{tikz}


\title{Scratchwork: Pythagoras Theorem}
\date{}
\begin{document}

\sffamily

\maketitle

\noindent Let $\vec{a}, \vec{b}, \vec{c} \in \mathbb{R}^3$ be vectors with $|\vec{a}|^2 + |\vec{b}|^2 + |\vec{c}|^2 = 1$.  Then $\vec{a} \times \vec{b}, \vec{b} \times \vec{c}, \vec{c} \times \vec{a}$ span the vector space $\mathbb{R}^3 \wedge \mathbb{R}^3 \simeq \mathbb{R}^3$.
$$
|  v \wedge w|^2 = |v_1 w_2 - v_2 w_1|^2 + |v_2 w_3 - v_3 w_1|^2 + |v_3 w_1 - w_3 v_1|^2  $$
This looks an awful lot like the cross product with coordinates to me.  $v = v_1 \mathbf{i} + v_2 \mathbf{j} + v_3 \mathbf{k}$ and $w = w_1 \mathbf{1} + w_2 \mathbf{j} + w_3 \mathbf{k}$.  
$$ 
v \times w = \left|
\begin{array}{ccc}
v_1 & v_2 & v_3 \\ w_1 & w_2 & w_3 \\ \mathbf{i} & \mathbf{j} & \mathbf{k} \end{array} \right|
= 
(v_1 w_2 - v_2 w_1) \mathbf{i} + (v_2 w_3 - v_3 w_2)\mathbf{j} + (v_3 w_1 - v_1 w_2)\mathbf{k}
$$
Now we have the Pythagoras Theorem in 3-space again.  Dually this can be read as De Gua's Theorem on areas of Triangles.  The area of the \textit{parallelogram} squared is te sum of the squares of the areas of the projetions. \hfill $\square$. \\ \\
This could be more meaningful in four dimensions.  Here, ``four dimensions" could mean two objects in flat space:
$$ \mathbb{R}^4 \simeq \mathbb{R}^2 \oplus \mathbb{R}^2 $$
Then we could take the wedge of two 4-vectors - this would model $([pt] \oplus [pt]) \wedge ([pt] \oplus [pt]) $.  It's hard to count the six degrees of freedom at the moment: 
$$ \mathbb{R}^4 \vee \mathbb{R}^4 \simeq \mathbb{R}^6 $$
What does Pythagoras' theorem say here? We have a Euclidean distance function on the set of pairs of points in the plane. Just two people standing in a room, with no restrictions.
$$ |v \wedge w|^2 = 
( v_1 w_2 - v_2 w_1)^2 +
( v_1 w_3 - v_3 w_1)^2 +
( v_1 w_4 - v_4 w_1)^2 +
( v_2 w_3 - v_3 w_2)^2 +
( v_2 w_4 - v_4 w_2)^2 +
( v_3 w_4 - v_4 w_3)^2   $$
The expression $|v \wedge w|$ still measures the area of the parallelogram spanned by 4-vectors $\vec{v}$ and $\vec{w}$. \\ \\ \\
\textbf{Proposition} The vectors $A,B,C \in \mathbb{R}^2$ are collinear iff $A \wedge B + B \wedge C + C \wedge A = 0$.
\vfill
\begin{thebibliography}{} 
\item \dots 
\end{thebibliography}
\end{document}