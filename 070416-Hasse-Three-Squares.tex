\documentclass[12pt]{article}
%Gummi|065|=)
\usepackage{amsmath, amsfonts, amssymb}
\usepackage[landscape, margin=0.5in]{geometry}
\usepackage{xcolor}
\usepackage{graphicx}
\newcommand{\off}[1]{}
\DeclareMathSizes{20}{30}{21}{18}

\title{\textbf{ Sum of 3 Squares Theorem, Hasse Principle, Banach-Tarski Paradox }}
\author{John D Mangual}
\date{}
\begin{document}

\fontfamily{qag}\selectfont \fontsize{25}{30}\selectfont

\maketitle

Scaling back, a much more ambitious project, we try to address three problems from number theory to measure theory:

\begin{itemize}
\item Lagrange showed $n = a^2 + b^2 + c^2$ iff $n \neq 4^a (8k+7)$
\end{itemize}
One quicky way to solve this eq is to solve in congruences:
\begin{itemize}
\item Solve equation in 2-adic numbers $n = a^2 + b^2 + c^2$ , so $n \not \equiv 0 \mod 4$
\item Solve $n = a^2 + b^2 + c^2 \textrm{ mod } p$ for all $p > 2$
\item Solve $n = a^2 + b^2 + c^2$ in $\mathbb{R}$ ( this just says $n > 0$ )
\end{itemize}
Hasse-Minkowski principle tells us this is sufficient, but why does Hasse-Minkowski principle work at all?

\newpage

\noindent\textbf{How do we know the Hasse-Minkowski principle?  }
\newline \newline
Case $n = a^2 + b^2 + c^2$ 
\newline \newline
Reading's Serre's \textit{Course on Arithmetic} we can solve in $\mathbb{Q}$:

$$ a^2 + b^2 + c^2 - n \, d^2 = 0$$

\noindent We are able to find an $x \neq 0$ in $\mathbb{Q}$ solving two quadratic eqs:

$$ a^2 + b^2 = x = c^2 - n \, d^2 $$

\noindent This works because $x \in \mathbb{Q}$ not just $ x \in \mathbb{Z}$, we'd better write

$$ a^2 + b^2 - x\, e^2=  c^2 - n \, d^2 - x \, f^2 = 0$$

\noindent Once we have solved for $x \in \mathbb{Q}$ we solve $(a,b,c), (d,e,f) \in \mathbb{Q}^3 $ \newline

\noindent $\circ$ $\circ$ $\circ$ This is reduction from 4 variables to 3 variables  

\newpage

\noindent\textbf{How do we know the Hasse-Minkowski principle?  }
\newline \newline
The only case $a^2 + b^2 + c^2 - n \, d^2 = 0$ 
\newline \newline
Reading's Serre's \textit{Course on Arithmetic} we can solve in $\mathbb{Q}$
\newline \newline
Reduce 4 variables to 3...
\newline \newline
Why is solving all congruences $n = a^2 + b^2 + c^2 \textrm{ mod }p$ enough?

\newpage

\fontfamily{qag}\selectfont \fontsize{12}{10}\selectfont

\begin{thebibliography}{}

\item JP Serre \textbf{Course on Arithmetic} Springer-Verlag



\end{thebibliography}


\end{document}
