\documentclass[12pt]{article}
%Gummi|065|=)
\usepackage{amsmath, amsfonts, amssymb}
\usepackage[landscape, margin=0.5in]{geometry}
\usepackage{xcolor}
\usepackage{graphicx}
\newcommand{\off}[1]{}
\DeclareMathSizes{20}{30}{21}{18}

\title{\textbf{ Sum of 3 Squares Theorem, Hasse Principle, Banach-Tarski Paradox }}
\author{John D Mangual}
\date{}
\begin{document}

\fontfamily{qag}\selectfont \fontsize{25}{30}\selectfont

\maketitle

Scaling back, a much more ambitious project, we try to address three problems from number theory to measure theory:

\begin{itemize}
\item Lagrange showed $n = a^2 + b^2 + c^2$ iff $n \neq 4^a (8k+7)$
\end{itemize}
One quicky way to solve this eq is to solve in congruences:
\begin{itemize}
\item Solve equation in 2-adic numbers $n = a^2 + b^2 + c^2$ , so $n \not \equiv 0 \mod 4$
\item Solve $n = a^2 + b^2 + c^2 \textrm{ mod } p$ for all $p > 2$
\item Solve $n = a^2 + b^2 + c^2$ in $\mathbb{R}$ ( this just says $n > 0$ )
\end{itemize}
Hasse-Minkowski principle tells us this is sufficient, but why does Hasse-Minkowski principle work at all?

\newpage

\noindent\textbf{How do we know the Hasse-Minkowski principle?  }
\newline \newline
Case $n = a^2 + b^2 + c^2$ 
\newline \newline
Reading's Serre's \textit{Course on Arithmetic} we can solve in $\mathbb{Q}$:

$$ a^2 + b^2 + c^2 - n \, d^2 = 0$$

\noindent We are able to find an $x \neq 0$ in $\mathbb{Q}$ solving two quadratic eqs:

$$ a^2 + b^2 = x = c^2 - n \, d^2 $$

\noindent This works because $x \in \mathbb{Q}$ not just $ x \in \mathbb{Z}$, we'd better write

$$ a^2 + b^2 - x\, e^2=  c^2 - n \, d^2 - x \, f^2 = 0$$

\noindent Once we have solved for $x \in \mathbb{Q}$ we solve $(a,b,c), (d,e,f) \in \mathbb{Q}^3 $ \newline

\noindent $\circ$ $\circ$ $\circ$ This is reduction from 4 variables to 3 variables  

\newpage

\noindent\textbf{How do we know the Hasse-Minkowski principle?  }
\newline \newline
The only case $a^2 + b^2 + c^2 - n \, d^2 = 0$ 
\newline \newline
Reading's Serre's \textit{Course on Arithmetic} we can solve in $\mathbb{Q}$
\newline \newline
Reduce 4 variables to 3...
\newline \newline
Why is solving all congruences $n = a^2 + b^2 + c^2 \textrm{ mod }p$ enough?

\newpage

\noindent\textbf{How do we know the Hasse-Minkowski principle?  }
\newline \newline
The only case $a^2 + b^2 + c^2 - n \, d^2 = 0$ 
\newline \newline
Reading's Serre's \textit{Course on Arithmetic} we can solve in $\mathbb{Q}$
\newline \newline
Reduce 4 variables to 3...
\newline \newline
Why is solving all congruences $n = a^2 + b^2 + c^2 \textrm{ mod }p$ enough?


\newpage

\noindent\textbf{How do we know the Hasse-Minkowski principle?  }
\newline \newline
The only case $a^2 + b^2 + c^2 - n \, d^2 = 0$ 
\newline \newline
Serre's \textit{Course on Arithmetic} is written a little bit out of squence:
\newline \newline
Let $f = x^2 + y^2 + z^2 - n \, w^2 $ and we are solving\footnote{Sorry for the variable change, to make it look more like Serre.} $f = 0$ in $\mathbb{Q}$.
\newline \newline
For any prime\footnote{Think of $p = 12721$ or $ 32323$ or $70507$ or $94949$ etc.} $p \in P$ we can solve in $\mathbb{Q}_p$ two equations at once\footnote{Therefore $n = x^2 + y^2 + z^2$ solves both \textbf{sum of two squares} and \textbf{Pell's equation} simultaneously.}:
$$x_p = x^2 + y^2 \text{ and }x_p = z^2 - n\, w^2 $$
This can be writen in \textbf{Hilbert symbols} as a shorthand:
$$(x, -ab)_p = (a,b)_v \text{ and }(x,-cd)_v=(c,d)_v $$
Then, the field of fractions $\mathbb{Q}$ of the integers $\mathbb{Z}$ is exceedingly rich, and we can find a single $x \in \mathbb{Q}$ solving all these Hilbert equations at once, for all $p \in P$.

\newpage

\noindent\textbf{Merging solutions over $\mathbb{Q}_p$ to solutions over $\mathbb{Q}$  }
\newline \newline
\noindent There is a single $x \in \mathbb{Q}$ such that\footnote{Please excuse all the quantifiers here? You deserve an better explanation!  It's just that Serre is a genius.} for every prime $p \in P$
$$a x_1^2 + b x_2^2 - x z^2 = 0 $$ 

\noindent can be solved $(x_1, x_2, z) \in \mathbb{Q}_p^3$.  There is a single $(x_1, x_2, z) \in \mathbb{Q}$ 
\newline \newline
solving $(\ast)$ over the rational numbers.  Similarly for $x_3^2 - n x_4^2 - x w^2$
\newline \newline
Therefore we can solve $f = 0$ over $\mathbb{Q}$

\newpage

\noindent\textbf{Sweeping Information under the Rug  }
\newline \newline
In the process of doing our Hilbert symbol calculation we are likely to have used\footnote{After reading the much shorter 3-pages proof by Ankeny using \textbf{Geometry of Numbers} I decided to read this for context}:
\begin{itemize}
\item Dirichlet's prime number theorem in arithmetic sequences $|P \cap (a \, \mathbb{Z} + b)|=\infty$.
\item Quadratic Reciprocity \hspace{0.75in}$(\frac{p}{q})(\frac{q}{p}) = (-1)^{\frac{p-1}{2}\cdot\frac{q-1}{2}}$
\end{itemize} \vspace{12pt}
I don't understand \textbf{diophanine approximation} or \textbf{unique factorization} well enough to understand all the Hilbert symbols used. 
\newline \newline
\textbf{Bonus problem} show there exists $x \in \mathbb{Z}$ such that Pell equation and sum of two squares can both be solved\footnote{There could be enough ``room" since we have both $x \in \mathbb{Z}$ and also $(x_1, x_2, x_3, x_4)\in \mathbb{Z}^4$ but can we satisfy constraints on $n$ for \textbf{all} primes $p \in P$ ?} (over $\mathbb{Z}$)
$$ x_1^2 + x_2^2 = x \text{ and } -x_3^2 + n \, x_4^2 = x $$

\newpage

\noindent\textbf{Strong Approxmtion}\footnote{Possibly, weak approximation...}
\newline\newline
Let $p,q$ be two prime numbers then $\mathbb{Z}/pq\mathbb{Z}\simeq 
\mathbb{Z}/p\mathbb{Z}\oplus
\mathbb{Z}/q\mathbb{Z}$
\newline\newline
It is also very reasonable that $\mathbb{Z}/m\mathbb{Z}\simeq 
\mathbb{Z}/m_1\mathbb{Z}\oplus \dots \oplus
\mathbb{Z}/m_k\mathbb{Z}$ as long as $m_1, \dots, m_k$ are pairwise relatively prime.
\newline\newline
Let $S$ be a finite subset of primes.  Then $\displaystyle \mathbb{Q}\to \prod_{p \in S} \mathbb{Q}_p$ is dense.  Something like this is sure to work when $S$ is infinite (or $S=P$), and this object will be called $\mathbb{A}$.
\newline\newline
$\mathbf{\ast}$ Let $a_i$ with $i \in I$ be some numbers.  Then we can solve $(a_i,x)_p = \varepsilon_{i,p}$ for any sequence\footnote{this notation is truly dreadful and unreadable.  And this statement is \textbf{false} as written.  If you multiply over all primes $(a_i,x)_p=1$.  Almost all of the spins $\varepsilon_{i,p}\equiv 1$.  
These equations should be solvable for each prime $p$, $(a_i, x_p) = \varepsilon_{i,p}$.  Between these three we have used Dirichlet's theorem on primes in arithmetic progressions. } of spins $\varepsilon = \pm 1$.

\newpage
\fontfamily{qag}\selectfont \fontsize{12}{10}\selectfont

\begin{thebibliography}{}

\item JP Serre \textbf{Course on Arithmetic} Springer-Verlag



\end{thebibliography}


\end{document}
