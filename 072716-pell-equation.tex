\documentclass[12pt]{article}
%Gummi|065|=)
\usepackage{amsmath, amsfonts, amssymb}
\usepackage[landscape, margin=0.5in]{geometry}
\usepackage{xcolor}
\usepackage{graphicx}
\newcommand{\off}[1]{}
\DeclareMathSizes{20}{30}{21}{18}

\title{\textbf{ Cubic Pell Equation }}
\author{John D Mangual}
\date{}
\begin{document}

\fontfamily{qag}\selectfont \fontsize{25}{30}\selectfont

\maketitle

\noindent On Math.StackExchange I learned we can solve Pell's Equation purely from Pigeonhole Principle.  Let's try\footnote{Here is also }
$$ x^2 - 17 y^2 = 1 $$
\textbf{\#1} Using Pigeonhole, there are infinely many pairs $(x,y)$ with $$x^2 - 17y^2 \leq 2 \sqrt{17}$$
This stems from attempting to use the Euclidean algorithm to find the GCD of $\sqrt{17}$ and $1$:
$$  \sqrt{17} = 4 \times 1 + (\sqrt{17}-4) $$

\newpage

\noindent The number $\sqrt{17}-4$ is the remainder when we use the division algorithm.   But this is getting ahead of ourselves.  \newline

\noindent The 18 multiples of $\sqrt{17}$ starting from $0$ can be put into groups:
$$ 0, \sqrt{17}, 2\sqrt{17}, \dots , 10\sqrt{17} $$
How do I compute the best integer approximations, I might square these:
$$ 0, 17, 68, \dots, 4913 $$
In the second case $68 = 8\times 8 + 4$ so that $2 \sqrt{7} - 8 < 1$.  And one more:
$$ 9 \times 17 = 153 = 12 \times 12 + 9 \longrightarrow 3 \sqrt{17}-12 < 1 $$
I know I shouldn't generalize, but probably we can take any multiple of $\sqrt{17}$ and find a number very close to it.  One more:
$$ 10 \times 10 \times 17 = 
1700 = 41\times 41 + 19 \longrightarrow 10 \sqrt{17} - 19 < 1$$
I forgot that in this case we get very lucky $ 4^2 - 17 \times 1^2 = - 1$
\newpage

\noindent I wanted to solve $x^2 - 17 y^2 = 1$ (with $+1$ instead of $-1$) \newline

\noindent Can we find two numbers $(p,q)$ which give a very small remainder and $q < 10$?
$$ 0< q \sqrt{17}-p < \frac{1}{10} $$
I think our initial guess still works.  I hate when I get it on the first try.  Not dramatic enough.
$$ \sqrt{17}-4  = 
\frac{ (\sqrt{17}-4)(\sqrt{17}+4)}{\sqrt{17}+4} = \frac{1}{\sqrt{17}+4} < \frac{1}{8}$$
And worse of all it looks correct because like I said:
$$ 4^2 - 17 \times 1^2 = 1 \textbf{ not } -1$$
Technical point: in the words of Bill Clinton we have to meditate of the meaning of the word ``is" \dots

\newpage

\noindent I just realized our example is not good enough.  I asked for $\frac{1}{10}$ and not $\frac{1}{8}$.  I went and looked at a calculator:
$$  8 \sqrt{17} = 32.984845\dots \longrightarrow -8 \sqrt{17} + 32 < \frac{1}{10}$$
What's funny about that\dots even if we asked $0 \leq q \leq 10$ we found a $q = - 8$ which is out of our range\footnote{Acceptable just not what I originally had in mind}.  This is why we can use absolute value sign:
$$ \big| 8 \sqrt{17} - 32 \big| < \frac{1}{10} $$
If we were really good, we'd do all the steps without a calculator.  With a slide rule or something.  \newline

\noindent Lastly how bad is the error?
$$ 33\times 33 - 8\times 8 \times 17 =
1089 - 1088 = 1 $$
\newpage




\newpage

\noindent \textbf{Solving Pell Equation with Pigeonhole} \newline

\noindent Let's try to approximate $\sqrt{19}$ as a fraction.  We can list multiples of this number and see which one is nearest a whole number:
$$ 0, \sqrt{19}, 
 2\sqrt{19},
  3\sqrt{19},
  \dots
  10 \sqrt{19}
$$ 
This is kind of like interlacing the perfect squares $\square = \{ 1,4,9,16,\dots\}$ and $19 \times \square$:
$$   \textbf{0},    1,    4,    9,   16,   \textbf{19},   25,   36,   49,   64,
         \textbf{76},   81,  100,  121,  144,  169,  \textbf{171},  196,
           225,  256 $$
$$  289,  \textbf{304},  324,  361,  400,  441,  \textbf{ 475},  484,  529,  576,  625,  676,
        \textbf{684},  729,  784,  841, $$
$$900, \textbf{931},  961, 1024, 1089, 1156, \textbf{1216},
       1225, 1296, 1369, 1444, 1521, \textbf{1539} $$
Interleaving these two squences of numbers, we can see nont of them are next to a square:
$$ x^2 - 19y^2 \neq 1 $$
So far.  How are we going to generate an answer if none of these small numbers work?
\newline

\noindent $4^2 - 19 \times 1^2 = -3 $\newline

\noindent $31^2 - 19 \times 7^2 = 30 $ \newline

\noindent We have no guarantee these differences should be small, but pigeonhole-principle has found us infinitely many. \newline

\noindent Once we see that $ 0 < \sqrt{19} - 4 < 1 $, let's save ourselves some time by just multiplying by this number instead.
$$ \sqrt{19} -4 = \frac{(\sqrt{19} -4)(\sqrt{19} +4)}{\sqrt{19} +4} = \frac{3}{\sqrt{19} +4} > \frac{1}{3}$$
So if I multiply this number by $4$ I get a new solution slightly larger than $1$. 
$$ 3(\sqrt{19} - 4)-1 = 3 \sqrt{19} - 13$$
In fact $4^2 \times 19 - 17^2 = 15$ which is rather large but still less than $19$.\newline

\noindent Also, $3^2 \times 19 - 13^2 = 2$

\newpage

\noindent $ 1421^2 - 19 \times 326^2 = 3$ \newline

\noindent $ 4^2 - 19 \times 1^2 = 3$ \newline

\noindent Are these enough to solve Pell's equation?
$$ \frac{1421 - 326\sqrt{19}}{4 - \sqrt{19}}\times \frac{4 + \sqrt{19}}{4 + \sqrt{19}}
= \frac{\text{complicated number}}{-3} $$
For some reason $3$ seems to be a common remainder for $\sqrt{19}$
\begin{itemize}
\item (1,4)
\item (14,61)
\item (326, 1421)
\item (4759, 20744)
\end{itemize}
This algorithm is really slow since I am tediously\footnote{A more judicious use of Pigenhole uses \textbf{renormalization} but I think we have demonstrated that $p^2 - 19q^2 = 3 $ has infinitely many solutions.  Next we check for solutions which are give the same remainder upon division by 19.  These will divide into each other and return a solution to the Pell equation $p^2 - 19q^2 = 1$.} checking that
$$ 0< \big| q \sqrt{19} - p \big| < \frac{1}{N} $$

\newpage

\noindent Really slow algorithm for solving Pell equation in some cases.  Solves $p^2 - 19q^2 = 3$ a lot.
\vspace{6pt}
\hrule

\begin{verbatim}
sol = []

for N in 1 + np.arange(1000):

    q = 1
    while( q*np.sqrt(19) % 1 > 1.0/N):
        q += 1
    p = int(q*np.sqrt(19) // 1)
    
    if (q,p) not in sol:
        sol += [(q,p)]
        print q, p, q*q*19 - p*p
\end{verbatim}

\newpage

\noindent Really slow algorithm for solving Pell equation in some cases.  Solves $p^2 - 19q^2 = 3$ a lot.
\vspace{6pt}
\hrule

\begin{verbatim}
   1     4   3
   3    13   2
  14    61   3
  53   231  10
  92   401  15
 131   571  18
 170   741  19
 209   911  18
 248  1081  15
 287  1251  10
 326  1421   3
1017  4433   2
\end{verbatim}
\vspace{6pt}
\hrule
Algorithm gets really slow after that\dots Sorry!

\newpage

\noindent \textbf{Finding units of $\mathbb{Z}[\sqrt[3]{2}]$ vis Pigeonhole} \newline

\noindent There's no continued fraction algorithm for cube roots.  Some people think $\sqrt[3]{2}$ is a sequence that never repeats\footnote{Think about that... how we we know a sequence avoids not just any pattern but all patterns.  That seems like a rather odd thing.} -- but nobody knows for sure. \newline

\noindent Hale and Trotter print out the first 1000 digits (they fit neatly on a page), the paper is from the 1970's and the numbers are type-written.  I don't know how they got a computer to do all of that... I can do it on my laptop with some effort. \newline

\noindent There is a continued fraction algorithm due to Brun, which starts from the vector $(1, \sqrt[3]{2}, \sqrt[3]{4})$ and leads to approximate vectors in $\mathbb{Z}^3$ (up to a proportionate factor).  \newline 

\noindent In a comuter simulation, the algorithm began repeating after 19 steps \newpage

\noindent but how do we know our Euclidean algorithm finished? \newline

\noindent -- insert discussion here -- 

\newpage

\includegraphics[width=5in]{cf-01.png}

\newpage

\fontfamily{qag}\selectfont \fontsize{12}{10}\selectfont

\begin{thebibliography}{}

\item Math.StackExchange \textbf{What is your favorite application of the Pigeonhole Principle?} \texttt{http://math.stackexchange.com/q/62565/4997}

\item Serge Lang, Hale Trottr \textbf{Continued fractions for some algebraic numbers.} \texttt{https://eudml.org/doc/151239}



\end{thebibliography}


\end{document}

