\documentclass[12pt]{article}
%Gummi|065|=)
\usepackage{amsmath, amsfonts, amssymb}
\usepackage[margin=0.5in]{geometry}
\usepackage{xcolor}
\usepackage{graphicx}
\usepackage{slashed}
%\usepackage{graphicx}
\newcommand{\off}[1]{}
\DeclareMathSizes{20}{30}{20}{18}
\newcommand{\myhrule}{}

\newcommand{\two }{\sqrt[3]{2}}
\newcommand{\four}{\sqrt[3]{4}}

\newcommand{\dash}{
\begin{tikzpicture}[scale=1]
\draw (0,0)--(19,0);
\end{tikzpicture}
}

\newcommand{\sq}[3]{
\node at (#1+0.5,#2+0.5) {#3};
\draw (#1+0,#2+0)--(#1+1,#2+0)--(#1+1,#2+1)--(#1+0,#2+1)--cycle;
}

\usepackage{tikz}
\usetikzlibrary{decorations.markings}

\title{\textbf{Proposal: Wallis Product}}
\author{John D Mangual}
\date{}
\begin{document}

\fontfamily{qag}\selectfont \fontsize{15}{20}\selectfont

\maketitle

\noindent Two particle physicists came up with a new proof of the Wallis Product for $\pi$:
$$ \frac{\pi}{2} = \frac{2\cdot 2}{1 \cdot 3}\;\frac{4\cdot 4}{3 \cdot 5} \;\dots $$
The Wallis product proof is very old, but their argument at the end of 2015 was fairly new. \\ \\
There are many ways to argue this without Quantum Mechanics.  Reasoning about the foundations is always a bit circular.  
\begin{itemize}
\item Theoretical physicists, who generate product formulas like these on their way to more serious objectives.  The best I can do can do is take them out of context and remark on how wonderful they are.  
\item If we study the proofs carefully we can get \textit{many} product formulas for all kinds of numbers, but mostly they will always involve numbers like $\pi^n$.   Or they are elements of the ring $\mathbb{Q}[\pi]$.
\end{itemize}
Let's simplify the Friedman-Hagen proof, we are asking about the eigenvalues (or ``spectrum" or ``Energy levels") of the Hydrogen atom:
$$ \Big( - \frac{\hbar}{2m}\nabla  + \frac{e}{r} \Big) \psi = E \psi $$
The hydrogen atom is ``kind of like" the Free Particle (setting electric charge to $e=0$) and the eigenfunctions will be the spherical harmonics
$$ \hspace{0.35in} \Big( - \frac{\hbar}{2m}\nabla  \Big) \psi = E \psi $$
Empty space is rotationally symmetric.  A light bulb in the center of the room gets dimmer as you move further away. \\ \\
There are exact formulas for eigenfunctions (a mystery in itself why we get so lucky). No only is there $SO(3)$ symmetry, but $SO(4)$ and even $SO(4,2)$.

\newpage

\noindent Because of the symmetry we have very detailed knolwedge of the energy levels here.
$$ E_{0, \ell } = - \frac{m \hbar^4}{2 \hbar^2} \frac{1}{(\ell + 1)^2} $$
If you want to know a bit more, there is \textbf{quantum chemistry}; anything more than a single Hydrogen atom is unsolvable. \\ \\
If you look far enough away you should not notice the electric charge.
$$  - \frac{\hbar}{2m}\nabla  + \frac{e}{r}  \approx  - \frac{\hbar}{2m}\nabla  $$
How do eigenvalues of $E$ change as a function of the electric charge $e$? 
$$ E_\ell  \leq \langle \psi_\ell | H | \psi_\ell \rangle  $$
1st-order perturbation theory has a way to estimate the energy levels. \\ \\
Friedman-Hagen compute ``statistical uncertainty" to be $0$ (which is nice):
$$ \frac{ [\langle r^4 \rangle - \langle r^2 \rangle^2 ]^{\frac{1}{2}} }{\langle r^2 \rangle} = \Big(\ell + \frac{1}{2}\Big)^{-\frac{1}{2}}$$
that the radius of the electron doesn't deviate too much.  \\ \\
An electron with high energy is really far away from the center of the atom, so the charge shouldn't matter that much.  So the orbit is nearly a perfect sphere.  I think $\langle r \rangle = \ell$. \\\\
Before you know it we have growing list of inner product spaces
$$ \langle \,H(r)\, \rangle\,, \quad \langle H \rangle_{\alpha\ell}\,, \quad \langle H \rangle_{min}^\ell $$
Do we even know over which subspace the trace was taken? 
$$ \langle H \rangle = \langle \psi | H | \psi \rangle $$
If we assume $\Gamma(\frac{1}{2}) = \sqrt{\pi}$ then we get a relationship between the Hydrogen 
$$ \lim_{\ell \to \infty} \frac{\langle H \rangle_{min}^\ell }{ E_{0, \ell}} 
= \lim_{\ell \to \infty}  \frac{(\ell+1)^2}{\ell + \frac{3}{2} }\left[ \frac{ \Gamma (\ell + 1) }{   \Gamma (\ell + \frac{3}{2} ) }\right]^2 = 1$$
ground state energy and the free particle .

\newpage

\noindent Maybe what they have in mind is this version of the Wallis product formula:
$$ \log det \Delta = \zeta'(0) = - \frac{1}{2} \log (2\pi)  $$
and we're computing the ``Redeimister torsion" of the sphere $S^2$.

\vfill

\begin{thebibliography}{}

\item Tamar Friedmann, C. R. Hagen, \textbf{Quantum Mechanical Derivation of the Wallis Formula for $\pi$}, \texttt{arXiv:1510.07813}

\item Stephen W. Hawking \textbf{Zeta function regularization of path integrals in curved spacetime} Comm. Math. Phys.
Volume 55, Number 2 (1977), 133-148.


\end{thebibliography}

\end{document}