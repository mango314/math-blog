\documentclass[12pt]{article}
%Gummi|065|=)
\usepackage{amsmath, amsfonts, amssymb}
\usepackage[margin=0.5in]{geometry}
\usepackage{xcolor}
\usepackage{graphicx}
\usepackage{slashed}
%\usepackage{graphicx}
\newcommand{\off}[1]{}
\DeclareMathSizes{20}{30}{20}{18}
\newcommand{\myhrule}{}

\newcommand{\two }{\sqrt[3]{2}}
\newcommand{\four}{\sqrt[3]{4}}

\newcommand{\dash}{
\begin{tikzpicture}[scale=1]
\draw (0,0)--(19,0);
\end{tikzpicture}
}

\newcommand{\sq}[3]{
\node at (#1+0.5,#2+0.5) {#3};
\draw (#1+0,#2+0)--(#1+1,#2+0)--(#1+1,#2+1)--(#1+0,#2+1)--cycle;
}

\usepackage{tikz}
\usetikzlibrary{decorations.markings}

\title{\textbf{Proposal: Wallis Product}}
\author{John D Mangual}
\date{}
\begin{document}

\fontfamily{qag}\selectfont \fontsize{15}{20}\selectfont

\maketitle

\noindent Two particle physicists came up with a new proof of the Wallis Product for $\pi$:
$$ \frac{\pi}{2} = \frac{2\cdot 2}{1 \cdot 3}\;\frac{4\cdot 4}{3 \cdot 5} \;\dots $$
The Wallis product proof is very old, but their argument at the end of 2015 was fairly new. \\ \\
There are many ways to argue this without Quantum Mechanics.  Reasoning about the foundations is always a bit circular.  
\begin{itemize}
\item Theoretical physicists, who generate product formulas like these on their way to more serious objectives.  The best I can do can do is take them out of context and remark on how wonderful they are.  
\item If we study the proofs carefully we can get \textit{many} product formulas for all kinds of numbers, but mostly they will always involve numbers like $\pi^n$.   Or they are elements of the ring $\mathbb{Q}[\pi]$.
\end{itemize}
Let's simplify the Friedman-Hagen proof, we are asking about the eigenvalues (or ``spectrum" or ``Energy levels") of the Hydrogen atom:
$$ \Big( - \frac{\hbar}{2m}\nabla  + \frac{e}{r} \Big) \psi = E \psi $$
The hydrogen atom is ``kind of like" the Free Particle (setting electric charge to $e=0$) and the eigenfunctions will be the spherical harmonics
$$ \hspace{0.35in} \Big( - \frac{\hbar}{2m}\nabla  \Big) \psi = E \psi $$
Empty space is rotationally symmetric.  A light bulb in the center of the room gets dimmer as you move further away. \\ \\
There are exact formulas for eigenfunctions (a mystery in itself why we get so lucky). No only is there $SO(3)$ symmetry, but $SO(4)$ and even $SO(4,2)$.

\newpage

\noindent Because of the symmetry we have very detailed knolwedge of the energy levels here.
$$ E_{0, \ell } = - \frac{m \hbar^4}{2 \hbar^2} \frac{1}{(\ell + 1)^2} $$
If you want to know a bit more, there is \textbf{quantum chemistry}; anything more than a single Hydrogen atom is unsolvable. \\ \\
If you look far enough away you should not notice the electric charge.
$$  - \frac{\hbar}{2m}\nabla  + \frac{e}{r}  \approx  - \frac{\hbar}{2m}\nabla  $$
How do eigenvalues of $E$ change as a function of the electric charge $e$? 
$$ E_\ell  \leq \langle \psi_\ell | H | \psi_\ell \rangle  $$
1st-order perturbation theory has a way to estimate the energy levels. \\ \\
Friedman-Hagen compute ``statistical uncertainty" to be $0$ (which is nice):
$$ \frac{ [\langle r^4 \rangle - \langle r^2 \rangle^2 ]^{\frac{1}{2}} }{\langle r^2 \rangle} = \Big(\ell + \frac{1}{2}\Big)^{-\frac{1}{2}}$$
that the radius of the electron doesn't deviate too much.  \\ \\
An electron with high energy is really far away from the center of the atom, so the charge shouldn't matter that much.  So the orbit is nearly a perfect sphere.  I think $\langle r \rangle = \ell$. \\\\
Before you know it we have growing list of inner product spaces
$$ \langle \,H(r)\, \rangle\,, \quad \langle H \rangle_{\alpha\ell}\,, \quad \langle H \rangle_{min}^\ell $$
Do we even know over which subspace the trace was taken? 
$$ \langle H \rangle = \langle \psi | H | \psi \rangle $$
If we assume $\Gamma(\frac{1}{2}) = \sqrt{\pi}$ then we get a relationship between the Hydrogen 
$$ \lim_{\ell \to \infty} \frac{\langle H \rangle_{min}^\ell }{ E_{0, \ell}} 
= \lim_{\ell \to \infty}  \frac{(\ell+1)^2}{\ell + \frac{3}{2} }\left[ \frac{ \Gamma (\ell + 1) }{   \Gamma (\ell + \frac{3}{2} ) }\right]^2 = 1$$
ground state energy and the free particle .

\newpage

\noindent Maybe what they have in mind is this version of the Wallis product formula:
$$ \log det \Delta = \zeta'(0) = - \frac{1}{2} \log (2\pi)  $$
and we're computing the ``Redeimister torsion" of the sphere $S^2$. Mathworld has:
$$ \frac{\pi}{2} = \left[ 4^{\zeta(0)} e^{- \zeta'(0)} \right]^2 $$
which is looking pretty good.

\vfill

\begin{thebibliography}{}

\item Tamar Friedmann, C. R. Hagen, \textbf{Quantum Mechanical Derivation of the Wallis Formula for $\pi$}, \texttt{arXiv:1510.07813}

\item Stephen W. Hawking \textbf{Zeta function regularization of path integrals in curved spacetime} Comm. Math. Phys.
Volume 55, Number 2 (1977), 133-148.


\end{thebibliography}

\newpage

\noindent Reading the Hagen-Friedman proof got me to think about a lot of things.  There are a few formulas we learn in undergrad, where the miracle is that we can fine the exact number:
\begin{itemize}
\item $\zeta(2) = \sum \frac{1}{n^2} = \frac{\pi^2}{6}$
\item $\Gamma(\frac{1}{2}) = \sqrt{\pi}$
\item $n! \approx \sqrt{2\pi n } (n/e)^n$
\end{itemize}
The area of the unit circle \begin{tikz} \draw (0,0) circle (0.2cm); \end{tikz} -- we call it $\pi$ --  poses a constraint on all the other things we are able to figure out.  There are lots of other ``miracles" but basically there are mostly combinations of these few. 
\begin{itemize}
\item look for other ``miracles"
\item verify these miracles are what they claim to be
\end{itemize} 
The second one is about checking all the details we sweep under the rug, inspecting a very common calculation carefully.  \\ \\
In calculus, engineers can do mostly correct calculations, and still getting the wrong answer.  This lead to a branch of mathematics called Analysis.  In my opinion, it is quite tedious to construct the real numbers $\mathbb{R}$ as the limit of fractions $\mathbb{Q}$. \\ \\ Then, one day I got kicked out of graduate school.  I got to look at the street, the supermarket, the basketball court, whatever.  They are using some exotic number system they didn't teach us about yet. \\ \\
The goal of the rest of this note is mostly to construct more miracles (as already appear in the literature) and they will be dull ``miracles", but nonetheless extraordinary.  \\ \\
But first a bit of what I think might be the content of Hagen-Friedman.  Instead of computing the determinant of $\Delta = \frac{\partial^2}{\partial x^2}+\frac{\partial^2}{\partial y^2} $ they estimate the determinant of the Hamiltonian (``energy") of the hydrogen atom
$$ - \frac{\hbar}{2m}\nabla  + \frac{e}{r} $$
so\dots Could there be ``exact" information encoded in the matrix elements of the operators? Yes.

\newpage

\noindent What I keep seeing in the physics literature are operators - or path integrals - that we can never compute / have no hope of computing, and we immediately approximate them with something we're more familiar with. \\ \\


\vfill

\begin{thebibliography}{}

\item Kentaro Ihara, Masanobu Kaneko and Don Zagier \textbf{Derivation and double shuffle relations for multiple zeta values} Compositio Math. 142 (2006) 307-338

\item Anton Kapustin, Brian Willett, Itamar Yaakov \textbf{Exact Results for Wilson Loops in Superconformal Chern-Simons Theories with Matter} \texttt{arXiv:0909.4559}

\end{thebibliography}

\newpage

\fontfamily{qag}\selectfont \fontsize{15}{20}\selectfont

\noindent Let's review some of Zagier's article.  If I set this on a homework, it will get solved:
$$ \zeta(k) \zeta(l) = \zeta(k,l) + \zeta(l,k) + \zeta(k+l) $$
I would write out the definition of a zeta series:
$$ \left[ \sum_{m>0} \frac{1}{m^k} \right]\left[ \sum_{n>0} \frac{1}{n^l} \right]
= \sum_{m>n} \frac{1}{m^k n^l}
+ \sum_{m>n} \frac{1}{m^l n^k}
+ \sum_{m>0}      \frac{1}{m^{k+l}} $$
In the product on the left one of two things will happen.  Either $m=n$ and we get the last term or $m > n$ or $m < n$.  Zagier summarizes this short hand with a ``harmonic product"
$$ z_k \ast z_l = z_k z_l + z_l z_k + z_{k+l} $$
There is also a \textit{shuffle product}  and a double shuffle.  I never read past the intro\dots Zagier writes all sorts of relations in terms of polynomial algebra (even though he is talking about zeta).  And his paper is already a bit old, the researching dating to the 1980's  -- looking for relations between value of the zeta function at 2. \\ \\
And we never get past these basic issues.  Maybe the best way to approach this is to review the modern articles and observe that Zagier is always in the background (if only by construction).  There is always $\zeta(2)$, $\Gamma(\frac{1}{2})$, $n!$ and I'd like there to be $\zeta(-1)$. \\ \\ And we are fitting all of this into the computation of $\det \Delta$, there must be a new paper every week. How can there not be something to say??

\vfill

\begin{thebibliography}{}

\item Kentaro Ihara, Masanobu Kaneko and Don Zagier \textbf{Derivation and double shuffle relations for multiple zeta values} Compositio Math. 142 (2006) 307-338

\item Anton Kapustin, Brian Willett, Itamar Yaakov \textbf{Exact Results for Wilson Loops in Superconformal Chern-Simons Theories with Matter} \texttt{arXiv:0909.4559}

\end{thebibliography}

\end{document}