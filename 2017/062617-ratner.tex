\documentclass[12pt]{article}
%Gummi|065|=)
\usepackage{amsmath, amsfonts, amssymb}
\usepackage[margin=0.5in]{geometry}
\usepackage{xcolor}
\usepackage{graphicx}

\newcommand{\off}[1]{}
\DeclareMathSizes{20}{30}{20}{18}

\newcommand{\two }{\sqrt[3]{2}}
\newcommand{\four}{\sqrt[3]{4}}
\newcommand{\red}{\begin{tikz}[scale=0.25]
\draw[fill=red, color=red] (0,0)--(1,0)--(1,1)--(0,1)--cycle;\end{tikz}}
\newcommand{\blue}{\begin{tikz}[scale=0.25]
\draw[fill=blue, color=blue] (0,0)--(1,0)--(1,1)--(0,1)--cycle;\end{tikz}}
\newcommand{\green}{\begin{tikz}[scale=0.25]
\draw[fill=green, color=green] (0,0)--(1,0)--(1,1)--(0,1)--cycle;\end{tikz}}

\usepackage{tikz}

\newcommand{\susy}{{\bf Q}}
\newcommand{\RV}{{\text{R}_\text{V}}}

\title{Unipotent Flows}
\author{John D Mangual}
\date{}
\begin{document}

\fontfamily{qag}\selectfont \fontsize{12.5}{15}\selectfont

\maketitle

\noindent I audited a course on homogeneous systems 
\begin{itemize}
\item Roth's Theorem
\item Szemeredi Theorem
\item Ratner Theorem
\end{itemize}
It was much too fast to absorb.  And, year later, I still have questions.  The teacher was brilliant and that may have been $\frac{1}{2}$ a design feature and $\frac{1}{2}$ my own curiosity.  \\ \\
Now, when I read the actual papers I notice the material is much more complicated than he had shown us; that he had (rather calmly) manoeuvered around various technicalities in order to present the material to us.  Also, there are now textbooks covering many features. \\ \\
In particular, I tried to understand the Oppenheim conjecture, as usual thinking it would be very simple to resolve.
$$  \overline{ \Big\{  x^2 + y^2 - \sqrt{2} \, z^2 : (x,y,z) \in \mathbb{Z}^3 \Big\} } = \mathbb{R} $$
My strategoy would have been to say here is $a \in \mathbb{R}$ and $\epsilon \ll 1$ and we'd like to find $(x,y,z)$ such that $q(x,y,z) \approx a$.
$$ \Big| (x^2 + y^2 - \sqrt{2} \, z^2) - a \Big| < \epsilon  $$
This is $\epsilon$-$\delta$ definition the way we might see in calculus class.  I have drawn the picture a few times.  Instead, the main step is to deform the equation itself:
$$ $$
The saying goes: \textbf{Algebra is generous: she often gives more than is asked for}; it is also hiding things from us as well.  Things that are quite complicated.\footnote{This is due to Jean d'Alembert (1717-1783) and I thought it was David Hilbert (1862-1943).}   This is also part of a common strategy in mathematics, to make the problem more severe and more terrifying, but we also see the big picture. \\ \\
In this case, we know a few things that are going on:
\begin{itemize}
\item Lie groups over $\mathbb{Q}$ are more complicated than Lie groups over $\mathbb{R}$ (e.g. it's possible to take limits in $SO(3, \mathbb{Q})$ that go outside the group).
\item ?? ( I forgot ) 
\end{itemize}
Cognitively, I see a blind spot \textit{among professors} about Ratner's theorem.  We treat it as a black box.  \textbf{By Ratner's Orbit-Closure Theorem the values of $Q$ are dense.} We have no idea inside, I have learned nothing. It makes asking for help very difficult. \\ \\
When procedure comes along that covers \textit{all} unipotent flows.  First of all, we are going to Oppenheim conjucture into a single giant procedure that will absorb all equidistribution problems, including ones we haven't though of.  We need two pieces of information:
\begin{itemize}
\item $G = \mathrm{SL}(3,\mathbb{R})$
\item $H = \mathrm{SO}(Q) \simeq \mathrm{SO}(1,2) \simeq \mathrm{PSL}(2, \mathbb{R})$
\end{itemize}
and the algebra says the closure of a certain orbit is the entire group: $\overline{G_\mathbb{Z}H}= G$.  That's excellent.  What does that even mean? \\ \\
Another similar statement $Q(H \mathbb{Z}^3) = Q(\mathbb{R}^3) $. (Yes, that's a 4th root of $2$) :
$$ [Q] = \big[ x^2 + y^2 - \sqrt{2} \, z^2  \big] \leftrightarrow \left[\begin{array}{ccc}1 & & \\ & 1 & \\ & & -\sqrt[4]{2} \end{array} \right] $$
The $H$ could literally be thought of a group of substitutions.\footnote{There might even be a professional-soundeding name for it, like ``representation"} If $a_+^2 + b_+^2 = 1$ then:
$$ \Big[ (a_+\, x + b_+ \, y)^2 + (-b_+\, x + a_+ \, y)^2 - \sqrt{2} \, z^2  \Big] \leftrightarrow 
\left[\begin{array}{rcc}a_+ & b_+ & \\ -b_+ & a_+ & \\ & & 1 \end{array} \right]
\left[\begin{array}{ccc}1 & & \\ & 1 & \\ & & -\sqrt[4]{2} \end{array} \right] $$
and we need the other substitution:
If $a_-^2 -  \sqrt{2}\, b_-^2 = 1$ then:
$$ \Big[ x^2 + \big( a_- \, y + \sqrt{2} \, b_-  \, z \big)^2 
- \sqrt{2} \, \big( b_-  \, y+ a_- \ z \big)^2  \Big] \leftrightarrow 
\left[\begin{array}{ccr}1 &  & \\ & a_- & \sqrt{2} \, b_-  \\ & b_- & a_- \end{array} \right]
\left[\begin{array}{ccc}1 & & \\ & 1 & \\ & & -\sqrt[4]{2} \end{array} \right] $$
and hopefully the $\sqrt{2}$'s are correct. This is as far as I got on my previous attempts. \\ \\ I think I'll go try something else.
\\ \\ \\
Later\dots how do we solve Lagrange's 3 squares problem as a unipotent flow?
$$  x^2 + y^2 + z^2 = n $$
This equation is \textbf{positive-definite}.
\vfill

\begin{thebibliography}{}

\item Dave Witte Morris \textbf{Introduction to Arithmetic Groups} \texttt{arXiv:math/0106063}

\end{thebibliography}

\end{document}