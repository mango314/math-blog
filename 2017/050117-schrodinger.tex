\documentclass[12pt]{article}
%Gummi|065|=)
\usepackage{amsmath, amsfonts, amssymb}
\usepackage[margin=0.5in]{geometry}
\usepackage{xcolor}
\usepackage{graphicx}

\newcommand{\off}[1]{}
\DeclareMathSizes{20}{30}{20}{18}

\newcommand{\two }{\sqrt[3]{2}}
\newcommand{\four}{\sqrt[3]{4}}


\usepackage{tikz}

\title{the Schr\"{o}dinger Equation}
\author{John D Mangual}
\date{}
\begin{document}

\fontfamily{qag}\selectfont \fontsize{12.5}{15}\selectfont

\maketitle

\noindent When I opened up Quantum Mechanics by Enrico Fermi, I just expected just some kind of review.  These course notes are facisimile of a class at University of Chicago in 1954 written by one of the ``greats". \\ \\
There are some things that are unusual.  He opens with a \textbf{optics-mechanics} analogy:
\begin{center}
\begin{tabular}{l|l} 
mass point & wave packet \\ \hline  \\ 
trajectory & ray \\  \hline \\
velocity & group velocity \\ \hline  \\
??? & phase velocity \\  \hline \\
potential function & refractive index \\ \hline  \\
energy & frequency \\  \hline \\
\end{tabular}
\end{center}
Hopefully, if you've taken a quantum mechanics course, you may have heard of ``particle-wave duality" even if you can't qualify what it meant.  \\ \\
What is\dots \textbf{particle-wave duality}? 
\begin{center}
\begin{tabular}{lcl}
Trajectory &=& Ray \\ 
$\downarrow$ & & $\downarrow$ \\
from Maupertuis & & from Fermat \\ 
$\downarrow$ & & $\downarrow$ \\
$\int (W - U)\, ds = min $ & & $\int \frac{ds}{v} = min$
\end{tabular}
\end{center}
He then proceeds to prove the Maupertuis and Fermat principles (of optics). \\
A beam of light ``searches for" the optimal path in one of two different ways:
\begin{itemize}
\item principle of least action
\item principle of least time
\end{itemize}
\vfill

\begin{thebibliography}{}

\item Richard Feynman.  \textbf{Quantum Mechanics and Path Integrals} Dover, 2010.

\item Enrico Fermi.  \textbf{Quantum Mechanics} University of Chicago Press, 1961.

\end{thebibliography}

\end{document}