\documentclass[12pt]{article}
%Gummi|065|=)
\usepackage{amsmath, amsfonts, amssymb}
\usepackage[margin=0.5in]{geometry}
\usepackage{xcolor}
\usepackage{graphicx}
%\usepackage{graphicx}
\newcommand{\off}[1]{}
\DeclareMathSizes{20}{30}{20}{18}
\newcommand{\myhrule}{}

\newcommand{\two }{\sqrt[3]{2}}
\newcommand{\four}{\sqrt[3]{4}}

\newcommand{\dash}{
\begin{tikzpicture}[scale=1]
\draw (0,0)--(19,0);
\end{tikzpicture}
}

\newcommand{\sq}[3]{
\node at (#1+0.5,#2+0.5) {#3};
\draw (#1+0,#2+0)--(#1+1,#2+0)--(#1+1,#2+1)--(#1+0,#2+1)--cycle;
}

\usepackage{tikz}

\title{\textbf{Proposal: The Factorial Function}}
\author{John D Mangual}
\date{}
\begin{document}

\fontfamily{qag}\selectfont \fontsize{20}{25}\selectfont

\maketitle

\noindent It would not be a stretch to categorize my work under the topics in my 3 previous blog posts and this one.  I talk about the Stirling formula:
$$ n! \approx \sqrt{2 \pi e } \left( \frac{n}{e} \right)^n $$
Think about it.  If you don't know much math, there are only so many starting points.  So no matter how complicated our analysis gets, there shold only be a few principles at play. \\ \\
Here we can get away with just the trapezoid rule:
\begin{eqnarray*} \log n! &=& \log 1 + \log 2 + \dots + \log n \\ \\
& \approx &
\int_1^n \log x \, dx = n \log n - n \end{eqnarray*}
The goal of this project is to state and make towards specific conjectures, built from these. \\ \\
This concludes my proposals for the Spring.

\newpage

\noindent \textbf{A} My frustration with physics literature or math literature, I am unlikely to come up with observation of interest to others. \\ \\
For one things, since my interest is about $n!$ I am maybe interested in counting things.  Within combinatorics there is:
 \begin{itemize}
 \item counting things
 \item finding rare and exotic objects
 \item studying how things are connected
 \item words
 \end{itemize}
 and many other things.  I am mainly focusing on the first one here.  It is called \textbf{enumrative combinatorics}.  There are two kinds of applications:
 \begin{itemize}
 \item counting things predicted by mathematics
 \item counting things related to the outside world
 \end{itemize}

\newpage


The permutation group can be used to prove quadratic reciprocity. \\ \\
There's a nice example of card shuffling, where measures on the permutation group converges to the uniform distribution. \\ \\
The domino shuffling of the Aztec diamond can be expressed in terms of measures on the permutation group. \\ \\
Binomial coefficients can be completed to geometric objects in the field of math called ``integral geometry" \\ \\
Galois theory, literally permutes the zeros of a polynomial and I always liked those inequalities from contests (which I could never solve)\\ \\
The Taylor series uses factorials.  That's by far the strangest one.  What is being permuted there? \\ \\
I would say combinatorics is a very healthy field and not much I could contribute there.  Nothing profound or exciting.

\newpage

\noindent \textbf{B}  \\
The modern theory of factorials I can't even read the formulas. \\
\includegraphics[width=7in]{factorial-01.png} \\ \\
Here is a very important random process \\
\includegraphics[width=7in]{factorial-02.png} \\ \\ 
Here is another discussion of that random process \\ 
\includegraphics[width=7in]{factorial-03.png} \\ \\
here is a physics partition function \\ 
\includegraphics[width=7in]{factorial-04.png} 

\newpage

\noindent Reality is complex, physical systems ultimately describe a very complicated reality.\footnote{  An iPhone is a very complicated thing and we don't know how to use it, except for basic features.} I am betting these formulas are very important and useful.  If I could read them! \\ \\
Here is one more just in as of yesterday: \\
\includegraphics[width=7in]{factorial-05.png} \\ 
It's really great there is all these wonderful formulas.  arXiv really is a LIFO queue.  Every day there is something new and you forget about what happened just before. \\ \\
Now to find some common ground:
\begin{itemize}
\item why are they doing all of this?  
\item there are lots of generalized factorials (and we have found some) these formulas are very specific and what had motivated the authors to build them
\item what's in it for the rest of us?  what are they counting?
\end{itemize}
Just like an elementary school reading comprehension exercise.   What is the main idea?
\newpage

\noindent \textbf{C}  These people are intrested in the \textbf{superconformal index} or \textbf{partition function} of certain supersymmetric gauge theories.   \\ \\
Here's a claim:
\begin{itemize}
\item superconformal indices generalize the factorial function
\end{itemize}
There's immediate defects about this idea.  Uh...
\begin{itemize}
\item (supersymmetric) gauge theories are geometric objects
\item and the answer we are getting is a ``cardinality" of [unknown object]. 
\item So {\color{green} we have discarded most of the information available to us} I believe we are counting fixed points of torus orbits, or various combinations of $SU(2)$.
\end{itemize}    
We have no idea what they are talking about anyway... so why not start with that? \\ \\
One mantra floating around is that \textbf{combinatorics objects are shadows of geometric objects} and it's corollary \textbf{geometric objects are shadows of integrable systems, K-theory, etc}  \\ \\
And there are new-age one's like ``{\color{blue} entropy is cardinality}" or even ``{\color{blue}entropy is area}".  And we can cross our legs and meditate on that.

\newpage

\noindent \textbf{C-1} To summarize part C, some physicists are generating sequences of numbers, which behave like the factorial, to arbitrary accuracy but with unclear meaning or patterns.  \\ \\
One thing I notice is the factorial and the Vandermonde matrix can go under the same style
$$
\left| \begin{array}{ccc} 1 & a & a^2 \\ 
1 & b & b^2 \\ 
1 & c & c^2 \end{array} \right|
 = (a-b)(b-c)(c-a)
 $$
and the Wikipedia page on Schur polynomials has other formulas of this type:
$$
\frac{1}{(a-b)(b-c)(c-a)} \left| \begin{array}{ccc} a & a^2 & a^4 \\ 
b & b^2 & b^4 \\ 
c & c^2 & c^4 \end{array} \right|
 = abc (a+b+c)
 $$
To me this is the oldest style of problem.  Take a sequence of numbers and say something about them.
 
\newpage

\noindent \textbf{Existential Crisis}  Research level mathematics is written in a very modern language\footnote{because math was not abstract enough already!} where it's hard for the non-specialist (me) to discern what the original problem was.  \\ \\
And having spent (on the order of years) asking for a direct statement.  Take alegebraic geometry for example.  They got rid of all the equations.  If need a \textbf{specific} modular form, or variety, I want the sequence of coefficients,  \\ \\
Physics and engineering are written in the old-fashioned language.  If you look past the first pages of the math papers -- they are also still mostly written with classical invariant theory.  In business / Econ / CS we are lucky if we are using the old-fashioned language.  \\ \\ Many peoeple even feel that numbers and other mathematical reasoning are too brittle for use.  Therefore if we look into ``the outside world" there will be objects that will defy current mathematical classification, or theories that the ouside world hasn't seen. \\ \\
And hopefully these are helpful exchanges.  \\ \\ I will argue (maybe not here) is that Cohomology organizes computations into ``trivial" and ``non-trivial" parts.  Yet, everyone finds \textit{ something }  trivial.  And trivial, basically just means \textbf{zero}, so really we're just explorning the meaning of \textbf{0}.


\newpage

\noindent \textbf{D} On to business.  Let's name a few things to work on.
\begin{itemize}
\item shuffling cards (and carries in arithmetic) define measures on $S_n$
\item Domino tiling of special shapes (e.g. Aztec Diamond) 
\item The greatest common divisor of a polynomial $f(a)$ always divides $k!$ as $a \in \mathbb{Z}$.  And there's a theory of factorials due to Bhargava that solves this nicely.
\item Taylor series involves a factorial: $f(x) = \sum \frac{f^{n}(x_0)}{k!} (x- x_0)^k$ and Taylor's theorem fails in various circumstances
\item The volumes of spheres always involve factorial $$ \mathrm{Vol}(S^N \times S^N) = \frac{4\pi^n}{ \frac{n}{2}! \times \frac{n}{2}! }$$
\item By fundamental therem of algebra $$p(x) = (x-a_1)\dots (x - a_n)$$ then I can permute the $(a_1, \dots, a_n) \in \mathbb{C}^n$.  \\
\end{itemize}
Cayley's Representation Theorem is the observation that all finite groups embed in a permutation group $G \subseteq S_n$. These are like coordinates.  And Schur-Weyl duality says $S_N \leftrightarrow U(N)$. \\ \\
If I look at empirical data, ``real-world" we will never have a group structure.  Things will always be off.  So there's something called an \textbf{almost-group} or we'll find some other way to reason about life.

\newpage

\fontfamily{qag}\selectfont \fontsize{12}{10}\selectfont

\begin{thebibliography}{}

\item \textbf{Wikipedia} ``Factorial Function" \texttt{https://en.wikipedia.org/wiki/Factorial}

\item Manjul Bhargava \textbf{The Factorial Function and Generalizations}  The American Mathematical Monthly, Vol. 107, No. 9 (Nov., 2000), pp. 783-799

\item Kurt Johansson \textbf{Discrete orthogonal polynomial ensembles and the Plancherel measure} \texttt{arXiv:math/9906120}

\end{thebibliography}

\fontfamily{qag}\selectfont \fontsize{20}{25}\selectfont


\newpage

\noindent \textbf{E} One of the earliest examples of mathematics I was exposed to was the Arctic Circle phenomenon\footnote{At the PROMYS summer program in Boston, where we learned Continued Fractions, Quadratic Reciprocity and a few other results in Number Theory.} showing that random tilings have a limit shape (a circle).  \\ \\
\includegraphics[width=7.5in]{AD-01.png} \\
That was in 1998.  By 2017 there are several complete proofs.  

\newpage

\noindent In order to argue mathematically:
\begin{itemize}
\item how do we know the number of tilings is $2^{\binom{n}{2}}$ ? This particular shape has an ingenious ``shuffling" algorithm, which comes out of a hat
\item how do we pick one at random?   (this is Markov Chain Monte Carlo) and write a computer simulation
\item How do we express the notion of ``convergence" to a ``limit shape" ? this is the theory of large deviations
\end{itemize} 

\noindent In order to argue like a theoretical physicist
\begin{itemize}
\item these numbers are related to the XXZ spin chain
\item Domino tilings fall under the umbrella of integrable systems and spectral curves
\end{itemize}
and so the discussion is done, right?  At least, that group of people consider it done.\footnote{There's a duality between exposition and research.  I would like to hope that you can tell the story better, that can be channeled into a better technical result.}  \\ \\
There's an even dumber problem, which is to count the number of domino tilings of a rectangle.  
$$ \prod_{j=1}^{m}\prod_{j=1}^{n} \left( 4 \cos^2 \frac{\pi j}{2n+1} 
+ 4 \cos^2 \frac{\pi k}{2n+1} \right) \in \mathbb{Z}$$
This doesn't look like a whole number. And I know that patterns in tilings occur with special frequencies, also related to $\pi$.


\newpage

\noindent One good reason to review, is that I would speak to an expert and he would not know this or know that.  Here are some excerpts:

\includegraphics[width=7.5in]{AD-02.png} \\
(\textbf{1998} $\uparrow$) Two year later, the result was qualified a bit more.\\
\includegraphics[width=7.5in]{AD-03.png} \\
The integrable case was considered ``resolved".  (\textbf{2000} $\uparrow$)  A decade later a few more details emerge \\
\includegraphics[width=7.5in]{AD-04.png} \\
(\textbf{2010} $\uparrow$) by 2017 we know Aztec Diamond is part of a larger group of problems.

\newpage

\noindent Different experts know different things, if you are an expert probabilist, the larger group is called \textbf{KPZ}.  If you are a physicist\footnote{I should specify which part of mathematical physics.} you might consider \textbf{brane tilings}, neglecting all the calculus of variations and large deviations (since they should all work out perfectly). Or if you only focus on enumeration and duality the theory of \textbf{cluster algebras}. \\ \\
Will you make my word for it? The rest of us have to \textbf{prove} it works out perfectly. \\ \\ 
What if you are in statistical mechanics (as I originally studied - and begrudgingly wrote my thesis)?  And what if you interests shifted to
\begin{itemize}
\item convex geometry 
\item information theory
\item low-dimensional topology
\item high school algebra 
\item everything else?
\end{itemize}
What will we do about all of that?  I found the probability papers difficult to read.  In other words:
\begin{itemize}
\item how things were rearranged and counted
\item the equations because too many symbols
\end{itemize}
\textbf{Combinatorics} and \textbf{Algebra} have names for these things. \\ \\ They're called ``sections" and ``bijections" and ``partitions".

\newpage

\noindent My mentor David Speyer was surprised when I told him that domino tilings converge to the Gaussian Unitary Ensemble.  It started to give me a complex.  It's just Stirling formula.
$$ n! \approx \sqrt{2\pi n} \;(n/e)^n \; \big(1 + o(1) \big)$$
Does the large deviations part of the proof have to be so unwieldy?  
\begin{itemize}
\item The paper from 2010 [Forester-Fleming] says 
\begin{itemize} 
\item the Arctic Circle Theorem was proven in 1998 [JSP] 
\item the Krawtchouk Ensemble, as outlined by [J] in 2000.
\item domino tilings are an example of a Gelfand-Tsetlin process, which converges to the GUE ``watermelon" 
\end{itemize}  
\item  There is lots of literature on determinantal processes, 
\begin{itemize}
\item except [JSP] never mentions the Krawtchouk ensemble or any determinantal process, at all.
\item Kurt Johansson deduces the Aztec diamond theorem as part of other Large deviations results
\item No discussion of \textbf{sub-additive ergodic theory}!
\end{itemize}
\end{itemize}
No paper has explicitly shown the math-induction proof in Jockusch-Shor-Propp leads to a Gelfand-Tsetlin process.  \\ \\ There are steps in [JSP] where the authors admit, could not be simplified, due to lack of knowledge.  Perhaps I can re-write [JSP] using the determinants?
\\ \\
Also, the XXZ spin chain relation to dominoes.

\newpage

\noindent Emphasis:  \textbf{integrable systems}


\fontfamily{qag}\selectfont \fontsize{12}{10}\selectfont

\begin{thebibliography}{}

\item \textbf{Wikipedia} ``Factorial Function" \texttt{https://en.wikipedia.org/wiki/Factorial}

\item William Jockusch, James Propp , Peter Shor\textbf{Random Domino Tilings and the Arctic Circle Theorem} \texttt{arXiv:math/980168}

\item Kurt Johansson \textbf{Discrete orthogonal polynomial ensembles and the Plancherel measure} \texttt{arXiv:math/9906120}

\item Eric Nordenstam \textbf{On the Shuffling Algorithm for Domino Tilings} \texttt{arXiv:0802.2592}

\item Benjamin J. Fleming, Peter J. Forrester \\ \textbf{Interlaced particle systems and tilings of the Aztec diamond} \texttt{1004.0474}

\item Alexei Borodin, Vadim Gorin \textbf{Lectures on Integrable Probability} \texttt{arXiv:1212.3351}

\item Sunil Chhita, Kurt Johansson \\ \textbf{Domino statistics of the two-periodic Aztec diamond} \texttt{arXiv:1410.2385}

\item Alexey Bufetov, Alisa Knizel \textbf{Asymptotics of random domino tilings of rectangular Aztec diamonds} \texttt{arXiv:1604.01491}

\end{thebibliography}

\fontfamily{qag}\selectfont \fontsize{20}{25}\selectfont

\noindent Emphasis:  \textbf{calculus of variations}


\fontfamily{qag}\selectfont \fontsize{12}{10}\selectfont

\begin{thebibliography}{}

\item 

\end{thebibliography}

\fontfamily{qag}\selectfont \fontsize{20}{25}\selectfont

\noindent Emphasis:  \textbf{Counting}


\fontfamily{qag}\selectfont \fontsize{12}{10}\selectfont

\begin{thebibliography}{}

\item Noam Elkies, Greg Kuperberg, Michael Larsen, James Propp  \\ \textbf{Alternating sign matrices and domino tilings} \texttt{arXiv:math/9201305}

\item Greg Kuperberg \textbf{Kasteleyn Cokernels} \texttt{arXiv:math/0108150}

\item Dylan Thurston \textbf{From Dominoes to Hexagons} \texttt{arXiv:math/0405482}

\item Fred\'{e}deric Bosio, Marc Van Leeuwen. \textbf{A bijection proving the Aztec diamond theorem by combing lattice paths} \texttt{arXiv:1209.5373}

\item Tri Lai. \textbf{A generalization of Aztec diamond theorem, part I} \texttt{arXiv:1310.0851}

\item Manuel Fendler, Daniel Grieser. \textbf{A new simple proof of the Aztec diamond theorem} \texttt{arXiv:1410.5590}

\item David E. Speyer \textbf{Variations on a theme of Kasteleyn, with application to the totally nonnegative Grassmannian} \texttt{arXiv:1510.03501}

\end{thebibliography}

\fontfamily{qag}\selectfont \fontsize{20}{25}\selectfont

\end{document}