\documentclass[12pt]{article}
%Gummi|065|=)
\usepackage{amsmath, amsfonts, amssymb}
\usepackage[margin=0.5in]{geometry}
\usepackage{xcolor}
\usepackage{graphicx}
%\usepackage{graphicx}
\newcommand{\off}[1]{}
\DeclareMathSizes{20}{30}{20}{18}
\newcommand{\myhrule}{}

\newcommand{\two }{\sqrt[3]{2}}
\newcommand{\four}{\sqrt[3]{4}}

\newcommand{\dash}{
\begin{tikzpicture}[scale=1]
\draw (0,0)--(19,0);
\end{tikzpicture}
}

\newcommand{\sq}[3]{
\node at (#1+0.5,#2+0.5) {#3};
\draw (#1+0,#2+0)--(#1+1,#2+0)--(#1+1,#2+1)--(#1+0,#2+1)--cycle;
}

\usepackage{tikz}

\title{\textbf{Proposal: The Factorial Function}}
\author{John D Mangual}
\date{}
\begin{document}

\fontfamily{qag}\selectfont \fontsize{20}{25}\selectfont

\maketitle

\noindent It would not be a stretch to categorize my work under the topics in my 3 previous blog posts and this one.  I talk about the Stirling formula:
$$ n! \approx \sqrt{2 \pi e } \left( \frac{n}{e} \right)^n $$
Think about it.  If you don't know much math, there are only so many starting points.  So no matter how complicated our analysis gets, there shold only be a few principles at play. \\ \\
Here we can get away with just the trapezoid rule:
\begin{eqnarray*} \log n! &=& \log 1 + \log 2 + \dots + \log n \\ \\
& \approx &
\int_1^n \log x \, dx = n \log n - n \end{eqnarray*}
The goal of this project is to state and make towards specific conjectures, built from these. \\ \\
This concludes my proposals for the Spring.

\newpage


\newpage



\fontfamily{qag}\selectfont \fontsize{12}{10}\selectfont

\begin{thebibliography}{}

\item \textbf{Wikipedia} ``Factorial Function" \texttt{https://en.wikipedia.org/wiki/Factorial}

\end{thebibliography}

\end{document}