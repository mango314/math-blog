\documentclass[12pt]{article}
%Gummi|065|=)
\usepackage{amsmath, amsfonts, amssymb}
\usepackage[margin=0.5in]{geometry}
\usepackage{xcolor}
\usepackage{graphicx}
%\usepackage{graphicx}
\newcommand{\off}[1]{}
\DeclareMathSizes{20}{30}{20}{18}
\newcommand{\myhrule}{}

\newcommand{\two }{\sqrt[3]{2}}
\newcommand{\four}{\sqrt[3]{4}}

\newcommand{\dash}{
\begin{tikzpicture}[scale=1]
\draw (0,0)--(19,0);
\end{tikzpicture}
}

\newcommand{\sq}[3]{
\node at (#1+0.5,#2+0.5) {#3};
\draw (#1+0,#2+0)--(#1+1,#2+0)--(#1+1,#2+1)--(#1+0,#2+1)--cycle;
}

\usepackage{tikz}

\title{\textbf{Proposal: The Factorial Function}}
\author{John D Mangual}
\date{}
\begin{document}

\fontfamily{qag}\selectfont \fontsize{20}{25}\selectfont

\maketitle

\noindent It would not be a stretch to categorize my work under the topics in my 3 previous blog posts and this one.  I talk about the Stirling formula:
$$ n! \approx \sqrt{2 \pi e } \left( \frac{n}{e} \right)^n $$
Think about it.  If you don't know much math, there are only so many starting points.  So no matter how complicated our analysis gets, there shold only be a few principles at play. \\ \\
Here we can get away with just the trapezoid rule:
\begin{eqnarray*} \log n! &=& \log 1 + \log 2 + \dots + \log n \\ \\
& \approx &
\int_1^n \log x \, dx = n \log n - n \end{eqnarray*}
The goal of this project is to state and make towards specific conjectures, built from these. \\ \\
This concludes my proposals for the Spring.

\newpage

\noindent \textbf{A} My frustration with physics literature or math literature, I am unlikely to come up with observation of interest to others. \\ \\
For one things, since my interest is about $n!$ I am maybe interested in counting things.  Within combinatorics there is:
 \begin{itemize}
 \item counting things
 \item finding rare and exotic objects
 \item studying how things are connected
 \item words
 \end{itemize}
 and many other things.  I am mainly focusing on the first one here.  It is called \textbf{enumrative combinatorics}.  There are two kinds of applications:
 \begin{itemize}
 \item counting things predicted by mathematics
 \item counting things related to the outside world
 \end{itemize}

\newpage


The permutation group can be used to prove quadratic reciprocity. \\ \\
There's a nice example of card shuffling, where measures on the permutation group converges to the uniform distribution. \\ \\
The domino shuffling of the Aztec diamond can be expressed in terms of measures on the permutation group. \\ \\
Binomial coefficients can be completed to geometric objects in the field of math called ``integral geometry" \\ \\
Galois theory, literally permutes the zeros of a polynomial and I always liked those inequalities from contests (which I could never solve)\\ \\
The Taylor series uses factorials.  That's by far the strangest one.  What is being permuted there? \\ \\
I would say combinatorics is a very healthy field and not much I could contribute there.  Nothing profound or exciting.

\newpage

\noindent \textbf{B}  \\
The modern theory of factorials I can't even read the formulas. \\
\includegraphics[width=7in]{factorial-01.png} \\ \\
Here is a very important random process \\
\includegraphics[width=7in]{factorial-02.png} \\ \\ 
Here is another discussion of that random process \\ 
\includegraphics[width=7in]{factorial-03.png} \\ \\
here is a physics partition function \\ 
\includegraphics[width=7in]{factorial-04.png} 

\newpage

\noindent Reality is complex, physical systems ultimately describe a very complicated reality.\footnote{  An iPhone is a very complicated thing and we don't know how to use it, except for basic features.} I am betting these formulas are very important and useful.  If I could read them!

\newpage

\fontfamily{qag}\selectfont \fontsize{12}{10}\selectfont

\begin{thebibliography}{}

\item \textbf{Wikipedia} ``Factorial Function" \texttt{https://en.wikipedia.org/wiki/Factorial}

\end{thebibliography}

\end{document}