\documentclass[12pt]{article}
%Gummi|065|=)
\usepackage{amsmath, amsfonts, amssymb}
\usepackage[margin=0.5in]{geometry}
\usepackage{xcolor}
\usepackage{graphicx}
%\usepackage{graphicx}
\newcommand{\off}[1]{}
\DeclareMathSizes{20}{30}{20}{18}
\newcommand{\myhrule}{}

\newcommand{\two }{\sqrt[3]{2}}
\newcommand{\four}{\sqrt[3]{4}}

\newcommand{\dash}{
\begin{tikzpicture}[scale=1]
\draw (0,0)--(19,0);
\end{tikzpicture}
}

\newcommand{\sq}[3]{
\node at (#1+0.5,#2+0.5) {#3};
\draw (#1+0,#2+0)--(#1+1,#2+0)--(#1+1,#2+1)--(#1+0,#2+1)--cycle;
}

\usepackage{tikz}

\title{\textbf{Proposal: The Factorial Function}}
\author{John D Mangual}
\date{}
\begin{document}

\fontfamily{qag}\selectfont \fontsize{20}{25}\selectfont

\maketitle

\noindent It would not be a stretch to categorize my work under the topics in my 3 previous blog posts and this one.  I talk about the Stirling formula:
$$ n! \approx \sqrt{2 \pi e } \left( \frac{n}{e} \right)^n $$
Think about it.  If you don't know much math, there are only so many starting points.  So no matter how complicated our analysis gets, there shold only be a few principles at play. \\ \\
Here we can get away with just the trapezoid rule:
\begin{eqnarray*} \log n! &=& \log 1 + \log 2 + \dots + \log n \\ \\
& \approx &
\int_1^n \log x \, dx = n \log n - n \end{eqnarray*}
The goal of this project is to state and make towards specific conjectures, built from these. \\ \\
This concludes my proposals for the Spring.

\newpage

\noindent \textbf{A} My frustration with physics literature or math literature, I am unlikely to come up with observation of interest to others.\footnote{
This blog is not very goal-oriented and does not accomplish anything in particular.  It is a waste of time.  \\ \\   I have been treating arXiv like ``monkeys on a typewriter".  Every day people write about different things, and I find some of them relevant.  \\ \\
Here is a crude explanation from social network analysis of why nothing important can happen here:
\begin{itemize}
\item most people have no idea what they want/need
\item a few people can navigate a constantly changing, and poorly understood terrain and identify valuable opportunities resources etc
\item even fewer people have the stature / seniority to affect any meaningful change in that space
\end{itemize}
Therefore, without any idea who or what or where, it starts to look like an uphill battle.  \textbf{Change is slow.}  More importantly, they way we decide who is good or what is important is a collective phenomenon with a few ``key players".  \\ \\
I can even prove there aren't too many key players, because we instinctively want the best that we can find.  That process happens essentially by chance. \\ \\ In fact, it is a very deliberate process based on hard work!!} 
 It's just not happening. \\ \\
For one things, since my interest is about $n!$ I am maybe interested in counting things.  Within combinatorics there is:
\begin{itemize}
\item counting things
\item finding rare and exotic objects
\item studying how things are connected
\item words
\end{itemize}
and many other things.  I am mainly focusing on the first one here.  It is called \textbf{enumrative combinatorics}.  There are two kinds of applications:
\begin{itemize}
\item counting things predicted by mathematics
\item counting things related to the outside world
\end{itemize}
\newpage \noindent The factorial function appears in the Laplace transform and it's unclear what is being counted there:
$$ \frac{n!}{t^{n+1}} = \int_0^\infty x^{n+1} \, e^{-tx} \, dx $$ 
who cares about the exponential function anyway?  or the derivative?  it's all just made up anyway.


\newpage

\fontfamily{qag}\selectfont \fontsize{12}{10}\selectfont

\begin{thebibliography}{}

\item \textbf{Wikipedia} ``Factorial Function" \texttt{https://en.wikipedia.org/wiki/Factorial}

\end{thebibliography}

\end{document}