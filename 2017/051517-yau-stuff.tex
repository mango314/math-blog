\documentclass[12pt]{article}
%Gummi|065|=)
\usepackage{amsmath, amsfonts, amssymb}
\usepackage[margin=0.5in]{geometry}
\usepackage{xcolor}
\usepackage{graphicx}

\newcommand{\off}[1]{}
\DeclareMathSizes{20}{30}{20}{18}

\newcommand{\two }{\sqrt[3]{2}}
\newcommand{\four}{\sqrt[3]{4}}


\usepackage{tikz}

\title{Miscellaneous Shing-Tung Yau Stuff}
\author{John D Mangual}
\date{}
\begin{document}

\fontfamily{qag}\selectfont \fontsize{12.5}{15}\selectfont

\maketitle

\noindent Here is just a short review of Yau stuff\dots \\ \\
ST Yau is prolific, teaches math at Harvard, for decades.  Obviously knows his stuff. So all of his work, comes with a guarantee that it's going to work.  There's not just one reading. \\ \\ 
A good, well-crafted paper is like a prism, you put stuff in and you get stuff out. By necessity these reviews have to be superficial and I picked 4 of them, all of which tie to stuff I care about. \\ \\
\textbf{\#1} Dynamics of D-Branes \\ \\
I have no idea what these things really are.  Yau offers a formalism and more importantly, enticing images.  The lack of compelling images in the math literature.  

\includegraphics[width=6in]{Yau-01.png}\\
\includegraphics[width=6in]{Yau-02.png}\\
\includegraphics[width=6in]{Yau-03.png}\\
Yau must have better things to do than draw pictures, but I don't!
\newpage

\noindent \textbf{\#2} The images are seductive.  Hopefully not misleading! \\
\includegraphics[width=7in]{Yau-04.png}\\
\includegraphics[width=7in]{Yau-05.png}\\
\includegraphics[width=7in]{Yau-06.png}\\

\newpage

\noindent \textbf{\#3} I can't do much more than admire. The formulas conjure powerful memories and response and yet I can't take a more tangible action. \\
\includegraphics[width=7in]{Yau-07.png}\\


\newpage


\noindent \textbf{\#4} Unfinished business.  How can you not look at this table of formulas and feel something? \\
\includegraphics[width=8in]{Yau-08.png}\\

\newpage

\noindent \dots

\vfill

\begin{thebibliography}{}

\item Chien-Hao Liu, Shing-Tung Yau. \textbf{Dynamics of D-branes II. The standard action --- an analogue of the Polyakov action for (fundamental, stacked) D-branes} \texttt{arXiv:1704.03237}

\item Yang-Hui He, Rak-Kyeong Seong, Shing-Tung Yau. \textbf{Calabi-Yau Volumes and Reflexive Polytopes} \texttt{arXiv:1704.03462}


\item Pavel Putrov, Juven Wang, Shing-Tung Yau. \textbf{Braiding Statistics and Link Invariants of Bosonic/Fermionic Topological Quantum Matter in 2+1 and 3+1 dimensions} \texttt{arXiv:1612.09298}

\item Dan Xie, Shing-Tung Yau \textbf{4d N=2 SCFT and singularity theory Part I: Classification} \texttt{arXiv:1510.01324} 





\end{thebibliography}

\end{document}