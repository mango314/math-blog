\documentclass[12pt]{article}
%Gummi|065|=)
\usepackage{amsmath, amsfonts, amssymb}
\usepackage[margin=0.5in]{geometry}
\usepackage{xcolor}
\usepackage{graphicx}

\usepackage{pifont}
\usepackage{amsmath}

\newcommand{\off}[1]{}
\DeclareMathSizes{20}{30}{20}{18}

\newcommand{\two }{\sqrt[3]{2}}
\newcommand{\four}{\sqrt[3]{4}}
\newcommand{\red}{\begin{tikz}[scale=0.25]
\draw[fill=red, color=red] (0,0)--(1,0)--(1,1)--(0,1)--cycle;\end{tikz}}
\newcommand{\blue}{\begin{tikz}[scale=0.25]
\draw[fill=blue, color=blue] (0,0)--(1,0)--(1,1)--(0,1)--cycle;\end{tikz}}
\newcommand{\green}{\begin{tikz}[scale=0.25]
\draw[fill=green, color=green] (0,0)--(1,0)--(1,1)--(0,1)--cycle;\end{tikz}}

\newcommand{\sq}[3]{\draw[#3] (#1,#2)--(#1+1,#2)--(#1+1,#2+1)--(#1,#2+1)--cycle;}

\usepackage{tikz}

\newcommand{\susy}{{\bf Q}}
\newcommand{\RV}{{\text{R}_\text{V}}}

\title{Scratchwork: Averages}
\author{John D Mangual}
\date{}
\begin{document}

\fontfamily{qag}\selectfont \fontsize{12.5}{15}\selectfont

\maketitle

\noindent In the process of trying to solve the \textbf{Averged Chowla Conjecture} two young mathematicians Radzi\l\l and M\"{a}tomaki (accompanied by Tao) are producing novel range of averaging techniques.   For years (since about high school) I have told myself I would learn to do inequalities. \\ \\
These results are somewhat high-end\footnote{this is Wal-Mart vs Bergdorf Goodman} and we shall take a somewhat ``compendious" approach and merely collect the results in a suggestive way.  They are insterested in averages like this:
$$ \sum_{X < n \leq 2X} \Lambda(n) \Lambda(n+h) = c \, X + o \,( \, X^{\frac{1}{2} + o(1)}\,) $$
these are related to the ``Twin Primes" conjecture.  They are interested in the Van Mangoldt function, which is basically $\log p$ whenever $p$ is a prime:
$$ \Lambda(n) = \left\{ 
\begin{array}{cc} \log p & n = p^k\\
0 & n \neq p \end{array}
 \right.  $$
These function definition has to do with unique factorization.  
$$ n = p_1^{k_1}p_2^{k_2} \dots p_m^{k_m} \to \log n = k_1 \log p_1 + \dots k_m \log p_m \to ``\log"n \stackrel{?}{=} \log p_1 + \dots + \log p_m $$
I dislike \textbf{analytic} techniques because we don't . The proofs don't have to be elementary, they could be quite advanced, but they should look like number theory.  I had hoped. \\ \\
Conjectures like these are a little dry for my taste, but as a disciplined person I know they are very important.
$$ c = 0 \text{ if } h \text{ is odd}$$
I can prove this part.  If $h$ is odd, then either one or the other of $n$ or $n+h$ is odd.  So either $\Lambda(n)= 0$ or $\Lambda(n+h) = 0$.  E.g. either $6$ is prime or $7$ is prime, not both.  If $h$ is even, there is content:
$$ \dots $$
This is classic Tao (who learned it from Erd\H{o}s) to juxtapose the tautological and the impossible.  We might wonder what kind of stuff is lurking around the number $\color{green!50!blue}{\mathbf{2}}$ ? They will also be interested in the divisor function. $d(n)$. E.g. 
$$d(6) = \# \{1,2,3,6  \}= d(2) \times d(3) = \# \{ 1 , 2 \} \times \# \{ 1, 3  \} = 4 $$
I've concluded this type of ``hardness" is also protecting us?  What if it was easy to open up our house?  Our safety rests on somebody not reproducing the notches of our key -- just a few bits of information. 

\newpage

\noindent The value of $c$ when $h$ is even has to so with the twin primes constant:
$$ c = 2 \times \prod_{p > 2} \Big( 1 - \frac{1}{(p-1)^2}\Big) \times \prod_{p | h } \frac{p-1}{p-2}  $$
Although\dots I might threaten to find structure there as well.  If we shift the twin-primes constant just a bit, we get one of my favorite constants:
$$ \frac{6}{\pi^2} = \prod_{p > 2} \Big( 1 - \frac{1}{p^2}\Big)  $$
The rest of these conjectures as so schematic.  You can't just say ``for some polynomial" -- this kind of numbr theory is a bit myopic in my taste.  \\ \\
Lastly, there are lots and lots of multiplicative or well-behaved number theoretic functions that could be derived from modular forms.  These people reason about classes of modular forms ``cuspidal" or ``automorphic"\footnote{ or $E_k(\Gamma)$ or $H^1(\Gamma, \mathbb{C})$ or $L^2(G(Q)\backslash G(A))$ or what-have-you.} but we don't get any translation to classical number theory.  Perhaps that's up to us. \\ \\
Just to give you a taste.  Here's the sum of squares function:
$$ r_k(n) = \big|  \{ (a_1, \dots, a_k) \in \mathbb{Z}^k : a_1^2 + \dots + a_k^2 = n \}  \big|  $$
It doesn't look that way, but it is multiplicative.  This sum of squares formula can be expressed in terms of the divisor function.
$$ r_2(n) = 4 \, \big[ \, d_{1(4)}(n) - d_{3(4)}(n) \, \big] $$ 
and maybe there is a formula for $r_3(n)$ or $r_4(n)$.  Maybe not?  This would mean the sum-of-squares type functions have somthing that's not contained in the divisors of $n$.  That could make sense too\dots \\ \\
The estimates then become very meaningful.  Some type of twin-prime type estimate; very difficult to obtain:
$$\sum_{X < n \leq 2X} d_2(n) d_2(n+h) = P_{2,2,h} \Big( \log X \Big) \, X + O( X^{\frac{2}{3} + o(1)} ) $$
For some anonymous polynomial logarithm.  That ends my complaining.  Onto formulas !

\newpage

\noindent All of these equations they cite are new and look very different:
$$ \int_T^{2T} |\zeta(\frac{1}{2} + it) |^6 \, dt \ll T^{\frac{5}{4} + \epsilon} $$
and for rather mysterious reasons they compare to averages of the divisor function \textit{in a specific range}
$$  \sum_{h \leq H} \sum_{n \leq X} d_3(n) d_3(n+h)$$
with $H = X^\frac{1}{3}$.  Even though in theory, studying the zeta function is a kind of ``multiplicative" Fourier analysis -- the exponent map changes the Fourier transform to the Mellin transform, but I don't have much practice with the Mellin transform version.\footnote{for a brand-new perspective on the Mellin transform from AdS/CFT \texttt{https://arxiv.org/abs/1107.1499} }
$$ \int_{-\infty}^\infty e^{-sx} f(x) \, dx \longleftrightarrow \int_0^\infty x^{s-1} f(x) \, dx $$
Here's an amazing totally useless integral.  It's what we're here for.
$$ \int_0^X \left(  \left| \sum_{m=1}^M {\color{red!50!orange}a}_m \, e(t \, \phi(m) )  \right|^* \right)^4  \, dt \ll \frac{X \log^4 X}{Y} \int_0^Y \left( \left| \sum_{m=1}^M e(t \, \phi(m) )  \right| \right)^4 \, dt$$
and there's a mean-value-theorm that I liked:
$$ \int_{T_0}^{T_0 + T} \left|\mathcal{D}[f](\frac{1}{2} + it)  \right|^2 \ll \frac{T+X}{X} ||f||_{\ell^2}  \text{ where } \mathcal{D}[f](s) := \sum_{n=1}^\infty \frac{f(n)}{n^s} $$
One thing that's different from their Fourier analysis vs the textbooks, is they are not integrating over all of $\mathbb{R}$ instead only on a segment $[0,X]$ or $[T, 2T]$ for specific, rather chaotic functions.   This is kind of like a tennis player playing grass-court instead of hard-court.
\vfill



\begin{thebibliography}{}

\item Kaisa Matom\"{a}ki, Maksym Radziwi\l\l, Terence Tao 
\textbf{Correlations of the von Mangoldt and higher divisor functions I. Long shift ranges} \texttt{arXiv:1707.01315}


\end{thebibliography}


\end{document}