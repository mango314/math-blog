\documentclass[12pt]{article}
%Gummi|065|=)
\usepackage{amsmath, amsfonts, amssymb}
\usepackage[margin=0.5in]{geometry}
\usepackage{xcolor}
\usepackage{graphicx}

\newcommand{\off}[1]{}
\DeclareMathSizes{20}{30}{20}{18}

\newcommand{\two }{\sqrt[3]{2}}
\newcommand{\four}{\sqrt[3]{4}}


\usepackage{tikz}

\title{What is Quiver W-Algebra?}
\author{John D Mangual}
\date{}
\begin{document}

\fontfamily{qag}\selectfont \fontsize{12.5}{15}\selectfont

\maketitle

\noindent Here is another situation where I must play ``follow the leader".  Between Pestun, and Kimura, and Nekrasov there's a discussion fo the $qq$-characters and these W-algebras.\footnote{An ``algebra" is just anything with a reasonable addition and multiplication. There are lots and lots of algebras we can use and study. It could be anything.  Yet, I'm confident Vasily knows what he's doing.  Why these?  and more importantly, how can I link these to the stuff that I care about? or to the big discussion?} \\ \\
Why does Pestun and Kimura extend W-algebra in this way?  The only sentence that resonates with me in any way from the intro is ``{\color{red!50!orange}our construction is orthogonal to AGT construction}".  What is AGT?  It is the definition of a conformal field theory, but it is also the literature -- all the research papers that are called ``AGT".  Tody, our discussion will be orthogonal to that.
\vfill

\begin{thebibliography}{}

\item Taro Kimura, Vasily Pestun \textbf{Fractional Quiver W-algebras} \texttt{arXiv:1705.04410}

\item Taro Kimura, Vasily Pestun \textbf{Quiver Elliptic W-algebras} \texttt{arXiv:1608.04651}

\item Taro Kimura, Vasily Pestun \textbf{Quiver W-algebras} \texttt{arXiv:1512.08533}

%\item Nikita Nekrasov. \textbf{BPS/CFT correspondence: non-perturbative Dyson-Schwinger equations and qq-characters.} \texttt{arXiv:1512.05388}





\end{thebibliography}

\end{document}