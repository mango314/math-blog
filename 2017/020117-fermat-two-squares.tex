\documentclass[12pt]{article}
%Gummi|065|=)
\usepackage{amsmath, amsfonts, amssymb}
\usepackage[margin=0.5in]{geometry}
\usepackage{xcolor}
\usepackage{graphicx}
%\usepackage{graphicx}
\newcommand{\off}[1]{}
\DeclareMathSizes{20}{30}{20}{18}
\newcommand{\myhrule}{}

\newcommand{\two }{\sqrt[3]{2}}
\newcommand{\four}{\sqrt[3]{4}}

\newcommand{\dash}{
\begin{tikzpicture}[scale=1]
\draw (0,0)--(19,0);
\end{tikzpicture}
}

\newcommand{\sq}[3]{
\node at (#1+0.5,#2+0.5) {#3};
\draw (#1+0,#2+0)--(#1+1,#2+0)--(#1+1,#2+1)--(#1+0,#2+1)--cycle;
}

\usepackage{tikz}

\title{\textbf{Proposal: Fermat Two Squares}}
\author{John D Mangual}
\date{}
\begin{document}

\fontfamily{qag}\selectfont \fontsize{20}{25}\selectfont

\maketitle

\noindent The last number theory project I want to propose is the Fermat Two-Squares problem.
$$ p = x^2 + y^2 \longleftrightarrow p \equiv 1 \text{ mod }4 $$
It took me years to decide these are separate projects, compared to the 3- and 4- squares projects. 
\begin{itemize}
\item every positive number is the sum of four squares
\item every prime number (with remainder of 1 after dividing by 4) is the sume of two perfect squares
\end{itemize}
Our perspective will be slightly advaned.  If you go on Wikipedia, our prime number theorem is a type of \textbf{Fermat Descent}.  Other examples are:
\begin{itemize}
\item $\sqrt{2}$ is not a fraction
\item Viet\'{e} Jumping (e.g. $x^2 + y^2 + z^2 = 3xyz$)
\item $a^2 + b^4 = c^4$ has no integer solutions
\end{itemize}
I am less interested in the last one.  It is very specific and very technical.  Hopefully I can find the specific name for the reason I don't like it very much. 

\newpage

\noindent In a way, all of my projects are the same\dots no matter which equation I pick it all becomes hard-core analysis and cohomology. \\ \\
Here is the from the Wikipedia page on Fermat's Infinite Descent:
\begin{quotation}
{\color{blue} To extend this to the case of an abelian variety, Andr\'{e} Weil had to make more explicit the way of quantifying the size of a solution, by means of a \textbf{\color{black}height function} -- a concept that became foundational. \\\\ To show that $A(Q)/2A(Q)$ is finite, which is certainly a necessary condition for the finite generation of the group $A(Q)$ of rational points of $A$, one must do calculations in what later was recognised as \textbf{\color{black}Galois cohomology}. In this way, abstractly defined cohomology groups in the theory become identified with \textit{\color{green!85!white!90!black}\textbf{descents}} in the tradition of Fermat. }
\end{quotation}
One example of Fermat decent is that the solutions to 
$$ x^2 + y^2 + z^2 = 3xyz $$
form a tree.  In the two squares case descent means that:
$$ p = 102761 = 19 \times 19 + 320 \times 320 $$
can be found in a systematic way.  Our goal is to understand why his is an instance of Galois Cohomology.

\newpage

Markoff Tree:

\includegraphics[width=5in]{MarkoffNumberTree.png}

\newpage



\fontfamily{qag}\selectfont \fontsize{12}{10}\selectfont

\begin{thebibliography}{}

\item 

\end{thebibliography}

\end{document}