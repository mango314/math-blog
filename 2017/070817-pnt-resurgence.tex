\documentclass[12pt]{article}
%Gummi|065|=)
\usepackage{amsmath, amsfonts, amssymb}
\usepackage[margin=0.5in]{geometry}
\usepackage{xcolor}
\usepackage{graphicx}

\usepackage{pifont}
\usepackage{amsmath}

\newcommand{\off}[1]{}
\DeclareMathSizes{20}{30}{20}{18}

\newcommand{\two }{\sqrt[3]{2}}
\newcommand{\four}{\sqrt[3]{4}}
\newcommand{\red}{\begin{tikz}[scale=0.25]
\draw[fill=red, color=red] (0,0)--(1,0)--(1,1)--(0,1)--cycle;\end{tikz}}
\newcommand{\blue}{\begin{tikz}[scale=0.25]
\draw[fill=blue, color=blue] (0,0)--(1,0)--(1,1)--(0,1)--cycle;\end{tikz}}
\newcommand{\green}{\begin{tikz}[scale=0.25]
\draw[fill=green, color=green] (0,0)--(1,0)--(1,1)--(0,1)--cycle;\end{tikz}}

\newcommand{\sq}[3]{\draw[#3] (#1,#2)--(#1+1,#2)--(#1+1,#2+1)--(#1,#2+1)--cycle;}

\usepackage{tikz}

\newcommand{\susy}{{\bf Q}}
\newcommand{\RV}{{\text{R}_\text{V}}}

\title{Scratchwork: Divergences}
\author{John D Mangual}
\date{}
\begin{document}

\fontfamily{qag}\selectfont \fontsize{12.5}{15}\selectfont

\maketitle

\noindent It's time to get serious.  I can nearly put it together myself.  \\ \\
How do I review this proof without it degenerating into some kind of recitation fo facts? One of my critiques of analytic number theory is that\dots it doesn't look like number theory.  If I spend all this effort to prove the Prime Number Theory\dots I already believed it was true! \\ \\
I started to look for arguments where the connection to prime factorization or to Geometry or Probablity or any other branch of Math.\footnote{If I express someone of my aggravation, and provide a partial demonstration / answer, that could be new.}  Talking to professors, I'm pretty much out of luck. They are satisfied with the current arguments.  They are professionals, I'm not.\\ \\
Try to imagine if Hollywood or a Hip-Hop label adopted the garbled narrative style of a Math Textbook.  To me, Mathematics has been a giant bait-and-switch.  They sold me one result, I got completely another.   \\ \\ 
I'm trying not to do that to you.  \\ \\


\vfill



\begin{thebibliography}{}

\item Terence Tao 
\\ \textbf{The Euler-Maclaurin formula, Bernoulli numbers, the zeta function, and real-variable analytic continuation} 
\texttt{http://bit.ly/2cyNB3H}
\item GH Hardy \textbf{Divergent Series}  Oxford University Press, 1949/1973.
\item EC Titchmarsh \textbf{Theory of Functions} \texttt{https://archive.org/details/TheTheoryOfFunctions}
\item Simon Rubinstein-Salzedo \textbf{Could Euler have conjectured the prime number theorem?} \texttt{ arXiv:1701.04718}
\item Andrew Granville, Adam J Harper, K. Soundararajan \textbf{A more intuitive proof of a sharp version of Hal\'{a}sz's theorem} \texttt{arXiv:1706.03755}

\item Serge Lang \textbf{Algebraic Number Theory} (Graduate Texts in Mathematics \# 110) Springer, .

\item J\"{u}rgen Neukirch. \textbf{Algebraic Number Theory} (Grundlehren der mathematischen Wissenschaften \#322) Springer, 1999.

\item David Sauzin \textbf{Introduction to 1-summability and resurgence} \texttt{arXiv:1405.0356}

\item Don Zagier \textbf{On Newman's Short Proof of the Prime Number Theorem} \texttt{http://people.mpim-bonn.mpg.de/zagier/files/doi/10.2307/2975232/fulltext.pdf}

\end{thebibliography}


\end{document}