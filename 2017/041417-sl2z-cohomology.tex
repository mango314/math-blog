\documentclass[12pt]{article}
%Gummi|065|=)
\usepackage{amsmath, amsfonts, amssymb}
\usepackage[margin=0.5in]{geometry}
\usepackage{xcolor}
\usepackage{graphicx}

\newcommand{\off}[1]{}
\DeclareMathSizes{20}{30}{20}{18}

\newcommand{\two }{\sqrt[3]{2}}
\newcommand{\four}{\sqrt[3]{4}}


\usepackage{tikz}

\title{Item: \textbf{Cohomology of Arithmetic Groups}}
\author{John D Mangual}
\date{}
\begin{document}

\fontfamily{qag}\selectfont \fontsize{12.5}{15}\selectfont

\maketitle

\noindent The Eichler-Shimura isomorphism let's you express weight-2 modular forms theory into cohology of $\text{SL}_2(\mathbb{Z})$. This will take a lot of effort to unpack.  One version I have found says these two are the same:
\begin{itemize}
\item $f(z) \, dz$ is a $\Gamma$-invariant form on $\mathbb{H}$
\item $[ \, f(z) dz \, ] \in H^1 (\mathbb{H}/\Gamma, \mathbb{C}) = H^1(\Gamma, \mathbb{C})$
\end{itemize}
but these are two different kinds of cohomology.  One of them is a hyperbolic space $\mathbb{H}/\Gamma$ and the other is a group of $2 \times 2$ matrices $\Gamma \subseteq \text{SL}_2(\mathbb{Z})$.  How can we not have a complete understanding of both of these objects? \\ \\
There's a trade-off between generality and our ability to supply details.  I never told you what $\Gamma$ was, and the entire textbook writes the discussion without naming a specific answer.  How can they have the best possible answer?  If, I decide to focus on one $\Gamma$, let's say $\Gamma_0(4) = \langle z \mapsto z + 1, z \mapsto - \frac{1}{4z}\rangle$ maybe I will say things that don't generalize.  \\ \\
In between, would be story where I examine many possible $\Gamma$ and a statement will be true in cases and not others (in many cases and not others). I might even be able to express this type of meta-logic using a small amount of category theory. \\ \\
Let's find a modular form of weight 2.  The first one I can think of is a theta-function raised to the 4th power:
$$  \theta(z) = \big( \sum q^{n^2} \big)^4 = \sum r_4(n) \, q^n $$
and here $\Gamma_0(4) \neq \text{SL}_2(\mathbb{Z})$. How do we know it is modular form of weight 2?  This is a great example if we keep in mind the following recipe:
$$ \text{M}_2(\Gamma_0(N)) = \text{S}_2(\Gamma_0(N)) \oplus \text{E}_2(\Gamma_0(N)) $$
for all congruence groups, not just $N = 4$.  This says every weight two modular form splits into to parts:
\begin{itemize}
\item Eisenstein series
\item Cusp forms
\end{itemize}
The jargon gets worse and worse.  Eichler-Shimura theory, Atkin-Lehmer theory.   If I have an interestng number theory problem, maybe I can turn it into a modular forms problem:
$$ \text{modular forms} \stackrel{?}{\not =} \text{number theory} $$
I cannot find any modular forms of weight 2 that are not Eisenstein series until $\Gamma_0(11)$

\newpage

\noindent \dots

\vfill

\fontfamily{qag}\selectfont \fontsize{12}{10}\selectfont

\begin{thebibliography}{}

\item John Cremona \textbf{The L-functions and modular forms database project}  \texttt{arXiv:1511.04289}  
\texttt{http://www.lmfdb.org/}

\item William Stein \textbf{Modular Forms, A Computational Approach} Modular forms of Weight 2 \texttt{http://wstein.org/books/modform/modform/index.html}

\end{thebibliography}

\end{document}