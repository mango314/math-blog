\documentclass[12pt]{article}
%Gummi|065|=)
\usepackage{amsmath, amsfonts, amssymb}
\usepackage[margin=0.5in]{geometry}
\usepackage{xcolor}
\usepackage{graphicx}


\usepackage{amsmath}

\newcommand{\off}[1]{}
\DeclareMathSizes{20}{30}{20}{18}

\newcommand{\two }{\sqrt[3]{2}}
\newcommand{\four}{\sqrt[3]{4}}
\newcommand{\red}{\begin{tikz}[scale=0.25]
\draw[fill=red, color=red] (0,0)--(1,0)--(1,1)--(0,1)--cycle;\end{tikz}}
\newcommand{\blue}{\begin{tikz}[scale=0.25]
\draw[fill=blue, color=blue] (0,0)--(1,0)--(1,1)--(0,1)--cycle;\end{tikz}}
\newcommand{\green}{\begin{tikz}[scale=0.25]
\draw[fill=green, color=green] (0,0)--(1,0)--(1,1)--(0,1)--cycle;\end{tikz}}

\newcommand{\sq}[3]{\draw[#3] (#1,#2)--(#1+1,#2)--(#1+1,#2+1)--(#1,#2+1)--cycle;}

\usepackage{tikz}

\newcommand{\susy}{{\bf Q}}
\newcommand{\RV}{{\text{R}_\text{V}}}

\title{Scratchwork: Theta Functions}
\date{}
\begin{document}

%\fontfamily{qag}\selectfont \fontsize{12.5}{15}\selectfont

\sffamily

\maketitle

\noindent Haivng tried it a bunch of times, I am surprised to see the sum of three square turn into many things.  It may even be unfinished business from high school.  In Chemistry class, the problem was to find:
$$ \int_{-\infty}^\infty x^2\,e^{-ax^2 }dx  $$
We did this very carefully over the first week of class. The idea is to \textbf{differentiate under the integral sign}. 
$$  \int_{-\infty}^\infty x^2\,e^{-ax^2 }dx
=  -\frac{d}{da} \int_{-\infty}^\infty e^{-ax^2 }dx
=  -\frac{d}{da} \int_{-\infty}^\infty e^{-a(x/\sqrt{a})^2 }d(x/\sqrt{a})
=  -\frac{d}{da}  \frac{1}{\sqrt{a}}\int_{-\infty}^\infty e^{-x^2 }dx
$$
and now this could be split into two easy problems from calculus class:
\begin{eqnarray*}  \frac{d}{da} \frac{1}{\sqrt{a}} &=& (-1/2) \frac{1}{a \sqrt{a}} \\
\int_{-\infty}^\infty e^{-x^2 }dx &=& \sqrt{2\pi}
\end{eqnarray*}
These were used to study the statistical mechanics of a free gas (basically \dots air). And many thermodynamic relations were found between entropy $S$, enthalpy $H$, free energy $G$ and helmoltz free energy $A$, such as 
$$ dA = - S \, dT - p \, dV $$ 
The Gaussian (otherwise known as the ``bell curve") appears in \textit{so} many branches of Mathematics, and it continues to hold a central role in \textit{theoretical} discussions.  There are huge problems with it:
\begin{itemize}
\item real probablities may not be bell curves (there just might no be enough numbers)
\item whenever we try to be ``statistical" we immediately assume it's bell curve
\end{itemize}
I didn't become a chemist, these days I'm not doing much of anything, and formulas like these gathered dust.  In probability class we do have the \textbf{Law of Large Numbers}:
$$ \mathbb{P}\big( \big| \frac{1}{n}(X_1 + \dots + X_n) - \mathbb{E}[X] \big| > \epsilon \big) = 0 $$
This is a mathy way of saying if I make a lot of obserations it should tend to the therotical mean.  This is pretty meaningless because in most cases there's no theoretical distribution to figure out.  \\ \\
Also I confuse the Law of Large Numbers with the \textbf{Central Limit Theorem}.
$$ \lim_{n \to \infty} \mathbb{P} \Big[  \frac{1}{n}(X_1 + \dots + X_n) - \mathbb{E}[X]  \leq \frac{z}{\sqrt{n}} \Big] = \int_{-\infty}^z e^{-x^2/\sigma} dx $$
where $\sigma$ is the variance of $X$, $\sigma^2 = \mathbb{E}[X^2] - \mathbb{E}[X]^2$. These formulas and considerations might be great for technicians, but can the end-user like me - no expensive laboratory or anything - put them to use as well? \\ \\
So we are led to (an ongoing) crisis about the usefulness of math.  My original intent is to study solutions to $n = x^2 + y^2 + z^2$.  To keep myself busy.

\newpage

\noindent Especially for quadratic forms theta functions are the tool of choice (as a kind of generating function).
$$ \theta(z)^3 = \left[ \sum q^{n^2} \right]^3 = \sum \left[\, \sum_{a^2 + b^2 + c^2 = m} 1 \right] q^m 
= \sum r_3(m) \, q^m $$
and we observe that each term in the theta function looks like the Gaussian:
$$ \theta(z) = \sum_{n \in \mathbb{Z}} e^{2\pi i \, n^2 z} $$
Therefore, trying to solve integer quadratic equations leads to the Gaussian as well.  Maybe there is some vague link between random walk and integer quadratic forms.  \\ \\
If I try to prove the basic symmetry properties there are two:
\begin{itemize}
\item $\theta(z+1) = \theta(z)$
\item $\theta(-1/z) =  (1/\sqrt{z}) \theta(z)$
\item The symmetry group generated by $\langle z  \rangle $ is $\Gamma_0(4)$ and the index of the subgroup is $[SL_2(\mathbb{Z}):\Gamma_0(4) ] = 6$.
\end{itemize}
I think the triple product identity is also worth observing:
$$ \theta_{00}(a, q) = \sum_{n \in \mathbb{Z}} a^{2n} q^{n^2} = 
\prod_{m=1}^\infty \big( 1 - q^{2m} \big) \big( 1 + a^2 q^{2m-1} \big) \big( 1 + a^{-2} q^{2m-1} \big)$$
The theta function occurs in the heat equation and in number theory.  These manipulations seem rather artificial (if beautiful) and the more you read, the harder they are thrown at you.  \\ \\
Of the two basic symmetry properties, the first looks self-evident since $q^{n(z+1)} = q^{n}$ but the second one is Poisson summation:
$$ \sum_{n \in \mathbb{Z}} e^{2\pi i n^2 z} = \frac{1}{\sqrt{2\pi i \, z}}\sum_{n \in \mathbb{Z}} e^{2\pi i n^2 /z} $$
Did I write the formula correctly?  The summation formla says for $f(x) \in \mathcal{S}(\mathbb{R})$ (in the ``Schwartz class") then
$$ \sum_{n \in \mathbb{Z}} f(z) = \sum_{n \in \mathbb{Z}} \widehat{f}(z) $$
so we need to do a careful evaluation of the Fourier transform. I have no idea what is $\mathcal{S}(\mathbb{R})$ except to write a function down and check it.  While there are other elements of Schwartz class, the main building block we have to far are functions weighted by Gaussians $p(x) e^{-x^2}$ and off we go.
\vfill

\begin{thebibliography}{}

\item PW Atkins, Julio de Paula. \textbf{Physical Chemistry} (10th ed.) WH Freeman, 2014.

\end{thebibliography}

\newpage

\noindent \textbf{11/13} To focus our discussion we need to pick an example.  Let $u(a,b) = a^2 - b^2$ be harmonic function.
$$ \theta(z;u) = \sum (a^2 - b^2) \, e^{2\pi i (a^2 + b^2 )\,z } $$
This is a theta function, and cusp form on $\Gamma_0(4)$.  That is the very first thing I learned.
\begin{center}
$\text{ theta functions } \subseteq \text{ modular forms }$  
\end{center}
I don't really know what a ``modular form" is anyway or a ``cusp form" anyway.\footnote{but they've been studied for decades at a level of abstraction that is out of reach}  For now, it means the values $z = 0, \frac{1}{2}, \infty$ matter \textit{a lot}.  Our exercise is to show - in example other than the original theta function - is that Poisson summation is holding:
$$ \sum_{n \in \mathbb{Z}} e^{- \pi n^2 \, t} =  \frac{1}{\sqrt{t}} \sum_{n \in \mathbb{Z}} e^{-\pi n^2 /t} $$
and modular invariance follow from \textbf{analytic continuation}, $t \in \mathbb{R}$ to $z \in \mathbb{C}$.  How do I teach - quickly and without too much complication - the modular invariance of $\theta(z;u)$? \\ \\
Not knowing anything about Fourier transforms we could write down the two-variables formula:
$$ 
\sum_{(m,n) \in \mathbb{Z}^2} e^{- \pi ( m^2 + n^2 ) \, t} =  \frac{1}{t} 
\sum_{(m,n) \in \mathbb{Z}^2} e^{- \pi ( m^2 + n^2 ) /t} $$
and - waving my hands a lot - if we weight by a harmonic polyonomial $u(m,n) = m^2 - n^2$ has degree $\deg u = 2$ then 
$$ 
\sum_{(m,n) \in \mathbb{Z}^2} (m^2 - n^2 )\, e^{- \pi ( m^2 + n^2 ) \, t} =  \frac{1}{t^2} 
\sum_{(m,n) \in \mathbb{Z}^2} (m^2 - n^2 )\, e^{- \pi ( m^2 + n^2 ) /t} $$
and whenever we have a harmonic polynomial $u(x,y)$.  And we get an endless supply of cusp forms to test other conjectures on.  For example, $\theta(z;u) \in L^2 (\Gamma_0(4) \backslash \mathbb{H}) $. \\ \\
We could deduce this formula from \textbf{Poisson summation} and the ``rules" of Fourier transform.
\begin{itemize}
\item If $f \in \mathcal{S}(\mathbb{R})$ \hspace{1.2em} then $\sum_{n \in \mathbb{Z}} f(n) = \sum_{n \in \mathbb{Z}} \widehat{f}(n)$.
\item If $f(x) = e^{-\pi x^2 }$ then $\widehat{f}(\xi) = f(\xi)$ and $\int_\mathbb{R} e^{-\pi x^2} \, dx = 1$.
\item If $f \in \mathcal{S}(\mathbb{R})$ \hspace{1.2em} then $(-2\pi i x) f(x) \to \frac{d}{d\xi} \widehat{f}(\xi)$
\end{itemize}
I don't know if Schwartz class is the best object for Fourier analysis -- certainly real signals are not guaranteed to decay in the way it resuires -- but it's the most natural in this particular niche.  \\ \\
Later I might try to extend to Adeles, $\mathcal{S}(\mathbb{A})$, and we'd have to ask how to extend the modular surface to adeles:
$$ \mathbb{H}\simeq \text{SL}(2, \mathbb{R})/\text{SO}(2, \mathbb{R}) 
\hspace{0.25in}\text{ and }\hspace{0.25in}
\text{SL}(2, \mathbb{Z})  \backslash \mathbb{H}\simeq \text{SL}(2, \mathbb{Z}) \backslash \text{SL}(2, \mathbb{R})/\text{SO}(2, \mathbb{R})$$
Number theory also take a very broad view of what ``Fourier coefficients" actually are -- views which have no elementary or engineering counterparts.  If we use ``\textit{Strong Approximation}":
$$ \text{T}^1 \big( \text{SL}(2, \mathbb{Z} ) \backslash \mathbb{H} \big) \simeq \text{SL}(2, \mathbb{Z})   \backslash \text{SL}(2, \mathbb{R})  
\simeq \text{SL}(2, \mathbb{Q})/\text{SO}(2, \mathbb{A})   $$
This is fine and wonderful: the theory of modular forms has been extending way beyond our original problem.  To the point of getting completely lost.

\newpage

\noindent \textbf{Ex.} Let's set $f(x,y) = (x^2 - y^2) \, e^{- \pi (x^2 + y^2 )\,t}$ and let's build the statement of Poisson summation in this case:
$$\sum_{(m,n) \in \mathbb{Z}^2} f(m,n) = 
\sum_{(m,n) \in \mathbb{Z}^2} \widehat{f}(m,n)$$
I know the general ``shape" of the equation I should get, but not the constants.\footnote{It may not even be true, but I admired Richard Feynman for getting all of this constants right, including factors of $\pm 1$.  He was also a Putnam Fellow, before doing Physics graduate school at Princeton.  These says we are critical of this same guy, Richard Feynman for his ``path integral" which cannot be defined\dots} At least, let $g(x) = e^{-\pi x^2}$:
$$ \left[ \, \int_\mathbb{R} e^{-\pi x^2} \, dx = 1 = \hat{g}(0) \right] \to 
\left[ \, \int_\mathbb{R} e^{-\pi x^2} \, e^{2\pi i \, n x} \, dx = e^{- \pi n^2} = \hat{g}(n) \right] $$
Hopefully that wasn't too circular,  but if we rescale, $g(x \sqrt{t}) = e^{\pi x^2 \, t} $ the Fourier transform behaves nicely under dilation:
$$ \left[ \, \int_\mathbb{R} e^{-\pi x^2} \, e^{2\pi i \, n x} \, dx = e^{- \pi n^2} = \hat{g}(n) \right] \to 
\left[ \, \int_\mathbb{R} e^{-\pi x^2 \, t} \, e^{2\pi i \, n x} \, dx = \frac{1}{\sqrt{t}} e^{- \pi n^2 / t} = (1/\sqrt{t}) \, \hat{g}(n/\sqrt{t}) \right] $$
and out comes the Poisson Summation formula (in one variable):
$$ \sum_{n \in \mathbb{Z}} g(n) =   \sum_{n \in \mathbb{Z}} e^{\pi  n^2 \, t} =    \sum_{n \in \mathbb{Z}} \hat{g}(n) =  \frac{1}{\sqrt{t}} \sum_{n \in \mathbb{Z}} e^{\pi  n^2 / t} $$
we need a \textit{weighted} Poisson summation formula in \textit{two} variables. \\ \\
Can we use the Fourier transform rules to solve our main problem?
\begin{eqnarray*} 
f_0(x,y) = e^{- \pi (x^2 + y^2 )} \hspace{0.7in} &\to& \hat{f}_0(\xi,\eta) = e^{- \pi (\xi^2 + \eta^2 )}\\
f_1(x,y) = e^{- \pi (x^2 + y^2 )\,t} \hspace{0.65in} &\to& 
\hat{f}_1(\xi,\eta) =  \left( \frac{1}{\sqrt{t}} \right)^2 \hat{f}_0(\xi/\sqrt{t},\eta/\sqrt{t}) = \frac{1}{t} e^{- \pi (\xi^2 + \eta^2 )/t} \\
f_2(x,y) = (x^2 - y^2) \, e^{- \pi (x^2 + y^2 )\,t} &\to& 
\hat{f}_2(\xi, \eta) = \frac{1}{(-2\pi i)^2}\left( \frac{d^2}{d\xi^2} - \frac{d^2}{d\eta^2} \right)\hat{f}_1(\xi, \eta)
= \frac{1}{t^2} (\xi^2 - \eta^2) e^{- \pi (\xi^2 + \eta^2 )/t}
\end{eqnarray*}
Then we obtain the formula on the previous page:
$$ 
\sum_{(m,n) \in \mathbb{Z}^2} (m^2 - n^2 )\, e^{- \pi ( m^2 + n^2 ) \, t} =  \frac{1}{t^2} 
\sum_{(m,n) \in \mathbb{Z}^2} (m^2 - n^2 )\, e^{- \pi ( m^2 + n^2 ) /t} $$
Note there are two indexes floating around.  $(m,n) \in \mathbb{Z}^2$ counting lattice points and $t \in \mathbb{R}$.  These are Fourier series in \textit{one} variable.  Our question is about the decay of Fourier coefficients in 4-variables (and 3 variables):
$$ \theta(u;z) = \sum_{m \in \mathbb{Z}^4} u(m) e^{- \pi  |m|^2 \,z} = f(z) = \sum a_n \, e^{2\pi i \, z } $$
Can we get $\theta(u;z)$ to look like an $f(z)$ ? We'd like to show that $a_n = o(n)$ or even $O(n^{-1/28})$.  And so our formulas will go under even more strutiny.  For now let's look on the right side: Stein's Fourier analysis textbook is holding strong even as I whack it on quite advanced material. 
\vfill

\begin{thebibliography}{}

\item Elias Stein, Rami Shakarchi.  \textbf{Fourier Analysis: An Introduction}. (Princeton Lectures in Analysis I), Princeton University Press, 2003.

\item Don Zagier ``Elliptic Modular Forms and Their Applications" from \textbf{1-2-3 of Modular Forms} (Universitext) Springer, 2008.

\item Phillip Fleig, Henrik Gustaffson, Axel Kleinschmidt, Daniel Persson.  \textbf{Eisenstein Series and Automorhpic Reresentations} \texttt{arXiv:1511.04265}

\end{thebibliography}

\end{document}