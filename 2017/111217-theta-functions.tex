\documentclass[12pt]{article}
%Gummi|065|=)
\usepackage{amsmath, amsfonts, amssymb}
\usepackage[margin=0.5in]{geometry}
\usepackage{xcolor}
\usepackage{graphicx}


\usepackage{amsmath}

\newcommand{\off}[1]{}
\DeclareMathSizes{20}{30}{20}{18}

\newcommand{\two }{\sqrt[3]{2}}
\newcommand{\four}{\sqrt[3]{4}}
\newcommand{\red}{\begin{tikz}[scale=0.25]
\draw[fill=red, color=red] (0,0)--(1,0)--(1,1)--(0,1)--cycle;\end{tikz}}
\newcommand{\blue}{\begin{tikz}[scale=0.25]
\draw[fill=blue, color=blue] (0,0)--(1,0)--(1,1)--(0,1)--cycle;\end{tikz}}
\newcommand{\green}{\begin{tikz}[scale=0.25]
\draw[fill=green, color=green] (0,0)--(1,0)--(1,1)--(0,1)--cycle;\end{tikz}}

\newcommand{\sq}[3]{\draw[#3] (#1,#2)--(#1+1,#2)--(#1+1,#2+1)--(#1,#2+1)--cycle;}

\usepackage{tikz}

\newcommand{\susy}{{\bf Q}}
\newcommand{\RV}{{\text{R}_\text{V}}}

\title{Scratchwork: Theta Functions}
\date{}
\begin{document}

%\fontfamily{qag}\selectfont \fontsize{12.5}{15}\selectfont

\sffamily

\maketitle

\noindent Haivng tried it a bunch of times, I am surprised to see the sum of three square turn into many things.  It may even be unfinished business from high school.  In Chemistry class, the problem was to find:
$$ \int_{-\infty}^\infty x^2\,e^{-ax^2 }dx  $$
We did this very carefully over the first week of class. The idea is to \textbf{differentiate under the integral sign}. 
$$  \int_{-\infty}^\infty x^2\,e^{-ax^2 }dx
=  -\frac{d}{da} \int_{-\infty}^\infty e^{-ax^2 }dx
=  -\frac{d}{da} \int_{-\infty}^\infty e^{-a(x/\sqrt{a})^2 }d(x/\sqrt{a})
=  -\frac{d}{da}  \frac{1}{\sqrt{a}}\int_{-\infty}^\infty e^{-x^2 }dx
$$
and now this could be split into two easy problems from calculus class:
\begin{eqnarray*}  \frac{d}{da} \frac{1}{\sqrt{a}} &=& (-1/2) \frac{1}{a \sqrt{a}} \\
\int_{-\infty}^\infty e^{-x^2 }dx &=& \sqrt{2\pi}
\end{eqnarray*}
These were used to study the statistical mechanics of a free gas (basically \dots air). And many thermodynamic relations were found between entropy $S$, enthalpy $H$, free energy $G$ and helmoltz free energy $A$, such as 
$$ dA = - S \, dT - p \, dV $$ 
The Gaussian (otherwise known as the ``bell curve") appears in \textit{so} many branches of Mathematics, and it continues to hold a central role in \textit{theoretical} discussions.  There are huge problems with it:
\begin{itemize}
\item real probablities may not be bell curves (there just might no be enough numbers)
\item whenever we try to be ``statistical" we immediately assume it's bell curve
\end{itemize}
I didn't become a chemist, these days I'm not doing much of anything, and formulas like these gathered dust.  In probability class we do have the \textbf{Law of Large Numbers}:
$$ \mathbb{P}\big( \big| \frac{1}{n}(X_1 + \dots + X_n) - \mathbb{E}[X] \big| > \epsilon \big) = 0 $$
This is a mathy way of saying if I make a lot of obserations it should tend to the therotical mean.  This is pretty meaningless because in most cases there's no theoretical distribution to figure out.  \\ \\
Also I confuse the Law of Large Numbers with the \textbf{Central Limit Theorem}.
$$ \lim_{n \to \infty} \mathbb{P} \Big[  \frac{1}{n}(X_1 + \dots + X_n) - \mathbb{E}[X]  \leq \frac{z}{\sqrt{n}} \Big] = \int_{-\infty}^z e^{-x^2/\sigma} dx $$
where $\sigma$ is the variance of $X$, $\sigma^2 = \mathbb{E}[X^2] - \mathbb{E}[X]^2$. These formulas and considerations might be great for technicians, but can the end-user like me - no expensive laboratory or anything - put them to use as well? \\ \\
So we are led to (an ongoing) crisis about the usefulness of math.  My original intent is to study solutions to $n = x^2 + y^2 + z^2$.  To keep myself busy.

\newpage

\noindent Especially for quadratic forms theta functions are the tool of choice (as a kind of generating function).
$$ \theta(z)^3 = \left[ \sum q^{n^2} \right]^3 = \sum \left[\, \sum_{a^2 + b^2 + c^2 = m} 1 \right] q^m 
= \sum r_3(m) \, q^m $$
and we observe that each term in the theta function looks like the Gaussian:
$$ \theta(z) = \sum_{n \in \mathbb{Z}} e^{2\pi i \, n^2 z} $$
Therefore, trying to solve integer quadratic equations leads to the Gaussian as well.  Maybe there is some vague link between random walk and integer quadratic forms.  \\ \\
If I try to prove the basic symmetry properties there are two:
\begin{itemize}
\item $\theta(z+1) = \theta(z)$
\item $\theta(-1/z) =  (1/\sqrt{z}) \theta(z)$
\item The symmetry group generated by $\langle z  \rangle $ is $\Gamma_0(4)$ and the index of the subgroup is $[SL_2(\mathbb{Z}):\Gamma_0(4) ] = 6$.
\end{itemize}
I think the triple product identity is also worth observing:
$$ \theta_{00}(a, q) = \sum_{n \in \mathbb{Z}} a^{2n} q^{n^2} = 
\prod_{m=1}^\infty \big( 1 - q^{2m} \big) \big( 1 + a^2 q^{2m-1} \big) \big( 1 + a^{-2} q^{2m-1} \big)$$
The theta function occurs in the heat equation and in number theory.  These manipulations seem rather artificial (if beautiful) and the more you read, the harder they are thrown at you.  \\ \\
Of the two basic symmetry properties, the first looks self-evident since $q^{n(z+1)} = q^{n}$ but the second one is Poisson summation:
$$ \sum_{n \in \mathbb{Z}} e^{2\pi i n^2 z} = \frac{1}{\sqrt{2\pi i \, z}}\sum_{n \in \mathbb{Z}} e^{2\pi i n^2 /z} $$
Did I write the formula correctly?  The summation formla says for $f(x) \in \mathcal{S}(\mathbb{R})$ (in the ``Schwartz class") then
$$ \sum_{n \in \mathbb{Z}} f(z) = \sum_{n \in \mathbb{Z}} \widehat{f}(z) $$
so we need to do a careful evaluation of the Fourier transform. I have no idea what is $\mathcal{S}(\mathbb{R})$ except to write a function down and check it.  While there are other elements of Schwartz class, the main building block we have to far are functions weighted by Gaussians $p(x) e^{-x^2}$ and off we go.
\vfill

\begin{thebibliography}{}

\item PW Atkins \textbf{Physical Chemistry}

\end{thebibliography}

\end{document}