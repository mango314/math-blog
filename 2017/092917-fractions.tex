\documentclass[12pt]{article}
%Gummi|065|=)
\usepackage{amsmath, amsfonts, amssymb}
\usepackage[margin=0.5in]{geometry}
\usepackage{xcolor}
\usepackage{graphicx}

%\usepackage{pifont}
\usepackage{amsmath}



\newcommand{\off}[1]{}
\DeclareMathSizes{20}{30}{20}{18}

\newcommand{\two }{\sqrt[3]{2}}
\newcommand{\four}{\sqrt[3]{4}}
\newcommand{\red}{\begin{tikz}[scale=0.25]
\draw[fill=red, color=red] (0,0)--(1,0)--(1,1)--(0,1)--cycle;\end{tikz}}
\newcommand{\blue}{\begin{tikz}[scale=0.25]
\draw[fill=blue, color=blue] (0,0)--(1,0)--(1,1)--(0,1)--cycle;\end{tikz}}
\newcommand{\green}{\begin{tikz}[scale=0.25]
\draw[fill=green, color=green] (0,0)--(1,0)--(1,1)--(0,1)--cycle;\end{tikz}}

\newcommand{\sq}[3]{\draw[#3] (#1,#2)--(#1+1,#2)--(#1+1,#2+1)--(#1,#2+1)--cycle;}

\usepackage{tikz}
\usetikzlibrary{decorations.markings}

\newcommand{\susy}{{\bf Q}}
\newcommand{\RV}{{\text{R}_\text{V}}}

\title{Scratchwork: Farey Fractions}
\date{}
\begin{document}

%\fontfamily{qag}\selectfont \fontsize{12.5}{15}\selectfont

\sffamily

\maketitle

\noindent Here's a nice question about Farey Fractions:

\newpage

\includegraphics[width=6in]{fractions-01.png} 

\newpage

\noindent What's good or bad about this question?  These ``elementary" questions tend to be the most applicable.  If you think of a ``number" you're probably thinking of $\mathbb{Z}$.  However, if we're more empirical, that object behaves like a number with a few common-sense exceptions.  Well, we have exited the realm of $\mathbb{Z}$ - it is some other object.  If we continue into the pristine world of number theory where everything is known to infinite accuracy, the \textbf{visible points} of $\mathbb{Z}^2$ might be part of a family of  sets of points for each number field $K/\mathbb{Q}$, or maybe there is variant of Euler $\phi$ function associated to modular form.  \\ \\
If we remain in $\mathbb{Z}$ we are asking for a push towards the Riemann Hypothesis.  There's no rush.  However, in the vaguely-titled \textbf{On the error term of a lattice counting problem} they considere the Farey Fractions:
$$ \mathcal{F}(T) = \{ \frac{a}{b}: (a,b) \in \mathbb{Z}^2 , 0 \leq a \leq b \leq T , \mathrm{gcd}(a,b) = 1 \} $$
The subset of the Farey Fractions he chooses to measure is rather specfic.  Less than $\frac{1}{2}$
$$ \mathcal{I}(T) = \mathcal{F}(T) \cap [0, \frac{1}{2}) $$
For each Farey Fraction, we define a subset rather close to $1$:
$$ \mathcal{C}_{a,b} (T) = \mathcal{F}(T) \cap [1 - a^2/b^2, 1] $$
and we define some kind of counting measure as the sum over all these fractions:
$$ C(T) = \sum_{a/b \in \mathcal{I}(T)} \# \mathcal{C}_{a,b}(T) $$
He tells you an interpretation of these fractions: {\color{purple!50!green}s the number of similarity classes of semi-stable arithmetic planar lattices of height
at most $T$}.  And there's a lot of number theory based on that, using dynamial systems.  What was his result?
$$ C(T) = \frac{3}{8 \pi^4 }T^4 + O(T^3 \, \log T) $$
I've always wanted to ``interpret" these error terms.  At least the constant, $3/8\pi^4$ I feel I understand better.  Maybe not even that.  They improve it to:
$$ C(T) = \frac{3}{8 \pi^4 }\, T^4 + O\big(T^3 \, (\log T)^{2/3} \,(\log \log T)^{4/3} \big) $$ 
and they proceed to do whatever transformatins they are going to do.  \\ \\
What is a fraction?   Is it a proportion? Is it the direction of a ray in space?  If the fraction is 63\% can we get a way with saying ``two-thirds"?  Etc.  These fractions are generated by some kind of \textbf{process} and modeling that process could lean to an argment that feels more concrete. Have we pushed towards the deeper issues?\\
\includegraphics[width=6in]{fractions-02.png} \\
From 1938, showing Markov's theorem that $\alpha \notin \mathbb{Q}$ implies that $|\alpha - p/q| < 1/\sqrt{5}q^2 $ has infinitely many solutions.  There just happens to be enough ``room" in configuration space.

\vfill

\begin{thebibliography}{}

\item MathOverflow \textbf{Error to sum of Euler phi-functions} \texttt{https://mathoverflow.net/q/95836/1358}

\item Noam D. Elkies and Curtis T. McMullen \textbf{Gaps in $\sqrt{n}$ mod 1 and Ergodic Theory} \\ Duke Math. J. Volume 123, Number 1 (2004), 95-139.

\item Lester Ford \textbf{Fractions} American Mathematical Monthly. Vol. 45, No. 9 (Nov., 1938), pp. 586-601. 

\item Olivier Bordell\`{e}s, Florian Luca, Igor E. Shparlinski \textbf{On the error term of a lattice counting problem} Journal of Number Theory Volume 182, January 2018, Pages 19-36
\end{thebibliography}



\end{document}