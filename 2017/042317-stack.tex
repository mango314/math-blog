\documentclass[12pt]{article}
%Gummi|065|=)
\usepackage{amsmath, amsfonts, amssymb}
\usepackage[margin=0.5in]{geometry}
\usepackage{xcolor}
\usepackage{graphicx}

\newcommand{\off}[1]{}
\DeclareMathSizes{20}{30}{20}{18}

\newcommand{\two }{\sqrt[3]{2}}
\newcommand{\four}{\sqrt[3]{4}}


\usepackage{tikz}

\title{Item: the $\textbf{Bun}_G$ \textbf{Stack}}
\author{John D Mangual}
\date{}
\begin{document}

\fontfamily{qag}\selectfont \fontsize{12.5}{15}\selectfont

\maketitle

\noindent I don't even know where to begin.  There's a discussion of stacks and they talk about $\mathrm{Bun}(G)$. \\ \\  I don't know what it is, or what it's elements are or why it is important.  Google and wikipedia don't really help since they pre-suppose lots of algebraic geometry and category theory.  \\ \\
One resource\footnote{https://web.stanford.edu/~ebwarner/uniformizationofBunG.pdf} says $\mathrm{Bun}(G)$ is the **moduli stack of $G$-bundles** where $G$ is an affine algebraic group over a field $k$. 
\begin{itemize}
\item the embedding $G \to GL_n$ induces a morphism of stacks $\mathrm{Bun}_G \to \mathrm{Bun}_{{GL}_n}$
\item $\mathrm{Bun}_G$ is depends on the space $X$, $\mathrm{Bun}_G(X)$ is a groupoid
\item $\mathrm{Bun}_G$ is a functor, meaning that it is well behaved under maps of spaces.  A map $Y \to X$ yields another "induced" map:
$$ \mathrm{Bun}_G(X) \to \mathrm{Bun}_G(Y) $$
\end{itemize}
I'm concluding it's simply not time yet.

\end{document}