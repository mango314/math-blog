\documentclass[12pt]{article}
%Gummi|065|=)
\usepackage{amsmath, amsfonts, amssymb}
\usepackage[margin=0.5in]{geometry}
\usepackage{xcolor}
\usepackage{graphicx}

\newcommand{\off}[1]{}
\DeclareMathSizes{20}{30}{20}{18}

\newcommand{\two }{\sqrt[3]{2}}
\newcommand{\four}{\sqrt[3]{4}}
\newcommand{\red}{\begin{tikz}[scale=0.25]
\draw[fill=red, color=red] (0,0)--(1,0)--(1,1)--(0,1)--cycle;\end{tikz}}
\newcommand{\blue}{\begin{tikz}[scale=0.25]
\draw[fill=blue, color=blue] (0,0)--(1,0)--(1,1)--(0,1)--cycle;\end{tikz}}
\newcommand{\green}{\begin{tikz}[scale=0.25]
\draw[fill=green, color=green] (0,0)--(1,0)--(1,1)--(0,1)--cycle;\end{tikz}}

\usepackage{tikz}

\newcommand{\susy}{{\bf Q}}
\newcommand{\RV}{{\text{R}_\text{V}}}

\title{Fibonacci Numbers}
\author{John D Mangual}
\date{}
\begin{document}

\fontfamily{qag}\selectfont \fontsize{12.5}{15}\selectfont

\maketitle

\noindent I've been trying to figure out why in the middle of a rather serious paper, Curtis McMullen writes about the Fibonacci numbers.  $F_0 = 0, F_1 = 1$:
$$ F_n = \mathrm{tr}_\mathbb{Q}^K \big[ \epsilon^m / \sqrt{D} \big]  \text{ so that } F_m \asymp \epsilon^m $$ 
Or we can write in the usual, recursive bunny-rabit format:
$$ F_{m+1} = t F_m - n F_{m-1} $$
here $t$ and $n$ are the \textbf{trace} and the \textbf{norm} of the number field unit $\epsilon$.  The number field is just adjoining the $\sqrt{D}$ to the rational numbers:
$$ K = \mathbb{Q}(\sqrt{D})  $$
Then $\epsilon$ is a unit in this quadratic field and it satisfies a quadratic equation:
$$ \epsilon^2 - t \, \epsilon + n = 0 \text{ with } t = \mathrm{tr}_\mathbb{Q}^K (\epsilon) \text{ and } n = \mathrm{N}_\mathbb{Q}^K (\epsilon)  = \pm 1 $$
McMullen does something (standard) but a bit freaky, lifing the field $\mathbb{Q}(\sqrt{D})$ off the number line into two dimensions:
$$ \mathbb{Z}[\epsilon] = \mathbb{Z} \oplus \mathbb{Z}\, \epsilon = (1, \epsilon) $$
The actions of $\epsilon$ and $\sqrt{D}$ get promoted to $2 \times 2$ matrices:
$$ \epsilon \sim U = \left( \begin{array}{cr} 0 & -n \\ 1 & t \end{array}  \right)
\text{ and } \sqrt{D} \sim S = 2U - tI =  \left( \begin{array}{cr} -t & -n \\ 2 & t \end{array}  \right) $$
The Fibonacci identity lifts to one for $2 \times 2$ matrices:
$$ U^m = f_m \, U - n \, f_{m-1} \, I  $$


\newpage

\noindent Curtis McMullen's objectives were to prove three results: \\ \\
\textbf{\#1} Any real quadratic field $\mathbb{Q}(\sqrt{d})$ contains infinitely many periodic continued fractions $x = \overline{[a_0, \dots, a_{p-1}]}$ with $1 \leq a_i \leq M_d$. \\ \\
\textbf{\#2} For any fundamental geodesic $\gamma \subset \mathbb{H}/\mathrm{SL}(2, \mathbb{Z})$ there is a compact subset of $M$ that contains infinitely many primitive, closed geodesics whose lengths are multiples of $L(\gamma)$.  ( A geodesic $\gamma$ is primitive if it's indivisible in $\pi_1(M)$. \\ \\
The meaning of the word ``compact" here in this context exceeds my geometric intuition:
\begin{itemize}
\item Complete geodesics lying in $Z$ form a closed set $G(Z)$ of measure zero.
\item Geodesics of length $m \, L(\gamma)$ become uniformly distributed in $\mathbb{H}/\mathrm{SL}(2, \mathbb{Z})$ as $m \to \infty$.
\item Most geodesics whose lengths are multiples $m \, L(\gamma)$ are not contained in $Z$, but infinitely many of them are.
\item Hausdorff dimension of $G(Z)$ can be made arbitrarily close to $2$ if $Z$ is large enough.
\end{itemize}
\textbf{\#3} In any real quadratic field $K$, there are infinitey many ideal classes with $\delta(I) > \delta_K > 0$.  
\begin{itemize}
\item $\delta(I) = \frac{N^\ast(I)}{\det (I) }$ ``packing density"
\item $\det (I) = \sqrt{\mathrm{disc}(I) }$ 
\item $N^\ast(I) = \min \big\{ | N^K_\mathbb{Q}(x) |: x \in I \text{ and } N^K_\mathbb{Q}(x) \neq 0 \big\}  $ \\
\end{itemize}
\textbf{\#2a} ({ \color{black!30!white} extension of result \#2 for Bianchi groups}) \\ For any fundamental geodesic in the hyperbolic orbifold $\mathbb{H}/ \mathrm{SL}_2(\mathcal{O}_d)$, there is a compact set that contains infinitely many closed primitive geodesics whose lengths are multiples of $L(\gamma)$. 

\newpage

\noindent McMullen's main construction: \\ \\
Start with matrix $A \in \mathrm{GL}_2(\mathbb{Z})$ such that
\begin{itemize}
\item $A^2 = I$
\item $\mathrm{tr}(A) = 0$
\item $\mathrm{tr}(A^\dagger U) = \pm 1$
\end{itemize}
and let $L_m = U^m + U^{-m} A $.  Then for all $m \geq 0$:
\begin{itemize}
\item $|\det (L_m)| = f_{2m}$ is a Fibonacci number
\item The lattice $[L_m]$ is fixed by $U^{2m}$ 
\item $L_{-m}= L_m \, A$ 
\item $\big|\big| U^i L_m U^{-i}\big|\big| , \big|\big| U^i L_m U^{-i}\big|\big|   \leq C \sqrt{|\det{L_m}|} $ 
\end{itemize}

\newpage 

\noindent This discussion has not used any modular forms, even though Duke's equidistribution theorem was proven using the elusive \textbf{Maass forms}.  Not sure what gives. \\ \\
Likely: many open questions about the geometry of the hyperbolic surfaces and 3-manifolds.

\vfill

\begin{thebibliography}{}

\item Curtis McMullen \\ \textbf{Uniformly Diophantine numbers in a
Fixed Real Quadratic Field} \texttt{http://www.math.harvard.edu/\~{}ctm/papers/home/text/papers/cf/cf.pdf} \\
\textbf{Horocycles in Hyperbolic 3-manifolds} \\ \texttt{http://www.math.harvard.edu/\~{}ctm/papers/home/text/papers/horo/horo.pdf}

\item Alex Brandts, Tali Pinsky, Lior Silberman \\ \textbf{Volumes of hyperbolic three-manifolds associated to modular links} \texttt{arXiv:1705.04760}

\end{thebibliography}
These references are more advanced, but also they are using the theory of modular forms, which I have deliberately left out of the previous discussion.  Wouldn't it be nice to find mmore elementary stuff to go here?
\begin{thebibliography}{}

\item Paul D. Nelson
\textbf{Quantum variance on quaternion algebras, I} \;\texttt{arXiv:1601.02526}

\item Paul D. Nelson
 \textbf{Quantum variance on quaternion algebras, II} \texttt{arXiv:1702.02669}

\end{thebibliography}

\newpage

\noindent \textbf{06/09} Following this digression a little bit, Paul Nelson has been able to explain to me how to ``unpack" the very difficult and abstract definitions in the number theory literature. \\ \\
For starters, a dilemma. I keep reading these 4 subjects are related.  There's a generali idea, but it's really up to \textbf{you} to connect them.  \\ \\
\begin{tikzpicture}
\node[text width=5cm] at ( 0, 0) {number theory};
\node[text width=5cm] at (10, 0) {modular forms};
\node[text width=5cm] at ( 0, 3) {ergodic theory};
\node[text width=5cm] at (10, 3) {harmonic analysis};

\draw (1,0)--(7,0);
\draw (1,3)--(7,3);
\draw ( 0,0.5)--( 0,2.5);
\draw (10,0.5)--(10,2.5);

\node at (4,-0.5) {?};
\node at (4,3.5) {?};
\node at (-0.5,1.5) {?};
\node at (10.5,1.5) {?};

\end{tikzpicture} \\ \\ 
Our favorite example, the equidistribution of the sum-of-three-squares problem quickly becomes very very difficult, one of the hardest in the literature:
$$ n = a^2 + b^2 + c^2 \text{ if and only if } n \neq 4^m (8n+7) $$
The first step was to turn into a question about modular forms.  It's hard enough to prove the existence of one solution, this is called \textbf{strong approximation}.  
\begin{itemize}
\item \textbf{A} $2 = 1^2 + 1^2 + 0^2$
\item \textbf{B} $5 = 2^2 + 1^2 + 0^2$
\item \textbf{C} $50 = 5^2 + 5^2 + 0^2$
\item \textbf{D} $100 = 10^2 + 0^2 + 0^2$
\item \textbf{E} $7 \neq \square + \square + \square $ but $49 = 7^2 + 0^2 + 0^2$
\end{itemize}
I would like to be able to ``multiply" solutions like this $\textbf{A} \times \textbf{B} = \textbf{C}$.  We can do this with 2 squares, or 4 squares. \\ \\
We need some way to argue that certain solutions are ``trivial".  And observe our equations $x^2 + y^2 + z^2$ are all of second degree, so perhaps some ingredients could be:
\begin{itemize}
\item second cohomology $H^2$
\item the rotation group $SO(3, \mathbb{Z})$
\item a scheme\footnote{not just a variety} $\{ x^2 + y^2 + z^2 : x,y,z \in \mathbb{Z} \}$
\end{itemize}
Whatever idea we have, we're going to carry it out.  Equidistribution means certain averages must tend to zero:
$$ \frac{1}{r_3(n)} \sum_{x^2 + y^2 + z^2 = n } \phi( \tfrac{x}{\sqrt{n}},\tfrac{y}{\sqrt{n}},\tfrac{z}{\sqrt{n}}) \stackrel{?}{\to} 0$$
The proofs we have right now, attach all of these aveages into a single modular form:
$$ \theta( z ; \phi) :=  \left[ \frac{1}{r_3(n)} \sum_{x^2 + y^2 + z^2 = n } \phi( \tfrac{x}{\sqrt{n}},\tfrac{y}{\sqrt{n}},\tfrac{z}{\sqrt{n}}) \right] e^{2\pi i nz}  $$
Self-evident to the expert, I am quite susprised this is compared a theta function:
$$ \theta(z) = \sum q^{n^2} = \sum e^{2\pi i \; n^2z} $$
and these have nice transformation properties.  It is a $\Gamma_0(4)$ modular form. \\ \\
I think there is lots of things wrong. We really compare apples and oranges here:
\begin{itemize}
\item the integer (``Diophantine") equation $n = a^2 + b^2 + c^2$
\item the hyperbolic space $\mathbb{H}/\Gamma_0(4)$
\item the space of modular forms $\theta(z, \phi)$
\end{itemize}
why do we need to use modular forms, can't we study the points on the sphere directly? \\ \\
The next step is another doozy, we're not going to study these modular forms too directly.  We merely observe, they call can be expanded in terms of Poincare series:
$$ P_m(z; k, \Gamma) = \sum_{\tau \in \Gamma_\infty \backslash \Gamma } j(\tau, z)^{-2k} e(m \tau z) = $$
We can find problem reduces to finding averages of Kloosterman sums:
$$ \hat{P}_m (n; k , \Gamma) = \left( \frac{n}{m} \right)^{\frac{k-1}{2}} 
\left\{ \delta_{mn} + 2\pi i^{-k} \sum_{c \equiv 0 \pmod n }^\infty 
c^{-1} J_{k-1}\left( \frac{4\pi \sqrt{mn}}{c} \right)  K(m,n;c) \right\} $$
and we spend the rest of the paper estimating $K$.  I think \textbf{we lose sight of the original goal}

\begin{tikzpicture}
\node[text width=5cm] at ( 0, 0) {$n = a^2 + b^2 + c^2$};
\node[text width=5cm] at (10, 0) {estimate $P_m$};
\node[text width=5cm] at ( 0, 3) {?? not mentioned };
\node[text width=5cm] at (10, 3) {estimate $K(m,n;c)$};

\draw[<-] (1,0)--(7,0);
\draw (2,3)--(7,3);
\draw ( 0,0.5)--( 0,2.5);
\draw (10,0.5)--(10,2.5);

\node at (4,-0.5) {?};
\node at (4,3.5) {?};
\node at (-0.5,1.5) {?};
\node at (10.5,1.5) {?};

\end{tikzpicture} \\ \\
There are lots and lots of modular forms.  Mathematicians know they exist, but we can't find write down the numbers.  Or if we find a sequence of numbers we care about, the modular form expansion is rather murky. \\ \\
The guy never said anything to me.  Yes, he did tell me to look around there a long time ago, and maybe now I'm feeling these solutions leave someting to be desired.  \\ \\So\dots I'm going to put my had in there? Most importantly, the dynamical systems part is currently \text{empty} and we need to fill something in there\dots

\newpage  

\begin{thebibliography}{}

\item Paul D. Nelson \textbf{The spectral decomposition of $|\theta|^2$ } \;\texttt{arXiv:1601.02526}

\end{thebibliography}

\end{document}