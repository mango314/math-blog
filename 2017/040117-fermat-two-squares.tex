\documentclass[12pt]{article}
%Gummi|065|=)
\usepackage{amsmath, amsfonts, amssymb}
\usepackage[margin=0.5in]{geometry}
\usepackage{xcolor}
\usepackage{graphicx}
%\usepackage{graphicx}
\newcommand{\off}[1]{}
\DeclareMathSizes{20}{30}{20}{18}
\newcommand{\myhrule}{}

\newcommand{\two }{\sqrt[3]{2}}
\newcommand{\four}{\sqrt[3]{4}}

\newcommand{\dash}{
\begin{tikzpicture}[scale=1]
\draw (0,0)--(19,0);
\end{tikzpicture}
}

\newcommand{\sq}[3]{
\node at (#1+0.5,#2+0.5) {#3};
\draw (#1+0,#2+0)--(#1+1,#2+0)--(#1+1,#2+1)--(#1+0,#2+1)--cycle;
}

\usepackage{tikz}

\title{\textbf{Proposal: Sum of Two Squares}}
\author{John D Mangual}
\date{}
\begin{document}

\fontfamily{qag}\selectfont \fontsize{20}{25}\selectfont

\maketitle

\noindent The beginning of the \'{E}tale cohomology book says we did it whenver we used \textbf{height functions} and \textbf{descent} to prove Fermat's Theorem:
$$ p = 4k+1 \leftrightarrow p = a^2 + b^2 $$ 
It's not fair.  You can't sell me on that and then I never see so much as an equation again.  \\ \\
There are several arguments -- all of which I enjoy. Here's one:\\ \\
\textbf{\#1} The equation $a^2 + b^2 \equiv 0 \text{ mod }p$ has two solutions:
\begin{itemize} 
\item $(a,b) = (p,0)$
\item $(a,b) = (x,1)$ with $x \equiv \sqrt{-1} \text{ mod } p$
\end{itemize}
The two solutions generate a lattice - some element of $SL(2, \mathbb{Z}) \backslash SL(2, \mathbb{R})$ - I can't say much more, just a random element. The area of the rhombus is $A = p$.\\ \\
Next we intersect with a circle. $\{ x^2 + y^2 = 2p\}\subset \mathbb{R}^2 $ with area $A = \pi p$.

\newpage

\noindent This argument solved the equation $x^2 + y^2 = p$ in two different venues:
\begin{itemize}
\item finite fields $\mathbb{F}_p \times \mathbb{F}_p$ with $\mathbb{F}_p= \mathbb{Z}/p\mathbb{Z}$
\item the Euclidean plane $\mathbb{R} \times \mathbb{R}$
\end{itemize}
What geometric object merges them both? \\ \\
No matter which proof you choose, these arguments all seem to come from nowhere.\footnote{While this problem is almost totally useless, the argument style.  Real problems are difficult, have no symmetry}  I have found 3 types:
\begin{itemize}
\item descent
\item heights
\item geometry of numbers
\item factorials
\item Viete Jumping
\end{itemize}
and all of these will fall out of Etale Cohomology.  \\ \\
Somehow, when we write all these equations we are already doing this wonderful thing. \\ \\ Now we read the book - which has virtually no equations whatsoever - hoping we can recover Fermat.

\newpage



\fontfamily{qag}\selectfont \fontsize{12}{10}\selectfont

\begin{thebibliography}{}

\item Christoph Soul\'{e} \textbf{Lectures on Arakelov Geometry} Cambridge Studies in Advanced Mathematics (Book 33) CUP, 1995.

\end{thebibliography}

\end{document}