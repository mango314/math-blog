\documentclass[12pt]{article}
%Gummi|065|=)
\usepackage{amsmath, amsfonts, amssymb}
\usepackage[margin=0.5in]{geometry}
\usepackage{xcolor}
\usepackage{graphicx}

%\usepackage{pifont}
\usepackage{amsmath}



\newcommand{\off}[1]{}
\DeclareMathSizes{20}{30}{20}{18}

\newcommand{\two }{\sqrt[3]{2}}
\newcommand{\four}{\sqrt[3]{4}}
\newcommand{\red}{\begin{tikz}[scale=0.25]
\draw[fill=red, color=red] (0,0)--(1,0)--(1,1)--(0,1)--cycle;\end{tikz}}
\newcommand{\blue}{\begin{tikz}[scale=0.25]
\draw[fill=blue, color=blue] (0,0)--(1,0)--(1,1)--(0,1)--cycle;\end{tikz}}
\newcommand{\green}{\begin{tikz}[scale=0.25]
\draw[fill=green, color=green] (0,0)--(1,0)--(1,1)--(0,1)--cycle;\end{tikz}}

\newcommand{\sq}[3]{\draw[#3] (#1,#2)--(#1+1,#2)--(#1+1,#2+1)--(#1,#2+1)--cycle;}

\usepackage{tikz}
\usetikzlibrary{decorations.markings}

\newcommand{\susy}{{\bf Q}}
\newcommand{\RV}{{\text{R}_\text{V}}}

\title{Scratchwork: Prime Number Theorem}
\date{}
\begin{document}

%\fontfamily{qag}\selectfont \fontsize{12.5}{15}\selectfont

\sffamily

\maketitle

\noindent Let's go straight for the Analytic Theorem. 
\begin{quotation}Let $f(t)$ with $t \geq 0$ be a bounded and locally integrable function and suppose that the function
$$ g(z) = \int_0^\infty f(t) \, e^{-zt} \, dt \text{ with }\mathrm{Re}(z) > 0 $$
extends holomorphically to $\mathbb{Re}(z) \geq 0$.  Then $\int_0^\infty f(t)\, dt$ and it equals $g(0)$.
\end{quotation}
On the one hand the language is standard, but the wording might be of some concern.  
\begin{itemize}
\item $f: \mathbb{R} \to \mathbb{C}$ is defined only on the real axis $\mathbb{R}$ and it is bounded $ |f(t)| < M$ always.
\item $g(z) = \int_0^\infty f(t) e^{-zt} dt$ is the \textbf{Laplace transform} extended to complex arguments $z \in \mathbb{C}$. 
\item ``Locally integrable" is what we mean by ``integrable" when we forget the domain of definition is $\mathbb{R}$ or $\mathbb{R}_{\geq 0}$;  Our example of a \textbf{locally integrable} function that is not \textbf{integrable}  is:
$$ f(x) = 1 \text{ for all }x \in \mathbb{R} $$
In fact, $\int_a^b f(x) = b - a$ however $\int_\mathbb{R} f(x)= \infty$.  Another excellent example is:
$$ f(x) = \left\{
\begin{array}{cc}
\frac{1}{x} & x \neq 0 \\ \\
0 & x = 0 \end{array}
 \right. $$
This funciton is not locally integrable at $x = 0$, no neighborhood has of $x=0$ has a integral $\int_{[-\epsilon, \epsilon]} f(x) \notin \mathbb{R} $ does not even converge to a number. \\ \\ Could $\int_{[-\epsilon, \epsilon]} f(x) $ be something more exotic than a number?  It is a \textbf{distribution} and in fact there is the \textbf{Cauchy Principal Value}. 
\item The theorem is designed with a specific choice of $f(t)$ in mind, it is:
$$ f(t) = \theta(e^t) e^{-t} - 1 = \frac{1}{e^{t}} \bigg[\, \sum_{p \leq e^{t}} \log p \, \bigg] -1 $$
and then the Laplace transform is the zeta function (rather Dirichlet series) that we know stuff about:
$$ g(z) =  \frac{1}{z+1}\,\Phi(z+1) - \frac{1}{z} = 
\frac{1}{z+1} \sum_p \frac{\log p}{p^s} - \frac{1}{z} $$
so the question is whether we can extract information about $f(t)$ from $g(z)$:
$$ \left[ \sum_p \frac{\log p}{p^s} \right] \to \left[ \sum_{p \leq x} \log p \right]$$


\end{itemize}

\newpage

\noindent Hopefully that explains a bit what Zagier Theorem (or Newman's Theorem) was designed to do and why where was a need in the first place.  There are these two measures of ``size" and we'd like to explain why one is related to the other. \\ \\
What could go wrong with the limit:
$$ \left[ g(z) = \int_0^\infty f(t) \, e^{-zt} \, dt\right] \stackrel{?}{\to} \left[ g(0) = \int_0^\infty f(t) \, dt \right] $$
\textbf{What goes wrong with plugging in $z = 0$?}   There was a great textbook called ``Counterexamples in Analysis" and every sentence in this discussion requires about a dozen counterexamples.  At least I need them, this counterfactual reasoning is very very difficult to me.\footnote{Analysis would become my best subject because I knew I had to allocate the most time.} When I did analysis proofs, I would write a ``wishful thinking" proof or a ``bad" argument, and then gradually turn all the wrong steps into correct steps by accounting for all the exceptions that could occur. \\ \\
\textbf{Bad Proof \#1} Laplace transforms are always \textit{very} well-behaved.  
$$ g(z) = \int_0^\infty f(t) \, e^{-zt} \, dt \;\; \tikz { \draw[thick,->] (0,0)--(0.75,0);} \;\; g(0) = \int_0^\infty f(t) \, dt $$
This is what we just said.  By {\color{blue!50!orange} continuity }, this limit should hold, for all $f: \mathbb{R} \to \mathbb{C}$. \\ \\
\textbf{CounterExample \#1} $z = 0$ seems just slightly outside of the domain of $\{ \text{Re}(z) > 0\}$.  How could it matter? \\
Let's build a counterexamples: two functions: $g_1, g_2 : \Omega \subseteq \mathbb{C} \to \mathbb{C}$ such that:
\begin{itemize}
\item $g_1(z)$ and $g_2(z)$ are holomorhpic on $\{ \text{Re}(z) > 0 \}$
\item $g_1(z) = g_2(z)$ agreeing on $\{ \text{Re}(z) > 0 \}$
\item $g_1(0) \neq g_2(0)$
\end{itemize} 
Amazingly this counterexample does not exist. If two holomorphic functions agree on any open, however small, they are identical everywhere.  \\
\textbf{Example \#2} The log function can be extended to complex numbers and has a definition as Laplce transform:
$$ \Gamma(z+1) = \int_0^\infty e^{-w} w^z \, dw  $$
This has a pole at $z = -1$ since $\log 0 = -\infty$.  Then $\log 1 = 0$ but we have:
$$ \log e^{i\theta} = i\theta \text{ for }\theta \in \mathbb{R} \text{ so that } \log 1 = 0, 2\pi i, 4\pi i, \text{etc.}$$
\textbf{Example \#3} The zeta function is an infinite series, but it can also be analytic continued:
\begin{eqnarray*}
\zeta(z) &=& \frac{1}{1^z} + \frac{1}{2^z} + \dots \\ \\
\zeta(z) &=& \frac{1}{\Gamma(z)} \int_0^\infty \frac{w^z}{e^w - 1} \, \frac{dw}{w}
\end{eqnarray*}
This formula is only true for $\mathrm{Re}(z) > 1$ for example:
$$ \zeta(\tfrac{1}{2}) \neq 1 + \frac{1}{\sqrt{2}} + 
+ \frac{1}{\sqrt{3}} + \dots  = \infty $$
It's very easy to take such a thing for granted we will take the series for $\zeta(\frac{1}{2}) \in \mathbb{R}$ and assign it a number.

\newpage 

\noindent \textbf{Special Case \#3} Example \#3 is important because of our choice of $g$:
\begin{eqnarray*} g(z) &=& \frac{1}{z} \sum_p \frac{\log p}{p^z} - \frac{1}{z-1} 
 \;\;\;\Bigg|_{z=1} = \sum_p \frac{\log p}{p} - \big[ \,\infty \,\big] \\ \\
&\stackrel{pnt}{=}& \int_0^\infty \left[ \frac{1}{e^t} \sum_{p < e^t} \log p - 1 \right] \, dt \\ \\
&=& \int_0^\infty \left[\;\;\;\;\,  \sum_{p < x} \log p - 1 \right] \, dx = \int_0^\infty f(t) \, dt\end{eqnarray*}
A big part of PNT was to ``regularize" a modest $[\infty] - [ \infty]$ diverent infinite sum, and then relate it to an average over all $p < x$ as $x \to \infty$. \\ \\
\textbf{Example \#2} If $z \in \mathbb{R}_{\geq 0}$ a real positive number, then freshman calculus integral holds like this:
$$ \int_0^\infty e^{-zt} \, dt = \frac{1}{z} $$
This integral can be extended to all of $z \in \mathbb{C}$ or at least $z \in \{  \mathrm{Re}(z) \geq 0\}$.  Then
$$ F(z) = \int_0^\infty e^{-zt} \, dt - \frac{1}{z}  \stackrel{?}{=} 0 $$
and this function is regular for $\mathrm{Re(z)} > 0$ and $F(z) = 0$ for all $z \in \mathbb{R}$.   We can't quite say ``=" zero, but ``equal, under appropriate condition" which could mean anything, really. \\ \\
These analytic continuation discussions, trying to get formulas to take values outside of their original domain, lead to the concept of a \textbf{sheaf} of holomorphic functions.  We get the notions like \textbf{Riemann surfaces} and \textbf{germs} and the branch cover or the \textbf{universal cover}.  \\ \\
Newmna's proof is slightly oblique, and it's not quantitative, but it's short and it does lead to the Prime Number Theorem (with no rate of convergence).  To me, the big challenge is relate information abou the zeta function $\zeta(s)$ to information about the primes $\sum_{p < x} \log p$ and to continue we need to read past the ``proof" sign. \\ \\
\textbf{Bad Proof \#1a} Clearly $g(z)$ is holomorphic.  Let's write out the Laplace transform:
$$ g(z) = \int_0^\infty f(t) e^{-zt} \, dt $$
If $f(z)$ is bounded and holomorphic, then $g(z)$ is holomorphic.  By continuity, $g(0) = \lim_{z \to 0} g(z)$. 
$$ \lim_{z \to 0} g(z) = \lim_{z \to 0} \int_0^\infty f(t) e^{-zt} \, dt = \int_0^\infty f(t) \, dt $$
The first equality is trivially true since $z \neq 0$ (or $\mathrm{Re}(z) \geq 0$) and the second integral is clearly true.\footnote{since it's also holomorphic} \\ \\
\textbf{Counterexample \#2a} Our counterexample is $\boxed{f(z) = 1}$ If $z \in \mathbb{R}_{\geq 0}$ a real positive number, then freshman calculus integral holds like this:
$$ \int_0^\infty \mathbf{1} \, e^{-zt} \, dt = \frac{1}{z} $$
This integral can be extended to all of $z \in \mathbb{C}$ or at least $z \in \{  \mathrm{Re}(z) \geq 0\}$.  Then
$$ F(z) = \int_0^\infty \mathbf{1} \, e^{-zt} \, dt - \frac{1}{z}  \stackrel{?}{=} 0 $$
and this function is regular for $\mathrm{Re(z)} > 0$ and $F(z) = 0$ for all $z \in \mathbb{R}$. I'm being slightly awful. \\ \\
\textbf{Bad Proof \#2} If we integrate over a finite domain, that Laplace transform will be holomorphic as well:
$$ g_T(z) = \int_0^T f(t) e^{-zt} \, dt  $$
We could say ``compact support" and integrate over all $t \geq 0$ and later it may be useful to write it out in this formal way. \\ \\
If $f(t)$ is bounded and holomorphic, $g_T(z)$ is holomorphic as well.  And finally:
$$ \lim_{z \to 0} g_T(z) = g_T(0) $$
This is \textbf{Cauchy integral formula} or possibly Morera theorem.  It remains to prove the exceedingly obvious:
$$ \lim_{T \to \infty} g_T(0) = g(0) $$
It seems to be a modest change of limits.  If $f(t)$ is holomorphic, i.e. $f(z + dz) = f(z) + f'(z) \, dz$ and bounded $|f(z)| < M$ then we could have:
$$ \lim_{T \to \infty} \, \lim_{z \to 0} \int_0^T f(t) \, e^{-zt} \, dt
= \lim_{z \to 0} \, \lim_{T \to \infty} \int_0^T f(t) \, e^{-zt} \, dt $$ 
and this is obviously true. \\ \\
\textbf{Bad Proof \#3} The previous proof is problematic since we are trying to connect $g(0)$ to the limit of integrals 
\begin{itemize}
\item near $z = 0$ we could $g(z) \to g(0) $
\item near $z = 0$ we could say  $\int_0^\infty f(t)\, e^{-zt} \, dt \to \int_0^\infty f(t) \, dt $
\item away from $z \neq 0$ we could say $\int_0^\infty f(t)\, e^{-zt} \, dt = g(z) $
\end{itemize}
and we can't connect information \textit{away} from $z = 0$ to information \textit{near} $z=0$.  This is why we could use Cauchy integral formula:
$$ g(0) = \frac{1}{2\pi i } \oint_C g(z) \, \frac{dz}{z}$$
where $C = 0 + \epsilon S^1 = \big\{ \epsilon \, e^{2\pi i} \, \theta : 0 < \theta < 2\pi \big\}$ and we'll let $\epsilon = T$ and $T \to \infty$ and we could be done. \\ \\
( to be continued ... )
\vfill

\begin{thebibliography}{}

\item Don Zagier ``On Newman's Short Proof of the Prime Number Theorem" American Mathematical Monthly, Vol 104 No 8, Oct 1997.

\item Gelbaum, Olmstead. \textbf{Counterexamples in Analysis} Holden Day, 1965 / Dover , 2003.

\end{thebibliography}

\end{document}