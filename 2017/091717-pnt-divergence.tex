\documentclass[12pt]{article}
%Gummi|065|=)
\usepackage{amsmath, amsfonts, amssymb}
\usepackage[margin=0.5in]{geometry}
\usepackage{xcolor}
\usepackage{graphicx}

%\usepackage{pifont}
\usepackage{amsmath}



\newcommand{\off}[1]{}
\DeclareMathSizes{20}{30}{20}{18}

\newcommand{\two }{\sqrt[3]{2}}
\newcommand{\four}{\sqrt[3]{4}}
\newcommand{\red}{\begin{tikz}[scale=0.25]
\draw[fill=red, color=red] (0,0)--(1,0)--(1,1)--(0,1)--cycle;\end{tikz}}
\newcommand{\blue}{\begin{tikz}[scale=0.25]
\draw[fill=blue, color=blue] (0,0)--(1,0)--(1,1)--(0,1)--cycle;\end{tikz}}
\newcommand{\green}{\begin{tikz}[scale=0.25]
\draw[fill=green, color=green] (0,0)--(1,0)--(1,1)--(0,1)--cycle;\end{tikz}}

\newcommand{\sq}[3]{\draw[#3] (#1,#2)--(#1+1,#2)--(#1+1,#2+1)--(#1,#2+1)--cycle;}

\usepackage{tikz}
\usetikzlibrary{decorations.markings}

\newcommand{\susy}{{\bf Q}}
\newcommand{\RV}{{\text{R}_\text{V}}}

\title{Scratchwork: Prime Number Theorem}
\date{}
\begin{document}

%\fontfamily{qag}\selectfont \fontsize{12.5}{15}\selectfont

\sffamily

\maketitle

\noindent Let's go straight for the Analytic Theorem. 
\begin{quotation}Let $f(t)$ with $t \geq 0$ be a bounded and locally integrable function and suppose that the function
$$ g(z) = \int_0^\infty f(t) \, e^{-zt} \, dt \text{ with }\mathrm{Re}(z) > 0 $$
extends holomorphically to $\mathbb{Re}(z) \geq 0$.  Then $\int_0^\infty f(t)\, dt$ and it equals $g(0)$.
\end{quotation}
On the one hand the language is standard, but the wording might be of some concern.  
\begin{itemize}
\item $f: \mathbb{R} \to \mathbb{C}$ is defined only on the real axis $\mathbb{R}$ and it is bounded $ |f(t)| < M$ always.
\item $g(z) = \int_0^\infty f(t) e^{-zt} dt$ is the \textbf{Laplace transform} extended to complex arguments $z \in \mathbb{C}$. 
\item ``Locally integrable" is what we mean by ``integrable" when we forget the domain of definition is $\mathbb{R}$ or $\mathbb{R}_{\geq 0}$;  Our example of a \textbf{locally integrable} function that is not \textbf{integrable}  is:
$$ f(x) = 1 \text{ for all }x \in \mathbb{R} $$
In fact, $\int_a^b f(x) = b - a$ however $\int_\mathbb{R} f(x)= \infty$.  Another excellent example is:
$$ f(x) = \left\{
\begin{array}{cc}
\frac{1}{x} & x \neq 0 \\ \\
0 & x = 0 \end{array}
 \right. $$
This funciton is not locally integrable at $x = 0$, no neighborhood has of $x=0$ has a integral $\int_{[-\epsilon, \epsilon]} f(x) \notin \mathbb{R} $ does not even converge to a number. \\ \\ Could $\int_{[-\epsilon, \epsilon]} f(x) $ be something more exotic than a number?  It is a \textbf{distribution} and in fact there is the \textbf{Cauchy Principal Value}. 
\item The theorem is designed with a specific choice of $f(t)$ in mind, it is:
$$ f(t) = \theta(e^t) e^{-t} - 1 = \frac{1}{e^{t}} \bigg[\, \sum_{p \leq e^{t}} \log p \, \bigg] -1 $$
and then the Laplace transform is the zeta function (rather Dirichlet series) that we know stuff about:
$$ g(z) =  \frac{1}{z+1}\,\Phi(z+1) - \frac{1}{z} = 
\frac{1}{z+1} \sum_p \frac{\log p}{p^s} - \frac{1}{z} $$
so the question is whether we can extract information about $f(t)$ from $g(z)$:
$$ \left[ \sum_p \frac{\log p}{p^s} \right] \to \left[ \sum_{p \leq x} \log p \right]$$


\end{itemize}

\newpage

\noindent Hopefully that explains a bit what Zagier Theorem (or Newman's Theorem) was designed to do and why where was a need in the first place.  There are these two measures of ``size" and we'd like to explain why one is related to the other. \\ \\
What could go wrong with the limit:
$$ \left[ g(z) = \int_0^\infty f(t) \, e^{-zt} \, dt\right] \stackrel{?}{\to} \left[ g(0) = \int_0^\infty f(t) \, dt \right] $$
\textbf{What goes wrong with plugging in $z = 0$?}   There was a great textbook called ``Counterexamples in Analysis" and every sentence in this discussion requires about a dozen counterexamples.  At least I need them, this counterfactual reasoning is very very difficult to me.\footnote{Analysis would become my best subject because I knew I had to allocate the most time.} When I did analysis proofs, I would write a ``wishful thinking" proof or a ``bad" argument, and then gradually turn all the wrong steps into correct steps by accounting for all the exceptions that could occur. \\ \\
\textbf{Bad Proof \#1} Laplace transforms are always \textit{very} well-behaved.  
$$ g(z) = \int_0^\infty f(t) \, e^{-zt} \, dt \;\; \tikz { \draw[thick,->] (0,0)--(0.75,0);} \;\; g(0) = \int_0^\infty f(t) \, dt $$
This is what we just said.  By {\color{blue!50!orange} continuity }, this limit should hold, for all $f: \mathbb{R} \to \mathbb{C}$.
\vfill

\begin{thebibliography}{}

\item Don Zagier ``On Newman's Short Proof of the Prime Number Theorem" American Mathematical Monthly, Vol 104 No 8, Oct 1997.

\item Gelbaum, Olmstead. \textbf{Counterexamples in Analysis} Holden Day, 1965 / Dover , 2003.

\end{thebibliography}

\end{document}