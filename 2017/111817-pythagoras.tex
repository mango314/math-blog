\documentclass[12pt]{article}
%Gummi|065|=)
\usepackage{amsmath, amsfonts, amssymb}
\usepackage[margin=0.5in]{geometry}
\usepackage{xcolor}
\usepackage{graphicx}

%\usepackage{pifont}
\usepackage{amsmath}

\newcommand{\off}[1]{}
\DeclareMathSizes{20}{30}{20}{18}

\newcommand{\two }{\sqrt[3]{2}}
\newcommand{\four}{\sqrt[3]{4}}
\newcommand{\red}{\begin{tikz}[scale=0.25]
\draw[fill=red, color=red] (0,0)--(1,0)--(1,1)--(0,1)--cycle;\end{tikz}}
\newcommand{\blue}{\begin{tikz}[scale=0.25]
\draw[fill=blue, color=blue] (0,0)--(1,0)--(1,1)--(0,1)--cycle;\end{tikz}}
\newcommand{\green}{\begin{tikz}[scale=0.25]
\draw[fill=green, color=green] (0,0)--(1,0)--(1,1)--(0,1)--cycle;\end{tikz}}

\newcommand{\sq}[3]{\draw[#3] (#1,#2)--(#1+1,#2)--(#1+1,#2+1)--(#1,#2+1)--cycle;}

\usepackage{tikz}

\newcommand{\susy}{{\bf Q}}
\newcommand{\RV}{{\text{R}_\text{V}}}

\title{Scratchwork: Pythagorean Quadruples}
\date{}
\begin{document}

%\fontfamily{qag}\selectfont \fontsize{12.5}{15}\selectfont

\sffamily

\maketitle

\noindent If real-world data and computations are noisy, full of errors, and \textit{never} integers, why study Number Theory?  Is there a way to sift out $\mathbb{Z}$ from the chaos of the world? \\ \\
From ancient times, we know to solve Pythagoras equation in integers:
$$ \big[ x^2 + y^2 = z^2 \big] \to \big[ x = m^2 - n^2 , y = 2mn , z = m^2 + n^2 \big]  $$
and we could merge this into a single formula in $\mathbb{Z}[m,n]$ that ``works" for all $m,n \in \mathbb{Z}$:
$$  \big(m^2 - n^2\big)^2 + \big(2mn\big)^2 = \big(m^2 + n^2 \big)^2$$
The left and right sides are homogenous polynoials of 4-th degree.  And one could ask if we have two equations $f(\vec{x}) = g(\vec{x})$ that agree for all $\vec{x} \in \mathbb{Z}^2$ if $f \equiv g$ as elements of $\mathbb{Z}[\vec{x}]$.  I think the answer is maybe false. \\ \\
\textbf{1903} We know how to solve the Pythagoras equation in 4 variables:
$$ a^2 + b^2 + c^2 = d^2 $$
and a solution can be obtained \\ \\
\textbf{2013} It was hard to believe that if we replace the coefficient with a $2$, the question was not resolved until four years ago:
$$ a^2 + b^2 + 2\, c^2 = d^2  $$
How is that even possible?  All they do is count solutions $r_{1,1,2}(n^2)$, using modular forms of some kind.  Another approach could have been using the Hasse principle, or String Approximation. \\ \\
My question about the ``tree" of solution is about the orthogonal group:
$$ O \big( a^2 + b^2 + 2\, c^2 - d^2 , \mathbb{Z} \big)  $$
and I guess this type of problem is wide open.  The Pythagorean triples looks like this:
$$ O(2,1, \mathbb{Z}) = O \big( a^2 + b^2 - c^2  , \mathbb{Z} \big)  \simeq \Gamma(2)$$
And they start to talk about Lorenzian spaces, since right triangles in $\mathbb{R}^2$ are null vectors (elements of the light cone)  in $\mathbb{R}^{2,1}$.  Here we have the exact sequence:
$$ 1 \to \Gamma(2) \to \text{SL}(2, \mathbb{Z}) \to SL(2, \mathbb{Z}_2) \to 1$$
Perhaps I should say $\mathbb{Z}/2\mathbb{Z}$ instead of $\mathbb{Z}_2$.

\vfill

\begin{thebibliography}{}

\item Shaun Cooper, Heung Yeung Lam.  \textbf{On the diophantine Equation $\mathbf{n^2 = x^2 + b\, y^2 + c\, z^2}$} Journal of Number Theory 133 (2013) 719–737
 
\end{thebibliography}

\end{document}