\documentclass[12pt]{article}
%Gummi|065|=)
\usepackage{amsmath, amsfonts, amssymb}
\usepackage[margin=0.5in]{geometry}
\usepackage{xcolor}
\usepackage{graphicx}
%\usepackage{graphicx}
\newcommand{\off}[1]{}
\DeclareMathSizes{20}{30}{20}{18}
\newcommand{\myhrule}{}

\newcommand{\two }{\sqrt[3]{2}}
\newcommand{\four}{\sqrt[3]{4}}

\newcommand{\dash}{
\begin{tikzpicture}[scale=1]
\draw (0,0)--(19,0);
\end{tikzpicture}
}

\newcommand{\sq}[3]{
\node at (#1+0.5,#2+0.5) {#3};
\draw (#1+0,#2+0)--(#1+1,#2+0)--(#1+1,#2+1)--(#1+0,#2+1)--cycle;
}

\usepackage{tikz}

\title{\textbf{Item: Carries}}
\author{John D Mangual}
\date{}
\begin{document}

\fontfamily{qag}\selectfont \fontsize{12.5}{15}\selectfont

\maketitle

\noindent The theory of symmetric functions is extensive.  The best I can do is run these new and rather interesting algebras through a rather uninteresting problem.   


\vfill

\fontfamily{qag}\selectfont \fontsize{12}{10}\selectfont

\begin{thebibliography}{}



\item  \dots

\end{thebibliography}


\end{document}