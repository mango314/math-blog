\documentclass[12pt]{article}
%Gummi|065|=)
\usepackage{amsmath, amsfonts, amssymb}
\usepackage[margin=0.5in]{geometry}
\usepackage{xcolor}
\usepackage{graphicx}

\usepackage{pifont}
\usepackage{amsmath}

\newcommand{\off}[1]{}
\DeclareMathSizes{20}{30}{20}{18}

\newcommand{\two }{\sqrt[3]{2}}
\newcommand{\four}{\sqrt[3]{4}}
\newcommand{\red}{\begin{tikz}[scale=0.25]
\draw[fill=red, color=red] (0,0)--(1,0)--(1,1)--(0,1)--cycle;\end{tikz}}
\newcommand{\blue}{\begin{tikz}[scale=0.25]
\draw[fill=blue, color=blue] (0,0)--(1,0)--(1,1)--(0,1)--cycle;\end{tikz}}
\newcommand{\green}{\begin{tikz}[scale=0.25]
\draw[fill=green, color=green] (0,0)--(1,0)--(1,1)--(0,1)--cycle;\end{tikz}}

\newcommand{\sq}[3]{\draw[#3] (#1,#2)--(#1+1,#2)--(#1+1,#2+1)--(#1,#2+1)--cycle;}

\usepackage{tikz}

\newcommand{\susy}{{\bf Q}}
\newcommand{\RV}{{\text{R}_\text{V}}}

\title{Scratchwork: Horocycles}
\author{John D Mangual}
\date{}
\begin{document}

\fontfamily{qag}\selectfont \fontsize{12.5}{15}\selectfont

\maketitle

\noindent While the other proofs are simmering, an alternative geometric style of number theory seems to have been emerging as of 1980.  The actual data is more could be more recent.  \\ \\
Zagier was able to show that evaluating certain integrals of Eisenstein series was related to the prime number theorem.  Whether that was a proof of PNT, I'm not 100\% sure.  The horocycle flow, just moving across a straight line, while it looks orderly and natural is one of the most violent flows in geometry.  The equidistribution and mixing of the horocycle flow on the modular surface could have the capacity to prove statements normally proven using ``analytic methods". \\ \\
I know there is certainly \textit{analogy} with ``spectral" methods or ``analytic" methods or ``geometric" methods but what is behind all of tha jargon?  Perhaps nothing.

\end{document}