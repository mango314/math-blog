\documentclass[12pt]{article}
%Gummi|065|=)
\usepackage{amsmath, amsfonts, amssymb}
\usepackage[margin=0.5in]{geometry}
\usepackage{xcolor}
\usepackage{graphicx}

\usepackage{pifont}
\usepackage{amsmath}

\newcommand{\off}[1]{}
\DeclareMathSizes{20}{30}{20}{18}

\newcommand{\two }{\sqrt[3]{2}}
\newcommand{\four}{\sqrt[3]{4}}
\newcommand{\red}{\begin{tikz}[scale=0.25]
\draw[fill=red, color=red] (0,0)--(1,0)--(1,1)--(0,1)--cycle;\end{tikz}}
\newcommand{\blue}{\begin{tikz}[scale=0.25]
\draw[fill=blue, color=blue] (0,0)--(1,0)--(1,1)--(0,1)--cycle;\end{tikz}}
\newcommand{\green}{\begin{tikz}[scale=0.25]
\draw[fill=green, color=green] (0,0)--(1,0)--(1,1)--(0,1)--cycle;\end{tikz}}

\newcommand{\sq}[3]{\draw[#3] (#1,#2)--(#1+1,#2)--(#1+1,#2+1)--(#1,#2+1)--cycle;}

\usepackage{tikz}

\newcommand{\susy}{{\bf Q}}
\newcommand{\RV}{{\text{R}_\text{V}}}

\title{Scratchwork: Decoupling}
\author{John D Mangual}
\date{}
\begin{document}

\fontfamily{qag}\selectfont \fontsize{12.5}{15}\selectfont

\maketitle

\noindent It seems very important we review the topic of \textbf{decoupling}\dots Many analytic number theorists seem to feel strongly this is the next big advance.  Me, being a fair bit outside this circle, observe that \textit{they} feel it's important.  It's like when Lebron James wears a certain sneaker and *everyone* feels they gotta have it. \\ \\
Even the statement of decoupling is very difficult to understand. We're greatful to Ms. Lillian Pierce for charging through impenetrable sections of the literature.  We include Bourgain's own ``study guide" and his original sources.  We do not include any primary sources. \\ \\
Why study?  Like any subject there are ``moments" of familiarity followed by mostly chaos.  If it is too much effort you simply give up and move to the next project. \\ \\
Here's one way to phrase the conjecture from Tao's blog:
\[ \int_{[0,1]^2}  | \sum_{j= 0}^N e(jx + j^2 y) |^6 \, dx \, dy \ll N^{3 + o(1)} \tag{\ding{37}}\]
as $N \to \infty$. It looks like a freshman Calculus exercise or something you will see on the Putnam exam.  It does not look very modern.   It looks very easy to compute.  If I recall, getting bogged down with the sixth power:
$$  | a + b + c|^6 = (a + b + c)^3 \; \overline{(a+b+c)^3}$$
I am less satisfied when I learned which equation they were solving:
\begin{eqnarray*}
j_1 + j_2 + j_3 &=& k_1 + k_2 + k_3 \\
j_1^2 + j_2^2 + j_3^2 &=& k_1^2 + k_2^2 + k_3^2 
\end{eqnarray*}
If these equations look intuitive their solution certainly does not.  Here's an exercise:
\begin{itemize}
	\item Let $N = 5$ or $N = 100$.  Evaluate the integral, \ding{37} to 10 decimal places.
\end{itemize}
At this level I really should qualify this term ``evaluate" to mean, as an element of $\mathbb{R}$ find 10 decimal places.  While I am reading (it may be some time) here's another way to look at this exercise:
\begin{itemize}
\item Which quadrant does $\sum_{j= 0}^N e(jx + j^2 y) $ live in? as a function of $x$ and $y$ ?  Find the entropy of the corresponding partition of $[0,1]^2$ into those 4 sets.
\item When does this function exhibit lots of ``cancellation" (if we can define such a thing) ?
\item Why is the {\color{red!50!white}\textbf{6}}th power important?
\end{itemize}

\newpage
\noindent Math is the luxurious world were we can ``evaluate" every function $f$ to arbitrary accuracy at every ``point" $x$ or $y$.  If I am skeptical of Tao's framework, or Bourgain's own devices, I can derive my own way of understanding this.  Very simply: we can't use decimals anymore is there a better language to describe this cacncellation?\footnote{One that's more concrete than what is being provided?} \\ \\
Like good librarians we have compiled all of our resources.  Now it's time to do some math! 

\vfill



\begin{thebibliography}{}

\item Lillian B. Pierce 
\textbf{
The Vinogradov Mean Value Theorem }[after Wooley, and Bourgain, Demeter and Guth] \texttt{arXiv:1707.00119}

\item Terence Tao \\
\textbf{Decoupling and the Bourgain-Demeter-Guth proof of the Vinogradov main conjecture}
\texttt{https://terrytao.wordpress.com/2015/12/10/decoupling-and-the-bourgain-demeter-guth-proof-of-the-vinogradov-main-conjecture/} \\ 
\textbf{The two-dimensional case of the Bourgain-Demeter-Guth proof of the Vinogradov main conjecture}
\texttt{https://terrytao.wordpress.com/2015/12/11/the-two-dimensional-case-of-the-bourgain-demeter-guth-proof-of-the-vinogradov-main-conjecture/}

\item Jean Bourgain, Ciprian Demeter. \textbf{A study guide for the $l^2$ Decoupling Theorem}  \texttt{arXiv:1604.06032}

\end{thebibliography}


\end{document}