\documentclass[12pt]{article}
%Gummi|065|=)
\usepackage{amsmath, amsfonts, amssymb}
\usepackage[margin=0.5in]{geometry}
\usepackage{xcolor}
\usepackage{graphicx}

\usepackage{pifont}
\usepackage{amsmath}

\newcommand{\off}[1]{}
\DeclareMathSizes{20}{30}{20}{18}

\newcommand{\two }{\sqrt[3]{2}}
\newcommand{\four}{\sqrt[3]{4}}
\newcommand{\red}{\begin{tikz}[scale=0.25]
\draw[fill=red, color=red] (0,0)--(1,0)--(1,1)--(0,1)--cycle;\end{tikz}}
\newcommand{\blue}{\begin{tikz}[scale=0.25]
\draw[fill=blue, color=blue] (0,0)--(1,0)--(1,1)--(0,1)--cycle;\end{tikz}}
\newcommand{\green}{\begin{tikz}[scale=0.25]
\draw[fill=green, color=green] (0,0)--(1,0)--(1,1)--(0,1)--cycle;\end{tikz}}

\newcommand{\sq}[3]{\draw[#3] (#1,#2)--(#1+1,#2)--(#1+1,#2+1)--(#1,#2+1)--cycle;}

\usepackage{tikz}

\newcommand{\susy}{{\bf Q}}
\newcommand{\RV}{{\text{R}_\text{V}}}

\title{Scratchwork: Divergences}
\author{John D Mangual}
\date{}
\begin{document}

\fontfamily{qag}\selectfont \fontsize{12.5}{15}\selectfont

\maketitle

\noindent Somewhat paranoid overview of PNT.  Very difficult to type of.  Scrapbook-style \\
Here is from the Wikipedia article on divergent series: \\ 
\includegraphics[width=3.5in]{divergent-01.png}  \\

\noindent Here's a discussion of the prime number theorem that has almost the same figure \\ 
\includegraphics[width=3.5in]{divergent-02.png}  

\newpage

\noindent Here is a formal discussion of divergences by Tao.  A few basic ones:  \\ \\
\includegraphics[width=5in]{divergent-07.png}  \\

\noindent and his explanation of what is going on  \\ \\
\includegraphics[width=5in]{divergent-03.png}  

\newpage

\noindent Here how it works for the zeta function $\zeta(s)$:  \\ \\
\includegraphics[width=5in]{divergent-04.png}  \\

\noindent and he generalizes it to the van Mangoldt function   \\ \\
\includegraphics[width=5in]{divergent-05.png}  \\ \\
That pole is not sufficient to prove the prime number theorem, however we get many many intermediate result.  If we include $\zeta(1 + it)\neq 0$ it will be, this is already outside the scope of Tao's blog (he will cover it in other places, though).

\newpage

\noindent What happens if we tried it with the divisor function?  \\ \\
\includegraphics[width=5in]{divergent-06.png}  \\

\noindent I left out the biggest generalizations.  Can you see it?  \\ \\
\includegraphics[width=5in]{divergent-07.png}  \\

\vfill



\begin{thebibliography}{}

\item Terence Tao \textbf{The Euler-Maclaurin formula, Bernoulli numbers, the zeta function, and real-variable analytic continuation} 
\texttt{https://terrytao.wordpress.com/2010/04/10/the-euler-maclaurin-formula-bernoulli-numbers-the-zeta-function-and-real-variable-analytic-continuation/}
\item GH Hardy \textbf{Divergent Series}  Oxford University Press, 1949/1973.
\item Simon Rubinstein-Salzedo \textbf{Could Euler have conjectured the prime number theorem?} \texttt{ arXiv:1701.04718}

\end{thebibliography}


\end{document}