\documentclass[12pt]{article}
%Gummi|065|=)
\usepackage{amsmath, amsfonts, amssymb}
\usepackage[margin=0.5in]{geometry}
\usepackage{xcolor}
\usepackage{graphicx}
%\usepackage{graphicx}
\newcommand{\off}[1]{}
\DeclareMathSizes{20}{30}{20}{18}
\newcommand{\myhrule}{}

\newcommand{\two }{\sqrt[3]{2}}
\newcommand{\four}{\sqrt[3]{4}}

\newcommand{\dash}{
\begin{tikzpicture}[scale=1]
\draw (0,0)--(19,0);
\end{tikzpicture}
}

\newcommand{\sq}[3]{
\node at (#1+0.5,#2+0.5) {#3};
\draw (#1+0,#2+0)--(#1+1,#2+0)--(#1+1,#2+1)--(#1+0,#2+1)--cycle;
}

\usepackage{tikz}

\title{\textbf{Proposal: Sum of Three Squares}}
\author{John D Mangual}
\date{}
\begin{document}

\fontfamily{qag}\selectfont \fontsize{12.5}{15}\selectfont

\maketitle

\noindent In this section we cover Lagrange's sum of three squares theorem:
$$ n = a^2 + b^2 + c^2 \leftrightarrow n \neq 4^a(8b+7) $$
and we will go straight for the Eistenstein series and the Kloosterman sums.  However, we will defer all the \'{E}tale craziness until it is absolutely necessary (once we invoke the Weil conjectures).  \\ \\
I am guessing if we learn a modest amount of Etale cohomology we can answer a few important questions:
\begin{itemize}
\item Why is here a Hasse principle for $x^2 + y^2 + z^2= n$? and study $SO(3,\mathbb{A})$
\item do we have to black-box the Weil conjectures?  perhaps these encode arbitrary amounts of algebra\footnote{Hilbert's \textbf{nullstellensatz} is that way\dots} We know that $X = \{ x^2 + y^2 + z^2 - n w^2 = 0 \}$ what do the line bundles $\mathcal{O}_X$ look like?  Why is sheaf cohomology necessary to make that kind of estimate? 
\item Equidistrubiton of the sum of three squares seems to contain a lot of information.  How do we recover intermediate results?  
\item The Etale cohomology should real like an instruction manual.  I have no idea what a sheaf is or how something could fail to be a sheaf.  There is just algebra. 
\end{itemize}
There are books out there, but it seems beneficial just to do it myself.  The first step is there are all these theta functions:
$$ \theta_\phi(z) = \sum_{|d|\geq 1} \bigg( \sum_{a^2 + b^2 + c^2 = n}\phi(a,b,c) \bigg)e^{2\pi i \, |d|z} $$ 
To me it's bit surprise that all the sphere averages over solutions to the sum of three squares problem, combine so neatly into a modular form.  The transform over $z \mapsto z+1$ is clear.  The transform of $z \mapsto - \frac{1}{4z}$ is a shock.  How can I encode this surprise in the form of a theorem? \\ \\
There exists a decomposition of this series in terms of Eisenstein series (which I'd like to learn more about).  And the coefficients of these eisenstein series will decay quickly enough (and that used the Weil conjectures).  \\ \\
I may as well have accepted the entire proof on faith.  Science is supposed to be reproducible, but the argument we have so far is that someone smart told us these results are correct.

\newpage

\noindent \textbf{Theorem} We have a modular form of weight $\frac{3}{2} + r$ for the modular group $\Gamma_0(4)$
$$ \theta_\phi(z) = \sum_{|d|\geq 1} \bigg( \sum_{a^2 + b^2 + c^2 = n}\phi(a,b,c) \bigg)e^{2\pi i \, |d|z} $$ 
where $\phi$ is a (non-constant) Harmonic polynomial of degree $r$.  (So we're excluding $\phi \equiv 1$.) \\ \\
I feel there is details missing here -- stuff that I can find in textbooks
\begin{itemize}
\item Can we actually solve $a^2 + b^2 + c^2 = n$ (long digression)
\item What are some good $\phi(a,b,c) \in L^2(S^2)$ ? E.g. $\phi(a,b,c) = a^2$.
\item This $\theta_\phi(z)$ is holomorphic.  That's great.  What's a cusp form?
\item Can we draw the fundamental domain of $\Gamma_0(4)$ using lines and circles?
\end{itemize}
and these questions become more and more pressing the further along we read. I guess number theorists are seeing things on a bigger scale.  Modular forms, number theory -- it's all good!  
\begin{itemize}
\item Maass (theta) corespondence
\item Theta correspondence for dual pairs
\end{itemize}
I have no idea what these are, but Google turns up lots and lots of papers.  Maybe there's some insight into number theory problems, there.   The Weyl sums are related to the Fourier coefficients.
$$ \int \phi \,  d\mu_d =  \widehat{\theta}_\phi(d) $$
There's another proof I don't know much about involving the ``subconvexity" of L-functions.
$$ W(\phi, d) = \int_{T_d(\mathbb{Q}) \backslash T_d(\mathbb{A})/K_{T_d} } \phi( z_d . t) \, dt $$
with an inadequate explanation of the definition of K (or perhaps I can find it in a book).  This is an ``integral" over an ``adelic torus".  What could be the maximal compact subgroup of $SO(3, \mathbb{A}_f)$ of rotations over finite adeles?
$$  \mathrm{SO}( x^2 + y^2 + z^2, \mathbb{A}) $$
My knowledge of algebraic groups is rather limited because they take common-sense looking objects and do common-sense things to them and the result is counter-intuitive.  Here is their definition of a torus:
$$ \mathbf{T}_d = \text{res}_{K/Q} \mathbb{G}_m / \mathbb{G}_m \to \mathbf{G} = \mathrm{PG} (B^{2, \infty}) $$
If I recall, we'd like to map solutions to the three squares problem to the quaternions, and there is a very formal ways of expressing that.  
$$ \mathbf{T}_d(\mathbb{Q})\backslash z_d.\mathbf{T}_d(\mathbf{A}_\mathbb{Q}) /K_{\mathbf{T}_d} \subset \mathbf{G}(\mathbf{Q})\backslash \mathbf{G}(A_\mathbf{Q})/K $$
So there is a torus orbit for each equation $x^2 + y^2 + z^2 = d$ possibly the finite set we could obtain by plugging the eq into a calculator.

\newpage

\noindent The Waldspurger formula relates the square of these sphere averages (Weyl sums) to ``automorphic L-functions"
$$ \big|W(\phi, d)\big|^2 = c_{\phi, d} \; \frac{L(\pi, \frac{1}{2}) L(\pi \times \chi_d , \frac{1}{2})}{L(\chi_d, 1)\sqrt{|d|}} $$
and equidistribution of three squres follows from ``subconvexity" and this other estimate:
$$ L(\phi \otimes \chi_K, \frac{1}{2})  \ll \mathrm{disc}(\mathcal{O}_K)^{\frac{1}{2} - \epsilon} \text{ and } c \ll \left[\frac{\mathrm{disc}(\mathcal{O})\;\;\;}{\mathrm{disc}(\mathcal{O}_K)}\right]^{\frac{1}{2} - \epsilon} $$
In that case, we have to prove subconvexity estimates or borrow someone else.
\vspace{6pt}
\hrule
\vspace{6pt}
\noindent Step back a moment.  \\ \\
Experts now agree this problem is completely settled.  I totally disagree, I just have o idea what these things are.   \\ \\
When we ask ``who is this?  what is this?" we look for a name, or make some observation about them.  \\ \\
If I am interested in this one problem (three squares), for a minimal effort, maybe I get to solve a whole class of problems. 
\vspace{6pt}
\hrule
\vspace{6pt}
\noindent Theta functions are still manageable objects and we were just told there are such objects in abundance (even more than I had realized) that we can solve with the theory of automorphic forms.  What have we obtained this way?

\begin{itemize}
\item the theta correspondence has been around 40 years or so
\item the literature on automorphic forms is written in such lofty language, I could hardly tell they were talking about infinite series
\end{itemize}
The idea would be to ``calculate" the infinite series generated by one of these correspondences. 
$$ \frac{1}{4}\sqrt{\frac{5}{\pi}}\,\sum_{a,b,c} (2c^2 - a^2 - b^2)\,  q^{a^2 + b^2 + c^2} $$
 What is more important to you, these terms of an infinite series or the representation that generates them? 

\vfill


\fontfamily{qag}\selectfont \fontsize{12}{10}\selectfont

\begin{thebibliography}{}

\item Philippe Michel and Akshay Venkatesh \textbf{Equidistribution, L-functions and ergodic theory: on some problems of Yu. Linnik} Proceedings of the International Congress
of Mathematicians, Madrid, Spain, 2006.

\item  Wee Teck Gan, \textbf{Theta correspondence: recent progress and applications}, Procedings of
ICM (2014) Vol. 2, 343-366.

\end{thebibliography}


\newpage

\noindent \textbf{1} \quad Constructing series with good properties like the theta functions may be very difficult and I find the automorphic forms literature rather unfriendly.  \\ \\
Let $F$ be a local field (not of characteristic $2$).  This could be the $p$-adic numbers 
$$ \mathbb{Z}_p = \lim_{\longleftarrow} \mathbb{Z}/p\mathbb{Z} $$
We need the fractions of $p$-adic numbers $\mathbb{Q}_p$, and $\psi$ is a nontrivial additive character.  
$$ \text{character} = \text{exponential function}$$
and I don't know what non-trivial means here.  Every time the word is used it means something slightly different. 
$$ \psi(n) = e^{2\pi i a\; \frac{n}{p^k}} $$
The metaplectic group is an exact sequence (describing the transformation of a theta function):
$$ 1 \to S^1 \to Mp(W) \to Sp(W) \to 1 $$
with $W = X \oplus X^\ast$. This is very much like the bra-ket formulism in quantum mechanics, but we can use it to describe any vector space:
$$ |x_1 \rangle \langle x_2 | = X \oplus X^\ast$$
The metaplectic group is a non-trivial central extension of the symplectic group (LOL), so we do not expect ordinary representations here! \\ \\
\textbf{How is anyone supposed to read such garbage?} Continuing, there's asimilar story over $\mathbb{A}$\dots
$$
1 \to S^1 \to Mp_\psi (W_\mathbb{A}) 
\to Sp_\psi (W_\mathbb{A}) \to 1
 $$
This already poses a problem, how do I ``feel out" what adelic Schwartz space could be?  Over the real numbers 
$\mathbb{R}$ Schwartz functions are something like:
$$ x^2 e^{-x^2} $$
which are eigenfunctions of the Harmonic oscillator. Lastly there's this ``splitting" $r_F : Sp(W)_F \to Mp(W_\mathbb{A})$ and for every element in the sympletic group $\gamma$ there is:
$$ \sum_{\xi \in X_F} (r_F(\gamma) \circ \Phi) (\xi) = \sum_{\xi \in X_F} \Phi(\xi) $$ 
and we should have a very familiar formula!  I hardly recognized it.

\vfill


\fontfamily{qag}\selectfont \fontsize{12}{10}\selectfont

\begin{thebibliography}{}

\item Philippe Michel and Akshay Venkatesh \textbf{Equidistribution, L-functions and ergodic theory: on some problems of Yu. Linnik} Proceedings of the International Congress
of Mathematicians, Madrid, Spain, 2006.

\item  Wee Teck Gan, \textbf{Theta correspondence: recent progress and applications}, Procedings of
ICM (2014) Vol. 2, 343-366.

\item James Arthur, Stephen Gelbart \textbf{Lectures on Automorphic L-functions}

\end{thebibliography}

\end{document}