\documentclass[12pt]{article}
%Gummi|065|=)
\usepackage{amsmath, amsfonts, amssymb}
\usepackage[margin=0.5in]{geometry}
\usepackage{xcolor}
\usepackage{graphicx}

\usepackage{pifont}
\usepackage{amsmath}

\newcommand{\off}[1]{}
\DeclareMathSizes{20}{30}{20}{18}

\newcommand{\two }{\sqrt[3]{2}}
\newcommand{\four}{\sqrt[3]{4}}
\newcommand{\red}{\begin{tikz}[scale=0.25]
\draw[fill=red, color=red] (0,0)--(1,0)--(1,1)--(0,1)--cycle;\end{tikz}}
\newcommand{\blue}{\begin{tikz}[scale=0.25]
\draw[fill=blue, color=blue] (0,0)--(1,0)--(1,1)--(0,1)--cycle;\end{tikz}}
\newcommand{\green}{\begin{tikz}[scale=0.25]
\draw[fill=green, color=green] (0,0)--(1,0)--(1,1)--(0,1)--cycle;\end{tikz}}

\newcommand{\sq}[3]{\draw[#3] (#1,#2)--(#1+1,#2)--(#1+1,#2+1)--(#1,#2+1)--cycle;}
\newcommand{\linebrk}{----------------------------------------------------------------------------------------------------------------------------------}


\usepackage{tikz}

\newcommand{\susy}{{\bf Q}}
\newcommand{\RV}{{\text{R}_\text{V}}}

\title{Tutorial : Quadratic Reciprocity}
\date{}
\begin{document}

\fontfamily{qag}\selectfont \fontsize{12.5}{15}\selectfont

\maketitle

\noindent Reciprocity Theorems: why do they happen?  why are they important?  Every time I see a proof of quadratic reciprocity it's usually at the end of a textbook just to show off all the ``concepts" we have learned.  For me, the argument's tell me it's true but if we are trying to solve the equiation:
$$ x^2 \equiv p \pmod q $$
it doesn't explain why $p \leftrightarrow q$ should be possible, why these numbers should be on equal footing.  And if we look at other ``reciprocity theorems"  can their proofs be told in ``elementary" language i.e. with numbers! \\ \\
These days I find the Legendre symbol misleading.  Let's try another way:
$$ \Big[ x^2 \equiv p \pmod q \Big] \leftrightarrow \Big[ x^2 \equiv q \pmod p \Big]  $$
except this is not true when but $p = 4k+3$ and $q = 4k + 3$ then we have the opposite
$$ \Big[ x^2 \equiv p \pmod q \Big] \leftrightarrow \Big[ x^2 \not\equiv q \pmod p \Big]  $$
How is this applied in the real world?  I think that's an open question.  All I can say is that the concept of the integers $\mathbb{Z}$ becomes are scaffolding for understanding a chaotic world, that really has no intrinsic concept of number.\footnote{Why is there no \textbf{dynamical systems}  proof of Quadratic Reciprocity?  We will not discuss {\color{black!10!white}cryptrography} in any way, but maywe we could discuss \textbf{coding}  or \textbf{information theory}.  Just, number theory is to as ``applied" as one would like.  Philosophically it's about abstracting away from the real world something extremely pure.} \\ \\
Can you come up with a better concept than $\mathbf{0}$ or better than $\mathbb{Z}$?  No. \\ \\ \\ \\
Usually the justification is that Quadratic Reciprocity is connect to other branches of mathematics.  Let's try:
$$ \big[ \text{fermat's little theorem(s)} \big]
\to \big[ \text{QR} \big]
\to \big[ \text{three squares} \big] 
\to \big[ \text{Banach-Tarski paradox} \big]$$


\newpage

\noindent \textbf{A} Wikipedia has a separate page for \text{proofs} of Quadratic Reciprocity.  Let's try the willy-nilly Eisenstein proof, which I despise.

\vfill

\begin{thebibliography}{}

\item Masanori Morishita \textbf{Knots and Primes} (Universitext) Springer, 2012.

\item Nancy Childress \;\;\;\,\textbf{Class Field Theory} (Universitext) Springer, 2009.

\item  Jared Weinstein \textbf{Reciprocity Laws and Galois Representations: Recent Breakthroughs} \\ 
Bull. Amer. Math. Soc. 53 (2016), 1-39 \hfill \texttt{https://doi.org/10.1090/bull/1515} 

\end{thebibliography}



\end{document}