\documentclass[12pt]{article}
%Gummi|065|=)
\usepackage{amsmath, amsfonts, amssymb}
\usepackage[margin=0.5in]{geometry}
\usepackage{xcolor}
\usepackage{graphicx}

\usepackage{pifont}
\usepackage{amsmath}

\newcommand{\off}[1]{}
\DeclareMathSizes{20}{30}{20}{18}

\newcommand{\two }{\sqrt[3]{2}}
\newcommand{\four}{\sqrt[3]{4}}
\newcommand{\red}{\begin{tikz}[scale=0.25]
\draw[fill=red, color=red] (0,0)--(1,0)--(1,1)--(0,1)--cycle;\end{tikz}}
\newcommand{\blue}{\begin{tikz}[scale=0.25]
\draw[fill=blue, color=blue] (0,0)--(1,0)--(1,1)--(0,1)--cycle;\end{tikz}}
\newcommand{\green}{\begin{tikz}[scale=0.25]
\draw[fill=green, color=green] (0,0)--(1,0)--(1,1)--(0,1)--cycle;\end{tikz}}

\newcommand{\sq}[3]{\draw[#3] (#1,#2)--(#1+1,#2)--(#1+1,#2+1)--(#1,#2+1)--cycle;}
\newcommand{\linebrk}{----------------------------------------------------------------------------------------------------------------------------------}


\usepackage{tikz}

\newcommand{\susy}{{\bf Q}}
\newcommand{\RV}{{\text{R}_\text{V}}}

\title{Tutorial : Sum of Four Squares}
\date{}
\begin{document}

\fontfamily{qag}\selectfont \fontsize{12.5}{15}\selectfont

\maketitle

\noindent Zagier's proof that every integer is the sum of 4 perfect squares, is scattered in various parts of his first notes. 
$$ n = a^2 + b^2 + c^2 + d^2 \text{ for } n \geq 0$$
Here we try to present a different narrative.  Number Theory and Modular forms are separate fields of Mathematics, advancing in their own way, but they have many things in common. It's hard to create Number Theory results, that use modular forms and still look elemenatary. \\ \\
\textbf{Step \# 1} Observe the coefficients of $\theta(z)^4$ are the number of ways to express $n$ as the sum of four squares:
$$ \theta(z)^4 = \left[ \sum_{n \in \mathbb{Z}}  q^{n^2} \right]^4 = \sum_{n \geq 0}\; 
\#\Big\{ (a,b,c,d): a^2 + b^2 + c^2 + d^2 = n  \Big\} \; q^n
\equiv \sum_{n \geq 0} r_4(n) q^n $$
This is a modular form of weight $2$ on $\Gamma_0(4)$ which is generated by $z \mapsto z + 1$ and $z \mapsto  - \frac{1}{4z}$. \\
The full group $\text{SL}(2, \mathbb{Z})$ acts on several theta functions at once turning them into each other. \\ \\
This action is closely related to Poisson summation.  For $f \in \mathcal{S}(\mathbb{R})$ the Schwartz class:
$$ \sum_{n \in \mathbb{Z}} f(n) = \sum_{n \in \mathbb{Z}} \hat{f}(n) $$
In our case, $f(z) = q^{z^2}$, for $z \in i \mathbb{R}$ (and by analytic continuation\footnote{What properties of $\theta$ were used to define such a continuation?} to all of $\mathbb{C}$). \\ \\
\textbf{Step \# 2} The space of modular forms of weight $2$ on $\mathbb{H}/\Gamma_0(4)$ is at most two-dimensional. \\ \\
This is another whopper.

\vfill

\begin{thebibliography}{}


\item Don Zagier \textbf{Elliptic Modular Forms and their Applications} \\ \texttt{https://doi.org/10.1007/978-3-540-74119-0\_1}

\item Jan Bruinier, Gerard Geer, G\"{u}nter Harder, Don Zagier. \\ \textbf{The 1-2-3 of Modular Forms} (Universitext) Springer, 2008.


\end{thebibliography}


\end{document}