\documentclass[12pt]{article}
%Gummi|065|=)
\usepackage{amsmath, amsfonts, amssymb}
\usepackage[margin=0.5in]{geometry}
\usepackage{xcolor}
\usepackage{graphicx}

\newcommand{\off}[1]{}
\DeclareMathSizes{20}{30}{20}{18}

\newcommand{\two }{\sqrt[3]{2}}
\newcommand{\four}{\sqrt[3]{4}}


\usepackage{tikz}

\title{Item: \textbf{Solovay-Kitaev Theorem}}
\author{John D Mangual}
\date{}
\begin{document}

\fontfamily{qag}\selectfont \fontsize{12.5}{15}\selectfont

\maketitle

\noindent Can you make a living solving diophantine equations?  There is Peter Sarnak. \\ \\
What can the rest of us do?  How about showing an integer is the sum of four squares:
$$ n = a^2 + b^2 + c^2 + d^2 $$
and in recent work Mr. Sarnak makes a connection to quantum computing.  Even if, out of curiosity, you read Nielsen and Chuang, I got sort of shy that maybe my arithmetic questions were stupid or something. These titles like the \textbf{quantum fourier transform} sound rather enticing, but the number theory in the textbook is rather limited.  There is Shor's algorithm.\\ \\
If I read correctly their gates are unitary operators act on a copy of $\mathbb{C}^2 \otimes \mathbb{C}^2$ a copy of two ``qubits".  And we get the basis operators:
$$ 
\left( \frac{ | 0 \rangle + | 1 \rangle  }{\sqrt{2}} \right) \otimes 
\left( \frac{ | 0 \rangle + | 1 \rangle  }{\sqrt{2}} \right) = 
\frac{ | 00 \rangle + | 01 \rangle + | 10 \rangle + | 11 \rangle}{2}$$
Then we check there are 2 basis elements $\times$ 2 basis elements for a total of 4 ``states":
$$ \mathbb{C}^2 \otimes \mathbb{C}^2 \simeq \mathbb{C}^4 $$
I took a course on ``alternative modes of computation" and I found the discussion lacking.  Nielsen-Chuang make the common-sense derivation:
$$ \mathbb{C}^2 \otimes \mathbb{C}^2 \simeq \mathbb{C}^4 \hspace{0.125in}\text{ therefore }\hspace{0.125in} U \Big((\mathbb{C}^2)^{\otimes 2}\Big) \simeq U \big(\mathbb{C}^4\big) \text{ or }U(4) $$
In an abstract algebra class (maybe 3rd year undergrad or again in graduate school\footnote{or forever\dots}) there is also the special unitary group, where we factor out the various copies of $S^1$.  Sarnak has chosen not to. \\ \\
Solovay-Kitaev theorem says we can approximate every state, with a product of basic operators.  For me it's the simple $\otimes$ which makes me wonder about it's cousins (that I know much less about)
$$ \text{Ext}\quad \text{Tor}  $$
This is the part of abstract algebra class I blanked out on.  However, the more backflips I see Nielsen and Chuang do, makes me wonder if Ext and Tor are not far behind.

\newpage

\noindent \dots


\fontfamily{qag}\selectfont \fontsize{12}{10}\selectfont

\begin{thebibliography}{}

\item John Cremona \textbf{The L-functions and modular forms database project}  \texttt{arXiv:1511.04289}  
\texttt{http://www.lmfdb.org/}

\item Sofia Lindqvist \textbf{Weak approximation results for quadratic forms in four variables} Modular forms of Weight 2 \texttt{arXiv:1704.00502}

\item Ori Parzanchevski, Peter Sarnak.  \textbf{Super-Golden-Gates for PU(2)} \texttt{arXiv:1704.02106} \\ \\
Peter Sarnak.  \textbf{Letter to Aaronson and Pollington on the Solvay-Kitaev Theorem and Golden Gates} \texttt{https://publications.ias.edu/sarnak/paper/2637}

\item Michael Nielsen, Isaac Chaung. \\ \textbf{Quantum computation and quantum information}  Cambridge University Press , 2010

\item Leonard Susskind, Art Friedman. \\ \textbf{Quantum Mechanics: The Theoretical Minimum} Basic Books, 2015
\end{thebibliography}

\end{document}