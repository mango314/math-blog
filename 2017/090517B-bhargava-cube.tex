\documentclass[12pt]{article}
%Gummi|065|=)
\usepackage{amsmath, amsfonts, amssymb}
\usepackage[margin=0.5in]{geometry}
\usepackage{xcolor}
\usepackage{graphicx}

%\usepackage{pifont}
\usepackage{amsmath}

\newcommand{\off}[1]{}
\DeclareMathSizes{20}{30}{20}{18}

\newcommand{\two }{\sqrt[3]{2}}
\newcommand{\four}{\sqrt[3]{4}}
\newcommand{\red}{\begin{tikz}[scale=0.25]
\draw[fill=red, color=red] (0,0)--(1,0)--(1,1)--(0,1)--cycle;\end{tikz}}
\newcommand{\blue}{\begin{tikz}[scale=0.25]
\draw[fill=blue, color=blue] (0,0)--(1,0)--(1,1)--(0,1)--cycle;\end{tikz}}
\newcommand{\green}{\begin{tikz}[scale=0.25]
\draw[fill=green, color=green] (0,0)--(1,0)--(1,1)--(0,1)--cycle;\end{tikz}}

\newcommand{\sq}[3]{\draw[#3] (#1,#2)--(#1+1,#2)--(#1+1,#2+1)--(#1,#2+1)--cycle;}

\usepackage{tikz}

\newcommand{\susy}{{\bf Q}}
\newcommand{\RV}{{\text{R}_\text{V}}}

\title{Scratchwork: Geometry of Numbers}
\date{}
\begin{document}

%\fontfamily{qag}\selectfont \fontsize{12.5}{15}\selectfont

\sffamily

\maketitle

\noindent As I read attempting to read Manjul Bhargava's papers, a few problems emerge:
\begin{itemize}
\item I don't know some of the groups definitions that he refers to
\item I don't know what he is calling ``geometry of numbers"
\item I don't know what the big deal is about class numbers.  Or why his contrubtions were so crucial.
\end{itemize}
Geometry of Numbers has been around since Hermann Minkowski, and there's even the book, written in academic 19th century German \textit{Geometry der Zahlen}.  And who knows?  Perhaps that bounds have improved since then, \dots there is the book of Cassels in the mid 20th century.  This area - when I learned it - amounted to a ``cute proof" that should be solve using more robust methods - and then Manjul won a Field's medal with it in 2014.\footnote{1) that is my personal opinion, and usually when you ask someone suddenly everyone changes their position \dots 2) I have look at him in person, maybe one or two-odd times in my life. }\\ \\
The geometry of numbers proof (e.g. Davenport) that $n = a^2 + b^2 + c^2 + d^2$ uses lattices over $\mathbb{R}^4$. However the proof that $p = 4k+1 = a^2 + b^2$ uses a mix of mod $p$ arithmetic -- over $(\mathbb{Z}/p\mathbb{Z})^2$ to define (an essentially random) lattice over $\mathbb{R}^2$.  So, we are going to combine the two objects. \\\\
At this point $p$ is still generic, so it's safe to say we are solving the ``family" equations $x^2 + y^2 = n$ \textbf{at all primes p}.
$$ X = \{ x^2 + y^2 - n = 0 \} \to \big( \mathbb{Z}/p\mathbb{Z} \times \mathbb{R} \big)^2 $$
and we are looking for lattices in this mixed geometirc object.  And if we have $X(\mathbb{Z}/p\mathbb{Z})$ and $X(\mathbb{R})$ we could even have:
$$ X(\mathbb{Z}_p) \to \dots \to X(\mathbb{Z}/p^{k+1}\mathbb{Z}) \to X(\mathbb{Z}/p^k \mathbb{Z}) \to \dots \to X(\mathbb{Z}/p \mathbb{Z})  $$
While looking at the scheme $X(\mathbb{Z})$ at one place $p$ is not enough if we look at a generic place $p$ \textit{as well as} $p = \infty$ we can have
$$ X(\mathbb{Z}) ``\simeq" X(\mathbb{Z}_p) \cap X(\mathbb{R})$$
and maybe this is sufficient.  \\ \\
Even if you know and prove Fermat's result that $p = 4k+1$ prime is the sum of two squares what have you gained from that:
$$ p = a^2 + b^2 = (a+bi)(a-bi) \text{ then } a+bi \in \sqrt{a^2 + b^2}\,e^{i\theta} $$
so there is an angle, $\theta$ that's unpredictiable.  In more modern language we could say that primes in $\mathbb{Z}[i]$ are rotationally symmetric in large scale.  \\ \\
There are two good exampls of Fermat's descent:
\begin{itemize}
\item $p = a^2 + b^2$
\item $\sqrt{2} \notin \mathbb{Q}$
\end{itemize}
\newpage 
\noindent Wikipedia (and other books) indicate that descent can be considered a type of Galois Cohomology.  Since a result like this is ``clear" from elementary means, why should you build it from this complicated machinary?  Here's the basic argumen:
$$ x^2 = 2y^2 \to x = 2z \to (2z)^2 = 2y^2 \to 2z^2 = y^2 \to y^2 = 2z^2 $$ 
The rest of the steps amound to identifying a variety (or scheme) basically the equation we are trying to solve, and the appropriate cohomology groups.
$$ X = \{  x^2 - 2y^2  = 0 \}$$
That is the equation we are trying to solve.  Now it's our variety or Scheme.  We could consider $X(\mathbb{Z})$ or $X(\mathbb{Q})$ or $X(\mathbb{Q}_p)$ or whatever.  All these equations we just calculated can be summarized by saying we found a \textbf{map}. 
$$ T: X \to X \text{ with } (x,y) \mapsto (y, x/2)$$
Then we say something interesting... we say our solution has gotten ``smaller", e.g.
$$ x^2 + y^2 > y^2 + (x/2)^2 \in \mathbb{Z} $$
In what sense have our solutions gotten smaller?  We are not longer just using a field, we are using an \textit{ordered} field, either $\mathbb{Q}$ or $\mathbb{R}$:
\begin{eqnarray*} X(\mathbb{Z}) = \{  x^2 - 2y^2  = 0 \} &\subseteq&  X(\mathbb{Q}) =
\{  x^2 - 2y^2  = 0 \} \\ 
&\subseteq & X(\mathbb{R})
= \{ (x - \sqrt{2}y)(x + \sqrt{2}y) = 0 \} 
= \{  x- \sqrt{2}y = 0 \} \cap \{  x+ \sqrt{2}y = 0 \} \end{eqnarray*}
and finally we say that there are no infinitely descending sequences of integers. \\ \\
In a way, we have done nothing.  And we can go even futher .  We might do this when the elementary means are unavailable, and when our own sense fail us or lead us astray.  Now we consider $x$ as a \textit{function}:
$$ x, y : X(\mathbb{Z}) \to \mathbb{Z} \text{ so that } x, y \in H^1(X, \mathbb{Z})$$
I'm not sure if this is the torsor that the Number Theorists use.  In fact, they use 
$$ H^1(k, G) \text{ possibly } H^1(\mathbb{Q}, T) $$
setting $k = \mathbb{Q}$ and $G = \{ T^k: k \in \mathbb{Z}\}$ to be iterations of the `` $\times$2 " map we discussed.  Then the idea of local-to-global is that we find a solution over $k$ if we solve over all completions $k_v$:
$$ \text{if }[X]= 0 \in H^1(k_v, G) \text{ then }[X]=0 \in H^1(k, G) $$
or we could state it as some type of product
$$ H^1(k, G) \to \prod_v H^1(k_v, G) $$
In our case $k = \mathbb{Q}$, $[X] = \{  x^2 - 2y^2 = 0\}$ and $G $ is basically the map $(x,y) \mapsto (y/2, x)$.  This amonunts to saying that for two relatively prime numbers $m, n \in \mathbb{Z}$ we could try to solve by unique factorization.
$$  $$
All these homology theories are (sometimes useful) layers that we put over real calculations and real things that are happening.  \'{E}tale cohomology over $\mathrm{Spec} k$ is Galois cohomology, so we could stop here and I will. \\ \\
We ought to discuss $p = a^2 + b^2$ a bit (btw I'm using notes of Brian Conrad on desent).
\vfill

\begin{thebibliography}{}
\item Manjul Bhargava \dots
\end{thebibliography}



\end{document}