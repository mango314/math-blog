\documentclass[12pt]{article}
%Gummi|065|=)
\usepackage{amsmath, amsfonts, amssymb}
\usepackage[margin=0.5in]{geometry}
\usepackage{xcolor}
\usepackage{graphicx}

%\usepackage{pifont}
\usepackage{amsmath}

\newcommand{\off}[1]{}
\DeclareMathSizes{20}{30}{20}{18}

\newcommand{\two }{\sqrt[3]{2}}
\newcommand{\four}{\sqrt[3]{4}}
\newcommand{\red}{\begin{tikz}[scale=0.25]
\draw[fill=red, color=red] (0,0)--(1,0)--(1,1)--(0,1)--cycle;\end{tikz}}
\newcommand{\blue}{\begin{tikz}[scale=0.25]
\draw[fill=blue, color=blue] (0,0)--(1,0)--(1,1)--(0,1)--cycle;\end{tikz}}
\newcommand{\green}{\begin{tikz}[scale=0.25]
\draw[fill=green, color=green] (0,0)--(1,0)--(1,1)--(0,1)--cycle;\end{tikz}}

\newcommand{\sq}[3]{\draw[#3] (#1,#2)--(#1+1,#2)--(#1+1,#2+1)--(#1,#2+1)--cycle;}

\usepackage{tikz}

\newcommand{\susy}{{\bf Q}}
\newcommand{\RV}{{\text{R}_\text{V}}}

\title{Scratchwork: Geometry of Numbers}
\date{}
\begin{document}

%\fontfamily{qag}\selectfont \fontsize{12.5}{15}\selectfont

\sffamily

\maketitle

\noindent As I read attempting to read Manjul Bhargava's papers, a few problems emerge:
\begin{itemize}
\item I don't know some of the groups definitions that he refers to
\item I don't know what he is calling ``geometry of numbers"
\item I don't know what the big deal is about class numbers.  Or why his contrubtions were so crucial.
\end{itemize}
Geometry of Numbers has been around since Hermann Minkowski, and there's even the book, written in academic 19th century German \textit{Geometry der Zahlen}.  And who knows?  Perhaps that bounds have improved since then, \dots there is the book of Cassels in the mid 20th century.  This area - when I learned it - amounded to a ``cute proof" that should be solve using more robust methods - and then Manjul won a Field's medal with it in 2014.\footnote{1) that is my personal opinion, and usually when you ask someone suddenly everyone changes their position \dots 2) I have look at him in person, maybe one or two-odd times in my life. }\\ \\
The geometry of numbers proof (e.g. Davenport) that $n = a^2 + b^2 + c^2 + d^2$ uses lattices over $\mathbb{R}^4$. However the proof that $p = 4k+1 = a^2 + b^2$ uses a mix of mod $p$ arithmetic -- over $(\mathbb{Z}/p\mathbb{Z})^2$ to define (an essentially random) lattice over $\mathbb{R}^2$.  So, we are going to combine the two objects:
$$  $$
At this point $p$ is still generic, so it's safe to say we are solving the ``family" equations $x^2 + y^2 = n$ \textbf{at all primes p}.
$$ X = \{ x^2 + y^2 - n = 0 \} \to \big( \mathbb{Z}/p\mathbb{Z} \times \mathbb{R} \big)^2 $$
and we are looking for lattices in this mixed geometirc object.  And if we have $X(\mathbb{Z}/p\mathbb{Z})$ and $X(\mathbb{R})$ we could even have:
$$ X(\mathbb{Z}_p) \to \dots \to X(\mathbb{Z}/p^{k+1}\mathbb{Z}) \to X(\mathbb{Z}/p^k \mathbb{Z}) \to \dots \to X(\mathbb{Z}/p \mathbb{Z})  $$
While looking at the scheme $X(\mathbb{Z})$ at one place $p$ is not enough if we look at a generic place $p$ \textit{as well as} $p = \infty$ we can have
$$ X(\mathbb{Z}) ``\simeq" X(\mathbb{Z}_p) \cap X(\mathbb{R})$$
and maybe this is sufficient.  
\vfill

\begin{thebibliography}{}
\item Manjul Bhargava \dots
\end{thebibliography}



\end{document}