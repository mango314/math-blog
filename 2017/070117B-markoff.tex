\documentclass[12pt]{article}
%Gummi|065|=)
\usepackage{amsmath, amsfonts, amssymb}
\usepackage[margin=0.5in]{geometry}
\usepackage{xcolor}
\usepackage{graphicx}

\newcommand{\off}[1]{}
\DeclareMathSizes{20}{30}{20}{18}

\newcommand{\two }{\sqrt[3]{2}}
\newcommand{\four}{\sqrt[3]{4}}
\newcommand{\red}{\begin{tikz}[scale=0.25]
\draw[fill=red, color=red] (0,0)--(1,0)--(1,1)--(0,1)--cycle;\end{tikz}}
\newcommand{\blue}{\begin{tikz}[scale=0.25]
\draw[fill=blue, color=blue] (0,0)--(1,0)--(1,1)--(0,1)--cycle;\end{tikz}}
\newcommand{\green}{\begin{tikz}[scale=0.25]
\draw[fill=green, color=green] (0,0)--(1,0)--(1,1)--(0,1)--cycle;\end{tikz}}

\newcommand{\sq}[3]{\draw[#3] (#1,#2)--(#1+1,#2)--(#1+1,#2+1)--(#1,#2+1)--cycle;}

\usepackage{tikz}

\newcommand{\susy}{{\bf Q}}
\newcommand{\RV}{{\text{R}_\text{V}}}

\title{Worksheet: Markoff Triples}
\author{John D Mangual}
\date{}
\begin{document}

\fontfamily{qag}\selectfont \fontsize{12.5}{15}\selectfont

\maketitle

\noindent Politically, let's see what modern number theorists are doing.  At least initiall, it's kid stuff.  And when you follow it through it still feels like GH Hardy, with some extra stuff.   \\ \\ \\
Sarnak has been leading some discussion of the Markoff triples.  I'll start with what I know. And then see how he put it in context.
$$ q || q \theta || <  5^{-\tfrac{1}{2}} $$
This equation has infinitely many integer solutions $q \in \mathbb{Z}$.  This is the best we can do in case:
$$ \theta^2 - \theta + 1 = 0 $$
The next one down the line:
$$ q || q \theta || <  2^{-\tfrac{3}{2}} $$
This equation has infinitely many integer solutions $q \in \mathbb{Z}$.  This is the best we can do in case:
$$ \theta^2 +2\theta - 1 = 0 $$
Otherwise there are infinitely many solutions of:
$$ q || q \theta || <  5/221^{-\tfrac{1}{2}} $$
This equation has infinitely many integer solutions $q \in \mathbb{Z}$.  This is the best we can do in case:
$$ 5\theta^2 + 11\theta - 5 = 0 $$
This is (confusingly) called the \textbf{Markov chain} and somehow this is related to the Markov diophantine equation $ a^2 + b^2 + c^2  = 3abc $.  There is a descent:
$$ (a,b,c) \mapsto ( bc - a , b, c) $$
and we could build an entire ``tree" of solutions this way\dots \\ \\ \\
We could ask about strong approximation.  And we know Hasse's principle works for quadratic forms:
$$ x^2 + y^2 + z^2 $$
and now we just switch a 2 to a 3 and see of the answer still holds
$$ F(\mathbf{x}) = x^3 + y^3 + z^3 \hspace{0.125in}\text{solving}\hspace{0.125in}
\big\{ (x,y,z) : F(x,y,z) = k \big\} $$
In order to confuse ourselves, we could think of the order of the cube of a number field $\mathbb{Q}(\sqrt[3]{2})$:
$$ N\big( a + b \sqrt[3]{2} + c \sqrt[3]{4}\big) 
= a^3 + 2b^3 + 4 c^3 - 2abc$$
This looks very closed to the Markov equation we just solved.  These cases are solved by the \textbf{Dirichlet Unit Theorem}.

\newpage

\noindent I think it's a good time to try these two things:
\begin{itemize}
\item solve $ a^2 + b^2 + c^2  = 3abc $ as much as we can
\item examine strong approximation for $x^2 + y^2 + z^2$ in the context of all strong approximations.
\end{itemize}

\vfill



\begin{thebibliography}{}

\item Amit Ghosh, Peter Sarnak. \\ 
\textbf{Integral points on Markoff type cubic surfaces} \texttt{arXiv:1706.06712}

\item Jean Bourgain, Alex Gamburd, Peter Sarnak. \\
\textbf{Markoff Triples and Strong Approximation} \hspace{30pt}\texttt{arXiv:1505.06411} \\
\textbf{Markoff Surfaces and Strong Approximation: 1} \texttt{arXiv:1607.01530}

\item Ori Parzanchevski, Peter Sarnak. \textbf{Super-Golden-Gates for PU(2)} \texttt{arXiv:1704.02106}

\item J.W.S. Cassels. \\ \textbf{Diophantine Approximation} (Cambridge Tracts in Mathematical Physics, \#45) \\ Cambridge University Press, 1957. 

\item Vladimir Platonov, Andrei Rapinchuk. \\ \textbf{Algebraic Groups and Number Theory}
Academic Press, 1994.

\item Emmanuel Breuillard and Hee Oh (eds.) \textbf{Thin Groups and Superstrong Approximation} 
\texttt{http://library.msri.org/books/Book61/contents.html} Cambridge University Press, 2014.
\end{thebibliography}


\end{document}