\documentclass[12pt]{article}
%Gummi|065|=)
\usepackage{amsmath, amsfonts, amssymb}
\usepackage[margin=0.5in]{geometry}
\usepackage{xcolor}
\usepackage{graphicx}

\newcommand{\off}[1]{}
\DeclareMathSizes{20}{30}{20}{18}

\newcommand{\two }{\sqrt[3]{2}}
\newcommand{\four}{\sqrt[3]{4}}


\usepackage{tikz}

\title{\textbf{Item: Nekrasov Partition Function}}
\author{John D Mangual}
\date{}
\begin{document}

\fontfamily{qag}\selectfont \fontsize{12.5}{15}\selectfont

\maketitle

\noindent There are now literally hundreds of Neraksov Partition functions, all with more or less the same structure:
\begin{itemize}
\item lengthy discussions of supersymmetry and/or D-branes
\item hand-way justification of reduction to finite dimensional integral
\item evaluation of the very easiest cases
\end{itemize}
It's very hard to say how I can contribute.  I still haven't done the superhomework${}^{\text{TM}}$ so I can't do item one.  Item 2 is impossible.  Item 3, as soon as soon as I solve the easy case, there are many other solved cases on the math side, with no physical counterpart.  \\ \\
These people communicate well with each other, but maybe not as well with outsiders.  Hopefully, as we familiarize ourselves with Nekrasov Partition Function, we can read carefully and avoid dilemmas \#1, \#2 and \#3. \\ \\
Logically these \texttt{hep-th} papers have the following structure: \textbf{hat}, \textbf{cat}, \textbf{bat} have the same last two letters.  Let's examine further: 
\begin{itemize}
\item \textbf{b} + \textbf{at} = \textbf{bat}
\item \textbf{c} + \textbf{at} = \textbf{cat}
\item \textbf{h} + \textbf{at} = \textbf{hat}
\end{itemize}
Therefore, all words that end with ``at" are the same.   Including words like: \textbf{gate}, \textbf{chocolate} or \textbf{considerate}.  These also rhyme.\\ \\
Let's pick one a theory out of a hat.  Wait better\dots Let's read Nekrasov himself. 
$$ Z = Z^\text{tree} \times Z^\text{1-loop} \times Z^\text{inst} $$
where we have removed the names of the paramters, since we don't know what they are.
\begin{itemize}
\item $a$ is the adjoint Higgs field\footnote{\dots as in the ``Higgs boson".}
\item $m$ is the set of complex masses of matter multiplets (???)
\item $\tau \in \mathbb{H}$ is the complexified gauge coupling, $q = e^{2\pi i \, \tau}$
\end{itemize}
The power series expansion $\displaystyle Z = \sum_k q^k Z_k $ is complicated since $k$ is not a number.

\vfill

\fontfamily{qag}\selectfont \fontsize{12}{10}\selectfont

\begin{thebibliography}{}



\item Joseph Hayling, Constantinos Papageorgakis, Elli Pomoni, Diego Rodr\'{i}guez-G\'{o}mez. \textbf{Exact Deconstruction of the 6D (2,0) Theory} \texttt{arXiv:1704.02986}

\item Nikita \, Nekrasov \textbf{BPS/CFT correspondence: non-perturbative Dyson-Schwinger equations and qq-characters} \texttt{ arXiv:1512.0538}

\item Nikita \, Nekrasov \textbf{BPS/CFT correspondence II: Instantons at crossroads, Moduli and Compactness Theorem} \texttt{arXiv:1608.07272}

\item Nikita \, Nekrasov \textbf{BPS/CFT Correspondence III: Gauge Origami partition function and qq-characters} \texttt{arXiv:1701.00189}

\end{thebibliography}


\end{document}