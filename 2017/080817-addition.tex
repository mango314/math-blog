\documentclass[12pt]{article}
%Gummi|065|=)
\usepackage{amsmath, amsfonts, amssymb}
\usepackage[margin=0.5in]{geometry}
\usepackage{xcolor}
\usepackage{graphicx}

\usepackage{pifont}
\usepackage{amsmath}

\newcommand{\off}[1]{}
\DeclareMathSizes{20}{30}{20}{18}

\newcommand{\two }{\sqrt[3]{2}}
\newcommand{\four}{\sqrt[3]{4}}
\newcommand{\red}{\begin{tikz}[scale=0.25]
\draw[fill=red, color=red] (0,0)--(1,0)--(1,1)--(0,1)--cycle;\end{tikz}}
\newcommand{\blue}{\begin{tikz}[scale=0.25]
\draw[fill=blue, color=blue] (0,0)--(1,0)--(1,1)--(0,1)--cycle;\end{tikz}}
\newcommand{\green}{\begin{tikz}[scale=0.25]
\draw[fill=green, color=green] (0,0)--(1,0)--(1,1)--(0,1)--cycle;\end{tikz}}

\newcommand{\sq}[3]{\draw[#3] (#1,#2)--(#1+1,#2)--(#1+1,#2+1)--(#1,#2+1)--cycle;}
\newcommand{\linebrk}{----------------------------------------------------------------------------------------------------------------------------------}


\usepackage{tikz}

\newcommand{\susy}{{\bf Q}}
\newcommand{\RV}{{\text{R}_\text{V}}}

\title{Tutorial : Additive Combinatorics}
\date{}
\begin{document}

\fontfamily{qag}\selectfont \fontsize{12.5}{15}\selectfont

\maketitle

\noindent Given a set $A, B \subset \mathbb{Z}$ we can define $A + B = \{ a + b: a \in A , b \in B \}$.  \\ \\
As I read it this construction, leading to {\color{yellow!60!black}\text{additive combinatorics}} is equivalent to Fourier analysis, to PDE, parts of graph theory, dynmical systems and many other things.  I have never followed through with taking a Fourier anlysis proof and replacing them with additive combinatorics results.  It certainly does not look like they are powerful enough to replace one with the other. \\ \\
Melyvn Nathanson has written two nice books on Additive Number Theory.  I do have trouble connecting them to objects that I would call ``additive": 
$$ 149 = 81 + 2 \times 27 + 9 + 3 + 2 \times 1= 3^4 = 2 \times 3^3 + 3^2 + 3 + 2 \times 3^0 =  12112_3 $$
This is just me being na\"{i}ve.  The other problem is once I have result how to I place it into context.  Context for whom?
\begin{itemize}
\item the guy on the street
\item an engineer
\item other mathematicians
\end{itemize}
``Context" will mean something different to everbody.  Looking at Nathanson's work now, I realize addition is all that we really have and the main struggle is to but 2+2 together. \\ \\ \\
What did Melvyn Nathanson prove? \\ \\
\#1 Every set $A \subseteq \mathbb{Z}_{\geq 0}$ of positive integers with upper Banach density $1$ contains an infinite sequence of pairwise disjoint subsets $(B_i)_{i \in \mathbb{N}}$:
$$  B_i \cap B_j = \varnothing $$
such that each of the $B_i$ (also) has upper Banach density $1$ such that:
$$ \sum_{i \in I} B_i = \left\{ \sum_{i \in I} b_i : b_i \in B_i  \right\}  \subseteq A $$
This is not the same as base arithmetic.  We could have that $\mathbb{Z}$ is the sum of $B_k = \{ 0, 2^k \}$ for the various power of two, $k \in \mathbb{N}$:
$$ \mathbb{Z} = \sum_{k \geq 0} \big\{ 0, 2^k \big\} = \{ 0 , 1\} + \{ 0, 2 \} + \{ 0, 4 \} + \dots $$
These sets are not pairwise disjoint, nor do they have density $1$.  Erd\H{o}s' conjecture says that if $A$ has positive density then $B + C \subseteq A$ where $B$ and $C$ have positive density. 

\newpage

\noindent Nathanson doesn't specifically say how his theorem is relted to the Gromov, nor does Gromov mention him back, though he does talk about the Borsuk-Ulam theorem and Ramsey theory. \\ \\
In the mean, time Nathson's book does mention the sum of 3 square problem.  How about 4-squares?  Let $\square = \{ x^2 : x \in \mathbb{Z}\}$. That $\square + \square + \square + \square = \mathbb{Z}$.  Says nothing about number fields, or what if the integers $\mathbb{Z}$ have been slightly displaced?  There's a limit to how evenly you can space objects on the surface of a sphere, there's always a bit of discrepancy. \\ \\
Nathon's second book is on {inverse problems} and that's what we have here.  An arbitrary density = 1 set $B$ and we split into infinitely many parts each with density = 1, such that $\sum B_i = B$.   Some will be bigger than others and there must be something like a base.  How do we compare two sets of these kind?  {Which is bigger}? \\ \\
I guess it's up to us to {find} compelling instances of density = 1 sets, and maybe improve on his result.

\vfill

\begin{thebibliography}{}

\item  Melvyn B. Nathanson  \\
\\ \textbf{Sumsets contained in sets of upper Banach density 1} \texttt{arXiv:1708.01905} \\\\
\textbf{Additive Number Theory}
\begin{itemize}
\item  The Classical Bases (GTM \# 164) Springer, 1996. 
\item  Inverse Problems and the Geometry of Sumsets (GTM \# 165) Springer, 1997.
\end{itemize}

\item Mikhail Gromov. \textbf{Colourful Categories} \\
Uspekhi Mat. Nauk, 	\hspace{0.85in} 2015,Volume 70,	Issue 4	\\
Russian Mathematical Surveys 2015, ----- " ----

\item Terrence Tao, Van Vu \textbf{Additive Combinatorics} (Cambridge Studies in Advanced Mathematics \#105) Cambridge University Press, 2010. 

\end{thebibliography}

\end{document}