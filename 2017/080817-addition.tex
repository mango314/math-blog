\documentclass[12pt]{article}
%Gummi|065|=)
\usepackage{amsmath, amsfonts, amssymb}
\usepackage[margin=0.5in]{geometry}
\usepackage{xcolor}
\usepackage{graphicx}

\usepackage{pifont}
\usepackage{amsmath}

\newcommand{\off}[1]{}
\DeclareMathSizes{20}{30}{20}{18}

\newcommand{\two }{\sqrt[3]{2}}
\newcommand{\four}{\sqrt[3]{4}}
\newcommand{\red}{\begin{tikz}[scale=0.25]
\draw[fill=red, color=red] (0,0)--(1,0)--(1,1)--(0,1)--cycle;\end{tikz}}
\newcommand{\blue}{\begin{tikz}[scale=0.25]
\draw[fill=blue, color=blue] (0,0)--(1,0)--(1,1)--(0,1)--cycle;\end{tikz}}
\newcommand{\green}{\begin{tikz}[scale=0.25]
\draw[fill=green, color=green] (0,0)--(1,0)--(1,1)--(0,1)--cycle;\end{tikz}}

\newcommand{\sq}[3]{\draw[#3] (#1,#2)--(#1+1,#2)--(#1+1,#2+1)--(#1,#2+1)--cycle;}
\newcommand{\linebrk}{----------------------------------------------------------------------------------------------------------------------------------}


\usepackage{tikz}

\newcommand{\susy}{{\bf Q}}
\newcommand{\RV}{{\text{R}_\text{V}}}

\title{Tutorial : Additive Combinatorics}
\date{}
\begin{document}

\fontfamily{qag}\selectfont \fontsize{12.5}{15}\selectfont

\maketitle

\noindent Given a set $A, B \subset \mathbb{Z}$ we can define $A + B = \{ a + b: a \in A , b \in B \}$.  \\ \\
As I read it this construction, leading to {\color{yellow!60!black}\text{additive combinatorics}} is equivalent to Fourier analysis, to PDE, parts of graph theory, dynmical systems and many other things.  I have never followed through with taking a Fourier anlysis proof and replacing them with additive combinatorics results.  It certainly does not look like they are powerful enough to replace one with the other. \\ \\
Melyvn Nathanson has written two nice books on Additive Number Theory.  I do have trouble connecting them to objects that I would call ``additive": 
$$ 149 = 81 + 2 \times 27 + 9 + 3 + 2 \times 1= 3^4 = 2 \times 3^3 + 3^2 + 3 + 2 \times 3^0 =  12112_3 $$
This is just me being na\"{i}ve.  The other problem is once I have result how to I place it into context.  Context for whom?
\begin{itemize}
\item the guy on the street
\item an engineer
\item other mathematicians
\end{itemize}
``Context" will mean something different to everbody.  Looking at Nathanson's work now, I realize addition is all that we really have and the main struggle is to but 2+2 together.

\vfill

\begin{thebibliography}{}

\item  Melvyn B. Nathanson \textbf{Sumsets contained in sets of upper Banach density 1} \texttt{arXiv:1708.01905}

\item Mikhail Gromov. \textbf{Colourful Categories} \\
Uspekhi Mat. Nauk, 	\hspace{0.85in} 2015,Volume 70,	Issue 4	\\
Russian Mathematical Surveys 2015, ----- " ----

\item Terrence Tao, Van Vu \textbf{Additive Combinatorics} (Cambridge Studies in Advanced Mathematics \#105) Cambridge University Press, 2010. 

\end{thebibliography}

\end{document}