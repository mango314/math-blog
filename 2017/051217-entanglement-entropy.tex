\documentclass[12pt]{article}
%Gummi|065|=)
\usepackage{amsmath, amsfonts, amssymb}
\usepackage[margin=0.5in]{geometry}
\usepackage{xcolor}
\usepackage{graphicx}

\newcommand{\off}[1]{}
\DeclareMathSizes{20}{30}{20}{18}

\newcommand{\two }{\sqrt[3]{2}}
\newcommand{\four}{\sqrt[3]{4}}


\usepackage{tikz}

\title{What is Entanglement Entropy?}
\author{John D Mangual}
\date{}
\begin{document}

\fontfamily{qag}\selectfont \fontsize{12.5}{15}\selectfont

\maketitle

\noindent I have no idea what entangelement entropy really is.  I just think it's good time to review it.\footnote{This is a cause for discomfort: trying things out before we really know what we are doing or why.  It's the reverse order of many people.}  \\ \\
Appendix A through D cover different aspects of theta functions.  \\ \\ 
$$ \begin{array}{ccrcr}
\theta_1(z|\tau) & = & {\displaystyle (-i)\sum_{n \in \mathbb{Z} + \frac{1}{2}} }(-1)^{n - \frac{1}{2}} \, q^{n^2/2} \, e^{2\pi i \, n z} & = & 
{ \displaystyle 2 \sin (\pi z) \; q^{1/8} \prod_{n \in \mathbb{Z} } \big(1-q^n\big) \big( 1 - 2q^n \cos (2\pi z) + q^{2n} \big) } \\ 
\theta_2(z|\tau) & = & {\displaystyle \sum_{n \in \mathbb{Z} + \frac{1}{2}} } q^{n^2/2} \, e^{2\pi i \, n z} & = & 
{ \displaystyle 2 \cos (\pi z) \; q^{1/8} \prod_{n \in \mathbb{Z} } \big(1-q^n\big) \big( 1 + 2q^n \cos (2\pi z) + q^{2n} \big) } \\ 
\theta_3(z|\tau) & = & {\displaystyle \sum_{n \in \mathbb{Z} } \;q^{n^2/2} \, e^{2\pi i \, n z} } & = & 
{ \displaystyle  q^{1/8} \prod_{n \in \mathbb{Z} + \frac{1}{2}}  \big(1-q^n\big) \big( 1 + 2q^n \cos (2\pi z) + q^{2n} \big) } \\ 
\theta_4(z|\tau) & = & {\displaystyle \sum_{n \in \mathbb{Z} }(-1)^{n } \, q^{n^2/2} \, e^{2\pi i \, n z} } & = & 
{ \displaystyle  q^{1/8} \prod_{n \in \mathbb{Z} + \frac{1}{2}}  \big(1-q^n\big) \big( 1 - 2q^n \cos (2\pi z) + q^{2n} \big) } 
\end{array}$$
The review runs about 6 pages.  Yet, there are books and books on theta functions and modular forms.  I can't do new physics -- I'm just not qualified, but maybe I can give their equations a bit of a manicure\footnote{In certain situations, both the Math and Physics sides are highly developed.  It's very difficult to chime in.  I am hoping here there is less competition.  Bluntly.}.  Conformal Field Theory (CFT) is a bit of a vague object, but we can look at what they did.  That is the goal of of this writing exercise. 

\vfill

\begin{thebibliography}{}

\item Mukund Rangamani, Tadashi Takayanagi. \textbf{Holographic Entanglement Entropy} \texttt{arXiv:1609.01287}

\item Michael Gutperle, John D. Miller. \textbf{Entanglement entropy at CFT junctions} \texttt{arXiv:1701.08856}





\end{thebibliography}

\end{document}