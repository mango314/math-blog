\documentclass[12pt]{article}
%Gummi|065|=)
\usepackage{amsmath, amsfonts, amssymb}
\usepackage[margin=0.5in]{geometry}
\usepackage{xcolor}
\usepackage{graphicx}

\newcommand{\off}[1]{}
\DeclareMathSizes{20}{30}{20}{18}

\newcommand{\two }{\sqrt[3]{2}}
\newcommand{\four}{\sqrt[3]{4}}
\newcommand{\red}{\begin{tikz}[scale=0.25]
\draw[fill=red, color=red] (0,0)--(1,0)--(1,1)--(0,1)--cycle;\end{tikz}}
\newcommand{\blue}{\begin{tikz}[scale=0.25]
\draw[fill=blue, color=blue] (0,0)--(1,0)--(1,1)--(0,1)--cycle;\end{tikz}}
\newcommand{\green}{\begin{tikz}[scale=0.25]
\draw[fill=green, color=green] (0,0)--(1,0)--(1,1)--(0,1)--cycle;\end{tikz}}

\newcommand{\sq}[3]{\draw[#3] (#1,#2)--(#1+1,#2)--(#1+1,#2+1)--(#1,#2+1)--cycle;}

\usepackage{tikz}

\newcommand{\susy}{{\bf Q}}
\newcommand{\RV}{{\text{R}_\text{V}}}

\title{Worksheet: Exotic Number Systems}
\author{John D Mangual}
\date{}
\begin{document}

\fontfamily{qag}\selectfont \fontsize{12.5}{15}\selectfont

\maketitle

\noindent Something truly exotic challenges your norms a little bit\dots makes one feel uneasy. \\ \\
The goal of this project is to study new numbers and number-like things.  We only try a small fraction of the ideas available. And there's no guarantee this will work. \\ \\
Our decimal system is intimately related to the multiplication by ten shift map $\times 10$ we can envision a number as a sequence of decimal digits (kind of like a factory)
$$ \pi = 3. \to  1 \to 4 \to 1 \to 5 \to 9 \to \dots $$
Each digit is connected to the last by one either by multplying by ten or shifting by 1:
\begin{itemize}
\item $T: x \mapsto (10 \times x ) \pmod 1 $  with  $x \in \mathbb{R}$\\ 
\item $T: (x_0, x_1, x_2, \dots ) \mapsto (x_1, x_2, x_3, \dots ) $ with $x \in \{0,1\}^\mathbb{N}$
\end{itemize}
This framework kind of forgets that $\mathbb{R}$ is a ring $(\mathbb{R}, + , \times)$ or even a group $(\mathbb{R}, +)$.  Regardless of our choice of $T$ we can say that rings are isomorphic. Yet, in reality we must make a choice of $T$, and it can have nothing to do with the decimal system.
$$ (x_0, x_1, x_2, \dots ) \oplus (y_0, y_1, y_2, \dots ) = \;? $$
Our goal is to explain the different ways $\mathbb{R}$ could be partitioned and try to achieve an analogue of addition with carries.  Some partitions are more ordinary, others are more unique. 

\vfill 

\begin{thebibliography}{}

\item Michael Baake, Natalie Frank , Uwe Grimm , E. Arthur Robinson 
\textbf{Geometric properties of a binary non-Pisot inflation and absence of absolutely continuous diffraction} \texttt{arXiv:1706.03976} 

\item Roy L. Adler.  \textbf{Symbolic dynamics and Markov partitions}

Bull. Amer. Math. Soc. 35 (1998), 1-56 

\item Carlos Matheus \textbf{New Numbers in M-L}\\ \texttt{https://matheuscmss.wordpress.com/2017/04/05/new-numbers-in-m-l/}

\end{thebibliography}


\end{document}