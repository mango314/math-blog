\documentclass[12pt]{article}
%Gummi|065|=)
\usepackage{amsmath, amsfonts, amssymb}
\usepackage[margin=0.5in]{geometry}
\usepackage{xcolor}
\usepackage{graphicx}

%\usepackage{pifont}
\usepackage{amsmath}

\newcommand{\off}[1]{}
\DeclareMathSizes{20}{30}{20}{18}

\newcommand{\two }{\sqrt[3]{2}}
\newcommand{\four}{\sqrt[3]{4}}
\newcommand{\red}{\begin{tikz}[scale=0.25]
\draw[fill=red, color=red] (0,0)--(1,0)--(1,1)--(0,1)--cycle;\end{tikz}}
\newcommand{\blue}{\begin{tikz}[scale=0.25]
\draw[fill=blue, color=blue] (0,0)--(1,0)--(1,1)--(0,1)--cycle;\end{tikz}}
\newcommand{\green}{\begin{tikz}[scale=0.25]
\draw[fill=green, color=green] (0,0)--(1,0)--(1,1)--(0,1)--cycle;\end{tikz}}

\newcommand{\sq}[3]{\draw[#3] (#1,#2)--(#1+1,#2)--(#1+1,#2+1)--(#1,#2+1)--cycle;}

\usepackage{tikz}

\newcommand{\susy}{{\bf Q}}
\newcommand{\RV}{{\text{R}_\text{V}}}

\title{Scratchwork: Theta Functions}
\date{}
\begin{document}

%\fontfamily{qag}\selectfont \fontsize{12.5}{15}\selectfont

\sffamily

\maketitle

\noindent William Duke's proof that the solutions to $n = a^2 + b^2 + c^2$ become equidistributed as $n \to \infty$ takes a quarter of a page:
\begin{itemize}
\item Goro Shimura shows there are many theta functions, each invariant under $\Gamma_0(4)$
$$ \theta(z;u) = \sum_{m \in \mathbb{Z}^3} u(m) \, e(z |m|^2) =
\sum_{n > 0} n^{\ell / 2}\, r_3(n) \left[ \frac{1}{r_3(n)} \sum_{\xi \in V_3(n)} u(\xi) \right] e ( n z)  $$
one for each spherical Harmonic $u \in L^2( S^2)$. Here $|m|^2 = m_1^2 + m_2^2 + m_3^2$ and $V_3(n) = \#\{ (a,b,c): a^2 + b^2 + c^2 = m \}$.
\item Henryk Iwaniec offers a bound for the Fourier coffients of cusp forms. 
$$ a_n \ll_{k, \epsilon} n^{k/2 - 2/7 + \epsilon} $$
These are tending to zero if we fix a tolerance  ($\epsilon$) and a ``weight" of modular form ($k$).
\item Combining Iwaniec and Shimura's result\footnote{and an estimate of Siegel, which I haven't even looked at, $r_3(n) \gg_\epsilon \sqrt{n^{1-\epsilon}} $.} we obtain an estimate for the sphere averages
$$  \frac{1}{r_3(n)} \sum_{\xi \in V_3(n)} u(\xi) \ll_{u,\epsilon} n^{-1/28 + \epsilon} $$
This bound depends on the spherical harmonic ($u$) and the tolerance ($\epsilon$).  And we need $n \not \equiv 7 \pmod 8$.  
\end{itemize}
There have been many surprises along the way learning this topic.  And I have problems with a lot of this discussion, because these are professors talking to other professors.  Slowly turning these objections into contributions of my own! 
\begin{itemize}
\item $\theta(z)$ is $\Gamma_0(4)$ invariant not $SL(2, \mathbb{Z})$ invariant.   As subgroups the index $[SL(2, \mathbb{Z}): \Gamma_0(4)] = 6 $.  For the record $\Gamma_0(4) \simeq \langle z \mapsto z + 1, z \mapsto - \frac{1}{4z}\rangle$ while $SL(2, \mathbb{Z}) \simeq \langle z \mapsto z + 1, z \mapsto - \frac{1}{z}\rangle$.  These are like continued fractions $a = [ a_0; a_1, \dots, a_n] $ all of whose digits are multiples of $4$.
\item $\theta(z)$ is \textit{not} a cusp form, but if $\deg u > 0$ then $\theta(z;u)$ \textbf{is} a cusp form.  The Fourier series \textit{is} the $q$-series as $q = e^{2\pi i t}$ is the change of variables.
$$ \theta(z; u) = 0 + a_1 \, q + a_2 \, q^2 + \dots $$
\item Even though we are studying equidistribution of solutions to an equation $f(x) = n$ we don't necessarily study the existence of {\color{green} one} solution. After reading Serre's \textbf{Course on Arithemtic} (GTM \#85) we learn that the existence of one solution is a rather deep problem.  Lastly, I don't think it's ``done" by the time we discuss equidistribution. 
\item I didn't know what the symbol ``$\ll$" meant.  $1 \ll 100$ and also $100 \ll 1$.  Here's one more: $10^{100} \ll x$ since *eventually* $x > 10^{100}$.  These are things we learn in beginning math classes, but sometimes a brand new symbol is introduced and we have to do it again.
\end{itemize}

\newpage

\noindent Then I find out all of this is slightly out of date.  This one pricked my last nerve.  I believe, in the process of verifying Duke's claim (leaning me through Iwaniec and Shimura and Walspurger and even more) we have made new lemmas.  One of them is definitnely new.\\ \\
OK.  Here's a problem statement: let's try to find the constant to go with the $\ll$ sign:
$$  \left[ a_n  \ll n^{k/2- 2/7 + \epsilon} \right] \to \left[ a_n  <  C \, n^{k/2- 2/7 + \epsilon}\right]$$
The constant $C$ is not known (probably because nobody cares) if we fix ($\epsilon$) and  ($k$). \\ \\
Shimura's constructions are the start of the \textbf{theta correspondence}, and I'm choosing to use the out-of-date version that doesn't use any of Waldspurger's technology.  Duke makes no use of the adeles, $\mathbb{A}$; any strategy involving them will be new.  Iwaniec makes heavy use of Eisenstein series and it's rather mysterious:
$$ [\text{theta functions}] \to [\text{Eisenstein series}] $$
Iwaniec says it works, making such a kind of map, risk-free, but do we really understand it?  I don't immediately have a problem that is unkown about it.  The more we unpack, the scarier it gets.
$$ [\text{Kloosterman sums}] \quad\big|\big|\quad [\text{Bessel functions}] $$
Do we really understand this?  In the case of 4-squares, the proof involves the Weil conjectures, yet in 3-squares this just falls out of some very tricky averaging procedures, that I do not wish to replicate.\footnote{at this time} \\ \\
This discussion has created some deep and lingering doubts: do I understand Weyl's equidistrubiton criterion \textit{for spheres} or requirements for Poisson summation?  Do I know what it was so important that $\theta(z;u)$ was a \textbf{holomorphic} cusp form, or how to get number-theoretic information out of that?  Not really.
\end{document}