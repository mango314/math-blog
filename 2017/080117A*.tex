\documentclass[12pt]{article}
%Gummi|065|=)
\usepackage{amsmath, amsfonts, amssymb}
\usepackage[margin=0.5in]{geometry}
\usepackage{xcolor}
\usepackage{graphicx}

\newcommand{\off}[1]{}
\DeclareMathSizes{20}{30}{20}{18}

\newcommand{\two }{\sqrt[3]{2}}
\newcommand{\four}{\sqrt[3]{4}}
\newcommand{\red}{\begin{tikz}[scale=0.25]
\draw[fill=red, color=red] (0,0)--(1,0)--(1,1)--(0,1)--cycle;\end{tikz}}
\newcommand{\blue}{\begin{tikz}[scale=0.25]
\draw[fill=blue, color=blue] (0,0)--(1,0)--(1,1)--(0,1)--cycle;\end{tikz}}
\newcommand{\green}{\begin{tikz}[scale=0.25]
\draw[fill=green, color=green] (0,0)--(1,0)--(1,1)--(0,1)--cycle;\end{tikz}}

\newcommand{\sq}[3]{\draw[#3] (#1,#2)--(#1+1,#2)--(#1+1,#2+1)--(#1,#2+1)--cycle;}

\usepackage{tikz}

\newcommand{\susy}{{\bf Q}}
\newcommand{\RV}{{\text{R}_\text{V}}}

\title{Worksheet: Entanglement Entropy}
\author{John D Mangual}
\date{}
\begin{document}

\fontfamily{qag}\selectfont \fontsize{12.5}{15}\selectfont

\maketitle

\noindent I'd like to do a chops piece.  There is a lot of discussion of entanglement entropy and it's written in foreign language.  We just choose a point of contact and try it:
$$ \langle \langle S | q^{\sum_{i=1}^N (L^i_0 + \overline{L}^i_0 - \frac{1}{12} ) }   | S \rangle \rangle = g^2 \, \Theta_\Lambda (2it) \, \big[\eta(2it) \big]^{-N} $$
I don't know why they wrote a thing, or the meaning of certain symbols such as $| S \rangle\rangle$.   I have seen the symbol $L_0$ but I don't know quite what it's doing here.  $\Theta$ functions are very common place in math, but they are not the only modular forms in existence. \\ \\
It's very very difficult to say what these people are writing about.  I write a similar proposal before and never did anything with it. \\ \\
This is \textbf{Conformal Field Theory}.  Depending on who you ask, it is on solid or shaky foundation. We have this other buzzword that we like {\color{red!50!yellow!50!green}\textbf{Entanglement Entropy}}. \\ \\
Let's say it one more time, ``Entanglement Entropy".  Feels good, don't it? \\ \\
\textbf{07/06} What exactly does the paper say and how did it contribut to science?  I read the paper the way an student might do a critical thinking exercise.  Looking for words and cues.   Likely, missing the point entirely. Here's what I got out of it:
\begin{itemize}
\item the compute Entanglement entropy of ``permeable junctions" of Conformal Field Theories. Whatever that means, that number will be calulated through a procedure involving a mix of geometry, algebra and calculus.  
\item We know how to compute entanglement entropy for $N = 2$ junctions and they will consider $N \geq 3$.  They only consider the \textbf{free boson}  and \textbf{free fermion}.  At the end of the paper they ask how subset $A \subseteq \{ 1, 2, \dots N\}$. entangle with itself (with it's conjugates $B = \overline{A}$).
\item What do these junctures look like? They only exhibit $N = 3$.  Maybe someone might like to hear about $N=4$.
\end{itemize}
If I assume this paper is at the forefront of the literature, then we can guess what is new or interesting.  I'm likely to be way off.  Also, there was no analysis (in the sense of $\epsilon$-$\delta$ calculations.  You can always rattle at the CFT foundations, especially for the free boson and free fermion, one of the few cases we can understand. \\ \\
There's a lot of jargon I don't know such.  Ramand vs Neveu-Schwartz junctions. There are also discussions about ``rational" conformal field theory -- if we had any idea what those were. 

\newpage

\noindent So\dots what are ``junctions" exactly?

 
\vfill

\begin{thebibliography}{}

\item Michael Gutperle, John D. Miller \textbf{Entanglement entropy at CFT junctions} \texttt{arXiv:1701.08856}

\item Mukund Rangamani, Tadashi Takayanagi. \\ \textbf{Holographic Entanglement Entropy} (Book) \texttt{arXiv:1609.01287} \\ \\
Tatsuma Nishioka, Shinsei Ryu, Tadashi Takayanagi \\ \textbf{Holographic Entanglement Entropy: An Overview} \texttt{arXiv:0905.0932}

\item Adam R. Brown, Leonard Susskind \textbf{The Second Law of Quantum Complexity} \texttt{arXiv:1701.01107}

\item Sunil Mukhi, Sameer Murthy, Jie-Qiang Wu \textbf{Entanglement, Replicas, and Thetas} \texttt{arXiv:1706.09426}

\end{thebibliography}

\end{document}