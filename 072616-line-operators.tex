\documentclass[12pt]{article}
%Gummi|065|=)
\usepackage{amsmath, amsfonts, amssymb}
\usepackage[landscape, margin=0.5in]{geometry}
\usepackage{xcolor}
\usepackage{graphicx}
\newcommand{\off}[1]{}
\DeclareMathSizes{20}{30}{21}{18}

\title{\textbf{ Rauzy-Veech Identities }}
\author{John D Mangual}
\date{}
\begin{document}

\fontfamily{qag}\selectfont \fontsize{25}{30}\selectfont

\maketitle

\noindent I was reading a string theory paper and checking the references.  The formula didn't make much sense, and upon closer inspection the diagram from one set of papers I was reading looked exactly like th other.


\newpage

\noindent\textbf{S-duality} and \textbf{T-duality}

\includegraphics[width=7in]{/home/john/Documents/math/veech-01.png}


\newpage

\noindent\textbf{Rauzy-Veech Induction}

\includegraphics[width=7in]{/home/john/Documents/math/veech-02.png}


\newpage

\fontfamily{qag}\selectfont \fontsize{12}{10}\selectfont

\begin{thebibliography}{}


\item Ofer Aharony, Nathan Seiberg, Yuji Tachikawa. \newline \textbf{Reading between the lines of four-dimensional gauge theories} \texttt{arXiv:1305.0318}

\item  Artur Avila, Carlos Matheus, Jean-Christophe Yoccoz. \newline \textbf{Zorich conjecture for hyperelliptic Rauzy-Veech groups.}  \texttt{arXiv:1606.01227}

\item JC Yoccoz \textbf{Interval exchange maps and translation surfaces} from \newline 
\textbf{Homogeneous Flows, Moduli Spaes and Arithmetic} \texttt{http://www.claymath.org/library/proceedings/cmip010c.pdf}


\end{thebibliography}


\end{document}

