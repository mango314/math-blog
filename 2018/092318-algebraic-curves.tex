\documentclass[12pt]{article}
%Gumm{\color{blue}i}|065|=)
\usepackage{amsmath, amsfonts, amssymb}
\usepackage[margin=0.5in]{geometry}
\usepackage{xcolor}
\usepackage{graphicx}
\usepackage{amsmath}

\newcommand{\off}[1]{}
\DeclareMathSizes{20}{30}{20}{18}
\usepackage{tikz}


\title{Scratchwork: Algebraic Curves}
\date{}
\begin{document}

\sffamily

\maketitle

\noindent \textbf{9/23} Can we think of an algebraic curve besides the circle?  Let's try a polar coordinates circle:
\begin{eqnarray*}
x &=& \cos \theta \\
y &=& \sin \theta \\
z &=& 0
\end{eqnarray*} 
What if we add another circle, moving around the plane generated by the height and the radius of the circle. $e_{\vec{r}} = e_{\vec{x}} \cos \theta + e_{\vec{y}} \sin \theta $  This is a moving coordinate plane, and we can draw a circle that moving plane.
\begin{eqnarray*}
r &=& \epsilon \, \cos \phi \\
z &=& \epsilon \, \sin \phi
\end{eqnarray*} 
where $\epsilon \ll 1$. Then we can put the moving plane back into a 3D ambient space.  The new coordinate equation are
\begin{eqnarray*}
x &=& \cos \theta + \epsilon \, \cos \theta \, \cos \phi \\ 
y &=& \sin \theta + \epsilon \, \sin \theta \, \cos \phi \\ 
z &=& \hspace{0.25in}0 + \epsilon \, \sin \phi  
\end{eqnarray*}
We claim this is an ``algebraic" curve.  If we let $\phi = 3 \theta$ or anything like that, we could solve for one (or a few) equations $f \in \mathbb{R}[x,y,z]$ such that $f(x,y,z) = 0$. \\ \\
Unfortunately, algebraic geometry requres that $x,y,z \in \mathbb{C}$ meaning the curve lives in six dimensional space.  And we'll ignore them and ask for points on this curve with $x,y,z \in \mathbb{Q}$. So now this is an arithemtic scheme. \\ \\
\texttt{[draw picture]} \begin{tikzpicture} \end{tikzpicture}

\begin{thebibliography}{}

\item \dots

\end{thebibliography}
\end{document}