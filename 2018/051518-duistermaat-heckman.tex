\documentclass[12pt]{article}
%Gummi|065|=)
\usepackage{amsmath, amsfonts, amssymb}
\usepackage[margin=0.5in]{geometry}
\usepackage{xcolor}
\usepackage{graphicx}

% zeta functions of cubic fields

%\usepackage{pifont}
\usepackage{amsmath}

\newcommand{\off}[1]{}
\DeclareMathSizes{20}{30}{20}{18}

\newcommand{\two }{\sqrt[3]{2}}
\newcommand{\four}{\sqrt[3]{4}}
\newcommand{\red}{\begin{tikz}[scale=0.25]
\draw[fill=red, color=red] (0,0)--(1,0)--(1,1)--(0,1)--cycle;\end{tikz}}
\newcommand{\blue}{\begin{tikz}[scale=0.25]
\draw[fill=blue, color=blue] (0,0)--(1,0)--(1,1)--(0,1)--cycle;\end{tikz}}
\newcommand{\green}{\begin{tikz}[scale=0.25]
\draw[fill=green, color=green] (0,0)--(1,0)--(1,1)--(0,1)--cycle;\end{tikz}}

\newcommand{\sq}[3]{\draw[#3] (#1,#2)--(#1+1,#2)--(#1+1,#2+1)--(#1,#2+1)--cycle;}

\usepackage{tikz}

\newcommand{\susy}{{\bf Q}}
\newcommand{\RV}{{\text{R}_\text{V}}}

\title{Scratchwork: Duistermaat Heckman Formula}
\date{}
\begin{document}

%\fontfamily{qag}\selectfont \fontsize{12.5}{15}\selectfont

\sffamily

\maketitle

\noindent If Physics describes everything and Gauge Theory describes all of physics, then Gauge Theory should map to everything.  The original scope of Gauge Theory is particle physics and trying to describe everything going on there.  In order to build a computer or store a light bulb, we need to study the various fundamental particles. \\ \\ 
A ``particle" at this level of formality is a solution to a partial differential equation - which could have many unexpected symmetries.  Or these PDE solutions could appear in unexpected places.  One of the tenets of this theory is that studying such solutions is fundamentally a good thing. \\ \\
In 1982, we have the Duistermaat-Heckman formula\dots $M$ is a symplectic manifold of dimension $2n$. There's a symplectic form $\omega$.  There's a Hamiltonian $U(1)$ action and the moment map $\mu$.  The fixed points of the of the $U(1)$ action are called $x_i$ and $e(x_i)$ is the produt of the weights of the $U(1)$ action on the tangent space at $x_i$.
$$ \int_M \frac{\omega^n}{n!}\, e^{-\mu} = \sum \frac{e^{-\mu(x_i)}}{e(x_i)} $$ 
We get there was no Duistermaat-Heckman formula before 1980.  As for Symplectic geometry:
\begin{quotation}
Symplectic geometry has its origins in the Hamiltonian formulation of classical mechanics where the phase space of certain classical systems takes on the structure of a symplectic manifold. \\ \\
The term "symplectic" is a calque of "complex", introduced by Weyl (1939, footnote, p.165); previously, the "symplectic group" had been called the "line complex group".  (Wikipedia)
\end{quotation}
The most basic symplectic manifold is $\mathbb{R}^2$ with $\omega = dp \wedge dq$ and the $U(1)$ action could be a rotation 
$$\left[\begin{array}{cr} \cos \theta & -\sin \theta \\ \sin \theta & \cos \theta \end{array} \right] $$
The variable $p$ was called "\textit{position}" and the variable $q = \dot{p}$ was called "\textit{momentum}" and therefore $p \leftrightarrow q$ form a position-momentum pair.  Then in quantum mechanics this ``symplectic pairing" gets quantized:
$$ x \leftrightarrow i\hbar\frac{d}{dx} $$
One-dimensional particle motion is also described by something called a ``Hamiltonian" which is just the kinetic plus potential energy $H = T + U$.  In fact, for any manifold $M$ we could have a cotangent bundle $T^\ast M$ and these are always symplectic manifolds.   \\ \\
Hopefully, we can continue to unfurl the definitions of the Duistermaat-Heckman formula.
\vfill

\begin{thebibliography}{}

\item Vasily Pestun, Maxim Zabzine. 
"Introduction to localization in quantum field theory" \texttt{arXiv:1608.02953}

\item Yuji Tachikawa "N=2 supersymmetric dynamics for pedestrians" \texttt{arXiv:1312.2684}
\end{thebibliography}

\newpage

\noindent There's a few jargons floating around maybe I can list a few of them.
\begin{itemize}
\item complex manifold
\item symplectic manifold
\item spin manifold
\item contact manifold
\item Poisson manifold
\item K\"{a}hler manifold
\end{itemize}
and there's even more.  All these jargons, at least in the beginning should encode highly desirable traits of spaces of objects.  We need to consider spaces, because things move around, depend on more than one thing, and we're never 100\% sure which one we're going to pick.\footnote{
In the literature, the papers will be there, and the authors will be technically competent and far-reaching.  The typical person knows nothing except for some ``common sense" results.  Then there is a corona of readers who know just a enough to peek at the matertial, but no technical mastery.  Arguably these are the \textit{worst} because their errors are the most glaring. Then we have experts that are so far into it, that they lose sight of the beginning motivations.  \\ \\
There's nobody left then.} \\ \\
And now for a quick trip to the library (a.k.a. Google) and we find a few resources immediately:
\begin{itemize}
\item Anna Cannas da Silva \\ \textbf{Lectures on Symplectic Geometry}  (Lecture Notes in Mathematics \#1764) Springer, 2008.
\item Victor Guillemin and Sholomo Sternberg. \\
\textbf{Supersymmetry and equivariant de Rham theory} Springer, 1999
\item Richard Szabo \textbf{Equivariant Localization of Path Integrals} \texttt{ arXiv:hep-th/9608068}
\item Loring Tu \\ \textbf{What is... Equivariant Cohomology?} Notices of the American Mathematical Society, March 2011.
\end{itemize}
There are a few experts who do this every day and an overwhleming majority who have no idea.  There were 160 Geometry and Topology PhDs (129 men + 31 women) awarded in the US in 2013-2014 academic year out of 1926 Math PhD's total, that year.  That's it. \\ \\
Anna Cannas da Silva places her discussion of Sympletic Geometry (written for ``beginners") right and the end of the book, at the end of her discussion fo {\color{blue}symplectic toric manifolds}.  That makes total sense since we require a $U(1)$ actioin for the Duistermaat-Heckmann theorem to even be stated.
\begin{itemize}
\item $M$ is a sympletic manifold
\item $\omega$ is the symplectic form
\item $\mu$ is the moment map of the $U(1)$ action.
\end{itemize}
and I don't know of any example in enough detail to state the Duistermaat-Heckman formula in that case.  \\ \\ We don't get a formula for $M, \omega, \mu$.  I'd like to know what the polytope relates to $\mu$, or the differential equation for the Hamiltonian vector field generating the $U(1)$ action.  Lastly, does this ``hamiltonian" match the definition of the Hamiltonian in the physics textbook.  She doesn't say no:
\begin{quotation} 
{\color{green!75!black}\textbf{Two centuries ago, symplectic geometry provided a language for classical mechanics}}.
Through its recent huge development, it conquered an independent and
rich territory, as a central branch of differential geometry and topology. To mention
just a few key landmarks, one may say that symplectic geometry began to take its
modern shape with the formulation of the Arnold conjectures in the 60’s and with
the foundational work of Weinstein in the 70’s. A paper of Gromov [49] in the 80’s
gave the subject a whole new set of tools: pseudo-holomorphic curves. Gromov also
first showed that important results from complex K¨ahler geometry remain true in the
more general symplectic category, and this direction was continued rather dramatically
in the 90’s in the work of Donaldson on the topology of symplectic manifolds
and their symplectic submanifolds, and in the work of Taubes in the context of
the Seiberg-Witten invariants. Symplectic geometry is significantly stimulated by
important interactions with {\color{orange}global analysis}, {\color{blue}mathematical physics}, {\color{orange!75!black}low-dimensional
topology}, {\color{blue}dynamical systems}, {\color{orange!50!black}algebraic geometry}, {\color{blue!50!black}integrable systems}, {\color{orange}microlocal
analysis}, {\color{blue!50!black}partial differential equations}, {\color{orange}representation theory}, {\color{blue}quantization}, {\color{orange!50!black}equivariant cohomology}, {\color{blue}geometric combinatorics}, etc. \\ \\
As a curiosity, note that two centuries ago the name symplectic geometry did
not exist. If you consult a major English dictionary, you are likely to find that
symplectic is the name for a bone in a fish's head.
\end{quotation}
As long as you stay within Mathematics, there's a lot of interesting things going on.  I take as a given that Mathematics describes things you see every day and a few exotic things.
\end{document}