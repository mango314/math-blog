\documentclass[12pt]{article}
%Gumm{\color{blue}i}|065|=)
\usepackage{amsmath, amsfonts, amssymb}
\usepackage[margin=0.5in]{geometry}
\usepackage{xcolor}
\usepackage{graphicx}
\usepackage{amsmath}

\newcommand{\off}[1]{}
\DeclareMathSizes{20}{30}{20}{18}
\usepackage{tikz}


\title{Scratchwork: Fourier Series on $\mathbb{A}$}
\date{}
\begin{document}

%\fontfam{\color{blue}i}ly{qag}\selectfont \fonts{\color{blue}i}ze{12.5}{15}\selectfont

\sffamily

\maketitle

\noindent Let's try to come to terms with fourier series on $\mathbb{Q}$ and Fourier series on $\mathbb{A}$.  Fourier Analylsis textbooks quite thoroughly discuss Fourier series over the circle $S^1$ or over the real line $\mathbb{R}$.  They might even discuss Discrete Fourier Analysis on $\mathbb{Z}/p\mathbb{Z}$ and the Fast Fourier Transform.\\ \\
When we invoke a Fourier analysis theorem - in context - it's likely to reduce to properties of $\mathbb{R}$ (or another set of numbers) as an ordered set and the interval arithmetic that we are using.   E.g. when we say a $\widehat{f}(\xi)$ is continuous in $\xi$. \\ \\
Some people might consider it trivial that Pythagorean triples can be indexed by two numbers:
$$ a^2 + b^2 = c^2 \text{ then } a = m^2 - n^2, b = 2mn, c = m^2 + n^2 $$
This looks OK. It reduces to a single algebraic identity that works in just about any ring $m, n \in R$ 
$$ (m^2 - n^2)^2 + (2mn)^2 = (m^2 + n^2)^2 $$
This means we have a paramterization of rational points on the circle related to Pythagoras:
$$ x^2 + y^2 = 1 \text{ then } (x,y) = (\frac{a}{c}, \frac{b}{c}) \text{ with }a^2 +b^2 = c $$
This is even an affine variety over $\mathbb{Q}$ (possibly we already need to say ``scheme").  We could define the sheaf of regular functions:
$$ \mathcal{O}_X \text{ for } X(\mathbb{Q}) \text{ where } X = \{ x^2 + y^2 - 1 = 0 \}$$
These are but highly formal restatements of what we had done in high school.  To all it a variety or scheme is to invoke something like the Nullstellensatz:
$$ p(x) = 0 \text{ on }\{ x^2 + y^2 - 1 = 0 \} \to p(x) \in \mathbb{Q}[x,y](x^2 + y^2 - 1) $$
These might not even be true as stated.  \\ \\
Fourier Analysis on the Adeles is then no small feat.  Can we write down test functions to take the Fourier analysis.  What is $C^\infty$ or ``smooth" mean in this setting?
\begin{itemize}
\item $(x,y) = (a/c,b/c) \mapsto a^4 + b^4 + c^4$
\item $(x,y) = (a/c,b/c) \mapsto \sigma(a) + \sigma(b) - \sigma(c)$ where $\sigma(n) = \sum_{m|n} 1$ is the divisor function.
\end{itemize}
These funcitons would be very badly chatotic (e.g. \textit{nowhere} continuous) on our circle $X(\mathbb{Q})$, so how can we grapple with this? \\ \\
It seems that Fourier Analysis over $\mathbb{A}$ can be no less informative then that over the real number line $\mathbb{R}$ or the circle $S^1$ and yet we don't do as much in this setting.  It suggests something should drastically simplify here?

\newpage

\noindent To get us started, let's notice that $\mathbb{R}$ is finitely generated $S^1$ is finitely generated (just rotations and shifts).  Also $\mathbb{Z}$ is finintely generated. Yet $\mathbb{Q}$ is \textit{not} finitely generated and neither is $SO(2, \mathbb{Q})$. 
$$ SO(2, \mathbb{Q}) \simeq \big(\mathbb{Z}/2\mathbb{Z}\big)\times \prod_{p = 4k+1} \big(\mathbb{Z}/p\mathbb{Z}\big) $$
showing that there are infinitely many $p = 4k+1$ primes is from a non-trivial theorem of Dirichet.\\ 
$$  \zeta_{\mathbb{Z}[i]}(s) = \left( \frac{1}{1 - \frac{1}{2^s}}\right)
\times \left( \prod_{p = 4k+1} \frac{1}{1 - \frac{1}{p^s}}\right)^2
\times \left( \prod_{p = 4k+3} \frac{1}{1 - \frac{1}{p^{2s}}}\right)$$
And we're in good position to try some actual Fourier Analysis. \\ \\
\textbf{Exercise} What are the odds of two numbers $a = a_1 + i a_2$ and $b = b_1 + i b_2$ to be relatively prime, i.e. $\text{gcd}(a,b) = 1$.  We could try running through the Euclidean algorithm over many pairs of complex numbers. \\ \\
\textbf{Example} By Ostrowski Theorem (or unique factorization) we have the product of the $p$-adic valuations is equal to $1$:
$$ x^2 + y^2 = 1 \to (x,y) = (a/c,b/c) \text{ with } a^2 + b^2 = c^2 \to \prod_p |a^4 + b^4 + c^4 |_p = 1 $$
where we could let $p$ run over the primes in $\mathbb{Q} \cup \infty$ or even the primes in $\mathbb{Q}(i) \cup \infty$.  In $\mathbb{Q}(\sqrt{2})$ there are two infinite places since we could have:
$$ |a + b \sqrt{2}|_1 = a + b \sqrt{2} \text{ or } |a + b \sqrt{2}|_1 = a - b \sqrt{2}$$

\vfill
\begin{thebibliography}{}

\item Anton Dietmar\textbf{Automorphic Forms} (Universitext)  Springer, 2012.

\item Joe Harris \\ 
\textbf{Algebraic Geometry} \hspace{0.52in}(GTM \#133)  Springer, 1992. \\
\textbf{The Geometry of Schemes} (GTM \#197) Springer, 2000.

\item Elias Stein \textbf{Fourier Series} (Princeton Lectures in Analysis \#1) Princeton University Press, 2003.

\item Ernest J. Eckert \textbf{The Group of Primitive Pythagorean Triangles} Mathematics Magazine, Vol. 57, \#1  \\
Lin Tan \textbf{The Group of Rational Points on the Unit Circle}  Mathematics Magazine Vol. 69, \#3  


\end{thebibliography}



\end{document}