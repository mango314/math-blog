\documentclass[12pt]{article}
%Gummi|065|=)
\usepackage{amsmath, amsfonts, amssymb}
\usepackage[margin=0.5in]{geometry}
\usepackage{xcolor}
\usepackage{graphicx}

% zeta functions of cubic fields

%\usepackage{pifont}
\usepackage{amsmath}

\newcommand{\off}[1]{}
\DeclareMathSizes{20}{30}{20}{18}

\newcommand{\two }{\sqrt[3]{2}}
\newcommand{\four}{\sqrt[3]{4}}
\newcommand{\red}{\begin{tikz}[scale=0.25]
\draw[fill=red, color=red] (0,0)--(1,0)--(1,1)--(0,1)--cycle;\end{tikz}}
\newcommand{\blue}{\begin{tikz}[scale=0.25]
\draw[fill=blue, color=blue] (0,0)--(1,0)--(1,1)--(0,1)--cycle;\end{tikz}}
\newcommand{\green}{\begin{tikz}[scale=0.25]
\draw[fill=green, color=green] (0,0)--(1,0)--(1,1)--(0,1)--cycle;\end{tikz}}

\newcommand{\sq}[3]{\draw[#3] (#1,#2)--(#1+1,#2)--(#1+1,#2+1)--(#1,#2+1)--cycle;}

\usepackage{tikz}

\newcommand{\susy}{{\bf Q}}
\newcommand{\RV}{{\text{R}_\text{V}}}

\title{Scratchwork: Quaternionic Shimura Varieties}
\date{}
\begin{document}

%\fontfamily{qag}\selectfont \fontsize{12.5}{15}\selectfont

\sffamily

\maketitle

\noindent We are going to try to parse whatever we can in one of the latest editions of Advances in Mathematics.  My theory is that a lot of these papers at least start off about things we are familiar with.  Too much symbolic manipulation makes me a bit critical that we've lost track. 
\begin{quotation}
The aim of this paper is to compare arithmetic intersection numbers on Shimura varieties attached to inner forms of $\text{GL}_2$ over a real quadratic field $F$. \\ \\
To illustrate our approach, first consider the arithmetic volume of a Shimura variety attached to an inner form of $\text{GL}_2$ over $\mathbb{Q}$. \\ \\ 
\dots \\\\
We begin with a division quaternion algebra $B$ over $F$  whose discriminant $D_B := (p_1,\dots,p_r)\cdot\mathcal{O}_F$ is a non-empty product of split rational primes.
\end{quotation}
While I may be grossly underestimating the difficulty of such a task, I feel there should be instances where the problem is easy to state and describe.  Even if solving is a ton of work.\footnote{Then I have a job.} Even when they solve it, they understate the meaning so much it's just totally lost.\\ \\


\vfill

\begin{thebibliography}{}

\item Gerard Freixas i Montplet, Siddarth Sankaran "Twisted Hilbert modular surfaces, arithmetic intersections and the Jacquet–Langlands correspondence" Advances in Mathematics
Volume 329, 30 April 2018, Pages 1-84

\end{thebibliography}

\end{document}