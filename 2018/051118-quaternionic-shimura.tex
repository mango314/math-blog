\documentclass[12pt]{article}
%Gummi|065|=)
\usepackage{amsmath, amsfonts, amssymb}
\usepackage[margin=0.5in]{geometry}
\usepackage{xcolor}
\usepackage{graphicx}

% zeta functions of cubic fields

%\usepackage{pifont}
\usepackage{amsmath}

\newcommand{\off}[1]{}
\DeclareMathSizes{20}{30}{20}{18}

\newcommand{\two }{\sqrt[3]{2}}
\newcommand{\four}{\sqrt[3]{4}}
\newcommand{\red}{\begin{tikz}[scale=0.25]
\draw[fill=red, color=red] (0,0)--(1,0)--(1,1)--(0,1)--cycle;\end{tikz}}
\newcommand{\blue}{\begin{tikz}[scale=0.25]
\draw[fill=blue, color=blue] (0,0)--(1,0)--(1,1)--(0,1)--cycle;\end{tikz}}
\newcommand{\green}{\begin{tikz}[scale=0.25]
\draw[fill=green, color=green] (0,0)--(1,0)--(1,1)--(0,1)--cycle;\end{tikz}}

\newcommand{\sq}[3]{\draw[#3] (#1,#2)--(#1+1,#2)--(#1+1,#2+1)--(#1,#2+1)--cycle;}

\usepackage{tikz}

\newcommand{\susy}{{\bf Q}}
\newcommand{\RV}{{\text{R}_\text{V}}}

\title{Scratchwork: Quaternionic Shimura Varieties}
\date{}
\begin{document}

%\fontfamily{qag}\selectfont \fontsize{12.5}{15}\selectfont

\sffamily

\maketitle

\noindent We are going to try to parse whatever we can in one of the latest editions of Advances in Mathematics.  My theory is that a lot of these papers at least start off about things we are familiar with.  Too much symbolic manipulation makes me a bit critical that we've lost track. 
\begin{quotation}
The aim of this paper is to compare arithmetic intersection numbers on Shimura varieties attached to inner forms of $\text{GL}_2$ over a real quadratic field $F$. \\ \\
To illustrate our approach, first consider the arithmetic volume of a Shimura variety attached to an inner form of $\text{GL}_2$ over $\mathbb{Q}$. \\ \\ 
\dots \\\\
We begin with a division quaternion algebra $B$ over $F$  whose discriminant $D_B := (p_1,\dots,p_r)\cdot\mathcal{O}_F$ is a non-empty product of split rational primes.
\end{quotation}
While I may be grossly underestimating the difficulty of such a task, I feel there should be instances where the problem is easy to state and describe.  Even if solving is a ton of work.\footnote{Then I have a job.} Even when they solve it, they understate the meaning so much it's just totally lost.\\ \\
\textbf{5/16} Because we waited, we get for free, a 1000 pages book on Quaternions by John Voight\footnote{\textit{Quaternion algebras}, current version (v.0.9.12, March 29, 2018).
 https://math.dartmouth.edu/~jvoight/quat.html}.  Between sections 1 and 2 we obtain the definitions we are looking for:
\begin{itemize}
\item $B$ is a totally indefinite division quaternion algebra
\item $\mathbf{G}(\mathbb{R}) \simeq \{ (A_1, A_2) \in \text{GL}_2(\mathbb{R}) \times 
\text{GL}_2(\mathbb{R}): \det A_1 = \det A_2 \}$
\item $X = \{ (z_1, z_2) \in \mathbb{C}^2 : \text{Im}(z_1) \cdot \text{Im}(z_2) > 0 \}$
\item There's a group action of $\textbf{G}(\mathbb{R})$ by fractional linear transformations:
$$  (A_1, A_2) \cdot (z_1, z_2) = \left( \frac{a_1 z_1 + b_1}{c_1 z_1 + d_1}, 
\frac{a_2 z_2 + b_2}{c_2 z_2 + d_2} \right) $$
\end{itemize}
Now there's two objects.  The {\color{blue!10!orange}{\textbf{complex twisted Hilbert modular surface}}} and the {\color{blue!10!green!80!black}{\textbf{quaternionic Shimura curve}}}.
\[ M_K := \textbf{G}(\mathbb{Q}) \bigg\backslash \Big[ X \times \big(
\textbf{G}(\mathbb{A}_{\mathbb{Q},f})  / K\big) \Big] \tag{{\color{blue!10!orange}{$\ast$}}} \]
and we're not even ready to discuss the other space.  Many thing here the author has stated as defintions can be turned back into \textit{exercises}  or even \textit{theorems}.   This jargon ``arithmetic intersections" should mean two equations we have in the arithmetic object we are describing\dots arrangements of the fractions over various number systems.  \\ \\
What do you think he means by $\mathbb{R}/\log | \mathbb{Q}^\times|$ could this object exist outside of number theory?  \\ \\
Continuing, let $\mathbb{H}^\pm = \{ z \in \mathbb{C} : \text{im}(z) \neq 0 \}$ be the union of the upper and lower half planes.   
$$ \Big[ \textbf{G}_1 = \text{Res}_{F/\mathbb{Q}} B_1^\times \Big] \to \Big[ \textbf{G}_1(\mathbb{R}) \simeq \mathbb{H} \times \text{GL}_2(\mathbb{R}) \Big]$$
And for the time being we have to parrot these identities because the authors are moving so quickly.  There is a group action:
$$ (h, \left( \begin{array}{cc} a & b \\ c & d \end{array} \right)) z = \frac{az+b}{cz+d} $$
and then we can define the quaternionic Shimura variety:
\[ S = \textbf{G}_1(\mathbb{Q}) \Big\backslash \big[ \mathbb{H}^\pm \times 
( \textbf{G}_1(\mathbb{A}_{\mathbb{Q},f}) / K_1) \big]  \tag{{\color{blue!10!green!80!black}{$\ast$}}} \]
I'll concede the author sort-of knows what these definitions are.  When the authors talk about a ``canonical model" not only am I not sure, I'm not even interested.  However I do know what is \textbf{fractional linear transformation} and a \textbf{quaternion}.  I even know slightly what is a {\color{black!70!white}\textbf{Todd class}}.  And so we keep reading.

\vfill

\begin{thebibliography}{}

\item Gerard Freixas i Montplet, Siddarth Sankaran "Twisted Hilbert modular surfaces, arithmetic intersections and the Jacquet–Langlands correspondence" Advances in Mathematics
Volume 329, 30 April 2018, Pages 1-84

\end{thebibliography}

\end{document}