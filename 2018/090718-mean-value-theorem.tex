\documentclass[12pt]{article}
%Gumm{\color{blue}i}|065|=)
\usepackage{amsmath, amsfonts, amssymb}
\usepackage[margin=0.5in]{geometry}
\usepackage{xcolor}
\usepackage{graphicx}

% zeta funct{\color{blue}i}ons of cub{\color{blue}i}c f{\color{blue}i}elds

%\usepackage{p{\color{blue}i}font}
\usepackage{amsmath}

\newcommand{\off}[1]{}
\DeclareMathSizes{20}{30}{20}{18}

\newcommand{\two }{\sqrt[3]{2}}
\newcommand{\four}{\sqrt[3]{4}}





\usepackage{tikz}

\newcommand{\susy}{{\bf Q}}
\newcommand{\RV}{{\text{R}_\text{V}}}

\title{Scratchwork: Mean Value Theorem}
\date{}
\begin{document}

%\fontfam{\color{blue}i}ly{qag}\selectfont \fonts{\color{blue}i}ze{12.5}{15}\selectfont

\sffamily

\maketitle

\noindent Do we understand the Mean Value Theorem at all? Let's try studying an example with sine. 
$$ f(b) + f(a) = (b-a)f'(c) \text{ with } b > c > a $$
This is used when we want to do a linear approxmation, which we usually write as:
$$ f(x+h) \approx f(x) + f'(x)\times h$$
Do we know what these operations are $+$ and $\times$?  And can we qualify this symbol $\approx$? And we still haven't set $f(x) = \sin x$.
$$ \sin (\theta +h) \approx \sin \theta + h \times \cos \theta $$
and haven't told you that $\sin'\theta = \cos \theta$.  It's just something that we religiously beleive.  These are, for example, the $x$ and $y$ coordinate charts of a circle
$$ x^2 + y^2 = 1 \text{ let's write } x = \cos \theta \text{ and } y = \sin \theta $$
as if we've never seen these objects in our lives.\footnote{Later, we might try to replace this with another algebraic curve.  It could even be birational to a circle.  To an algebraic geometer this might be dull.  Here's an example $(x,y) = (\cos 2\theta, \sin 3\theta)$ is an algebraic curve.} \\ \\
The mean value theorem let's use quantitative versions:
$$ f'(x) \approx \frac{1}{2h}\big( f(x+h) - f(x-h) \big) + \frac{h^2}{2} \times \max_{[x-h , x+h]} |f''(c)|$$
Let's look at a table of exact values of sine:
\begin{itemize}
\item $ \sin 72^\circ = \sqrt{\frac{1}{8}(5 + \sqrt{5})}$
\item $ \sin 75^\circ = \frac{\sqrt{2}}{4} \big( \sqrt{3}+1 \big)$ and $ \cos 75^\circ = \frac{\sqrt{2}}{4} (\sqrt{3}-1) $
\item $ \sin 78^\circ = \frac{1}{8}(\sqrt{30 + 6 \sqrt{5}} + \sqrt{5}-1)$
\end{itemize}
These angles are in increments of $3^\circ  = \frac{\pi}{120}$ radians.  And very innocently put in the values:
\begin{eqnarray*}
\cos 75^\circ &=& \frac{60}{\pi}\times (\sin 78^\circ - \sin 72^\circ) + (\frac{\pi}{120})^2 \times |\sin(?)| \\ \\
\frac{\pi}{60} \times \frac{\sqrt{2}}{4} \times (\sqrt{3}-1) &=&   \Big[\frac{1}{8}(\sqrt{30 + 6 \sqrt{5}} + \sqrt{5}-1)\Big]
- \Big[\sqrt{\frac{1}{8}(5 + \sqrt{5})}\Big]  \pm \frac{1}{4} \times \frac{\pi}{60}
\end{eqnarray*}
Not that $|\sin \theta| < 1$. Not quite Ramanujan level, but it's less than obvious.\footnote{There will be times when $|\sin \theta| < 1$ less than sufficient.  See if we can find ways to bootstrap such inequalities?  For the moment\dots}
\begin{thebibliography}{}

\item \dots

\end{thebibliography}



\end{document}