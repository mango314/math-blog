\documentclass[12pt]{article}
%Gummi|065|=)
\usepackage{amsmath, amsfonts, amssymb}
\usepackage[margin=0.5in]{geometry}
\usepackage{xcolor}
\usepackage{graphicx}

% zeta functions of cubic fields

%\usepackage{pifont}
\usepackage{amsmath}

\newcommand{\off}[1]{}
\DeclareMathSizes{20}{30}{20}{18}

\newcommand{\two }{\sqrt[3]{2}}
\newcommand{\four}{\sqrt[3]{4}}
\newcommand{\red}{\begin{tikz}[scale=0.25]
\draw[fill=red, color=red] (0,0)--(1,0)--(1,1)--(0,1)--cycle;\end{tikz}}
\newcommand{\blue}{\begin{tikz}[scale=0.25]
\draw[fill=blue, color=blue] (0,0)--(1,0)--(1,1)--(0,1)--cycle;\end{tikz}}
\newcommand{\green}{\begin{tikz}[scale=0.25]
\draw[fill=green, color=green] (0,0)--(1,0)--(1,1)--(0,1)--cycle;\end{tikz}}

\newcommand{\sq}[3]{\draw[#3] (#1,#2)--(#1+1,#2)--(#1+1,#2+1)--(#1,#2+1)--cycle;}

\usepackage{tikz}

\newcommand{\susy}{{\bf Q}}
\newcommand{\RV}{{\text{R}_\text{V}}}

\title{Scratchwork: Pythagorean Triples over $\mathbb{Z}[i]$}
\date{}
\begin{document}

%\fontfamily{qag}\selectfont \fontsize{12.5}{15}\selectfont

\sffamily

\maketitle

\noindent The equation $x^2 + y^2 = 1$ defines what we might call a \textbf{variety}.  Here it's just a circle.  Our decision to use Cartesian coordinates, algebra and equations, to describe Euclidean geometry, leads to all sorts of complications.  Attempts to finalize what we might call a variety leads to all sorts of difficult \textbf{commutative algebra} and ultimately \textbf{schemes}.  \\ \\
Taking for granted a minute that circles are a meaningful concept and that we should use algebra and geometry, we observe the variety $X = \{ x^2 + y^2 - 1 = 0\}$ has a rational paramterization.
$$ t \in \mathbb{Q}  \longrightarrow (x,y) = \left( \frac{1-t^2}{1 + t^2}, \frac{2t}{1+t^2} \right) \in \mathbb{Q}^2 $$
And we can think of fractions as pairs of integers $\frac{m}{n} \in \mathbb{Q}$ or as pairs of integers up to proportion\footnote{Yet another word we can scrutinize.  What are we calling ``proportionate"? } $[m:n] \in P\mathbb{Z}^2$.  Therefore we can solve an equation over integers: $a^2 + b^2 = c^2$ over $\mathbb{Z}$:
$$ (a,b,c) = (m^2 - n^2, 2mn, m^2 + n^2) \in \mathbb{Z}^3$$ 
The machinery of algebra tells us that all we required is that $K = \mathbb{Q}$ is a field.  Therefore we could also try to solve the Pythagoras equation over $\mathbb{Z}[i]$ or $\mathbb{Z}[\sqrt{2}]$ and solve it in the same way. \\ \\
What happens if we write the thing down we could write $a = a_1 + i a_2$ etc. and find:
$$ (a_1 + i a_2)^2 + (b_1 + i b_2)^2 = (c_1 + i c_2)^2 $$
and we learned we could seprate the real and imaginary components and find two equations:
\begin{eqnarray*}
(a_1^2 - a_2^2) + (b_1^2 - b_2^2)  &=& (c_1^2 - c_2^2) \\
2a_1 a_2 + 2 b_1 b_2 &=& 2 c_1 c_2 
\end{eqnarray*}
which is the intersection of two conic sections.  And here because of the structure of $\mathbb{Z}[i]$ we can expect a spectactular reduction.  Is that all?
\end{document}