\documentclass[12pt]{article}
%Gummi|065|=)
\usepackage{amsmath, amsfonts, amssymb}
\usepackage[margin=0.5in]{geometry}
\usepackage{xcolor}
\usepackage{graphicx}

% zeta functions of cubic fields

%\usepackage{pifont}
\usepackage{amsmath}

\newcommand{\off}[1]{}
\DeclareMathSizes{20}{30}{20}{18}

\newcommand{\two }{\sqrt[3]{2}}
\newcommand{\four}{\sqrt[3]{4}}
\newcommand{\red}{\begin{tikz}[scale=0.25]
\draw[fill=red, color=red] (0,0)--(1,0)--(1,1)--(0,1)--cycle;\end{tikz}}
\newcommand{\blue}{\begin{tikz}[scale=0.25]
\draw[fill=blue, color=blue] (0,0)--(1,0)--(1,1)--(0,1)--cycle;\end{tikz}}
\newcommand{\green}{\begin{tikz}[scale=0.25]
\draw[fill=green, color=green] (0,0)--(1,0)--(1,1)--(0,1)--cycle;\end{tikz}}

\newcommand{\sq}[3]{\draw[#3] (#1,#2)--(#1+1,#2)--(#1+1,#2+1)--(#1,#2+1)--cycle;}

\usepackage{tikz}

\newcommand{\susy}{{\bf Q}}
\newcommand{\RV}{{\text{R}_\text{V}}}

\title{Scratchwork: Pythagorean Triples over $\mathbb{Z}[i]$}
\date{}
\begin{document}

%\fontfamily{qag}\selectfont \fontsize{12.5}{15}\selectfont

\sffamily

\maketitle

\noindent The equation $x^2 + y^2 = 1$ defines what we might call a \textbf{variety}.  Here it's just a circle.  Our decision to use Cartesian coordinates, algebra and equations, to describe Euclidean geometry, leads to all sorts of complications.  Attempts to finalize what we might call a variety leads to all sorts of difficult \textbf{commutative algebra} and ultimately \textbf{schemes}.  \\ \\
Taking for granted a minute that circles are a meaningful concept and that we should use algebra and geometry, we observe the variety $X = \{ x^2 + y^2 - 1 = 0\}$ has a rational paramterization.
$$ t \in \mathbb{Q}  \longrightarrow (x,y) = \left( \frac{1-t^2}{1 + t^2}, \frac{2t}{1+t^2} \right) \in \mathbb{Q}^2 $$
And we can think of fractions as pairs of integers $\frac{m}{n} \in \mathbb{Q}$ or as pairs of integers up to proportion\footnote{Yet another word we can scrutinize.  What are we calling ``proportionate"? } $[m:n] \in P\mathbb{Z}^2$.  Therefore we can solve an equation over integers: $a^2 + b^2 = c^2$ over $\mathbb{Z}$:
$$ (a,b,c) = (m^2 - n^2, 2mn, m^2 + n^2) \in \mathbb{Z}^3$$ 
The machinery of algebra tells us that all we required is that $K = \mathbb{Q}$ is a field.  Therefore we could also try to solve the Pythagoras equation over $\mathbb{Z}[i]$ or $\mathbb{Z}[\sqrt{2}]$ and solve it in the same way. \\ \\
What happens if we write the thing down we could write $a = a_1 + i a_2$ etc. and find:
$$ (a_1 + i a_2)^2 + (b_1 + i b_2)^2 = (c_1 + i c_2)^2 $$
and we learned we could seprate the real and imaginary components and find two equations:
\begin{eqnarray*}
(a_1^2 - a_2^2) + (b_1^2 - b_2^2)  &=& (c_1^2 - c_2^2) \\
2a_1 a_2 + 2 b_1 b_2 &=& 2 c_1 c_2 
\end{eqnarray*}
which is the intersection of two conic sections.  And here because of the structure of $\mathbb{Z}[i]$ we can expect a spectactular reduction.  Is that all? \\ \\ 
Let's find a few solutions.  The algebra is meaningful if we can find a few number solutions.  Let $m = i$ and $n = 1+i$.  Then:
$$ \begin{array}{ccccl} m^2 - n^2 &=& (i)^2 - (1+i)^2 &=& -1 - 2i \\
2mn &=& i \times (1+i) &=& -1 + i \\
m^2 + n^2 &=& (i)^2 + (1+i)^2 &=& -1 + 2i \end{array}$$
Then, using the same algebra identity we used over $\mathbb{Z}$ we obtain another Pythagoras formula:
$$ (1 +2i)^2 + (1 - i)^2 = (1 - 2i)^2  $$
What kind of shape is this?  Is it a right triangle?  What is a ``complexified" right triangle?  

\newpage \noindent

Also we can encode $i = \sqrt{-1}$ as a $2 \times 2$ matrix:
$$ \sqrt{-1} = \left[ \begin{array}{cr} 
0 & -1 \\ 1 & 0\end{array} \right] $$
Does our Pythagoras equation look any better as relation on $2 \times 2$ matrices?
$$
\left[ \begin{array}{rr} 
1 & -2 \\ 2 & 1\end{array} \right]^2 + 
\left[ \begin{array}{rr} 
1 & 1 \\ -1 & 1\end{array} \right]^2 =
\left[ \begin{array}{rr} 
1 & 2 \\ -2 & 1\end{array} \right]^2  $$
Exact matrix identities like this should be pretty rare.  Pythagoras theorem is a machine that consistently produces equations like this. The square-root operator has no guarantee of falling in the integer $2 \times 2$ matrices: $\sqrt{A} \stackrel{?}{\in} \text{GL}_2(\mathbb{Z})$.\\ \\
Are there other $2 \times 2$ matrices such that $A^2 = 1$?  I can find an invertible matrix $B \in \text{GL}_2(\mathbb{Z})$ and we always obtain: 
$$ (BAB^{-1})(BAB^{-1}) = (BA)(B^{-1}B)(A^{-1}B^{-1}) = (BA)I_{2 \times 2}(BA)^{-1} = I_{2 \times 2} $$
If these matrices were perfectly associative, if the transformations we were considering perfectly mapped two variables into two other variables, if $A^{-1}$ is exactly the inverse of $A$, then our formula always works. Let's try:
$$B = 
\left[ \begin{array}{rr} 
3 & 4 \\ 2 & 3\end{array} \right] \in \text{GL}_2(\mathbb{Z}) \text{ and }B^{-1} = \left[ \begin{array}{rr} 
3 & -4 \\ -2 & 3\end{array} \right] \in \text{GL}_2(\mathbb{Z})
 $$
Then we can multiply the matrices:
$$ BAB^{-1} = 
\left[ \begin{array}{rr} 
3 & 4 \\ 2 & 3\end{array} \right] \times
\left[ \begin{array}{cr} 
0 & -1 \\ 1 & 0\end{array} \right] \times
\left[ \begin{array}{rr} 
3 & -4 \\ -2 & 3\end{array} \right] = \dots = 
\left[ \begin{array}{rr} 
17 & -24 \\ 12 & -17\end{array} \right]  \in \text{GL}_2(\mathbb{Z})
$$
In the process of computing a square root, for $2 \times 2$ matrices, we obtain approximate square roots of integers over $\mathbb{Z}$:
$$ 1 = 17 \times 17 - 12 \times 24 = \bigg( 17^2 - 2 \times 12^2 \bigg)  $$
and these were the types of considerations that I'm hoping can motivate theory of modular forms and dynamical systems to other people.

\newpage \noindent Using {\tt numPy} we can multiply $2 \times 2$ matrices as a dot product operation:

\begin{verbatim}
Python 2.7.12 (default, Dec  4 2017, 14:50:18) 
[GCC 5.4.0 20160609] on linux2
Type "help", "copyright", "credits" or "license" for more information.
>>> import numpy as np

>>> B = np.array([[3,4],[2,3]])
>>> A = np.array([[0,-1],[1,0]])
>>> C = np.dot(B,A)

>>> D = np.dot(C, np.linalg.inv(B))
>>> D
array([[ 18., -25.],
       [ 13., -18.]])
>>> D.astype(int)
array([[ 17, -24],
       [ 12, -17]])
>>> np.dot(D,D)
array([[ -1.00000000e+00,  -2.84217094e-14],
       [ -1.06581410e-14,  -1.00000000e+00]])

\end{verbatim}
Using {\tt Sympy} we can obtain the same answer, staying within the realm of integers.
\begin{verbatim}
Python 2.7.12 (default, Dec  4 2017, 14:50:18) 
[GCC 5.4.0 20160609] on linux2
Type "help", "copyright", "credits" or "license" for more information.
>>> from sympy import *
>>> B = Matrix([[3,4],[2,3]])
>>> A = Matrix([[0,-1],[1,0]])
>>> B*A*B**(-1)
Matrix([
[18, -25],
[13, -18]])
\end{verbatim}
Our computers can never really behave like $\mathbb{R}$ or $\mathbb{Z}$ but for the moment $10^{-14}$ scale errors do not qualitatively change our answers. 

\newpage \noindent We do not even scratch the surface.  We notes that computing $\sqrt{2}$ motivates finding a circle on the modular surface $\text{SL}_2(\mathbb{Z}) \backslash \mathbb{H}$ or even $ \Gamma_0(4)\backslash \mathbb{H}$.  And then we can study Maa\b. forms or theta functions.
\end{document}