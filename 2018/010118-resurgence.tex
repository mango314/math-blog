\documentclass[12pt]{article}
%Gummi|065|=)
\usepackage{amsmath, amsfonts, amssymb}
\usepackage[margin=0.5in]{geometry}
\usepackage{xcolor}
\usepackage{graphicx}

%\usepackage{pifont}
\usepackage{amsmath}

\newcommand{\off}[1]{}
\DeclareMathSizes{20}{30}{20}{18}

\newcommand{\two }{\sqrt[3]{2}}
\newcommand{\four}{\sqrt[3]{4}}
\newcommand{\red}{\begin{tikz}[scale=0.25]
\draw[fill=red, color=red] (0,0)--(1,0)--(1,1)--(0,1)--cycle;\end{tikz}}
\newcommand{\blue}{\begin{tikz}[scale=0.25]
\draw[fill=blue, color=blue] (0,0)--(1,0)--(1,1)--(0,1)--cycle;\end{tikz}}
\newcommand{\green}{\begin{tikz}[scale=0.25]
\draw[fill=green, color=green] (0,0)--(1,0)--(1,1)--(0,1)--cycle;\end{tikz}}

\newcommand{\sq}[3]{\draw[#3] (#1,#2)--(#1+1,#2)--(#1+1,#2+1)--(#1,#2+1)--cycle;}

\usepackage{tikz}

\newcommand{\susy}{{\bf Q}}
\newcommand{\RV}{{\text{R}_\text{V}}}

\title{Scratchwork: Divergent Series}
\date{}
\begin{document}

%\fontfamily{qag}\selectfont \fontsize{12.5}{15}\selectfont

\sffamily

\maketitle

\noindent These gentlemen start off with a nice \textit{classical} analysis of the Bessel funtion.  These were surprising to me, but the formulas are quite old.
$$ Z(g) = \frac{1}{\sqrt{g}} \int_{-\frac{\pi}{2}}^{\frac{\pi}{2}} dx \, e^{- \frac{1}{2g} (\sin x)^2} 
= \frac{\pi}{\sqrt{g}} e^{-\frac{1}{4g}} I_0(\frac{1}{4g})$$
These formulas are related to the \textit{asymptotic expansion} of the Bessel funtion, but they can also considered a 0+0 dimensional quantum field theory. Later we have another variant of this expansion:
$$ Z(\xi) = \sqrt{\xi} \int_{-\pi}^{\pi}e^{-2\xi (\sin \phi)^2} d\phi 
= \int_{\mathcal{C}} \frac{e^{-\xi x}}{\sqrt{(1+x)(1-x)}}\, dx$$
We can show, using \textbf{variation of paramters} of some kind or another that $Z$ solves the Bessel differential equation:
$$ Z_*(\xi) = A \, I_0(\xi) + B \, K_0 (\xi)$$
This was revolutionary to me, but involves Riemann surface geomtry already discussed in the 19th century.  
\begin{itemize}
\item Do professors still study such integrals?
\item How is this different from ``asymptotic series"?  What is wrong with these formulas?
\item Is there a 0+1 dimensional quantum field theory? Where did these equations fit?
\end{itemize}
One hint we have is the next integral these gentelemen compute, curresponding to an elliptic curve:
$$ Z(\zeta, \xi) = \int \frac{e^{-x\xi}}{\sqrt{x(1 - (1 - \zeta)x)(1 + \zeta x)}} \, dx$$
These functions are still very versatile and could become a number of things.  And as they get more and more complex, we have theories about them, but their specific properties are less and less well-known. \\ \\
The asymptotic series is just the Taylor-expansion around $g = 0$.   If $|g| \ll 1$ we have this formula:
$$ Z(g) \stackrel{?}{\simeq} \sqrt{2\pi} \left[ 1 + \frac{g}{2} + \frac{9}{8}g^2 + \dots \right]$$
except we're now concerned this formula is no longer true for $g \notin \mathbb{R}$.  We could hashtag this \texttt{[sequences-and-series]} and consult our freshman analysis textbooks.  And a world of complications will unfold\dots

\vfill

\begin{thebibliography}{}

\item Aleksey Cherman, Peter Koroteev, Mithat \"{U}nsal.  \textbf{Resurgence and Holomorphy: From Weak to Strong Coupling} \texttt{arXiv:1410.0388}
 
\end{thebibliography}

\newpage

\noindent \textbf{1/2} Instead of Abramowitz and Stegun, I searched through \textbf{Watson}'s \textit{Treatise on the Theory of Bessel Functions}, where they discuss $J_0(z)$ instead of $I_0(z)$.
$$ J_0(z) = \frac{1}{2\pi} \int_0^{\pi}  \cos \big(z \cos \theta \big) \, d\theta $$
and the Bessel function satisfies the Bessel differential equation:
$$ z^2 \frac{d^2 y}{dz^2} + z \frac{dy}{dz} + z^2 y = 0 $$
These differential equations are often present in problems with rotational symmetry.  But also they deal with the representation theory of $\text{SL}(2, \mathbb{R})$, which is not discussed in Watson's book.\\ \\ \\
The Bessel function can be expressed in terms of the Legendre polynomials (which are always about the sphere, $x^2 + y^2 + z^2 = 1$).
$$ \lim_{n \to \infty} P_n\big( \cos \frac{z}{n}\big) = J_0(z) $$
These always have to do with Laplaces equation on spherically symmtric functions:
$$ \nabla f = \frac{\partial^2 f}{\partial x^2}
+ \frac{\partial^2 f}{\partial y^2}
+ \frac{\partial^2 f}{\partial z^2} = 0 $$
and this way we have obtained elements of both the $\text{SO}(3)$ and $\text{SL}(2, \mathbb{R})$ functions.  Since Cherman and Koroteev and \"{U}nsal talk about $I_0(z)$ and not $J_0(z)$ we can find an asymptotic for that:
$$ \lim_{n\to \infty} P_n \Big( \frac{n^2+z^2 }{n^2 - z^2 } \Big) = I_0(2z) $$
These connections are important because I am trying to find other sequences of steps leading to their derivation.
$$ P_n( \cos \theta) = \cos^2 (\theta/2)\; {}_2F_1 (-n,-n;1;-\tan^2 \frac{1}{2}\theta)$$
Therefore classical functions are an endless labyrinth, and we can always obtain novel symmetries. \\ \\ \\
In order to look for a quick solution we have:
$$ \lim_{n \to \infty} P_n (\cos \theta/n) = \frac{1}{\pi} \int_0^\pi e^{iz \, \cos \phi} \, d\phi = J_0(z)$$
and hopefully this can lead to an asymptotic series.  Without further ado, we find on p.203 the asymptotic they derive:
$$ I_0(z) \sim \frac{e^z}{\sqrt{2\pi z}} \sum_{m=0}^\infty \frac{(-1)^m (0,m)}{(2z)^m}
+ i\, \frac{e^{-z}}{\sqrt{2\pi z}} \sum_{m=0}^\infty \frac{ (0,m)}{(2z)^m} $$
So wherever we aim, we can always find more formula of this type:
\begin{itemize}
\item How bad are the asymptotic series?  Can we fix them numbers in $\mathbb{R}$ ?
\item Is there continuity as we pass from $|z| \ll 1$ to $|z| \gg 1$ ?  
\item How about $z = \epsilon S^1$ moves around a small circle?
\end{itemize}
I have not been able to check why they feel Picard-Fuchs equations and Schwinger-Dyson equations are related.

\newpage

\noindent \"{U}nsal and his colleagues offer a good discussion of resurgence in several papers, and for exponential integrals there are notes and lectures of Konsevich that I have not looked at.  These high priests decree that a particular statement is ``old" or ``new", so what are they working on?  Resolving \textbf{divergences in quantum field theory} is the open problem which should lead to certifiably new insight.

\vfill

\begin{thebibliography}{}

\item Gerald V. Dunne, Mithat \"{U}nsal  \textbf{Resurgence and Trans-series in Quantum Field Theory: The $\mathbb{C}P^{N-1}$ Model} \texttt{arXiv:1210.2423}

\item Maxim Kontsevich \textbf{Exponential Integral} \texttt{https://www.youtube.com/watch?v=tM25X6AI5dY}
 
\end{thebibliography}

\end{document}