\documentclass[12pt]{article}
%Gummi|065|=)
\usepackage{amsmath, amsfonts, amssymb}
\usepackage[margin=0.5in]{geometry}
\usepackage{xcolor}
\usepackage{graphicx}

%\usepackage{pifont}
\usepackage{amsmath}

\newcommand{\off}[1]{}
\DeclareMathSizes{20}{30}{20}{18}

\newcommand{\two }{\sqrt[3]{2}}
\newcommand{\four}{\sqrt[3]{4}}
\newcommand{\red}{\begin{tikz}[scale=0.25]
\draw[fill=red, color=red] (0,0)--(1,0)--(1,1)--(0,1)--cycle;\end{tikz}}
\newcommand{\blue}{\begin{tikz}[scale=0.25]
\draw[fill=blue, color=blue] (0,0)--(1,0)--(1,1)--(0,1)--cycle;\end{tikz}}
\newcommand{\green}{\begin{tikz}[scale=0.25]
\draw[fill=green, color=green] (0,0)--(1,0)--(1,1)--(0,1)--cycle;\end{tikz}}

\newcommand{\sq}[3]{\draw[#3] (#1,#2)--(#1+1,#2)--(#1+1,#2+1)--(#1,#2+1)--cycle;}

\usepackage{tikz}

\newcommand{\susy}{{\bf Q}}
\newcommand{\RV}{{\text{R}_\text{V}}}

\title{Scratchwork: Motives}
\date{}
\begin{document}

%\fontfamily{qag}\selectfont \fontsize{12.5}{15}\selectfont

\sffamily

\maketitle

\noindent One computation we can all enjoy doing is the Basel problem, posed in 1644 by Pietro Mengoli and solved by Leonhard Euler in 1735 (with a ``rigorous" proof by 1741):
$$ 1 + \frac{1}{2^2} + \frac{1}{3^2} + \dots = \frac{\pi^2}{6} $$
It took about a century for the first full solution to come about and it was still just a mathematical oddity.  250 years later, it has avalanched into one of the most important formulae in the literature. \\\\
I have also been interested in solving $\zeta(-1)$ and we have this much newer formula:
$$ 1 + 2 + 3 + 4 + \dots ``=" - \frac{1}{12} $$
Do people study these objects as grown-ups?  Mostly, I have seen that any time we get an explicit anwser about an infinite series, either it ``factors" through this these two, or the geometric series formula and a handful or others.  \\ \\
There are about 15 proofs in the workbook of Robin Chapman called ``Evaluating $\zeta(2)$".  It's basically a guided tour of all of mathematics.  Maybe\dots all ``really symmetric" mathematics.  Every time you got an exact answer it went trough this one (or a few others). \\ \\
I was able to find a few conjectures about $\zeta_F(2)$ and $\zeta_F(4)$ and $\zeta_F(-1)$.  Evaluating these objects can be ``done" in the sense that you can get a numerical answer.  Yet, the conjectures (Bloch-Kato, Birch-Tate, Zagier,\dots) are very very difficult to state.  And in many cases we know they have ``special values" but we can't write down the series - ever.  Therefore \textit{statements} of these special-values formulas are still needed. \\ \\
I'm not driving the point home enough. OK.   Let $F = \mathbb{Q}(i)$.  I am going to wrtie own a Dirichlet series:
$$ \zeta_F(s) = \sum_{a,b \geq 0} \frac{1}{(a^2 + b^2)^s} $$ 
Discussions of this were not very constructive.  There is a quick way to evaluate the answer and then commentors have derided my question as ``basic" or ``repetetive".  There is no longer the variet of answers as before.\\ \\
Then I tried: $F = \mathbb{Q}(\sqrt{2})$, and more of the same problems occur:
$$ \zeta_{Q(\sqrt{2})} = \sum_{a,b \geq 0} \frac{1}{\big(a^2 - 2\, b^2\big)^2} $$
After asking around, I learned this has a major problem, since $a^2 - 2b^2 = 1$ infinitel many times, we have a divergent series.  This isn't the definition of $\zeta_F$ in this case anyway, instead we're supposed to sum over \textit{ideals} of  $(a + b \sqrt{2}) \subseteq \mathbb{Z}[\sqrt{2}]$.

\newpage

\noindent Ideal computations in abstract algebra start off looking off like the typical algebraic manipulation of symbols.  Maybe, these often encode things like \textbf{Euclidean geometry} or the \textbf{Pigeonhole principle}.  Haven't really tried to hard in this regard.  \\ \\
The conjectures I have seen regarding $\zeta_F(2)$ and $\zeta_F(-1)$ are very very difficult to state. \\ \\
\textbf{Q \& A} \\
Are these people doing all this work for their health?  No. \\
Are these people working too hard?  Somewhat. \\
Are there easier to state conjectures and examples within this framework?  Maybe. 

\vfill

\begin{thebibliography}{}

\item Chapman \textbf{Evaluating $\zeta(2)$} \texttt{[various google]}

\item Alexander Goncharov \textbf{Zagier's Conjecture on $\zeta_F(4)$} \texttt{https://youtu.be/GI\_{}ah1U0xjw}

\item Spencer Bloch, Kazuya Kato. \textbf{L-functions and the Tamagawa Numbers of Motives}
 
\end{thebibliography}

\end{document}