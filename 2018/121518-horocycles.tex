\documentclass[12pt]{article}
%Gumm{\color{blue}i}|065|=)
\usepackage{amsmath, amsfonts, amssymb}
\usepackage[margin=0.5in]{geometry}
\usepackage{xcolor}
\usepackage{graphicx}
\usepackage{amsmath}

\newcommand{\off}[1]{}
\DeclareMathSizes{20}{30}{20}{18}
\usepackage{tikz}


\title{Scratchwork: Horocycles}
\date{}
\begin{document}

\sffamily

\maketitle

\noindent Let's try to understand a little better what it means that the horocycle flow is mixing. \\ \\
\noindent \textbf{Thm 11.22} Let $\Gamma \leq \text{SL}_2(\mathbb{R})$ be a lattice.  Then the action of $\text{SL}_2(\mathbb{R})$ on $X = \Gamma \backslash \text{SL}_2(\mathbb{R})$ is mixing.  \\ \\
These groups represent an equivalence class of number systems.  For simplicity we choose two: $\text{SL}_2(\mathbb{Z})\backslash \text{SL}_2(\mathbb{R})$ and $\Gamma_0(4) \backslash \text{SL}_2(\mathbb{R})$ with:
$$ \text{SL}_2(\mathbb{Z}) = \left\{ \left( 
\begin{array}{cc} a & b \\ c & d \end{array}\right) : ad-bc = 1 \right\} \text{ and }\Gamma_0(4) = \text{SL}_2(\mathbb{Z}) \cap \left\{ 
\left( 
\begin{array}{cc} a & b \\ c & d \end{array}\right)  \equiv \left( 
\begin{array}{cc} * & * \\ 0 & * \end{array}\right) \pmod 4 \right\} $$
while it looks self-evidence to solve the equation in brackets, finding a solution over $\mathbb{Z}$ already requires continued fractions. Theorem 11.22 is about the entire $\text{SL}_2(\mathbb{R})$ which contains both the geodesic flow and the horocycle flow. Here is a statement about only horocycles.\\ \\
\textbf{11.15} Let $\Gamma \leq \text{SL}_2(\mathbb{R})$ be a lattice.  Let $g \in \text{SL}_2(\mathbb{R})$ be an element that is not conjugate to an element of $\text{SO}(2)$.  Then $R_g$ acts ergodically $(X, \mathcal{B}_X, m_x)$. \\ \\
Here $m_X$ is the Haar measure on $\text{SL}_2(\mathbb{Z})\backslash \text{SL}_2(\mathbb{R})$, and $\mathcal{B}_X$ is a Borel $\sigma$-algebra of measurable sets on $X$. \\ \\
It seems in the process of doing functional analysis and measure rigidity, we're not placing too much emphasis on basic Eucliean geometry or examples.  Or perhaps I'm missing something. \\ \\
What is a \textbf{horocycle} anyway?\footnote{Wikipedia \texttt{https://en.wikipedia.org/wiki/Horocycle}} 
\begin{quotation} \noindent In hyperbolic geometry, a horocycle (Greek: ὅριον + κύκλος — border + circle, sometimes called an oricycle, oricircle, or limit circle) is a curve whose normal or perpendicular geodesics all converge asymptotically in the same direction. It is the two-dimensional example of a horosphere (or orisphere).
\begin{itemize}
\item Through every pair of points there are two horocycles.
\item No three points of a horocycle are on a line, circle or hypercircle.
\item A straight line, circle, or hypercircle cuts a horocycle in at most two points
\item a regular \textbf{apeirogon} is circumscribed by either a horocycle or a hypercycle.
\item The perpendicular bisector of a chord of a horocycle is a normal of the horocycle and it bisects the arc subtended by the chord.
\end{itemize}
In the Poincar\'{e} disk model of the hyperbolic plane, horocycles are represented by circles tangent to the boundary circle, the centre of the horocycle is the ideal point where the horocycle touches the boundary circle.

The compass and straightedge construction of the two horocycles through two points is the same construction of the CPP construction for the Special cases of Apollonius' problem where both points are inside the circle.
\end{quotation}
A cursory look through Wikipedia gives a lot of information, and should motivate a look through a good Geometry textbook.  \\ \\

\noindent \textbf{12/17} What kind of issues could we address this way?  An element of $\text{SL}_2(\mathbb{Z})$ would have four integers $a,b,c,d \in \mathbb{Z}$ with $ad-bc = 1$.  
$$ \frac{a}{c} - \frac{b}{d} = \frac{1}{cd} < \frac{1}{\text{max}\{c,d\}^2} $$
This equation says, for example that always $\text{gcd}(a,c) = 1$ and we could solve with continued fractions  (another thing we'll call into question). Ergiodicity of the horocycle flow intuitively implies that:
$$ 
\overline{\left[ \begin{array}{cr} 1 & t \in \mathbb{R} \\ 0 & 1 \end{array} \right]
\left[ \begin{array}{cc} a & b \\ c & d \end{array} \right] \text{SL}_2(\mathbb{Z})} = \text{SL}_2(\mathbb{R}) $$
Because of human error, even if I have four integers $a,b,c,d \in \mathbb{Z}$ with a small amount of error.  Suppose I make a mistake in one of them.  
$$ (a + \epsilon) (d + \epsilon) - (b + \epsilon)( = 1 $$
What if the number $(a,c)$ were not quite relatively prime?  Could we take the derivative of the GCD function with respect to one of the variables? 
$$ \frac{d}{da} \Big[ \text{gcd}(a,c) \Big]  = \lim_{|\epsilon| \to 0} \frac{\text{gcd}(a + \epsilon,c) - \text{gcd}(a,c)}{\epsilon} $$
Does a statement like this even make sense?  It's good enough for now. \\ \\

\noindent \textbf{12/20} Why even bother stating formally the mixing of the horocycle flow.  Can we even state two lattices in $\text{SL}_2(\mathbb{R})$ that are close to one another (approximate).  Let's find two matrices that solve $ad-bc = 1$:
$$ \left[ \begin{array}{cc} \sqrt{2} & 0 \\ 0 & \frac{1}{\sqrt{2}} \end{array} \right] \quad\text{and}\quad
\left[ \begin{array}{cc} 2 & \sqrt{3} \\ \sqrt{3} & 2 \end{array} \right] $$
since we have $\sqrt{2} \times \frac{1}{\sqrt{2}} = 1$ and yet $2 \times 2 -3 = 1$.  These two matrices are not adjacent $\sqrt{2} \approx 1.414$ (at least, as an element of $\mathbb{R}$).  Perhaps we could find a number $\epsilon \approx 0$ with:
\begin{eqnarray*} \det  \left[ \begin{array}{cc} \sqrt{2}-\epsilon & \epsilon \\ \epsilon & \frac{1}{\sqrt{2} }+ \epsilon   \end{array} \right]
&=& 1 - \epsilon \, \big( \frac{1}{\sqrt{2}} - \sqrt{2} \big) - 2\;\epsilon^2 = 1 \\
\epsilon &=&  \frac{1}{2} \big( \sqrt{2} - \frac{1}{\sqrt{2}}  \big) \end{eqnarray*}
And the second one is even more confusing.  It's also easy.  Why don't we try a more systmatic approach.  Using the $NAK$ decomposition of the $2\times2$ invertible matrices:
$$
\left[ \begin{array}{cc} \sqrt{a} & 0 \\ 0 & \frac{1}{\sqrt{a}} \end{array} \right] \times
\left[ \begin{array}{cc} 1 & b \\ 0 & 1 \end{array} \right] \times
\left[ \begin{array}{rr} c & -d \\ d & c \end{array} \right] \text{ with }c^2 + d^2 =1$$
So now we have an easy way to construct $2 \times 2$ matrices with entires in $\mathbb{R} \backslash \mathbb{Z}$.  Let $b = 0$, and $c = \epsilon$ and $d = \sqrt{1 - \epsilon^2}$.  
$$ \left[ \begin{array}{cc} \sqrt{a} & 0 \\ 0 & \frac{1}{\sqrt{a}} \end{array} \right] \times
\left[ \begin{array}{rr} \sqrt{1 - \epsilon^2} & -\epsilon \\ \epsilon & \sqrt{1 - \epsilon^2} \end{array} \right] $$
where we observe that $\sqrt{1-\epsilon^2} \approx 1 - \frac{1}{2} \epsilon^2 + O(\epsilon^4)$ a statement which is also up for veritication.  In particular, when is $\sqrt{1 - \epsilon^2} \in \mathbb{Q}$.  In words, we want rational points on the circle that are close to $(1,0)$.  Notice that:
$$ 1^2 + 0^2 = 1^2 \text{ so then } (1-\epsilon^2)^2 + \epsilon^2 = 1 + O(\epsilon^4) $$
This is a not an acceptable result.  Not right now.  This also look like a bit of a scheme result, we could have $\mathbb{R}[\epsilon]/(\epsilon^4)$ and this is a \textbf{scheme}.  We could organize our request into a form or a cylinder set:
\begin{eqnarray*}
c^2 + d^2 &=& 1 \\
(c,d) &\in& \mathbb{Q}^2\\
(c,d) & \approx & (0,1) \\ \vspace{8pt}
&\text{or}& \\ \vspace{8pt}
|c| & < & \epsilon \\
|d| & > & 1 - \epsilon
\end{eqnarray*}
and we could be more specific with our choice of " $\approx$ " (what we'd call a topology).\footnote{There's an obvious topology on $\mathbb{R}$ but our attempts to smell out solutions over $\mathbb{Q}$ (where our circle is now a \textit{scheme}) will lead us to other completions such as $\mathbb{Q}_p$.}  Using the Pythagoras formula we could have:
$$ (c,d) = \left( \frac{m^2 - n^2}{m^2 + n^2}, \frac{2mn}{m^2 + n^2} \right) \stackrel{m \gg 1, n=1}{\longrightarrow} \left(  
\frac{m^2 - 1}{m^2 + 1}, \frac{2m}{m^2 + 1}\right)\approx (1,0) $$  
All that's left is to cobble together the numbers to get a result.  
$$ \left[ \begin{array}{cc} \sqrt{a} & 0 \\ 0 & \frac{1}{\sqrt{a}} \end{array} \right] 
\times
\left[ \begin{array}{rr} \frac{m^2 - 1}{m^2 + 1} & -\frac{2m}{m^2 + 1} \\ 
\frac{2m}{m^2 + 1} & \frac{m^2 - 1}{m^2 + 1}  \end{array} \right] \stackrel{a=3,m=10}{=}
 \left[ \begin{array}{cc} \sqrt{2} & 0 \\ 0 & \frac{1}{\sqrt{2}} \end{array} \right] 
\times
\left[ \begin{array}{rr} \frac{99}{101} & -\frac{20}{101} \\ \\
\frac{20}{101} & \frac{99}{101}  \end{array} \right]
$$
If we want to reason about the numbers we are getting we have to dig a little deeper.  At the moment all we want is two nearby elements of $\text{SL}(2, \mathbb{R})$.
\vfill
\begin{thebibliography}{} 

\item Brian Marcus \textbf{The horocycle flow is mixing of all degrees} Inventiones Mathematicae, Vol 46 \#3 201-209 (1978)

\item Manfred Einsiedler, Thomas Ward. \textbf{Ergodic Theory: with a view towards Number Theory} \\GTM \#259 Springer, 2011.

\item Marina Ratner \\ 
\textbf{Factors of Horocycle Flows}  Ergodic Theory and Dynamical Systems  Vol. 2, \#3-4 pp.465-489, 1982 \\
\textbf{Horocycle Flows, Joinings and Rigidity of Products} Annals of Mathematics Vol. 118, No. 2 pp. 277-313 (1983)

\end{thebibliography}




\end{document}