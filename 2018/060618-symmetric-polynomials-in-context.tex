\documentclass[12pt]{article}
%Gumm{\color{blue}i}|065|=)
\usepackage{amsmath, amsfonts, amssymb}
\usepackage[margin=0.5in]{geometry}
\usepackage{xcolor}
\usepackage{graphicx}

% zeta funct{\color{blue}i}ons of cub{\color{blue}i}c f{\color{blue}i}elds

%\usepackage{p{\color{blue}i}font}
\usepackage{amsmath}

\newcommand{\off}[1]{}
\DeclareMathSizes{20}{30}{20}{18}

\newcommand{\two }{\sqrt[3]{2}}
\newcommand{\four}{\sqrt[3]{4}}





\usepackage{tikz}

\newcommand{\susy}{{\bf Q}}
\newcommand{\RV}{{\text{R}_\text{V}}}

\title{Scratchwork: Symmetr{\color{blue}i}c Polynom{\color{blue}i}als}
\date{}
\begin{document}

%\fontfam{\color{blue}i}ly{qag}\selectfont \fonts{\color{blue}i}ze{12.5}{15}\selectfont

\sffamily

\maketitle

\noindent Where do symmetr{\color{blue}i}c polynom{\color{blue}i}als come from?  The start{\color{blue}i}ng po{\color{blue}i}nts are almost too obv{\color{blue}i}ous to even ment{\color{blue}i}on. \\ \\
\textbf{Ex.} F{\color{blue}i}nd a cub{\color{blue}i}c polynom{\color{blue}i}al $f(x) = x^3 {\color{green}\,\times\,} ax^2 {\color{green}+} bx {\color{green}+} c$ such that $f(0)=1$ and $f(1) = 2$, $f(2) = 3$. \\ \\
These constra{\color{blue}i}nts leads to s{\color{blue}i}multaneous equat{\color{blue}i}ons for the number $a,b,c$:
$$\begin{array}{rcrcrcrcc}
    &   &    &   &    &  & c &=& 1 \\
 1  & {\color{green}+} &  a & {\color{green}+} & b  &{\color{green}+} & c &=& 2 \\
 8  & {\color{green}+} & 4a & {\color{green}+} & 2b &{\color{green}+} & c &=& 3 
\end{array}$$
A polynom{\color{blue}i}al {\color{blue}i}s just a made{\color{red}-}up dev{\color{blue}i}ce that mathemat{\color{blue}i}c{\color{blue}i}ans use to solve equat{\color{blue}i}ons anwyay.  Why m{\color{blue}i}ght such a th{\color{blue}i}ng be natural?  {\color{blue}i}f you're a bel{\color{blue}i}ever {\color{blue}i}n the Newton Le{\color{blue}i}bn{\color{blue}i}z calculus, there was the Taylor ser{\color{blue}i}es expans{\color{blue}i}on from 1715 or so:
$$ f(x{\color{green}+}a) = f(x) {\color{green}+} f'(x) a {\color{green}+} f''(x) \frac{a^2}{2} {\color{green}+} f'''(x) \frac{a^3}{6}\dots $$
As long as you have enough der{\color{blue}i}vat{\color{blue}i}ves.  We are go{\color{blue}i}ng to extrapolate nearby values bas{\color{blue}i}c on what we know at a s{\color{blue}i}ngle po{\color{blue}i}nt, w{\color{blue}i}th \text{\color{blue}i}t{zero} knowledge of $f$. \\ \\
So we have matr{\color{blue}i}x equat{\color{blue}i}on:
$$
\left[ 
\begin{array}{rrr}  0 & 0 & 1 \\
 1 & 1 & 1 \\
 4 & 2 & 1 \end{array}\right]
\left[ 
\begin{array}{r} a \\
b  \\
c  \end{array}\right] = 
\left[ 
\begin{array}{r} 1 \\
1  \\
{\color{red}-}5 \end{array}\right] 
 $$
We seem to be on the r{\color{blue}i}ght track.  Cram\'{e}r's rule f{\color{blue}i}rst appears {\color{blue}i}n 1750 but we've l{\color{blue}i}kely had s{\color{blue}i}multaneous equat{\color{blue}i}ons before that.  F{\color{blue}i}rst of all there {\color{blue}i}s a s{\color{blue}i}ngle equat{\color{blue}i}on:
$$ c = 1  \quad\text{and}\quad 
\left[ 
\begin{array}{rr} 
 1 & 1  \\
 4 & 2 \end{array}\right]
\left[ 
\begin{array}{r} 
b  \\
c  \end{array}\right] = 
\left[ 
\begin{array}{r} 
0 \\
{\color{red}-}6 \end{array}\right] 
$$
We only have two var{\color{blue}i}ables. So let's just solve them:
$$ 
a = \frac{
\left| 
\begin{array}{rr} 
 0 & 1  \\
 {\color{red}-}6 & 2 \end{array}\right|
}{\left| 
\begin{array}{rr} 
 \;\;\;1 & 1  \\
 4 & 2 \end{array}\right|} = \frac{6}{{\color{red}-}2}={\color{red}-}3
\quad\text{and}\quad
 b = \frac{
 \left| 
\begin{array}{rr} 
 1 & 0  \\
 4 & {\color{red}-}6 \end{array}\right|
 }{\left| 
\begin{array}{rr} 
 1 & 1  \\
 4 & \;\;\,2 \end{array}\right|} = \frac{{\color{red}-}6}{{\color{red}-}2}=3$$
and now l{\color{blue}i}ke good students we should restate our answer.  The polynom{\color{blue}i}al should be:
$$ f(x) = x^3 {\color{red}-}3x^2 {\color{green}+} 3x {\color{green}+} 1$$
Add{\color{blue}i}t{\color{blue}i}onally, we observe that Taylor ser{\color{blue}i}es mot{\color{blue}i}vates order of operat{\color{blue}i}ons:
$$ f(3) = 1 {\color{blue}\,\times\,}\big(3 {\color{blue}\,\times\,} 3 {\color{blue}\,\times\,} 3\big) {\color{red}-} 3 {\color{blue}\,\times\,} \big(3 {\color{blue}\,\times\,} 3\big) {\color{green}+} 3 {\color{blue}\,\times\,} \big(3\big) {\color{green}+} 1 $$
Th{\color{blue}i}s {\color{blue}i}s just a sketch of how the grade school operat{\color{blue}i}ons could have emerged. 

\newpage 
\noindent 
We can talk for a moment about the formulas of V{\color{blue}i}ete. 
$$ (x{\color{red}-}a)(x{\color{red}-}b)(x{\color{red}-}c) = x^3 {\color{red}-} (a{\color{green}+}b{\color{green}+}c)x^2 {\color{green}+} (ab{\color{green}+}bc{\color{green}+}ca)x {\color{red}-}abc $$
Then {\color{blue}i}f we have a cub{\color{blue}i}c equat{\color{blue}i}on wh{\color{blue}i}ch {\color{blue}i}s very d{\color{blue}i}ff{\color{blue}i}cult to solve, we can st{\color{blue}i}ll get {\color{blue}i}nformat{\color{blue}i}on about the average behav{\color{blue}i}or of the numbers:
$$ x^3 {\color{red}-} 3x^2 {\color{green}+} 3x {\color{green}+} 1$$
and perhaps we can f{\color{blue}i}nd the sum of the squares of the roots:
$$ a^2 {\color{green}+} b^2 {\color{green}+} c^2 = 
(a{\color{green}+}b{\color{green}+}c)^2 {\color{red}-} 2 {\color{blue}\,\times\,} (ab {\color{green}+} bc {\color{green}+} ca) = ({\color{red}-}3)^2 {\color{red}-} 2 {\color{blue}\,\times\,} 3 = 3 $$
These th{\color{blue}i}ngs h{\color{blue}i}de  {\color{blue}i}n front of your face.  They're almost too obv{\color{blue}i}ous to state.\footnote{{\color{blue}i}s there a techn{\color{blue}i}cal term for th{\color{blue}i}s??} \\ \\
\textbf{Ex.} Are the roots of $f(x)$ all real numbers? (1 real) {\color{green}+} (2 {\color{blue}i}mag{\color{blue}i}nary)?  F{\color{blue}i}nd $a^4 {\color{green}+} b^4 {\color{green}+} c^4$. \\ \\
\textbf{Ex.} Der{\color{blue}i}ve Taylor's formula to 4th order.  $(x,y)=(0,0)$ {\color{blue}i}s a po{\color{blue}i}nt on the lemn{\color{blue}i}scate: $(x^2 {\color{green}+} y^2)^2 {\color{red}-} 2(x^2 {\color{red}-} y^2)=0$. \\ What are some nearby po{\color{blue}i}nts?  Can we get an exact answer?\footnote{Gabr{\color{blue}i}el Cram\'{e}r ``{\color{blue}i}ntrouct{\color{blue}i}on a L'Analyse des L{\color{blue}i}nes Courbes Algebra{\color{blue}i}ques" \\ \texttt{https://arch{\color{blue}i}ve.org/deta{\color{blue}i}ls/bub\_gb\_gtKvSzJPOOAC}} 

\newpage 

\noindent \textbf{6/8} While that's not nearly enough motivation, let's say I have a cubic polynomial, $f(x) = x^3 + ax^2 + bx + c$.  How do we distinguish between these two cases:
\begin{itemize}
\item $f(x) = 0$ for three numbers $x_1, x_2, x_3  \in \mathbb{R}$ (the ``totally real" case)
\item $f(x) = 0$ for one number $x_1 \in \mathbb{R}$ and two complex conjugate numbers $x_2, \overline{x_2} \in \mathbb{C}$.
\end{itemize}
Jumping ahead of ourselves, let's enumerate the possibilities for the Galois group:
\begin{itemize}
\item $S_3$ the group of permutations on three letters
\item $A_3$ the group of (even) permutations in $S_3$
\item $C_3$ the cyclic group on three symbols (like a wheel)
\item $I = \{ e \}$ the group with one element.
\end{itemize}
And just like a calculator, there are procedures for finding the information we want about any given field and any given prime number.  There's even a few thousand pages of paper if you want to read them giving a few patterns in the behavior of these numbers.  
\begin{itemize}
\item $[\mathbb{Q}(\sqrt[3]{2}):\mathbb{Q}]= 3$.  Every number can be written as $x = a + b \sqrt[3]{2} + c \sqrt[3]{4}$.  This is related to the equation $x^3 - 2 = 0$. This field is \textbf{not} Galois.  $\frac{6}{3} = 2$ so we need a \textit{quadratic} extension $K$ over $\mathbb{Q}(\sqrt{2})$ so that $[K:\mathbb{Q}] = 6$.
$$ x^2 + ax + b = 0$$
How do you feel about the quadratic formula if we know additionally that $a,b \in \mathbb{Q}(\sqrt[3]{2})$? 
\item  The polynomial $x^3 - x - 1$ is called the ``plastic number".  It is also a \textit{Pisot} number in that $x_1 \in \mathbb{R}$ and $|x_1| > 1$ while $|x_2|, |x_3| < 1$.  How do we verify such information for ourselves?  What is the Galois group of $\mathbb{Q}(x)/\mathbb{Q}$?  (It's in fact, $S_3$.)
\end{itemize}
We should be leary, that even for such basic examples, we have to use a computer.  And the entire existing literature is full of such a doubt. \\ \\
\textbf{Q} Could we feel out the Galois group of a cubic using the averages of the coefficints, e.g. $f_n= x_1^n + x_2^n + x_3^n \in \mathbb{Z}$.  Even thought we can't solve for $x_1, x_2, x_3 \in \mathbb{C}$ we could use the Vi\'{e}te polynomials to get the various averages of the roots.
\begin{itemize}
\item $a = x_1 + x_2 + x_3$
\item $b = x_1x_2 + x_2x_3+x_3x_1$
\item $c = x_1x_2x_3$
\end{itemize}
In the case of a quadratic, the sequence $f_n = x_1^n + x_2^n$ yields a kin of the Fibonacci numbers.  And there's an exhaustic literature there, which nonetheless has open questions (somewhat on the deep end).  So we even have a guess what the numbers are like. \\ \\
\textbf{Q}: We have that $\overline{\mathbb{Q}(\sqrt[3]{2})} = \mathbb{R}$ (the ``closure" of the set of rational numbers with respect to the absolute value $|\cdot|$ how can we best approximate numbers like $\sqrt{2}, \sqrt{3}, \sqrt[3]{5}$, $\pi$, etc.?
$$ \frac{a_1 + \sqrt[3]{2}b_1+ \sqrt[3]{4}c_1}{a_2 + \sqrt[3]{2}b_2 +\sqrt[3]{4}c_2} = a + \sqrt[3]{2}b +\sqrt[3]{4}c $$  
with $(a_1, b_1, c_1), (a_2, b_2, c_2) \in \mathbb{Z}^3$ and $a,b,c \in \mathbb{Q}$. Tautologically, the fractions can be written in the left way or the right way, but how do we feel about such a map?

\newpage

\begin{thebibliography}{}
\item Artur Avila, Vincent Delecroix. \textbf{Some monoids of Pisot matrices} \texttt{arXiv:1506.03692}
\end{thebibliography} 

\noindent \textbf{6/10} Other than Galois Theory and Numerical Analysis do we have other candidates for usages of symmetric polynomials\dots ? These arise in combinatorics and probability.  Let's have probability distribution $1 = p_1 + p_2 + \dots + p_n$.  Then perhaps we could try to compute the moments:
$$ \mathbb{E}[x_k] = p_1^k + \dots + p_n^k $$
This is a symmetric polynomial.  Therefore, all expectation values are symmetric in the probabilities.  If we have information about the distribution, the numbers $\{ 1, \dots, n\}$ start to look different.  If we had a random permutation:
$$ 1 \mapsto (1,2,3), 2 \mapsto (1,3,2), 3 \mapsto (2,1,3), 4 \mapsto (2,3,1), \dots $$
Then perhaps there is other symmetry.  These symmetries are decidedly bland and look trivial -- literally like a $\mathbf{0}$ -- and yet have profound consequences.
\end{document}