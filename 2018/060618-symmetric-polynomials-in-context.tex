\documentclass[12pt]{article}
%Gummi|065|=)
\usepackage{amsmath, amsfonts, amssymb}
\usepackage[margin=0.5in]{geometry}
\usepackage{xcolor}
\usepackage{graphicx}

% zeta functions of cubic fields

%\usepackage{pifont}
\usepackage{amsmath}

\newcommand{\off}[1]{}
\DeclareMathSizes{20}{30}{20}{18}

\newcommand{\two }{\sqrt[3]{2}}
\newcommand{\four}{\sqrt[3]{4}}
\newcommand{\red}{\begin{tikz}[scale=0.25]
\draw[fill=red, color=red] (0,0)--(1,0)--(1,1)--(0,1)--cycle;\end{tikz}}
\newcommand{\blue}{\begin{tikz}[scale=0.25]
\draw[fill=blue, color=blue] (0,0)--(1,0)--(1,1)--(0,1)--cycle;\end{tikz}}
\newcommand{\green}{\begin{tikz}[scale=0.25]
\draw[fill=green, color=green] (0,0)--(1,0)--(1,1)--(0,1)--cycle;\end{tikz}}

\newcommand{\sq}[3]{\draw[#3] (#1,#2)--(#1+1,#2)--(#1+1,#2+1)--(#1,#2+1)--cycle;}

\usepackage{tikz}

\newcommand{\susy}{{\bf Q}}
\newcommand{\RV}{{\text{R}_\text{V}}}

\title{Scratchwork: Symmetric Polynomials}
\date{}
\begin{document}

%\fontfamily{qag}\selectfont \fontsize{12.5}{15}\selectfont

\sffamily

\maketitle

\noindent Where do symmetric polynomials come from?  The starting points are almost too obvious to even mention. \\ \\
\textbf{Ex.} Find a cubic polynomial $f(x) = x^3 + ax^2 + bx + c$ such that $f(0)=1$ and $f(1) = 2$, $f(2) = 3$. \\ \\
These constraints leads to simultaneous equations for the number $a,b,c$:
$$\begin{array}{rcrcrcrcc}
    &   &    &   &    &  & c &=& 1 \\
 1  & + &  a & + & b  &+ & c &=& 2 \\
 8  & + & 4a & + & 2b &+ & c &=& 3 
\end{array}$$
A polynomial is just a made-up device that mathematicians use to solve equations anwyay.  Why might such a thing be natural?  If you're a believer in the Newton Leibniz calculus, there was the Taylor series expansion from 1715 or so:
$$ f(x+a) = f(x) + f'(x) a + f''(x) \frac{a^2}{2} + f'''(x) \frac{a^3}{6}\dots $$
As long as you have enough derivatives.  We are going to extrapolate nearby values basic on what we know at a single point, with \textit{zero} knowledge of $f$. \\ \\
So we have matrix equation:
$$
\left[ 
\begin{array}{rrr}  0 & 0 & 1 \\
 1 & 1 & 1 \\
 4 & 2 & 1 \end{array}\right]
\left[ 
\begin{array}{r} a \\
b  \\
c  \end{array}\right] = 
\left[ 
\begin{array}{r} 1 \\
1  \\
-5 \end{array}\right] 
 $$
We seem to be on the right track.  Cram\'{e}r's rule first appears in 1750 but we've likely had simultaneous equations before that.  First of all there is a single equation:
$$ c = 1  \quad\text{and}\quad 
\left[ 
\begin{array}{rr} 
 1 & 1  \\
 4 & 2 \end{array}\right]
\left[ 
\begin{array}{r} 
b  \\
c  \end{array}\right] = 
\left[ 
\begin{array}{r} 
0 \\
-6 \end{array}\right] 
$$
We only have two variables. So let's just solve them:
$$ 
a = \frac{
\left| 
\begin{array}{rr} 
 0 & 1  \\
 -6 & 2 \end{array}\right|
}{\left| 
\begin{array}{rr} 
 \;\;\;1 & 1  \\
 4 & 2 \end{array}\right|} = \frac{6}{-2}=-3
\quad\text{and}\quad
 b = \frac{
 \left| 
\begin{array}{rr} 
 1 & 0  \\
 4 & -6 \end{array}\right|
 }{\left| 
\begin{array}{rr} 
 1 & 1  \\
 4 & \;\;\,2 \end{array}\right|} = \frac{-6}{-2}=3$$
and now like good students we should restate our answer.  The polynomial should be:
$$ f(x) = x^3 -3x^2 + 3x + 1$$
Additionally, we observe that Taylor series motivates order of operations:
$$ f(3) = 1 \times\big(3 \times 3 \times 3\big) - 3 \times \big(3 \times 3\big) + 3 \times \big(3\big) + 1 $$
This is just a sketch of how the grade school operations could have emerged. 

\newpage 
\noindent 
We can talk for a moment about the formulas of Viete. 
$$ (x-a)(x-b)(x-c) = x^3 - (a+b+c)x^2 + (ab+bc+ca)x -abc $$
Then if we have a cubic equation which is very difficult to solve, we can still get information about the average behavior of the numbers:
$$ x^3 - 3x^2 + 3x + 1$$
and perhaps we can find the sum of the squares of the roots:
$$ a^2 + b^2 + c^2 = 
(a+b+c)^2 - 2 \times (ab + bc + ca) = (-3)^2 - 2 \times 3 = 3 $$
These things hide  in front of your face.  They're almost too obvious to state.\footnote{Is there a technical term for this??} \\ \\
\textbf{Ex.} Are the roots of $f(x)$ all real numbers? (1 real) + (2 imaginary)?  Find $a^4 + b^4 + c^4$. \\ \\
\textbf{Ex.} Derive Taylor's formula to 4th order.  $(x,y)=(0,0)$ is a point on the lemniscate: $(x^2 + y^2)^2 - 2(x^2 - y^2)=0$. \\ What are some nearby points?  Can we get an exact answer?\footnote{Gabriel Cram\'{e}r ``Introuction a L'Analyse des Lines Courbes Algebraiques" \\ \texttt{https://archive.org/details/bub\_gb\_gtKvSzJPOOAC}} 
\end{document}