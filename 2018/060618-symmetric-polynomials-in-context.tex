\documentclass[12pt]{article}
%Gumm{\color{blue}i}|065|=)
\usepackage{amsmath, amsfonts, amssymb}
\usepackage[margin=0.5in]{geometry}
\usepackage{xcolor}
\usepackage{graphicx}

% zeta funct{\color{blue}i}ons of cub{\color{blue}i}c f{\color{blue}i}elds

%\usepackage{p{\color{blue}i}font}
\usepackage{amsmath}

\newcommand{\off}[1]{}
\DeclareMathSizes{20}{30}{20}{18}

\newcommand{\two }{\sqrt[3]{2}}
\newcommand{\four}{\sqrt[3]{4}}





\usepackage{tikz}

\newcommand{\susy}{{\bf Q}}
\newcommand{\RV}{{\text{R}_\text{V}}}

\title{Scratchwork: Symmetr{\color{blue}i}c Polynom{\color{blue}i}als}
\date{}
\begin{document}

%\fontfam{\color{blue}i}ly{qag}\selectfont \fonts{\color{blue}i}ze{12.5}{15}\selectfont

\sffamily

\maketitle

\noindent Where do symmetr{\color{blue}i}c polynom{\color{blue}i}als come from?  The start{\color{blue}i}ng po{\color{blue}i}nts are almost too obv{\color{blue}i}ous to even ment{\color{blue}i}on. \\ \\
\textbf{Ex.} F{\color{blue}i}nd a cub{\color{blue}i}c polynom{\color{blue}i}al $f(x) = x^3 {\color{green}\,\times\,} ax^2 {\color{green}+} bx {\color{green}+} c$ such that $f(0)=1$ and $f(1) = 2$, $f(2) = 3$. \\ \\
These constra{\color{blue}i}nts leads to s{\color{blue}i}multaneous equat{\color{blue}i}ons for the number $a,b,c$:
$$\begin{array}{rcrcrcrcc}
    &   &    &   &    &  & c &=& 1 \\
 1  & {\color{green}+} &  a & {\color{green}+} & b  &{\color{green}+} & c &=& 2 \\
 8  & {\color{green}+} & 4a & {\color{green}+} & 2b &{\color{green}+} & c &=& 3 
\end{array}$$
A polynom{\color{blue}i}al {\color{blue}i}s just a made{\color{red}-}up dev{\color{blue}i}ce that mathemat{\color{blue}i}c{\color{blue}i}ans use to solve equat{\color{blue}i}ons anwyay.  Why m{\color{blue}i}ght such a th{\color{blue}i}ng be natural?  {\color{blue}i}f you're a bel{\color{blue}i}ever {\color{blue}i}n the Newton Le{\color{blue}i}bn{\color{blue}i}z calculus, there was the Taylor ser{\color{blue}i}es expans{\color{blue}i}on from 1715 or so:
$$ f(x{\color{green}+}a) = f(x) {\color{green}+} f'(x) a {\color{green}+} f''(x) \frac{a^2}{2} {\color{green}+} f'''(x) \frac{a^3}{6}\dots $$
As long as you have enough der{\color{blue}i}vat{\color{blue}i}ves.  We are go{\color{blue}i}ng to extrapolate nearby values bas{\color{blue}i}c on what we know at a s{\color{blue}i}ngle po{\color{blue}i}nt, w{\color{blue}i}th \text{\color{blue}i}t{zero} knowledge of $f$. \\ \\
So we have matr{\color{blue}i}x equat{\color{blue}i}on:
$$
\left[ 
\begin{array}{rrr}  0 & 0 & 1 \\
 1 & 1 & 1 \\
 4 & 2 & 1 \end{array}\right]
\left[ 
\begin{array}{r} a \\
b  \\
c  \end{array}\right] = 
\left[ 
\begin{array}{r} 1 \\
1  \\
{\color{red}-}5 \end{array}\right] 
 $$
We seem to be on the r{\color{blue}i}ght track.  Cram\'{e}r's rule f{\color{blue}i}rst appears {\color{blue}i}n 1750 but we've l{\color{blue}i}kely had s{\color{blue}i}multaneous equat{\color{blue}i}ons before that.  F{\color{blue}i}rst of all there {\color{blue}i}s a s{\color{blue}i}ngle equat{\color{blue}i}on:
$$ c = 1  \quad\text{and}\quad 
\left[ 
\begin{array}{rr} 
 1 & 1  \\
 4 & 2 \end{array}\right]
\left[ 
\begin{array}{r} 
b  \\
c  \end{array}\right] = 
\left[ 
\begin{array}{r} 
0 \\
{\color{red}-}6 \end{array}\right] 
$$
We only have two var{\color{blue}i}ables. So let's just solve them:
$$ 
a = \frac{
\left| 
\begin{array}{rr} 
 0 & 1  \\
 {\color{red}-}6 & 2 \end{array}\right|
}{\left| 
\begin{array}{rr} 
 \;\;\;1 & 1  \\
 4 & 2 \end{array}\right|} = \frac{6}{{\color{red}-}2}={\color{red}-}3
\quad\text{and}\quad
 b = \frac{
 \left| 
\begin{array}{rr} 
 1 & 0  \\
 4 & {\color{red}-}6 \end{array}\right|
 }{\left| 
\begin{array}{rr} 
 1 & 1  \\
 4 & \;\;\,2 \end{array}\right|} = \frac{{\color{red}-}6}{{\color{red}-}2}=3$$
and now l{\color{blue}i}ke good students we should restate our answer.  The polynom{\color{blue}i}al should be:
$$ f(x) = x^3 {\color{red}-}3x^2 {\color{green}+} 3x {\color{green}+} 1$$
Add{\color{blue}i}t{\color{blue}i}onally, we observe that Taylor ser{\color{blue}i}es mot{\color{blue}i}vates order of operat{\color{blue}i}ons:
$$ f(3) = 1 {\color{blue}\,\times\,}\big(3 {\color{blue}\,\times\,} 3 {\color{blue}\,\times\,} 3\big) {\color{red}-} 3 {\color{blue}\,\times\,} \big(3 {\color{blue}\,\times\,} 3\big) {\color{green}+} 3 {\color{blue}\,\times\,} \big(3\big) {\color{green}+} 1 $$
Th{\color{blue}i}s {\color{blue}i}s just a sketch of how the grade school operat{\color{blue}i}ons could have emerged. 

\newpage 
\noindent 
We can talk for a moment about the formulas of V{\color{blue}i}ete. 
$$ (x{\color{red}-}a)(x{\color{red}-}b)(x{\color{red}-}c) = x^3 {\color{red}-} (a{\color{green}+}b{\color{green}+}c)x^2 {\color{green}+} (ab{\color{green}+}bc{\color{green}+}ca)x {\color{red}-}abc $$
Then {\color{blue}i}f we have a cub{\color{blue}i}c equat{\color{blue}i}on wh{\color{blue}i}ch {\color{blue}i}s very d{\color{blue}i}ff{\color{blue}i}cult to solve, we can st{\color{blue}i}ll get {\color{blue}i}nformat{\color{blue}i}on about the average behav{\color{blue}i}or of the numbers:
$$ x^3 {\color{red}-} 3x^2 {\color{green}+} 3x {\color{green}+} 1$$
and perhaps we can f{\color{blue}i}nd the sum of the squares of the roots:
$$ a^2 {\color{green}+} b^2 {\color{green}+} c^2 = 
(a{\color{green}+}b{\color{green}+}c)^2 {\color{red}-} 2 {\color{blue}\,\times\,} (ab {\color{green}+} bc {\color{green}+} ca) = ({\color{red}-}3)^2 {\color{red}-} 2 {\color{blue}\,\times\,} 3 = 3 $$
These th{\color{blue}i}ngs h{\color{blue}i}de  {\color{blue}i}n front of your face.  They're almost too obv{\color{blue}i}ous to state.\footnote{{\color{blue}i}s there a techn{\color{blue}i}cal term for th{\color{blue}i}s??} \\ \\
\textbf{Ex.} Are the roots of $f(x)$ all real numbers? (1 real) {\color{green}+} (2 {\color{blue}i}mag{\color{blue}i}nary)?  F{\color{blue}i}nd $a^4 {\color{green}+} b^4 {\color{green}+} c^4$. \\ \\
\textbf{Ex.} Der{\color{blue}i}ve Taylor's formula to 4th order.  $(x,y)=(0,0)$ {\color{blue}i}s a po{\color{blue}i}nt on the lemn{\color{blue}i}scate: $(x^2 {\color{green}+} y^2)^2 {\color{red}-} 2(x^2 {\color{red}-} y^2)=0$. \\ What are some nearby po{\color{blue}i}nts?  Can we get an exact answer?\footnote{Gabr{\color{blue}i}el Cram\'{e}r ``{\color{blue}i}ntrouct{\color{blue}i}on a L'Analyse des L{\color{blue}i}nes Courbes Algebra{\color{blue}i}ques" \\ \texttt{https://arch{\color{blue}i}ve.org/deta{\color{blue}i}ls/bub\_gb\_gtKvSzJPOOAC}} 
\end{document}