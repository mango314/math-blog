\documentclass[12pt]{article}
%Gumm{\color{blue}i}|065|=)
\usepackage{amsmath, amsfonts, amssymb}
\usepackage[margin=0.5in]{geometry}
\usepackage{xcolor}
\usepackage{graphicx}
\usepackage{amsmath}

\newcommand{\off}[1]{}
\DeclareMathSizes{20}{30}{20}{18}
\usepackage{tikz}


\title{Scratchwork: ``Locally Compact Abelian" Groups}
\date{}
\begin{document}

\sffamily

\maketitle

\noindent This harmonic analysis jargon ``Locally Compact Abelian" groups appears basically all over number theory in a very abstract form.  Here are the two examples I can think of immediately: $\mathbb{R}$ and $\mathbb{Q}_p$.  Harmonic analysis is generalization of Fourier Analysis to abstract groups, such as $SO(3)$ the rotations of a three-dimensional object and many other kinds of symmetry.\footnote{When do we study a group action?  Usually this means the object was not \text{a priori} symmetric!  A cube can be rotated in three-dimensional space by $SO(3)$ elements, and we can study the remainder that's leftover.} \\ \\
In our case, there is an isomorphism (as group) in Number Theory
$$ \mathbb{Q}^\times = \prod_p p^{\mathbb{Z}} \quad\text{and}\quad
\mathbb{Q}(i)^\times = \prod_\mathfrak{p} \mathfrak{p}^{\mathbb{Z}} $$
In fact all the $F^\times$ are isomorphic for any number field $F$.  That cannot be right.  On the left the primes are $p \in \mathbb{Z}$ and on the right $\mathfrak{p} \in \mathbb{Z}[i]$ and for little more than grade school arithmetic, we have that primes $p \in 4\mathbb{Z}+1$ factor into $p = \mathfrak{p}_1\mathfrak{p}_2$ E.g. $13 = (2+3i)\times(2-3i)$ or $17 = (4+i)\times (4-i)$. Therefore both number fields share the primes in $p \in 4\mathbb{Z}+3$. So we need one extra thing to distinguish between $\mathbb{Q}^\times$ and $\mathbb{Q}(i)^\times$, that is a \textbf{topology}. \\ \\
Have you seen Fourier analysis on $\mathbb{Q}^\times$ or $\mathbb{Q}(i)^\times$?  These not compact, since we can find Cauchy sequences $x \to \sqrt{2}$ or $\sqrt{3} \notin \mathbb{Q}^\times$.  So how can we define a derivative $f'(x) = 2x$.
$$ f'(x) = \lim_{|\epsilon| \to } \frac{f(x + \epsilon) - f(x)}{\epsilon} $$
Therefore we need to find a space where this derivative makes any kind of sense at all.  I think we have some freedom here.  Why is this thing $f'(x)$ even a number? \\ \\
Since I didn't study Munkres' textbook carefully (and all the pathological but nifty examples) we just settle for asking, which number $x \in \mathbb{Q}^\times$ or $x \in \mathbb{Q}(i)^\times$ solve $x \approx 0$ ? Which numbers are close to zero?  And our number theory tool is called \textbf{strong approximation}.  So there is in fact some ambiguity in the statement $x \to \sqrt{2}$ in a manner that is important to Number Theory.\footnote{Don't just call their bluff, there should hopefully be a good reason.}
\vfill
\begin{thebibliography}{} 

\item 

\end{thebibliography}

\newpage

%\noindent \textbf{01/20/19} Let's try to write solutions to the sum of three squares as a ``torus" orbit of some kind.  The number $\sqrt{d}$ could be embedded into the quaternions.  $ \mathbb{Q}(\sqrt{n}) \to \mathbb{H}$. Let $a^2 + b^2 + c^2 = n$, then $a \mathbf{i} + b \mathbf{j} + c \mathbf{k} \in \mathbb{H} $, with the correct norm.  Then there is even a way to turn quaternions into $3 \times 3$ matrices.  $\mathbb{Q}(\sqrt{n}) \to \mathbb{H} \to \text{SO}(3)$.  The rotation action $ $.

\noindent \textbf{01/24/19} One very easy ``topological" group, or ``Locally Compact Abelian" group should be $\mathbb{Q}^\times/(\mathbb{Q}^\times)^2$, the rational numbers modulo the perfect squares.  Why such a thing.  Let's consider a variety of some kind:
$$ V_1  = \{ x^2 + y^2 = a \}  \text{ and } V_2 = \{ x^2 + y^2 = ab^2 \}$$
The map $(x,y) \mapsto (bx,by)$ takes the small circle to the larger circle, so we call them \textit{birational equivalent}.\footnote{Algebraic geometry is such that every time we make such a call, it merely becomes an invitation to check an endless hierarchy of exceptions.  And it's OK.  There are plenty of authoritive resources at varying levels.}  So we can say that $V(a) \equiv V(ab^2)$ for $a,b \in \mathbb{Q}$.  And more basically, $a  \equiv ab^2$ .  Therefore our conic sections (mostly circles) are equivalence classes $[a] \in \mathbb{Q}^\times/(\mathbb{Q}^\times)^2$. \\ \\
What kind of observables can we study here?  I'd like to count the rational points on this circle of a given height.  For example:
$$ \{ x^2 + y^2 = 1 \} $$
The rational points here correspond to \textbf{Pythagorian triples}  or to \textbf{right triangles}.  It's very unlikely that you've seen a proof of Pythagoras theorem.  There are several hundred and they all say slightly different things about triangles.  We would like the points in this circle of bounded height:
$$ \Big\{ (a,b,c) \in \mathbb{Q} : \Big( \frac{a}{c}\Big)^2 + \Big(\frac{b}{c}\Big)^2 = 1 \text{ and } |a|,|b|,|c| < N \text{ and } (a,b)=(a,c)=1\Big\} $$
We would have to ``solve" the equation - and at this level it's harder and hard to means - and then we have to do GCD computations over and over.  And we could let $f(N)$ be the number of points in this set.  \\ \\
Let's try one more.  Here's a circle which clearly has points over $\mathbb{R}$ and yet is vacuous over $\mathbb{Q}$:
$$ x^2 + y^2 = 3 $$
No rational points whatsover.  Since $\{ 0^2, 1^2\} + \{ 0^2 , 1^2 \}  = \{ 0,1,2\}$ using a tiny amount of sumset theory.  We've shown that $3 \notin \square$.  We have shown a circle in $\mathbb{R}^2$ that has successfully avoided all of $\mathbb{Q}^2$.  Nothing particularly wrong with that, but maybe we can quantify how close to points in $\mathbb{Q}^2$ we can get.

\vfill
\begin{thebibliography}{} 

\item Manfred Einsiedler, Thomas Ward.  \textbf{Ergodic Theory with a view towards Number Theory} (Graduate Texts in Mathematics \#259).  Springer, 2011.

\item Philippe Gille, Tam\'{a}s Szamuely. \textbf{Central Simple Algebras and Galois Cohomology}  (Cambridge Studies in Advanced Mathematics \#165).  Cambridge Uniersity Press, 2017.


\item Pierre Guillot. \textbf{A Gentle Course in Local Class Field Theory} Cambridge University Press, 2018. 


\end{thebibliography}

\end{document}