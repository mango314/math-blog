\documentclass[12pt]{article}
%Gummi|065|=)
\usepackage{amsmath, amsfonts, amssymb}
\usepackage[margin=0.5in]{geometry}
\usepackage{xcolor}
\usepackage{graphicx}

% zeta functions of cubic fields

%\usepackage{pifont}
\usepackage{amsmath}

\newcommand{\off}[1]{}
\DeclareMathSizes{20}{30}{20}{18}

\newcommand{\two }{\sqrt[3]{2}}
\newcommand{\four}{\sqrt[3]{4}}
\newcommand{\red}{\begin{tikz}[scale=0.25]
\draw[fill=red, color=red] (0,0)--(1,0)--(1,1)--(0,1)--cycle;\end{tikz}}
\newcommand{\blue}{\begin{tikz}[scale=0.25]
\draw[fill=blue, color=blue] (0,0)--(1,0)--(1,1)--(0,1)--cycle;\end{tikz}}
\newcommand{\green}{\begin{tikz}[scale=0.25]
\draw[fill=green, color=green] (0,0)--(1,0)--(1,1)--(0,1)--cycle;\end{tikz}}

\newcommand{\sq}[3]{\draw[#3] (#1,#2)--(#1+1,#2)--(#1+1,#2+1)--(#1,#2+1)--cycle;}

\usepackage{tikz}

\newcommand{\susy}{{\bf Q}}
\newcommand{\RV}{{\text{R}_\text{V}}}

\title{Scratchwork: Rankin-Selberg Method}
\date{}
\begin{document}

%\fontfamily{qag}\selectfont \fontsize{12.5}{15}\selectfont

\sffamily

\maketitle

\noindent Whenever I think of an automorphic form, I think of it in one of two ways: either as a functon that has certain symmetry properties (some kind of  invariance under $SL(2, \mathbb{Z})$) or using a sequence of numbers ``Fourier coefficents of automorphic forms".  Classically, these special functions were discovered by computing integrals that were particularly bad -- intractable and then we organized them into space. \\ \\
A few recurring themes are ``Hecke Operators" and ``the Rankin-Selberg method".  While there are lots of things known - and endless, nearly infinte supply of results - it's very difficult to obtain information about the thing you care about in language that is easy to understand.  I have found the discussions I've had from professors insuling and personally offensive.  I have concluded that my learning style is conflict with general procedure. \\ \\
The link between \textbf{automorphic functions}  and \textbf{L-functions} and to this end we'll use the \textbf{Rankin-Selberg method}.  Results in the literature tend to deal with ``shape" and leave the computation to others (or to computers).  I think that getting the numbers to match up is an important final step.  If someone tells you about the {\color{blue!50!white} Petersson inner product}:
$$ 
\langle h_1, h_2 \rangle 
= \int \int_{\Gamma \backslash H} 
h_1(z) \overline{h_2(z)} y^k \frac{dx\, dy}{y^2}$$
these read like set of instructions.  We go obtain two approriate functions and integrate them and reason about the steps.  In automorphic functions these theory have gotten very formal. \\ \\
Without referring to any text, the Rankin-Selberg method turns Fourier series into L-functions:
$$
\Big[ f(z) = \sum a_n \, q^n \Big] \to \Big[ 
L(f,s) = \sum a_n \, n^{-s} \Big] $$
Nothing about the automorphic properties of $f$ or the convergence of $L$-series.  There is a perfectly good theory of analytic functions or Dirichlet series.  They are related by a Mellin transform. \\ \\
Every time you want to find something out, the book is there. Just do it! \\ \\ 
- - - \\ \\
So what actually going on?
\vfill

\begin{thebibliography}{}

\item Anton Deitmar \textbf{Automorphic Forms} (Universitext) Springer, 2011.
\item Klaus Roth, Hein Halberstam \textbf{Sequences} OUP, 1966. Springer 1983 / 2012.
\item H. Montgomery, R. Vaughan \textbf{Multiplicative Number Theory}  Cambridge University Press, 2006.
\item Goro Shimura \\
\textbf{Elementary Dirichlet Series and Modular Forms} Springer, 2007. \\
\textbf{Modular Forms, Basics and Beyond}  Springer, 2012.

 
\end{thebibliography} 
\end{document}