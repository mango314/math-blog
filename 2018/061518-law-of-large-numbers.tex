\documentclass[12pt]{article}
%Gumm{\color{blue}i}|065|=)
\usepackage{amsmath, amsfonts, amssymb}
\usepackage[margin=0.5in]{geometry}
\usepackage{xcolor}
\usepackage{graphicx}

% zeta funct{\color{blue}i}ons of cub{\color{blue}i}c f{\color{blue}i}elds

%\usepackage{p{\color{blue}i}font}
\usepackage{amsmath}

\newcommand{\off}[1]{}
\DeclareMathSizes{20}{30}{20}{18}

\newcommand{\two }{\sqrt[3]{2}}
\newcommand{\four}{\sqrt[3]{4}}





\usepackage{tikz}

\newcommand{\susy}{{\bf Q}}
\newcommand{\RV}{{\text{R}_\text{V}}}

\title{Scratchwork: Law of Large Numbers}
\date{}
\begin{document}

%\fontfam{\color{blue}i}ly{qag}\selectfont \fonts{\color{blue}i}ze{12.5}{15}\selectfont

\sffamily

\maketitle

\noindent It's time to review the law of large numbers. Blankly: it says that our observations tend towards the mean.  That is a rather baroque statement, but I needed to do it cold.\footnote{
In the real world -- when I had a job, I wanted to model a process using ``coin-flips".  In fact my job was to guess at the outcome of the football season each week.  Each each week, there is an event.  And the outcome is either ``win" or ``loss" but what are the odds of success? Perhaps we'd use a Bernoulli random variable $X \in \{ 0,1\}$ with $\mathbb{P}(X=1) = p$ and $\mathbb{P}(X=0) = 1-p$.  And I said to myself all I had to do is estimate $p \in [0,1]$.  Within a year, I was fired. \\ \\
A football game is a fairly well-thought out process.  The actions are deliberate.  There's a mix of strategy and last-minute thinking.  I think we keep record of the result after each quarter and the final outcome of the game.  These things are not so random after all! \\ \\
Worse, there's no good way to separate the deliberate elements and the chance elements.  People are very good at aggregating information like news, collectively they can put express a range of opinions.\\} \\ \\
Now here's statement from the Terry Tao blog.\footnote{\texttt{https://terrytao.wordpress.com/2015/10/23/275a-notes-3-the-weak-and-strong-law-of-large-numbers/}}  He's very detailed about use usage and discussion of the construct of $\mathbb{R}$:
\begin{quotation}
\noindent Let $X_1, \dots, X_n$ be an iid sequence of copies of an absolutely integrable random variable $X$.  
\begin{itemize}
\item converges in probability (Weak)
\item converges weakly (Strong)
\end{itemize}
The emperical average converges to the statistical average: $\frac{1}{n}(X_1 + \dots + X_n) \to \mathbf{E}(X)$
\end{quotation}
Failing to apply LLN correctly has a name, and has been studied.  Applying Law of Large Numbers where it egregiously doesn't apply (and filling in the gaps later) has a name.  On and On and On. \\ \\
Where did we use Law of Large Numbers? The conclusions of LLN are so obvious, we hardly give it much thought.

\vfill

\begin{thebibliography}{}

\item 

\end{thebibliography}
\end{document}