\documentclass[12pt]{article}
%Gumm{\color{blue}i}|065|=)
\usepackage{amsmath, amsfonts, amssymb}
\usepackage[margin=0.5in]{geometry}
\usepackage{xcolor}
\usepackage{graphicx}
\usepackage{amsmath}

\newcommand{\off}[1]{}
\DeclareMathSizes{20}{30}{20}{18}
\usepackage{tikz}


\title{Scratchwork: Configuration Spaces and \'{E}tale Cohomology}
\date{}
\begin{document}

\sffamily

\maketitle

\noindent One way to get representations of the Symmetric group $S_n$ is to use configuration space, e.g. of Points or Lines.
\begin{eqnarray*}
X_n(\mathbb{C}) &=& \big\{ (z_1, \dots, z_n) \big| z_i \in \mathbb{C} \text{ and } z_i \neq z_j\big\} \\
X_n(\mathbb{C}) &=& \big\{ (L_1, \dots, L_n) \big| L_1 \text{ line in } \mathbb{C}^n \;\; L_1, \dots, L_n \text{ linearly independent} \big\} 
\end{eqnarray*}
Before I even try to learn the main ``new" results of the paper, the authors remark that it's easy to compute these homologies by hand:
$$ H^0(\mathbb{C}\mathbf{P}^2)= \mathbb{Q} \text{ and }
H^2(\mathbb{C}\mathbf{P}^2)= \mathbb{Q} \text{ and }
H^4(\mathbb{C}\mathbf{P}^2)= \mathbb{Q} $$
and from this they will deduce the number of points in the counting formula:
$$ |\mathbf{P}^2(\mathbb{F}_q)| = q^2 + q + 1  $$
and this is true for any prime power $q$.   A more interesting result about ``twisted cohomology" is that
$$ H^1\big(\text{Conf}_n(\mathbb{C}); \wedge^2 \mathbb{Q}^n \big) \simeq \mathbb{Q} \text{ for }n \geq 4 $$
This is already an astonishing amount of information.
\vfill

\begin{thebibliography}{}

\item Thomas Church, Jordan S. Ellenberg, Benson Farb. \textbf{Representation stability in cohomology and asymptotics for families of varieties over finite fields} \texttt{arXiv:1309.6038}

\end{thebibliography}

\end{document}