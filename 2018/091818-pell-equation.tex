\documentclass[12pt]{article}
%Gumm{\color{blue}i}|065|=)
\usepackage{amsmath, amsfonts, amssymb}
\usepackage[margin=0.5in]{geometry}
\usepackage{xcolor}
\usepackage{graphicx}
\usepackage{amsmath}

\newcommand{\off}[1]{}
\DeclareMathSizes{20}{30}{20}{18}
\usepackage{tikz}


\title{Scratchwork: Pell Equation over $F$}
\date{}
\begin{document}

\sffamily

\maketitle

\noindent \textbf{9/18} I always wondered what Pell Equation would look like over number fields other than $\mathbb{Q}$.  Let's try:
$$ x^2 - 2 y^2 = 1 \text{ with } x,y \in \mathbb{Z}[\sqrt{3}] $$
This ring is a Euclidean domain.  I haven't proved that.  For example:
\begin{itemize}
\item Show that $5$ and $17$ are prime in $\mathbb{Z}[\sqrt{3}]$.  Find the continued fraction of $\frac{17}{5}$ as a number in $\mathbb{Q}(\sqrt{3})$.
\item We could write rational numbers in two different ways $ \frac{a + b \sqrt{3}}{c + d \sqrt{3}} \in \mathbb{Q}(\sqrt{3}) $ or $\frac{a}{b} + \frac{c}{d} \sqrt{3} \in \mathbb{Q}$ with $a,b,c,d \in \mathbb{Z}$.
\end{itemize}
Let's na\"{i}vely write the real and imaginary parts of that:
$$ (x_1 + \sqrt{3}x_2)^2 - 2 (y_1 + \sqrt{3}y_2)^2
= \big[(x_1^2 + 3x_2^2) - (y_1^2 + 3y_2^2)\big] 
+ 2\sqrt{3} \big[ x_1 x_2 -  y_1 y_2 \big] = 1$$
So this could read as a simultaneous equations, the intersection of two conic sections in 4 dimensions:
\begin{eqnarray*}
(x_1^2 + 3x_2^2) - (y_1^2 + 3y_2^2) &=& 1 \\ 
 x_1 x_2 -  y_1 y_2  &=& 0
\end{eqnarray*}
This could also be read as a diophantine approximation problem.  What's the best approximation of $\sqrt{2}$ in the number field $\mathbb{Q}(\sqrt{3})$ ?
$$ \left| \frac{a}{b} - \sqrt{2} \right| < \frac{1}{ \big|\, b \mathbb{Z}[\sqrt{3}]:\mathbb{Z}[\sqrt{3}] \, \big|^m} \text{ with }  $$
I don't even know the correct exponent of $m$.  This could be found using the Pigeonhole principle (the ``Dirichlet principle").  
\begin{eqnarray*}
(x_1^2 + 3x_2^2) - (y_1^2 + 3y_2^2) &=& a \\ 
 x_1 x_2 -  y_1 y_2  &=& b
\end{eqnarray*}
I'm not even sure how to draw these equations.  If you have two quadrics $C_1$ and $C_2$ then the linear combination $\lambda C_1 + \mu C_2$ is also a quadratic.  And this family is called a ``divisor".  For specific values of $\lambda, \mu$ degenerates into the intersection of two lines. \\ \\
Before we do any of that, let's observe our Pell equation can be written as a $2 \times 2$ matrix:
$$ \det \left|\begin{array}{cc} x & 2y \\ y & x \end{array} \right| = 1$$
Sadly, matrices and determinants are old-fashioned objects and need to be one away with.  The determinant just the endomorphism of the exterior product of lattices anyway\footnote{https://ncatlab.org/nlab/show/determinant}. And a matrix is just an element of $V \otimes V^*$ with $V = \mathbb{Q}^2$.\\ \\
Next observe we can do a quadratic form in four variables, elements of a quaterion algebra:
$$ \det \left[ \begin{array}{cc} x_1 + x_2 \sqrt{3} & y_1 - y_2 \sqrt{3} \\
2(y_1 + y_2 \sqrt{3}) & x_1 - x_2 \sqrt{3} \end{array}\right] = x_1^2 - 3x_2^2 - 2y_1^2 + 6 y_2^2 = 1 $$
As soon as you write the thing carefully, Pell equation becomes quite categorical.  The matrices themselves become somwhat obsolete with no clear replacement. \\ \\
\textbf{Q}: How many integer solutions does this equation have?  $\mathbb{Z}$ or $\mathbb{Z} \oplus \mathbb{Z}$ or more?  We can solve:
\begin{eqnarray*}
a^2 - 3b^2 &=& 1 \\
a^2 - 2b^2 &=& 1
\end{eqnarray*}
This leads us astray somewhat.  I'd think the well-approximated ness of certain fraction algorithms can be turned into Pell-type equations. \\ \\
\textbf{Q}: It's unknown if the continued fraction of the $\sqrt[3]{2}$ has any patterns.  There might be other continued fraction algorithms, e.g. over vectors $(1, \sqrt[3]{2}, \sqrt[3]{4})$ and this would be related to Pell equation $a^3 + 2b^3 + 4c^3 - 6abc = 1$. \\ \\
\noindent We needed Pell equation in order to estimate square roots outside of our number field, e.g. find $x \in \mathbb{Q}(\sqrt{2})$ such that  
$$ \left|\,x - \sqrt{2 + \sqrt{3}}\,\right| < \epsilon $$
This is not quite the statement of well-approximatedness.  We need two integers $a + b \sqrt{3}$ with $a,b \in \mathbb{Z}$:
$$ \min_{a,b \in \mathbb{Z}[\sqrt{2}]} (b_1^2 - 3b_2^2)^{\color{blue}\mathbf{1}} \times \left|\,\left( \frac{a_1 + a_2\sqrt{3}}{b_1 + b_2\sqrt{3}}\right)^2 - \sqrt{2 + \sqrt{3}}\,\right| $$
The $\color{blue}1$ is a guess this is just the size of the lattice $|b_1^2 - 2b_2^2| =  \big[(b_1 + \sqrt{2}b_2)\mathbb{Z}[\sqrt{3}]: \mathbb{Z}[\sqrt{3}]\big] $.  \\ \\
\textbf{Q} The number of elements of $\mathbb{Z}[\sqrt{3}]$ with norm $ < m$ is roughly $m$. That's not true.  There are infinitely many points in the regin $|x^2 - 3y^2| < m$.  In that case since $ x^2 - 3y^2 = 1$, we can count points in $\mathbb{Z}[\sqrt{3}]^\times/(2 + \sqrt{3})$ \\ \\
\textbf{Ex} Using guess and check.  Let's write the primes $p = 2,3,5,7,11,13,17,19$. We have that $-11 =  1 \times 1 - 3 \times (2 \times 2)$ and $13 = 4 \times 4 - 3 \times (3 \times 3) $ and then $(11) = (1+\sqrt{2}) \times (1 + \sqrt{2})$ and $(13)$ so yeah.  Then we can write $$
\mathbb{A}_F = \mathbb{R} \times \mathbb{Q}_2 \times \mathbb{Q}_3 \times \mathbb{Q}_5 \times \mathbb{Q}_7 \times \mathbb{Q}_{1+\sqrt{2}} \times \mathbb{Q}_{1-\sqrt{2}} \times \dots $$ as part of the the Adeles.\footnote{It seems somewhat hypocritical to prove these results via Geometry of Numbers and to find these numbers by exhaustic search.  Even if we do that\dots \textbf{search problems} are a well-studied genre of computer science problem.  Ignoring all structore over $\mathbb{Z}$ - which wasn't too realistic anyway, we are evaluating a function $f(x,y) = x^2 - 3y^2$ over a 2D array} Quadratic reciprocity or Fermat's Little Theorem or Geometry of Numbers or Pigeonhole Principle.  This variety of techniques should lead to a variety of problems to solve.\\ \\
\textbf{Q} Solve $|x^2 - 2 |_{1 + \sqrt{2}} < \frac{1}{11}$ and $|x^2 - 3 |_{1 - \sqrt{2}} < \frac{1}{11}$ in $\mathbb{Q}(\sqrt{3})$. \\ \\
\textbf{Q} Why don't we use the Hasse principle to solve the Pell equation?
\vfill

\begin{thebibliography}{}

\item Jurgen Neukirch. \textbf{Algebraic Number Theory} (Grundlehren der mathematischen Wissenschaften) \\ Springer, 1999.
\item Vladimir Platonov, Andrei Rapinchuk, Rachel Rowen \textbf{Algebraic Groups and Number Theory}  \\ Academic Press, 1993. 
\item JWS Cassels. \textbf{An Introduction to Diophantine Approximation} (Cambridge Tracts in Mathematics and Mathematical Physics, No. 45) Cambridge University Press, 1957.

\end{thebibliography}

\noindent What happens when $F = \mathbb{Q}(\sqrt{5})$ and the class number is $\mathbf{2}$ and there's no Euclidean algorithm.    How about $F = \mathbb{Q}(\sqrt{14})$ which is Euclidean but not norm-Euclidean? \footnote{\texttt{https://math.stackexchange.com/q/1234305/4997}}. \\ \\
\textbf{Q} Solve $x^2 \approx 2$ in $\mathbb{Q}(\sqrt{5})$ or $\mathbb{Q}(\sqrt{8})$ or $\mathbb{Q}(\sqrt{14})$.  

\end{document}