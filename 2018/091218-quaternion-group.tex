\documentclass[12pt]{article}
%Gumm{\color{blue}i}|065|=)
\usepackage{amsmath, amsfonts, amssymb}
\usepackage[margin=0.5in]{geometry}
\usepackage{xcolor}
\usepackage{graphicx}
\usepackage{amsmath}

\newcommand{\off}[1]{}
\DeclareMathSizes{20}{30}{20}{18}
\usepackage{tikz}

\newcommand{\one}{\mathbf{1}}
\newcommand{\ii}{\mathbf{i}}
\newcommand{\jj}{\mathbf{j}}
\newcommand{\kk}{\mathbf{k}}

\title{Scratchwork: Symmetric Polynomials}
\date{}
\begin{document}

\sffamily

\maketitle

\noindent The quaternions are a number system defined by three rules of multiplication.  They generalize complex numbers:
$$ \one \times \one = \one \text{ and } \ii \times \jj = \kk \text{ and } \ii \times \ii = -\one $$
These multiplications can we completed to form a group of order $8$.
$$
\begin{array}{r||rr|rr||rr|rr}
 \times & \one & \ii & \jj & \kk & -\one & -\ii & -\jj & -\kk \\ \hline
\one  & \one & \ii & \jj & \kk & -\one & -\ii & -\jj & -\kk \\
\ii & \ii & -\one & \kk & -\jj & -\ii & \one & -\kk & \jj \\ \hline 
\jj & \jj & - \kk & -\one & \ii & - \jj & \kk & \one & -\ii  \\
\kk & \kk & -\jj & -\ii & \one & -\kk & \jj & \ii & -\one \\ \hline \hline
-\one & -\one & -\ii & -\jj & -\kk & \one & \ii & \jj & \kk\\
-\ii & -\ii & \one & -\kk & \jj & \ii & -\one & -\kk & -\jj\\ \hline
-\jj & -\jj &  \kk & \one & -\ii &  \jj & -\kk & -\one & \ii  \\
-\kk & -\kk & \jj & \ii & -\one & \kk & -\jj & -\ii & \one
\end{array}
 $$
It looks like there are eight things being multiplied, so we made an $8 \times 8$ table.  There are eight things being permuted in 8 possible ways:
$$ \{ \one , \ii , \jj, \kk, -\one, -\ii, -\jj, -\kk \} $$
It may even be possible to whittle this down to four - with the inclusion of a minus sign $(-\one)$.  
\begin{eqnarray*}
-\one \times \one &=& -\one \\
-\one \times \ii  &=& -\ii \\
-\one \times \jj  &=& -\jj \\
-\one \times \kk  &=& -\kk
\end{eqnarray*}
Cayley's Theorem says every group can be placed into a permutation group.  We could call the elements of this group $\{1, 2, \dots, 8\}$. 
$$
\one \mapsto \left[\begin{array}{cccccccc} 1 & 2 & 3 & 4 & 5 & 6 & 7 & 8 \\
 1 & 2 & 3 & 4 & 5 & 6 & 7 & 8  \end{array} \right]
 $$
 and now we replace with different rows of the multiplication table:
 
\begin{eqnarray*}
\ii &\mapsto& \left[\begin{array}{cccccccc} 1 & 2 & 3 & 4 & 5 & 6 & 7 & 8 \\
 2 & 5 & 4 & 7 & 6 & 1 & 8 & 3  \end{array} \right] \\ \\
\jj &\mapsto& \left[\begin{array}{cccccccc} 1 & 2 & 3 & 4 & 5 & 6 & 7 & 8 \\
3 & 8 & 5 & 2 & 7 & 4 & 1 & 6 \end{array} \right] \\ \\
\kk &\mapsto& \left[\begin{array}{cccccccc} 1 & 2 & 3 & 4 & 5 & 6 & 7 & 8 \\
4 & 7 & 6 & 1 & 8 & 3 & 2 & 5 \end{array} \right] 
\end{eqnarray*}
The rule for $(-\one)$ looks a little bit complicated.  For the time being switch the first and second half.
$$
\one \mapsto \left[\begin{array}{cccccccc} 1 & 2 & 3 & 4 & 5 & 6 & 7 & 8 \\
5 & 6 & 7 & 8 & 1 & 2 & 3 & 4   \end{array} \right]
 $$ 
\newpage 
\noindent There's even other ways of representing the quaternion group.  Here's the more usual $2 \times 2$ matrices (in case you're scared of Quaternion objects).
$$
\one \mapsto \left[ \begin{array}{cc} 1 & 0 \\ 0 & 1 \end{array} \right] \;
\ii \mapsto \left[ \begin{array}{cc} i & 0 \\ 0 & i \end{array} \right]  \;
\jj \mapsto \left[ \begin{array}{rr} 0 & 1 \\ -1 & 0 \end{array} \right]  \;
\kk \mapsto \left[ \begin{array}{rr} 0 & i \\ i & 0 \end{array} \right] $$
It could even be instructive to write out the full $8 \times 8$ matrices:
$$  \one \to \left[ \begin{array}{cccccccc} 
1 & \cdot & \cdot  & \cdot  & \cdot  & \cdot  & \cdot  & \cdot  \\
\cdot & 1 & \cdot & \cdot & \cdot & \cdot & \cdot & \cdot \\
\cdot & \cdot & 1 & \cdot & \cdot & \cdot & \cdot & \cdot \\
\cdot &\cdot & \cdot & 1 \cdot & \cdot & \cdot & \cdot & \cdot \\
\cdot & \cdot & \cdot & \cdot & 1 & \cdot & \cdot & \cdot \\
\cdot & \cdot & \cdot & \cdot & \cdot & 1 & \cdot & \cdot  \\
\cdot & \cdot &  \cdot & \cdot & \cdot & \cdot &  1 & \cdot \\
\cdot & \cdot & \cdot & \cdot & \cdot & \cdot & \cdot &  1 \end{array} \right]$$
That one was not too informative let's try the other three.   
$$\ii \to \left[ \begin{array}{cccc|cccc}
\cdot & 1 & \cdot & \cdot & \cdot & \cdot & \cdot & \cdot \\
\cdot & \cdot & \cdot & \cdot & 1 & \cdot & \cdot & \cdot \\
\cdot & \cdot & \cdot & 1 & \cdot & \cdot & \cdot & \cdot \\
\cdot & \cdot & \cdot & \cdot & \cdot & \cdot & 1 & \cdot \\ \hline 
\cdot & \cdot & \cdot & \cdot & \cdot & 1 & \cdot & \cdot \\
1 & \cdot & \cdot & \cdot & \cdot & \cdot & \cdot & \cdot \\
\cdot & \cdot & \cdot & \cdot & \cdot & \cdot & \cdot & 1 \\
\cdot & \cdot & 1 & \cdot & \cdot & \cdot & \cdot & \cdot  \end{array}\right] \text{ and } \jj \to 
\left[ \begin{array}{cccc|cccc}
\cdot & \cdot & 1 & \cdot & \cdot & \cdot & \cdot & \cdot \\
\cdot & \cdot & \cdot & \cdot & \cdot & \cdot & \cdot & 1 \\
\cdot & \cdot & \cdot & \cdot & 1 & \cdot & \cdot & \cdot \\
\cdot & 1 & \cdot & \cdot & \cdot & \cdot & \cdot & \cdot \\ \hline 
\cdot & \cdot & \cdot & \cdot & \cdot & \cdot & 1 & \cdot \\
\cdot & \cdot & \cdot & 1 & \cdot & \cdot & \cdot & \cdot \\
1 & \cdot & \cdot & \cdot & \cdot & \cdot & \cdot & \cdot \\
\cdot & \cdot & \cdot & \cdot & \cdot & 1 & \cdot & \cdot  \end{array}\right]
\text{ and }
\kk \to 
\left[ \begin{array}{cccc|cccc}
\cdot & \cdot & \cdot & 1 &  \cdot & \cdot & \cdot & \cdot \\
\cdot & \cdot & \cdot & \cdot & \cdot & \cdot &  1 & \cdot \\
\cdot & \cdot & \cdot & \cdot & \cdot & 1  & \cdot & \cdot \\
 1 & \cdot & \cdot & \cdot & \cdot & \cdot & \cdot & \cdot \\ \hline 
\cdot & \cdot & \cdot & \cdot & \cdot & \cdot & \cdot & 1  \\
\cdot & \cdot & 1 & \cdot & \cdot & \cdot & \cdot & \cdot \\
\cdot & 1 & \cdot & \cdot & \cdot & \cdot & \cdot & \cdot \\
\cdot & \cdot & \cdot & \cdot & 1 & \cdot & \cdot & \cdot  \end{array}\right]
$$
Do we lose any information by writing them as  $4 \times 4$ matrices?
$$\one \to \left[ \begin{array}{cccc} 1 & \cdot & \cdot & \cdot \\ 
\cdot & 1 & \cdot  & \cdot \\ 
\cdot & \cdot & 1 & \cdot \\
\cdot & \cdot & \cdot & 1  \end{array} \right] \text{and}\; -\one \to \left[ \begin{array}{cccc} -1 & \cdot & \cdot & \cdot \\ 
\cdot & -1 & \cdot  & \cdot \\ 
\cdot & \cdot & -1 & \cdot \\
\cdot & \cdot & \cdot & -1  \end{array} \right] \text{and}\;
\ii \to \left[
\begin{array}{cccc} \cdot & 1 & \cdot & \cdot \\ 
-1 & \cdot & \cdot & \cdot \\ 
\cdot & \cdot & \cdot & 1 \\
\cdot & \cdot & -1 & \cdot \end{array} \right] \text{and}\;
\jj \to \left[ \begin{array}{cccc} \cdot & \cdot & 1 & \cdot \\
\cdot & \cdot & \cdot & -1 \\ 
-1 & \cdot & \cdot & \cdot  \\ 
\cdot & 1 & \cdot & \cdot \end{array}\right] $$
So we've now found three different representations of the quaternion algbra as matrices of various sizes $2 \times 2$ and $4 \times 4$ and $8 \times 8$.  It seems like we can keep going\dots 
\begin{thebibliography}{}

\item \dots

\end{thebibliography}

\end{document}