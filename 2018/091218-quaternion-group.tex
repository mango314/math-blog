\documentclass[12pt]{article}
%Gumm{\color{blue}i}|065|=)
\usepackage{amsmath, amsfonts, amssymb}
\usepackage[margin=0.5in]{geometry}
\usepackage{xcolor}
\usepackage{graphicx}
\usepackage{amsmath}

\newcommand{\off}[1]{}
\DeclareMathSizes{20}{30}{20}{18}
\usepackage{tikz}

\newcommand{\one}{\mathbf{1}}
\newcommand{\ii}{\mathbf{i}}
\newcommand{\jj}{\mathbf{j}}
\newcommand{\kk}{\mathbf{k}}

\title{Scratchwork: Induced Representations}
\date{}
\begin{document}

\sffamily

\maketitle

\noindent The quaternions are a number system defined by three rules of multiplication.  They generalize complex numbers:
$$ \one \times \one = \one \text{ and } \ii \times \jj = \kk \text{ and } \ii \times \ii = -\one $$
These multiplications can we completed to form a group of order $8$.
$$
\begin{array}{r||rr|rr||rr|rr}
 \times & \one & \ii & \jj & \kk & -\one & -\ii & -\jj & -\kk \\ \hline
\one  & \one & \ii & \jj & \kk & -\one & -\ii & -\jj & -\kk \\
\ii & \ii & -\one & \kk & -\jj & -\ii & \one & -\kk & \jj \\ \hline 
\jj & \jj & - \kk & -\one & \ii & - \jj & \kk & \one & -\ii  \\
\kk & \kk & -\jj & -\ii & \one & -\kk & \jj & \ii & -\one \\ \hline \hline
-\one & -\one & -\ii & -\jj & -\kk & \one & \ii & \jj & \kk\\
-\ii & -\ii & \one & -\kk & \jj & \ii & -\one & -\kk & -\jj\\ \hline
-\jj & -\jj &  \kk & \one & -\ii &  \jj & -\kk & -\one & \ii  \\
-\kk & -\kk & \jj & \ii & -\one & \kk & -\jj & -\ii & \one
\end{array}
 $$
It looks like there are eight things being multiplied, so we made an $8 \times 8$ table.  There are eight things being permuted in 8 possible ways:
$$ \{ \one , \ii , \jj, \kk, -\one, -\ii, -\jj, -\kk \} $$
It may even be possible to whittle this down to four - with the inclusion of a minus sign $(-\one)$.  
\begin{eqnarray*}
-\one \times \one &=& -\one \\
-\one \times \ii  &=& -\ii \\
-\one \times \jj  &=& -\jj \\
-\one \times \kk  &=& -\kk
\end{eqnarray*}
Cayley's Theorem says every group can be placed into a permutation group.  We could call the elements of this group $\{1, 2, \dots, 8\}$. 
$$
\one \mapsto \left[\begin{array}{cccccccc} 1 & 2 & 3 & 4 & 5 & 6 & 7 & 8 \\
 1 & 2 & 3 & 4 & 5 & 6 & 7 & 8  \end{array} \right]
 $$
 and now we replace with different rows of the multiplication table:
 
\begin{eqnarray*}
\ii &\mapsto& \left[\begin{array}{cccccccc} 1 & 2 & 3 & 4 & 5 & 6 & 7 & 8 \\
 2 & 5 & 4 & 7 & 6 & 1 & 8 & 3  \end{array} \right] \\ \\
\jj &\mapsto& \left[\begin{array}{cccccccc} 1 & 2 & 3 & 4 & 5 & 6 & 7 & 8 \\
3 & 8 & 5 & 2 & 7 & 4 & 1 & 6 \end{array} \right] \\ \\
\kk &\mapsto& \left[\begin{array}{cccccccc} 1 & 2 & 3 & 4 & 5 & 6 & 7 & 8 \\
4 & 7 & 6 & 1 & 8 & 3 & 2 & 5 \end{array} \right] 
\end{eqnarray*}
The rule for $(-\one)$ looks a little bit complicated.  For the time being switch the first and second half.
$$
\one \mapsto \left[\begin{array}{cccccccc} 1 & 2 & 3 & 4 & 5 & 6 & 7 & 8 \\
5 & 6 & 7 & 8 & 1 & 2 & 3 & 4   \end{array} \right]
 $$ 
\newpage 
\noindent There's even other ways of representing the quaternion group.  Here's the more usual $2 \times 2$ matrices (in case you're scared of Quaternion objects).
$$
\one \mapsto \left[ \begin{array}{cc} 1 & 0 \\ 0 & 1 \end{array} \right] \;
\ii \mapsto \left[ \begin{array}{cc} i & 0 \\ 0 & i \end{array} \right]  \;
\jj \mapsto \left[ \begin{array}{rr} 0 & 1 \\ -1 & 0 \end{array} \right]  \;
\kk \mapsto \left[ \begin{array}{rr} 0 & i \\ i & 0 \end{array} \right] $$
It could even be instructive to write out the full $8 \times 8$ matrices:
$$  \one \to \left[ \begin{array}{cccccccc} 
1 & \cdot & \cdot  & \cdot  & \cdot  & \cdot  & \cdot  & \cdot  \\
\cdot & 1 & \cdot & \cdot & \cdot & \cdot & \cdot & \cdot \\
\cdot & \cdot & 1 & \cdot & \cdot & \cdot & \cdot & \cdot \\
\cdot &\cdot & \cdot & 1 \cdot & \cdot & \cdot & \cdot & \cdot \\
\cdot & \cdot & \cdot & \cdot & 1 & \cdot & \cdot & \cdot \\
\cdot & \cdot & \cdot & \cdot & \cdot & 1 & \cdot & \cdot  \\
\cdot & \cdot &  \cdot & \cdot & \cdot & \cdot &  1 & \cdot \\
\cdot & \cdot & \cdot & \cdot & \cdot & \cdot & \cdot &  1 \end{array} \right]$$
That one was not too informative let's try the other three.   
$$\ii \to \left[ \begin{array}{cccc|cccc}
\cdot & 1 & \cdot & \cdot & \cdot & \cdot & \cdot & \cdot \\
\cdot & \cdot & \cdot & \cdot & 1 & \cdot & \cdot & \cdot \\
\cdot & \cdot & \cdot & 1 & \cdot & \cdot & \cdot & \cdot \\
\cdot & \cdot & \cdot & \cdot & \cdot & \cdot & 1 & \cdot \\ \hline 
\cdot & \cdot & \cdot & \cdot & \cdot & 1 & \cdot & \cdot \\
1 & \cdot & \cdot & \cdot & \cdot & \cdot & \cdot & \cdot \\
\cdot & \cdot & \cdot & \cdot & \cdot & \cdot & \cdot & 1 \\
\cdot & \cdot & 1 & \cdot & \cdot & \cdot & \cdot & \cdot  \end{array}\right] \text{ and } \jj \to 
\left[ \begin{array}{cccc|cccc}
\cdot & \cdot & 1 & \cdot & \cdot & \cdot & \cdot & \cdot \\
\cdot & \cdot & \cdot & \cdot & \cdot & \cdot & \cdot & 1 \\
\cdot & \cdot & \cdot & \cdot & 1 & \cdot & \cdot & \cdot \\
\cdot & 1 & \cdot & \cdot & \cdot & \cdot & \cdot & \cdot \\ \hline 
\cdot & \cdot & \cdot & \cdot & \cdot & \cdot & 1 & \cdot \\
\cdot & \cdot & \cdot & 1 & \cdot & \cdot & \cdot & \cdot \\
1 & \cdot & \cdot & \cdot & \cdot & \cdot & \cdot & \cdot \\
\cdot & \cdot & \cdot & \cdot & \cdot & 1 & \cdot & \cdot  \end{array}\right]
\text{ and }
\kk \to 
\left[ \begin{array}{cccc|cccc}
\cdot & \cdot & \cdot & 1 &  \cdot & \cdot & \cdot & \cdot \\
\cdot & \cdot & \cdot & \cdot & \cdot & \cdot &  1 & \cdot \\
\cdot & \cdot & \cdot & \cdot & \cdot & 1  & \cdot & \cdot \\
 1 & \cdot & \cdot & \cdot & \cdot & \cdot & \cdot & \cdot \\ \hline 
\cdot & \cdot & \cdot & \cdot & \cdot & \cdot & \cdot & 1  \\
\cdot & \cdot & 1 & \cdot & \cdot & \cdot & \cdot & \cdot \\
\cdot & 1 & \cdot & \cdot & \cdot & \cdot & \cdot & \cdot \\
\cdot & \cdot & \cdot & \cdot & 1 & \cdot & \cdot & \cdot  \end{array}\right]
$$
Do we lose any information by writing them as  $4 \times 4$ matrices?
$$\one \to \left[ \begin{array}{cccc} 1 & \cdot & \cdot & \cdot \\ 
\cdot & 1 & \cdot  & \cdot \\ 
\cdot & \cdot & 1 & \cdot \\
\cdot & \cdot & \cdot & 1  \end{array} \right] \text{and}\; -\one \to \left[ \begin{array}{cccc} -1 & \cdot & \cdot & \cdot \\ 
\cdot & -1 & \cdot  & \cdot \\ 
\cdot & \cdot & -1 & \cdot \\
\cdot & \cdot & \cdot & -1  \end{array} \right] \text{and}\;
\ii \to \left[
\begin{array}{cccc} \cdot & 1 & \cdot & \cdot \\ 
-1 & \cdot & \cdot & \cdot \\ 
\cdot & \cdot & \cdot & 1 \\
\cdot & \cdot & -1 & \cdot \end{array} \right] \text{and}\;
\jj \to \left[ \begin{array}{cccc} \cdot & \cdot & 1 & \cdot \\
\cdot & \cdot & \cdot & -1 \\ 
-1 & \cdot & \cdot & \cdot  \\ 
\cdot & 1 & \cdot & \cdot \end{array}\right] $$
So we've now found three different representations of the quaternion algbra as matrices of various sizes $2 \times 2$ and $4 \times 4$ and $8 \times 8$.  It seems like we can keep going\dots 

\vfill 




\vfill

\begin{thebibliography}{}

\item \dots

\item Sir William Rowan Hamilton \\ 
\textbf{Elements of quaternions} \texttt{https://archive.org/details/elementsofquater00hamirich} \\
\textbf{Lectures on quaternions} \texttt{https://archive.org/details/lecturesonquater00hami}

\end{thebibliography}

\newpage

\noindent \textbf{9/13} At this moment, we're going to invoke the machinery of Group Representations.  We know for a fact there are $\mathbf{4}$ irreducible one-dimensional group representations.  
\begin{eqnarray*}
\one & \stackrel{\phi}{\mapsto} & 1 \in \mathbb{C} \\
\ii  & \mapsto & \pm 1 \text{ or } \pm i \in \mathbb{C}\\
\jj  & \mapsto & \pm 1 \text{ or } \pm i \in \mathbb{C}\\
\kk  & \mapsto & \phi(\ii) \times \phi(\kk)
\end{eqnarray*}
and one more representation as $2 \times 2$ matrices, which are defined over $\mathbb{C}$ as well.
$$ 
\one \mapsto \left[ \begin{array}{cc} 1 & 0 \\ 0 & 1 \end{array} \right] \;
\ii \mapsto \left[ \begin{array}{cc} i & 0 \\ 0 & i \end{array} \right]  \;
\jj \mapsto \left[ \begin{array}{rr} 0 & 1 \\ -1 & 0 \end{array} \right]  \;
\kk \mapsto \left[ \begin{array}{rr} 0 & i \\ -i & 0 \end{array} \right] $$ 
and then we are told, in a sense, we have all the group representations we will ever need. 
$$ |Q_8| = 8 = 2 \times 2 + 4 \cdot (1 \times 1) = \dim (2 \times 2) + 4 \cdot \dim (1 \times 1) $$
Here's what Wikipedia has to say about Schur's Lemma:
\begin{quotation}
\noindent In mathematics, Schur's lemma is an elementary but extremely useful statement in representation theory of groups and algebras. In the group case it says that if $M$ and $N$ are two finite-dimensional irreducible representations of a group $G$ and $\phi$ is a linear transformation from $M$ to $N$ that commutes with the action of the group, then either $\phi$ is invertible, or $\phi = 0$. An important special case occurs when $M = N$ and $\phi$ is a self-map.
\end{quotation}
To call Schur's lemma ``elementary" is to risk missing an opportunity, I think.  Nope I don't believe it for second. \\ \\
How can I cast doubt?  Here's a brand new representation I just made up, using polynomials.  Any $2 \times 2$ matrix becomes a map linear map
$$ 
 \left[ \begin{array}{cc} a & b \\ c & d \end{array} \right] : (x,y) \mapsto (ax+by, cx+dy)  $$ 
This can be done to polynomials as well.  Let's only allow quadratic terms $x^2, xy, y^2$.  Then:
\begin{eqnarray*}
x^2 &\mapsto& (ax+by)^2 \\
xy  &\mapsto& (ax+by)(cx+dy) \\
y^2 &\mapsto& (cx+dy)^2
\end{eqnarray*}
This linear map preserves the vector space of polynomials $\mathbb{C}[x^2, xy, y^2]$.  So we have a three-dimensional representation of the quaternions.
\begin{eqnarray*} \ii &\mapsto& 
\left[\begin{array}{ccc}
x^2 &\mapsto& -x^2 \\
xy  &\mapsto& -xy \\
y^2 &\mapsto& -y^2
\end{array}\right] = \left[ \begin{array}{rrr} -1 & 0 & 0 \\ 0 & -1 & 0 \\ 0 & 0 & -1\end{array}\right] \\ \\
\jj &\mapsto& 
\left[\begin{array}{ccc}
x^2 &\mapsto& y^2 \\
xy  &\mapsto& -xy \\
y^2 &\mapsto& x^2
\end{array}\right] \;\;\; = \left[ \begin{array}{rrr} 0 & 0 & 1 \\ 0 & -1 & 0 \\ 1 & 0 & 0\end{array}\right] \\ \\
\kk &\mapsto& 
\left[\begin{array}{ccc}
x^2 &\mapsto& y^2 \\
xy  &\mapsto& xy \\
y^2 &\mapsto& x^2
\end{array}\right] \;\;\; = \left[ \begin{array}{rrr} 0 & 0 & -1 \\ 0 & 1 & 0 \\ -1 & 0 & 0\end{array}\right]
\end{eqnarray*} 
Schur's Lemma tells us our $3 \times 3$ representation is a direct sum of 1D representations $\mathbb{C}[x^2, xy, y^2] \simeq \mathbb{C} \oplus \mathbb{C} \oplus  \mathbb{C}$.

\newpage  \noindent Let's explicitly compute the matrices for cubic polynomials:
\begin{eqnarray*} \ii &\mapsto& 
\left[\begin{array}{cccc}
x^3 &\mapsto&  -i\, x^3  \\
x^2y &\mapsto& -i\, x^2y \\ 
xy^2 &\mapsto& -i\, xy^2 \\
y^3 &\mapsto & -i\, y^3 & 
\end{array}\right] = \left[ \begin{array}{rr|rr} -i & 0 & 0 & 0\\ 
0 & -i & 0 & 0 \\ \hline 0 & 0 & -i & 0 \\
0 & 0 & 0 & -i \end{array}\right] \\ \\
\jj &\mapsto& 
\left[\begin{array}{cccc}
x^3 &\mapsto&   -y^3  \\
x^2y &\mapsto&  xy^2 \\ 
xy^2 &\mapsto&  -x^2y \\
y^3 &\mapsto &  x^3 & 
\end{array}\right]  \;\;\; = \left[ \begin{array}{rr|rr} 0 & 0 & 0 & 1 \\ 
0 & 0 & 1 & 0 \\ \hline
0 &  1 & 0 & 0 \\
1 & 0 & 0 & 0 \end{array}\right] \\ \\
\kk &\mapsto& 
\left[\begin{array}{cccc}
x^3 &\mapsto&   i\,y^3  \\
x^2y &\mapsto&  -i\,xy^2 \\
xy^2 &\mapsto&  i\,x^2y \\
y^3 &\mapsto &  -i\,x^3 & 
\end{array}\right]
\;\;\; = \left[ \begin{array}{rr|rr} 0 & 0 & 0 & i \\ 
0 & 0 & -i & 0 \\ \hline
0 &  i & 0 & 0 \\
-i & 0 & 0 & 0 \end{array}\right]
\end{eqnarray*} 
We are finding the \textbf{symmetric product} of the two representations.  In notation:
$$ \text{Sym}( \mathbf{2}) \simeq \mathbf{1} \oplus\mathbf{1} \oplus\mathbf{1}  \text{ and } \mathbf{2}\otimes\mathbf{2} \simeq \mathbf{1} \oplus \mathbf{1} \oplus \mathbf{1} \oplus \mathbf{1} $$
Even if the groups are comparatively straightforward, their \textit{plethysms}  are not.  \\ \\
\textbf{Ex} Is $\text{Sym}^3(\mathbf{2})$ isomorphic to $\mathbf{2} \oplus \mathbf{2}$ or $\mathbf{2} \oplus \mathbf{1} \oplus \mathbf{1}$ or $\mathbf{1} \oplus \mathbf{1} \oplus \mathbf{1} \oplus \mathbf{1}$ ? \\ \\
\textbf{Ans} Our ability to draw lines find the $2 \times 2$ matrices ourselves suggests the answer is  $\boxed{\mathbf{2} \otimes \mathbf{2}}$ . \\ \\
This game of Tetris goes on forever.  What we really have done is taken a representation of $SU(2)$ and found the induced representation.  These are called ``spin" and they are indexed by the half-integers (we could write $\frac{1}{2}\mathbb{Z}$ or $\mathbb{Z}$).  This is fairly old hat. \\ \\
The quaternions were also a subgroup of $S_8$ and we permuted $8$ objects.  This is called the regular representation (after we remove the trivial part). \hfill \textbf{Ex} Prove or disprove $\mathbf{8} = \mathbf{2} \oplus \mathbf{2} \oplus \mathbf{1} \oplus \mathbf{1} $ . \\ \\
There's an infinite group inside $SU(2)$.  Let's try $1 = (3/5)^2 + (4/5)^2$.  Then we can generate quaternions $a = \frac{3}{5} + \frac{4}{5}\mathbf{i} $ or $b = \frac{3}{5} + \frac{4}{5}\mathbf{j} $ or $c = \frac{3}{5} + \frac{4}{5}\mathbf{k} $ . These could be turned into $2 \times 2$ matrices:
$$ \left[
\begin{array}{cc} \frac{3}{5} + \frac{4}{5}\,i & 0 \\ \\
0 & \frac{3}{5} + \frac{4}{5}\, i\end{array} \right] \text{ or }
\left[
\begin{array}{cr} \frac{3}{5}  & -\frac{4}{5} \\ \\
\frac{4}{5} & \frac{3}{5}\end{array} \right] \text{ or }
\left[
\begin{array}{cc} \frac{3}{5} + \frac{4}{5}\,i & 0 \\ \\
0 & \frac{3}{5} + \frac{4}{5}\, i\end{array} \right] $$
and we can't stop there because all of this is well-trodden territory.  The two representations are called $\tikz{\begin{scope}[scale=0.25]\draw (0,0)--(1,0)--(1,1)--(0,1)--cycle;\draw (1,0)--(2,0)--(2,1)--(1,1)--cycle;\end{scope}}$ and 
$\tikz{\begin{scope}[scale=0.25]\draw (0,0)--(1,0)--(1,1)--(0,1)--cycle;
\draw (0,1)--(1,1)--(1,2)--(0,2)--cycle;\end{scope}}$ .  How about we turn these into the $3 \times 3$ matrices we found before.
$$ b = \frac{3}{5} + \frac{4}{5}\mathbf{k}
\; \mapsto \;
\frac{3}{5}\left[ \begin{array}{rrr} -1 & 0 & 0 \\ 0 & -1 & 0 \\ 0 & 0 & -1\end{array}\right] 
+ 
\frac{4}{5}\left[ \begin{array}{rrr} 0 & 0 & -1 \\ 0 & 1 & 0 \\ -1 & 0 & 0\end{array}\right]
= \left[ \begin{array}{rrr} -\frac{3}{5} & 0 & \frac{4}{5} \\ 0 & \frac{1}{5} & 0 \\ -\frac{4}{5} & 0 & -\frac{3}{5}\end{array}\right]$$
and the complex numbers get projected to a line?  These non-commuting matrices generate an infinite group. \\ \\
Somewhat more strict:
\begin{eqnarray*}
e_1 &\mapsto& \tfrac{3}{5} e_1 - \tfrac{4}{5} e_2 \\
e_2 &\mapsto& \tfrac{4}{5} e_1 + \tfrac{3}{5} e_2 \\ \\
e_1 \otimes e_2 &\mapsto& \big(\tfrac{3}{5} e_1 - \tfrac{4}{5} e_2\big) \otimes 
\big( \tfrac{4}{5} e_1 + \tfrac{3}{5} e_2 \big)
= \frac{12}{25} \big( e_1 \otimes e_1 \big)+  
\frac{9}{25} \big( e_1 \otimes e_2 \big)-
\frac{16}{25} \big( e_2 \otimes e_1 \big)- \frac{12}{25} \big(e_2 \otimes e_2 \big)
\end{eqnarray*}
and in addition -- since we have a symmetric product we really want:
$$ 
\frac{1}{2}\big[(e_1 \otimes e_2) + (e_2 \otimes e_1)\big]
\mapsto \frac{12}{25} \big( e_1 \otimes e_1 \big)+  
\frac{7}{25} \big[ \big( e_1 \otimes e_2 \big) + \big( e_2 \otimes e_1 \big) \big]- \frac{12}{25} \big(e_2 \otimes e_2 \big) $$
Shows you how long it's been since I've done this kind of algebra\dots 
\vfill

\begin{thebibliography}{}

\item William Fulton \textbf{Representation Theory: A First Course} (GTM \#129) Springer, 1991.

\item Benjamin Steinberg \textbf{Representation Theory of Finite Groups} (Universitext) Springer, 2012.

\item Ben Green \textbf{What is\dots an Approximate Group} Notices of the AMS. Volume 59 Number 5, 2012.

\end{thebibliography}

\end{document}