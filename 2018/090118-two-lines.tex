\documentclass[12pt]{article}
%Gumm{\color{blue}i}|065|=)
\usepackage{amsmath, amsfonts, amssymb}
\usepackage[margin=0.5in]{geometry}
\usepackage{xcolor}
\usepackage{graphicx}

% zeta funct{\color{blue}i}ons of cub{\color{blue}i}c f{\color{blue}i}elds

%\usepackage{p{\color{blue}i}font}
\usepackage{amsmath}

\newcommand{\off}[1]{}
\DeclareMathSizes{20}{30}{20}{18}

\newcommand{\two }{\sqrt[3]{2}}
\newcommand{\four}{\sqrt[3]{4}}





\usepackage{tikz}

\newcommand{\susy}{{\bf Q}}
\newcommand{\RV}{{\text{R}_\text{V}}}

\title{Scratchwork: Intersecton of Two Lines}
\date{}
\begin{document}

%\fontfam{\color{blue}i}ly{qag}\selectfont \fonts{\color{blue}i}ze{12.5}{15}\selectfont

\sffamily

\maketitle

\noindent In geometry class, we learn the Cramer rule for the intersection two lines.
\begin{eqnarray*}
a_1 x + b_1 y &=& c_1 \\
a_2 x + b_2 y &=& c_2
\end{eqnarray*}
And so the intersection of these two lines can be found with a \textbf{determinant} of a $2 \times 2$ matrix:
$$ x = \frac{\left|\begin{array}{cc} c_1 & b_1 \\ c_2 & b_2 \end{array} \right|}{\left|\begin{array}{cc} a_1 & b_1 \\ a_2 & b_2 \end{array} \right|} \quad\text{and}\quad 
y = \frac{\left|\begin{array}{cc} a_1 & c_1 \\ a_2 & c_2 \end{array} \right|}{\left|\begin{array}{cc} a_1 & b_1 \\ a_2 & b_2 \end{array} \right|} $$
In a Linear Algebra course - or a Geometry course - one might check that $a,b,c \in \mathbb{R}$ means our solutions $(x,y) \in \mathbb{R}^2$.  We don't have that for an integer problem $a,b,c \in \mathbb{Z}$ the solution remains in integers $(x,y) \in \mathbb{Z}^2$. \\ \\
Since the $+$ and $\times$ operations we do aren't too fancy, we can do Linear Algebra over a field such as $K = \mathbb{Q}$ or $K = \mathbb{C}$.   In addition, let's use a tiny bit of Exterior Algebra taken from a Geometry textbook. \\ \\
\textbf{Thm}
The points $A$, $B$ and $C$ are collinear if and only if $ A \wedge B + B \wedge C + C \wedge A = 0$. \\ \\
In our case, the equation has one line \boxed{$Ax + By = C$}\;.  In that case, we can write the Cramer rule in an more condensed way:
$$ Ax + By = C \to A \wedge (Ax + By)  = (A \wedge B) y = (A \wedge C) \to y = \frac{A \wedge C}{A \wedge B} $$
And a similar formula for $x$.  Is it okay to write the coordinate value of $x$ and $y$ as the ratio of two areas.  The geometric objects look kind of funny but OK.
$$ \text{[number]} = \frac{\text{[area]}}{\text{[area]}} $$
This is not outrageous.  Pedeo gives a careful derivation of the wedge product of two vectors:
$$ u \wedge v = (x_1 E_1 + x_2 E_2) \wedge (y_1 E_1 + y_2 E_2) = (x_1 y_2 - x_2 y_1) (E_1 \wedge E_2) $$
where $E_1, E_2 \in \mathbb{R}^2$ are unit vectors in the plane.  \\ \\
There are even more intersection formulas like this.  Two planes in Four dimensions intersect (generically) at a point. 
$$ \mathbb{R}^2 \cap \mathbb{R}^2 = \{ pt \} \text{ in } \mathbb{R}^4 $$
Since all we're doing is linear algebra, this still could work over $\mathbb{Q}$ we'd have $\mathbb{Q}^2 \cdot \mathbb{Q}^2 = [pt] \subseteq \mathbb{Q}^4$.  This is the beginnings of intersection theory and a lot of sheafy things could occur.

\vfill

\begin{thebibliography}{}

\item Joe Harris \textbf{Algebraic Geometry: A First Course} (GTM \#133) Springer, 1992.

\item  Gabriel Cram\`{e}r \textbf{Introduction \`{a} l'Analyse des Lignes Courbes Algébriques} \\ \texttt{https://archive.org/details/bub\_{}gb\_{}HzcVAAAAQAAJ}

\item Dan Pedoe.  \textbf{Geometry: A Comprehensive Course} Dover, 1970.

\end{thebibliography}



\end{document}