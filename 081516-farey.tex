\documentclass[12pt]{article}
%Gummi|065|=)
\usepackage{amsmath, amsfonts, amssymb}
\usepackage[margin=0.5in]{geometry}
\usepackage{xcolor}
\usepackage{graphicx}
\newcommand{\off}[1]{}
\DeclareMathSizes{20}{30}{21}{18}

\usepackage{tikz}

\title{\textbf{ Stumbling Across the Prime Number Theorem }}
\author{John D Mangual}
\date{}
\begin{document}

\fontfamily{qag}\selectfont \fontsize{25}{30}\selectfont

\maketitle

\noindent A few weeks ago I read that ratios of prime numbers are dense in the real number line. \\ \\
Using my script for primes I plotted a histogram of the values of $\frac{p}{q}$ for primes $0 < p < 2q < 1000$ (so there are about $10^6$ values in all).  \\ \\
There seems to be a cutoff around $p = q$ and a long tail.

\includegraphics[width=3.5in]{prime-01.png}

\newpage

\noindent What if we include all integers?  Our fractions have a nice flat line for $m < n$ and some type of curve for $n < m < 2n$. \\ \\
The set is $\{ \frac{m}{n}: 0 < n <  1000 \text{ and } 0 < \frac{m}{n} < 2 \}$.

\includegraphics[width=3.5in]{prime-02.png} \\
We should hope for a perfectly even histogram on the interval $[0,1]$ since our set of integrs include all fraction.  \\ \\If we are lucky the fractions have density $1$ in the set of real numbers.
$$ \overline{\mathbb{Q} \cap [0,1]} = [0,1] $$
We save the analysis of these two curves for another time.  The density of fractions $\frac{p}{q}\approx \alpha $ is equivalent to the \textbf{prime number theorem}.


\newpage

\noindent Restricting only to reduced fractions, $\frac{a}{b}$ relatively prime numbers have a perfectly even density
$$ \mathcal{F}=\big\{ \frac{a}{b}: 0 < a < 2b, \;\text{gcd}(a,b)=1 \big\} $$
These are the \textbf{Farey Fractions} - obviously these should have uniform density in the real number line $\mathbb{R}$ since they represent every fraction once and without repeat.

\includegraphics[width=4in]{ratio-02.png} \\
The Farey Fractions become uniformly even when the denominator is not too large:
$$ \sum_{0 \leq \frac{a}{b}< 1} e^{2\pi i \, m\frac{a}{b}} \ll \sqrt{N}$$
If we prove the cancellation is faster, it is equivalent to showing the \textbf{prime number theorem}.

\newpage

\noindent \textbf{Ramanujan's Sum}.  If $p$ is prime:
$$ \sum_{0 \leq \frac{a}{q} < 1} e\left( h \, \frac{a}{q} \right) = \sum_{ c | (h,q)} c\,  \mu \left(\frac{q}{c}\right)$$
If $h=1$ there is only one term on the right hand side:
$$ \sum_{0 \leq \frac{a}{q} < 1} e\left(  \, \frac{a}{q} \right) = \mu(q)$$
Here is the identity linking Ramanujan's sum to the Mobius averages:
$$ \sum_1^N e( h \, x_n) =  \sum_{d | h} d M(Q/d) $$
where $0 < x_n < 1$ ranges over all the Farey fractions $\mathcal{F}$. \vspace{12pt} \hrule \vspace{12pt}
\noindent I have not been able to find a proof of the prime number theorem in the book, but the results he's approving approach that of the \textbf{Riemann Hypothesis}.  It might be possible to prove PNT using the techniques in the book\footnote{I suspect that Weyl's Equidistribution theorem is strong enough to prove PNT.  Karamata's Theorem can be used to lead to PNT.  Both of these are consequences of Weierstrass Approximation Theorem\dots}.

\fontfamily{qag}\selectfont \fontsize{12}{10}\selectfont

\begin{thebibliography}{}

\item M. N. Huxley \textbf{Area, Lattice Points, and Exponential Sums} London Mathematical Society Monographs.  Oxford University Press, 1996.


\end{thebibliography}

\fontfamily{qag}\selectfont \fontsize{25}{30}\selectfont

\noindent \textbf{Part II} strategy proposals \\ \\
Somehow we need to improve our Weyl sums 
$$ \sum_1^N e( h \, x_n) =  \sum_{d | h} d M(Q/d)  = o(N^{1/2})$$
where $0 < x_n < 1$ are the Farey Fractions $\mathcal{F}$. \\ \\
However before we do that here we take some examples of theorems that may be equivalent to the Prime Number Theorem
$$ \sum_{n \leq x} \mu(n)^2 =  \frac{6}{\pi^2} \; x^2 + o(  x^{1/2}) $$
The topic of square-free numbers is very much developed by Cellarosi and Sinai in recent years and I have not closely examined.
$$ \sum_{n \leq x} d(n)
= x \log x + (2 C_0 - 1)x + o(x^{1/2}) $$
where $d(n) \approx x \log x $ counts the divisors of $n$.  \\ \\
In short, anything in Chapter 2 of Montgomery Vaughn is equivalent to PNT if $O(x) \Longrightarrow o(x) $ 

\newpage

\noindent \textbf{Large Sieve}

\newpage

\noindent \textbf{Horocycle Flow} \\ \\
How does the action $\times m $ act on all the Farey Fractions $0 < \frac{a}{d} < 1$ ? They get kind of mixed around a bit.   In fact, there are two maps:
$$ \frac{a}{d} \to \frac{a}{d} + (\dots) \hspace{0.25in}\text{or}\hspace{0.25in} \frac{a}{d} \to m \times \frac{a}{d} $$
Following any horocircle we can get an interesting ordering of fractions which might behave analagously to the Farey Fractions and still retain much of their behavior.  \\ \\
More importantly we can attempt to quantify how fast this horocycle flow is mixing.

\includegraphics{ford.png} \\
Are the Ford circles a discretization of the horocycle flow?

\newpage

\fontfamily{qag}\selectfont \fontsize{12}{10}\selectfont

\begin{thebibliography}{}

\item Francesco Cellarosi, Yakov G. Sinai \textbf{Ergodic Properties of Square-Free Numbers} \texttt{arXiv:1112.4691}

\item M. N. Huxley \textbf{Multiplicative Number Theory I: Classical Theory} Cambridge Studies in Advanced Mathematics.  Cambridge University Press, 2006.

\item  Ford, L. R. (1938), "Fractions", The American Mathematical Monthly, 45 (9): 586–601,

\end{thebibliography}


\end{document}

