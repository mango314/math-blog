\documentclass[12pt]{article}
%Gummi|065|=)
\usepackage{amsmath, amsfonts, amssymb}
\usepackage[margin=0.5in]{geometry}
\usepackage{xcolor}
\usepackage{graphicx}
\newcommand{\off}[1]{}
\DeclareMathSizes{20}{30}{21}{18}

\newcommand{\myhrule}{}

\usepackage{tikz}

\title{\textbf{ Examples: the Gamma Function }}
\author{John D Mangual}
\date{}
\begin{document}

\fontfamily{qag}\selectfont \fontsize{25}{30}\selectfont

\maketitle

\noindent \textbf{A} Let's show that $\Gamma(\frac{1}{2}) = \sqrt{\pi}$.  We can use the mirror formula:
$$ \Gamma(z) \Gamma(1-z) = \frac{\pi}{\sin \pi z} $$
and if we set $z = 1 - z = \frac{1}{2}$ our number pops out:
$$ \Gamma\left(\tfrac{1}{2}\right)^2 = \frac{\pi}{\sin \frac{\pi}{2}} = \pi $$
Why stop there set $z = \frac{1}{3}$ and we have:
$$ \Gamma(\tfrac{1}{3}) \Gamma(\tfrac{2}{3})
= \frac{\pi}{\sin \frac{\pi}{3}} = \frac{2\pi}{\sqrt{3}}$$
while shopping on MathWorld I found this formula, possibly generated by computer:
$$\frac{\Gamma(\frac{1}{24})\Gamma(\frac{11}{24})}{\Gamma(\frac{5}{24})\Gamma(\frac{7}{24})} = \sqrt{3}\cdot  \sqrt{2 + \sqrt{3}} $$
This is a lot more interesting!

\newpage

\noindent Noam Elkies shows that this formula can also be derived from the multiplication formula\footnote{Notice we never quite get away with a factor of $\sqrt{3^{1-6z}}$ this is similar to the doubling formula:
$$ \Gamma(z) \Gamma(z + \frac{1}{2}) = 2^{1-2z} \sqrt{\pi}\Gamma(2z)$$
The left side has poles at $z = -1, -2, -3, \dots$ as well as $z = -\frac{1}{2},  -\frac{3}{2},  -\frac{5}{2}, \dots$ and the right side has poles at all the negative half integers $z \in - \frac{1}{2}\mathbb{N}$.  While this kind of reasoning might make sense, it went under further scrutiny still.}:
$$ \Gamma(z)  \Gamma(z + \frac{1}{3}) 
 \Gamma(z + \frac{2}{3})
 = 2\pi \cdot 3^{\frac{1}{2} - 3z} \Gamma(3z)
$$
Let's follow Elkies' instructions at set $z = \frac{1}{24}$ and also $z = \frac{1}{8}$:
$$ \Gamma(\frac{1}{8})
\Gamma(\frac{11}{24}) \Gamma(\frac{19}{24})
= 2\pi \cdot 3^{\frac{1}{8}} \Gamma(\frac{3}{8})$$
but also
$$ \Gamma(\frac{1}{24})
\Gamma(\frac{3}{8}) \Gamma(\frac{17}{24})
= 2\pi \cdot 3^{\frac{5}{8}} \Gamma(\frac{1}{8})$$
and sure enough when you multiply the answer is:
$$ 
\Gamma(\frac{ 1}{24})
\Gamma(\frac{11}{24}) 
\Gamma(\frac{19}{24})
\Gamma(\frac{17}{24})
 = 4\pi^2 \sqrt{3}
$$

\newpage

\noindent Yet if we set $z = 5/24$ and $z = 7/24$ into the mirror formula:
$$ \Gamma(\frac{5}{24})
\Gamma(\frac{19}{24})
= \frac{\pi}{\sin 5\pi / 24}
$$
and also 
$$ \Gamma(\frac{7}{24})
\Gamma(\frac{17}{24})
= \frac{\pi}{\sin 7\pi / 24}
$$
and multiplying these we get:
$$ 
\Gamma(\frac{ 5}{24})
\Gamma(\frac{7}{24})
\Gamma(\frac{17}{24})
\Gamma(\frac{19}{24})
= 
\frac{\pi^2}{\sin \frac{5\pi}{24}
\sin \frac{7\pi}{24}}
 $$
 and there is even more cancellation:
 $$
 \frac{\Gamma(\frac{ 1}{24})
\Gamma(\frac{11}{24}) 
{\color{black!20!white}{\Gamma(\frac{17}{24})
\Gamma(\frac{19}{24})}} }
{\Gamma(\frac{ 5}{24})
\Gamma(\frac{7}{24})
{\color{black!20!white}{\Gamma(\frac{17}{24})
\Gamma(\frac{19}{24})}}} 
 = 4 \sqrt{3} \sin \frac{5\pi}{24}
\sin \frac{7\pi}{24} $$
It remains to show that:
$$ \sin \frac{5 \pi}{24} = 
\sin \frac{7 \pi}{24}
= \sqrt[4]{2 + \sqrt{3}} $$
I noticed immediately the action of:
\begin{itemize}
\item $z \mapsto 1 - z$
\item $z \mapsto z + \frac{1}{3}$
\item on $\frac{1}{24}\mathbb{Z}$
\end{itemize}
I never thought too much before of $\boxed{\frac{1}{24} + \frac{1}{3} = \frac{3}{8}}$ but here we are.

\newpage

\noindent \textbf{B} the second more complicated solution -- if we have no idea which cancelations might occur, we can just randomly use the triplication formula until we get someting that works:
$$ \Gamma(\tfrac{1}{3}x)
 \Gamma(\tfrac{1}{3}x+1)
  \Gamma(\tfrac{1}{3}x+2)
  = \frac{2\pi}{3^{x - \frac{1}{2}}}\Gamma(x) $$
In a way, I don't worry too much about the letter $\Gamma$ or the algebraic fractor of: $\frac{2\pi}{3^{x - \frac{1}{2}}}\Gamma(x) $.  \\ \\ 
I can just write a shorthand of brackets $[ \cdot ] $ so
$$ 
\Big[\tfrac{1}{3}x \Big]\oplus 
\Big[\tfrac{1}{3}x +1 \Big]\oplus 
\Big[ \tfrac{1}{3}x + 2 \Big]
\approx \big[ x \big]
 $$
 and I don't know which combination will work in advance so I keep writing them out:
 \begin{eqnarray}
\Big[\tfrac{1}{24} \Big]\oplus 
\Big[\tfrac{3}{\;8\;}\Big]\oplus 
\Big[\tfrac{17}{24} \Big]
&\approx& \Big[ \tfrac{1}{8} \Big] \\
\Big[\tfrac{1}{\;8\;} \Big]\oplus 
\Big[\tfrac{11}{24}\Big]\oplus 
\Big[\tfrac{19}{24} \Big]
&\approx& \Big[ \tfrac{3}{8} \Big] 
 \end{eqnarray}
 Then we can add these two equations and conclude:
$$
\Big[\tfrac{1}{24} \Big]\oplus 
\Big[\tfrac{11}{24}\Big]\oplus 
\Big[\tfrac{17}{24} \Big] \oplus
\Big[\tfrac{19}{24} \Big]
 \approx  \big[0\big]$$
These equations may wind up becoming faulty, but seem to do the bookkeeping for us.  At least part of it.
\newpage

\noindent \textbf{C} Can all Chowla-Selberg formulas be proven with careful use of the mirror + multiplication formulas?
$$ 
\log \Gamma(x)
= 
\left( \frac{1}{2}-x\right)(\gamma + log 2)
+ (1-x)\log \pi - \frac{1}{2} \log \sin \pi x
 \sum_{n=1}^\infty \frac{\log n}{n\pi}\sin 2 \pi n x
$$
Then multiply both sides by Legendre symbol:
$$
\sum_{n=1}^{p-1} \left(\frac{n}{p} \right)
\log \Gamma \left(\frac{n}{p} \right)
= - (\log + 2\pi ) \sum_{n=1}^{p-1} \left(\frac{n}{p} \right) n
+ \sqrt{p} \sum_{n=1}^\infty \left(\frac{n}{p} \right) \frac{\log n }{n \pi}
 $$
 and the Chowla-Selberg formula in a logarithmic form. \\ \\ 
 Here is an example:
$$  \Gamma(\tfrac{1}{4}) \Gamma(\tfrac{3}{4})\int_0^1  x^{-3/4}(1-x)^{-1/4}(1 - x/64)^{-1/4} \; dx $$
is equal to
$$ \left[  \frac{7\pi}{2} \times 
\frac{ \Gamma(1/7)\Gamma(2/7)\Gamma(4/7)}{
\Gamma(3/7)\Gamma(5/7)\Gamma(6/7)}
 \right]^{1/2}$$
Chowla's paper deserves a more careful reading than this\footnote{Somehow we needed the Fourier series to prove the multiplication formula and mirror formulas.  The Weierstrass product formula could yield a quick proof of $$ \Gamma(s) \Gamma(1-s) = \frac{\pi}{\sin \pi s} $$ It would be quick if we proved the Weierstrass product formula.  Fourier series also has the same fine print.}. 

\newpage

\noindent \textbf{D} For the moment we move on to the paper of Benedict Gross and David Rohrlich. \\ \\
Chowla and Selberg's paper starts off with a zeta function:
$$ Z(s) = \sum \frac{1}{(am^2 + bmn + cn^2)^s} $$
and they prove a functional equation relating $Z(s)$ and $Z(1-s)$ and somehow it leads to these $\Gamma$ function identities.  \\ \\
I'll take this moment to mention the Wallis Product formula:
$$ \frac{\Gamma(n + \frac{1}{2})}{\Gamma(n)} \to 1$$
sorry if the relation to the formula for $\frac{\pi}{2}$ is not clear. \\ \\
We've already shown the Chowla-Selberg formula and that $p=7$ does not follow from the 7-multiplication and mirror formulas. \\ \\
I guess the issue is resolved.  

\newpage

\fontfamily{qag}\selectfont \fontsize{12}{10}\selectfont

\begin{thebibliography}{}


\item MathOverflow  \textbf{show that $ \frac{\Gamma(\frac{1}{24})\Gamma(\frac{11}{24})}{\Gamma(\frac{5}{24})\Gamma(\frac{7}{24})} = \sqrt{3}\cdot \sqrt{2 + \sqrt{3}} $} \texttt{http://mathoverflow.net/q/249164}

\end{thebibliography}


\end{document}