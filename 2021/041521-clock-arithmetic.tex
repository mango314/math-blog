\documentclass[12pt]{article}
%Gumm{\color{blue}i}|065|=)
\usepackage{amsmath, amsfonts, amssymb}
\usepackage[margin=0.5in]{geometry}
\usepackage{xcolor}
\usepackage{graphicx}
\usepackage{amsmath}
\usepackage{hyperref}
\usepackage{ amssymb }
\usepackage{fontspec}
\usepackage{xcolor}

\usepackage{tikz-cd}

\newcommand{\off}[1]{}
\DeclareMathSizes{20}{30}{20}{18}
\usepackage{tikz}

%\setmainfont[Color=brown]{Linux Libertine}


\title{Tune-Up: Clock Arithmetic}
\date{}
\begin{document}

\sffamily

\maketitle

{\fontsize{16pt}{16pt}\selectfont 

\noindent \textbf{Thm} (2010) ``Non-Conventional" ergodic theorem:
\begin{itemize}
\item $T: \mathbb{Z}^r \curvearrowright (X, \Sigma, \mu)$ commuting probability-preserving $\mathbb{Z}^r$ actions
\item $(I_N)_{N \geq 1}$ F\o{}lner sequence of subsets of $\mathbb{Z}^r$.
\item $(a_N)_{N \geq 1}$ sequence of points in $\mathbb{Z}^r$
\item $f_1, f_2, \dots f_d \in L^\infty(\mu)$
\end{itemize}
The sequence of ``non-conventional" ergodic averages converges in $L^2(\mu)$.
$$ \frac{1}{|I_N|} \sum_{n \in I_n + a_N} \prod_{i=1}^d f_i \circ T_i^n $$
Let's try a re-phrasing of the theorem. \\ \\
\textbf{Thm} (2010) The sequence $x_n$ converges.  $x_n \to x$.  \\ \\
Let's try with a couple of more details.  We have the sequence of averages of other functions.\\ \\
\textbf{Thm} (2010) The sequence of functions $\text{Avg}_1(f), \dots \text{Avg}_n(f)$ converges in the function space $L^2(\mu)$.  Here $Avg_n = \sum_{i = 0}^{n-1} T^i$ or ``shuffle $n$ times", we have separated the procedure from the thing it's acting on.\\ \\
\textbf{Ex} $\mathbb{Q}^\times$ is a multiplicative group.  I am being sloppy we could write $\mathbb{Q}^\times \simeq \mathbb{Z}^\infty$.  This is a contradiction because we could write an equally true statement for any number field $F^\times \simeq \mathbb{Z}^\infty$.  The term seems to be ``loclly compact abelian group". \\ \\
\textbf{Ex} $\times 2 \times 3$ these are commuting operations.  The arithmetic makes it instantly clear even though 
\begin{itemize}
\item ``divide by 2 parts and shuffle"
\item ``divide by 3 parts and shuffle"
\end{itemize}
could be very different if we switch the operations. \\ \\
\textbf{d=1} This is the \textbf{Von Neumann ergodic theorem} which is already in the textbook\footnote{it's new textbook as of 2011}.
\vfill

\begin{thebibliography}{}

\item Tim Austin \textbf{On the Norm Convergence of non-conventional ergodic averges} \texttt{arXiv:0805.0320}

\end{thebibliography}

\end{document}