\documentclass[12pt]{article}
%Gumm{\color{blue}i}|065|=)
\usepackage{amsmath, amsfonts, amssymb}
\usepackage[margin=0.5in]{geometry}
\usepackage{xcolor}
\usepackage{graphicx}
\usepackage{amsmath}
\usepackage{hyperref}

\usepackage{fontspec}
\usepackage{xcolor}

\usepackage{tikz-cd}

\newcommand{\off}[1]{}
\DeclareMathSizes{20}{30}{20}{18}
\usepackage{tikz}

%\setmainfont[Color=brown]{Linux Libertine}


\title{Reminder: Group Theory}
\date{}
\begin{document}

\sffamily

\maketitle

{\fontsize{16pt}{16pt}\selectfont 

\noindent Let's get some help with computers.  What are the groups of order $|G| = 400$?  Here is the computer program.\footnote{\texttt{https://math.stackexchange.com/questions/4108993/the-221-groups-of-order-g-400} \\
\url{https://www.gap-system.org/}}

\begin{verbatim}
G := AllSmallGroups(400);;
List(G, g -> StructureDescription(g));
[ "C25 : C16", "C400", "C25 : C16", "C25 : Q16", "C8 x D50",
  "C25 : (C8 : C2)", "C25 : QD16", "D400", "C2 x (C25 : C8)",
  "C25 : (C8 : C2)", "C4 x (C25 : C4)", "C25 : (C4 : C4)",
  "C25 : (C4 : C4)", "C25 : ((C4 x C2) : C2)", "C25 : QD16", "C25 : D16",
  "C25 : Q16", "C25 : QD16", "C25 : ((C4 x C2) : C2)", "C100 x C4",
  "C25 x ((C4 x C2) : C2)", "C25 x (C4 : C4)", "C200 x C2",
  "C25 x (C8 : C2)", "C25 x D16", "C25 x QD16", "C25 x Q16",
  "C25 : (C8 x C2)", "C25 : (C8 : C2)", "C4 x (C25 : C4)",
  "C25 : (C4 : C4)", "C2 x (C25 : C8)", "C25 : (C8 : C2)",
  "C25 : ((C4 x C2) : C2)", "C2 x (C25 : Q8)", "C2 x C4 x D50",
  "C2 x D200", "C25 : ((C4 x C2) : C2)", "D8 x D50",
  "C25 : ((C4 x C2) : C2)", "Q8 x D50", "C25 : ((C4 x C2) : C2)",
  ...
]
\end{verbatim}
Notice there is is both $C_{25} \times QD_{16}$ and $C_{25}\ltimes QD_{16}$. \\ \\
In more commom notation we can write with the symbols $\times$ (direct product) and $\ltimes$ (indirect product):
\begin{itemize}
\item $G = (C_5 \ltimes Q_8 ) \times D_{10}$
\item $G = (C_5 \ltimes C_5) \ltimes (C_4 \times C_4) $
\item $G = C_2 \times ((C_5 \times C_5) \ltimes C_8)$
\item $G = D_8 \times ((C_5 \times C_5) \ltimes C_2)$
\end{itemize}

\newpage

\noindent {\color{green!80!black}\textbf{Induced Representations}}: \\ \\
$\text{Ind}^G_H \pi = \mathbb{C}[G] \otimes_{\mathbb{C}[H]} \otimes V$
\begin{itemize}
\item $G$ finite group
\item $H$ subgroup
\item $(\pi , V)$ representation of $H$
\end{itemize}
\textbf{Ex} Right regular representation.
$$ \textbf{reg} = \text{Ind}^G_1 1 = \mathbb{C}[G] \times_\mathbb{C} 1$$
Frobenius Reciprocity:
\begin{itemize}
\item $\langle \text{Ind}^G_H \psi, \phi \rangle_G = \langle \psi, \text{Res}^G_H \phi \rangle_H $ (category theory)
\item $\text{Hom}_{\mathbb{C}[G]} (\mathbb{C}[G] \otimes_{\mathbb{C}[H]}M, N) \simeq \text{Hom}_{\mathbb{C}[H]} (M, {}_{\mathbb{C}[H]} N ) $ (modules)
\item $\text{Ind}^G_H \dashv \text{Res}^G_H$ (adjunction)
\begin{itemize}
\item[o] $\text{Res}^G_H : \text{Rep}_G \to \text{Rep}_H$
\item[o] $\text{Ind}^G_H : \text{Rep}_H \to \text{Rep}_G$
\end{itemize}
\end{itemize}
\textbf{Ex} Adjoint functors: which categories and functors were used here?
$$ \text{hom}_\mathcal{C} (FY, X) \simeq \text{hom}_\mathcal{D} (Y, GX) $$
Here $F = \text{Ind}^G_H$ (induction) and $G = \text{Res}^G_H$ (restriction).  $\mathcal{C} = \text{Rep}_G$ and $\mathcal{D} = \text{Rep}_H$.
\vfill

\begin{thebibliography}{}

\item Wikipedia ``induced representation", ``"
\item \dots 

\end{thebibliography}

\end{document}