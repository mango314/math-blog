\documentclass[12pt]{article}
%Gumm{\color{blue}i}|065|=)
\usepackage{amsmath, amsfonts, amssymb}
\usepackage[margin=0.5in]{geometry}
\usepackage{xcolor}
\usepackage{graphicx}
\usepackage{amsmath}
\usepackage{hyperref}

\usepackage{fontspec}
\usepackage{xcolor}

\newcommand{\off}[1]{}
\DeclareMathSizes{20}{30}{20}{18}
\usepackage{tikz}

%\setmainfont[Color=brown]{Linux Libertine}


\title{Reading: Schemes}
\date{}
\begin{document}

\sffamily

\maketitle

{\fontsize{16pt}{16pt}\selectfont 

\noindent  If we grab the easier parts of the algebraic geometry textbook, we suggestions for some geometric constructions.\dots \\ \\
\textbf{Ex}: This example has to do with \textbf{parabola} and the Affrine plane  (something like $\mathbb{Q}^2$:
$$ \text{Spec} \, \mathbb{Q}[x,y]/(y^2 - x, x) \simeq \text{Spec}\, \mathbb{Q}[y]/(y^2) $$
Look like we can get the ``pre-image of -1" and get the complex numbers:
$$ \text{Spec}\mathbb{Q}[x,y]/(y^2 - x, x + 1) \simeq \text{Spec} \, \mathbb{Q}[y]/(y^2 + 1) \simeq \text{Spec}\mathbb{Q}[i] = \text{Spec} \, \mathbb{Q}(i) $$
So we are still trying to identify specific ``points" in our geometric object (which we are describing with rings and equation).  The pre-image over the  ``\text{generic point}":
$$ \text{Spec} \, \mathbb{Q}[x,y]/(y^2 - x)
 \otimes_{\mathbb{Q}[x]} \mathbb{Q}(x) \simeq  \text{Spec} \mathbb{Q}[y] \otimes_{\mathbb{Q}[y^2]} \mathbb{Q}(y^2)$$
 \textbf{Ex} ? What is th scheme-theoretic 
 { \color{blue} \textbf{fiber} } of 
 $ \text{Spec} \mathbb{Z}[i] \to \text{Spec} \mathbb{Z} $ over the prime $(p)$? \\ \\
\textbf{Ex}  Consider $\text{Spec} k[\epsilon]/(\epsilon^2) \to \text{Spec}\, k[x] = \mathbb{A}^1_k$ given by $x \mapsto \epsilon$.  The image of the \textbf{fuzzy point}. \\ \\
We'll consult with \texttt{math.Stackexchange} to see if these are correct.
\vfill

\begin{thebibliography}{}

\item The Rising Sea: Foundations of Algebraic Geometry \texttt{http://math.stanford.edu/\~{}vakil/216blog/}
\item \dots 

\end{thebibliography}


\end{document}