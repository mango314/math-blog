\documentclass[12pt]{article}
%Gumm{\color{blue}i}|065|=)
\usepackage{amsmath, amsfonts, amssymb}
\usepackage[margin=0.5in]{geometry}
\usepackage{xcolor}
\usepackage{graphicx}
\usepackage{amsmath}
\usepackage{hyperref}

\usepackage{fontspec}
\usepackage{xcolor}

\newcommand{\off}[1]{}
\DeclareMathSizes{20}{30}{20}{18}
\usepackage{tikz}

%\setmainfont[Color=brown]{Linux Libertine}

\newcommand{\red}{\tikz{\fill[ color=red] (0,0) circle (0.25);}}
\newcommand{\yellow}{\tikz{\fill [color=yellow] (0,0)--(0.5,0)--(0.5,0.5)--(0,0.5)--cycle;}}


\title{Reading: Euclid's Elements}
\date{}
\begin{document}

\sffamily

\maketitle

{\fontsize{16pt}{16pt}\selectfont 

\noindent  \textbf{BK 5 Prop 15} Magnitudes have the same ratio to one another as their equimultipes have. \\ \\
Let \red \;and \yellow \; be two magnitudes; \\ \\
then,  $\red : \yellow :: M'\,\red : M'\,\yellow $. \\ \\
For \begin{eqnarray*}
\red : \yellow &=& \red : \yellow \\
&=& \red : \yellow \\
&=& \red : \yellow 
\end{eqnarray*}
$\therefore \red: \yellow :: 4 \, \red: 4 \, \yellow $ (Book 5 Proposition 12 ). \\ \\
And as the same resoning is generaly applicable we have:
$$ \red: \yellow :: M'\red : M'\yellow $$
$\therefore$ Magnitudes have the same ratio, etc.
}

\newpage


\end{document}