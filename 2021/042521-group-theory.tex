\documentclass[12pt]{article}
%Gumm{\color{blue}i}|065|=)
\usepackage{amsmath, amsfonts, amssymb}
\usepackage[margin=0.5in]{geometry}
\usepackage{xcolor}
\usepackage{graphicx}
\usepackage{amsmath}
\usepackage{hyperref}

\usepackage{fontspec}
\usepackage{xcolor}

\usepackage{tikz-cd}

\newcommand{\off}[1]{}
\DeclareMathSizes{20}{30}{20}{18}
\usepackage{tikz}

%\setmainfont[Color=brown]{Linux Libertine}


\title{Reminder: Group Theory}
\date{}
\begin{document}

\sffamily

\maketitle

{\fontsize{16pt}{16pt}\selectfont 

\noindent Let's get some help with computers.  What are the groups of order $|G| = 400$?  Here is the computer program.\footnote{\texttt{https://math.stackexchange.com/questions/4108993/the-221-groups-of-order-g-400} \\
\url{https://www.gap-system.org/}}

\begin{verbatim}
G := AllSmallGroups(400);;
List(G, g -> StructureDescription(g));
[ "C25 : C16", "C400", "C25 : C16", "C25 : Q16", "C8 x D50",
  "C25 : (C8 : C2)", "C25 : QD16", "D400", "C2 x (C25 : C8)",
  "C25 : (C8 : C2)", "C4 x (C25 : C4)", "C25 : (C4 : C4)",
  "C25 : (C4 : C4)", "C25 : ((C4 x C2) : C2)", "C25 : QD16", "C25 : D16",
  "C25 : Q16", "C25 : QD16", "C25 : ((C4 x C2) : C2)", "C100 x C4",
  "C25 x ((C4 x C2) : C2)", "C25 x (C4 : C4)", "C200 x C2",
  "C25 x (C8 : C2)", "C25 x D16", "C25 x QD16", "C25 x Q16",
  "C25 : (C8 x C2)", "C25 : (C8 : C2)", "C4 x (C25 : C4)",
  "C25 : (C4 : C4)", "C2 x (C25 : C8)", "C25 : (C8 : C2)",
  "C25 : ((C4 x C2) : C2)", "C2 x (C25 : Q8)", "C2 x C4 x D50",
  "C2 x D200", "C25 : ((C4 x C2) : C2)", "D8 x D50",
  "C25 : ((C4 x C2) : C2)", "Q8 x D50", "C25 : ((C4 x C2) : C2)",
  ...
]
\end{verbatim}
Notice there is is both $C_{25} \times QD_{16}$ and $C_{25}\ltimes QD_{16}$. \\ \\
In more commom notation we can write with the symbols $\times$ (direct product) and $\ltimes$ (indirect product):
\begin{itemize}
\item $G = (C_5 \ltimes Q_8 ) \times D_{10}$
\item $G = (C_5 \ltimes C_5) \ltimes (C_4 \times C_4) $
\item $G = C_2 \times ((C_5 \times C_5) \ltimes C_8)$
\item $G = D_8 \times ((C_5 \times C_5) \ltimes C_2)$
\end{itemize}
Once we have these explicit descriptions of groups, we  look at the representation theory of finite groups.  Example, by linear algebra:
$$ \text{dim} \;\text{Ind}^G_H(\mathbf{1}_H) = [G:H] = \text{dim}(V) $$
At least theoretically we can relate the induced representation from the subgroups and the character theory of the subgroups.
$$ \sum_\rho (\dim \rho)^2 = |G| $$
This is called \textbf{Maschke's Theorem}.  We have 500 examples to check dimensions of representation and Harmonic analysis of finite groups.\footnote{The category is called $\text{Mod}_G$ so that functorial properties there could be related to character theoretic formulas here.}
\newpage

\noindent \textbf{Approximate Groups} 
Does "commutativity" matter? We've been studying the equation $ab = ba$.  It certainly works for numbers $2 \times 3 = 3 \times 2$  and we can describe when it fails:
$$
\left[ \begin{array}{cc} 1 & 1 \\ 0 & 1 \end{array} \right]
\left[ \begin{array}{cc} 1 & 1 \\ 1 & 0 \end{array} \right] = \left[ \begin{array}{cc} 2 & 1 \\ 1 & 0 \end{array} \right] \neq \left[ \begin{array}{cc} 1 & 2 \\ 1 & 0 \end{array} \right] = 
\left[ \begin{array}{cc} 1 & 1 \\ 1 & 0 \end{array} \right]
\left[ \begin{array}{cc} 1 & 1 \\ 0 & 1 \end{array} \right] $$
so we have lots of examples of non-commutativity.  \\ \\
\textbf{Lemma} Let $G$ be an aribtary group.  Let $A \subset G$ be a subset with $|A^2| \leq K|A|$.  Then $|A^{-1}A| \leq K^2 |A|$ and $|AA^{-1}|\leq K^2 A$. \\ \\
In my experience the proofs not very exciting.  Counting he possibilities on both sides and making sure both sides are equal.  Here's the counter-example the book has provided:
Let $H$ be a finite group and $G = H \ast \langle x \rangle$, the free-product of $H$ and the {infinite cyclic group} one generatory (basically $\mathbb{Z}$).  Set $A = H \cup \{  x \}$.  Then 
\[ |A^2 \leq 3 |A| \tag{$\ast$} \]
but $HxH \subseteq A^3$ and yet $|HxH| = |H|^2 \asymp |A|^2$.  \\ \\
This is their instance of \textbf{small tripling}.  We could imamgine $A \subseteq \mathbb{Z}$ then:
$$ A = A + A + A  \text{ or } |3A| \leq 3|A| $$
so once we throw away the exact relation $A + A = A$ (such as the arithmetic progression or a \textbf{subgroup} or \textbf{coset}  of $\mathbb{Z}$) we get to consider small-doubling or small-tripliing moves. \\ \\
\textbf{Ex} $(A \cup \{  1\} \cup A^{-1})^2 $ is an $O(K^9)$-approximate group.   

\newpage 
\noindent \textbf{representation theory}\footnote{\url{https://people.math.ethz.ch/~kowalski/representation-theory.pdf}} Just a reminder that $\int$ is just a short-hand of $\sum$ which just means "$+$".
$$ \int_G f(xg) \, d\mu(x) = \int_G f(x) d\mu(x) $$
we need Haar measure on group of choice $G = \mathbb{R}/\mathbb{Z}$ or $G = \mathbb{R}^d$ or (non-abelian) $G = \text{SO}(3)$ (the space of rotations of the sphere). We could get the approximate-groups by setting $f = 1_A$ ! \\ \\
\textbf{?} Where did we get these invariant measures and perfect lattices and spheres?   \\ \\
\textbf{Thm} (Peter-Weyl) Let $G$ be a compact topological group with Haar measure $\mu$.  Then the regular representation of $G$ on the space $L^2(G, \mu)$ decomposes to Hilbert spae direct sum:
$$ L^2(G, \mu) = \bigoplus_\rho M(\rho) $$
of isotypic components of the finite dimensional unitary representations, each $M(\rho)$ being isomorphic to $\dim(\rho)$ copies of $\rho$. \\ \\
\textbf{Q} What does it measn that $\text{SO}(3)$ is a \textbf{compact} topological group?
\vfill

\begin{thebibliography}{}

\item 

\end{thebibliography}

\end{document}