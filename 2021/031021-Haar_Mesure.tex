\documentclass[12pt]{article}
%Gumm{\color{blue}i}|065|=)
\usepackage{amsmath, amsfonts, amssymb}
\usepackage[margin=0.5in]{geometry}
\usepackage{xcolor}
\usepackage{graphicx}
\usepackage{amsmath}
\usepackage{hyperref}

\usepackage{fontspec}
\usepackage{xcolor}

\newcommand{\off}[1]{}
\DeclareMathSizes{20}{30}{20}{18}
\usepackage{tikz}

%\setmainfont[Color=brown]{Linux Libertine}


\title{Reading: Integrals}
\date{}
\begin{document}

\sffamily

\maketitle

{\fontsize{16pt}{16pt}\selectfont 

\noindent  Since these are \texttt{arXiv} papers written between different groups of professors and graduate students, we get references to the existence of objects without much supporting details, just references to textbooks.  \\ \\
Math papers phrase things as \textbf{existence} of objects, or objects that could exist.
The paper says something like:
$$  \left| \int_{A^\times} \Psi_A(x) \psi_s(x) \, d^\times_Ax \right| \ll_{n,e} \sqrt{UV} $$
We're told that an integral \textit{exists} (theoretically) and then they proceed to discuss function spaces (such as Schwartz spaces, or Hilbert spaces or $L^p$ spaces). Maybe we could write:
$$ \int_{k^\times} f(x) \, d^\times x $$ 
where $k$ is a field.  In fact, the paper says that a ``self-dual" measure is defined over a local Field.
\begin{itemize}
\item $\int_{\mathcal{O}^\times_{K,w}} d^\times_K x = \text{disc}_w(K)^{-\frac{1}{2}} $
\item $\int_{\mathcal{O}_{K,w}} d_K x = \text{disc}_w(K)^{-\frac{1}{2}} $
\item $d^\times x := \zeta_{F,v}(1) \frac{dx}{|x|_v} $ (normalization)
\end{itemize}
We can also define measures on $(\mathbb{A},+)$, $(\mathbb{A}_K,+)$ or $\mathbb{A}^\times$ and $\mathbb{A}_K^\times$ or $\mathbf{T}(\mathbb{A}) = \mathbb{A}_K^\times/\mathbb{A}^\times$, which is a torus and an example of a ``group scheme".  So once we carefully define the measure once, we theoretically get a definition of $\int$ over a \textbf{category} of objects.  
 
\vfill

\begin{thebibliography}{}

\item ELMV
\item \dots 

\end{thebibliography}


\end{document}