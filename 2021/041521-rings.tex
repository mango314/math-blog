\documentclass[12pt]{article}
%Gumm{\color{blue}i}|065|=)
\usepackage{amsmath, amsfonts, amssymb}
\usepackage[margin=0.5in]{geometry}
\usepackage{xcolor}
\usepackage{graphicx}
\usepackage{amsmath}
\usepackage{hyperref}

\usepackage{fontspec}
\usepackage{xcolor}

\usepackage{tikz-cd}

\newcommand{\off}[1]{}
\DeclareMathSizes{20}{30}{20}{18}
\usepackage{tikz}



\title{Lookup: Ring Theory}
\date{}
\begin{document}

\sffamily

\maketitle

{\fontsize{16pt}{16pt}\selectfont 
 
\noindent Our prototypical example of a ring is the number line, abstractly written as $\mathbb{Z}$.  This didn't happen until the 20th century with Emmy Noether (or something like that). \\ \\
\noindent We're pretty sure we need the tensor product.  A \textit{tensor} is just a \textbf{box}.  We needed tensors to build the theory of General Relativity and described ``curved" 3- and 4-dimensional spaces.  They are used to describe the \textbf{curvature} of different types of ``shapes" or ``spaces".  By the 1920's academics realized tensors would play a role in topology (the most \textit{qualitative} study of shapes -- invariant under high levels of distortion).  \\ \\
The na\"{i}ve way of constructing the tensor product would be merely to write $x \otimes y$ with $x, y \in R$ two elements of a ring.  We should have:
$$ x \otimes y \neq y \otimes x $$
Then we could have the basic properties of tensor produts:
\begin{itemize}
\item $x \otimes (y_1 + y_2) = (x \otimes y_1) + (x \otimes y_2)$
\item $(x_1 + x_2) \otimes y = x_1 \otimes y + x_2 \otimes  y$
\item $(r\, x ) \otimes y = x \otimes (r \, y)$
\end{itemize}
so we could have $x \in M$ and $y \in N$ elements of two $R$-modules (a generalization of matrices).  Then we continue the inspection of the properties of the ring.  Instead of building the tensor product element by element, there is a \textbf{bilinear map}:
$$ \otimes : M \times N \to M \otimes N $$
In our case, $x = \vec{x} \in M$ and $ y = \vec{y} \in N$ are vector spaces. \\ \\  
\textbf{Ex} if $M = \mathbb{R}$ and $N = \mathbb{R}$ then $M \times N = R \otimes R$. \\ \\
\textbf{Ex} $\mathbb{C} \otimes_{\mathbb{R}} \mathbb{C} = \mathbb{C}^2$. \\ \\
\textbf{Ex} $\mathbb{R} \otimes \mathbb{Z}[i] = \mathbb{R}[i] = \mathbb{C}$. \\ \\
\textbf{Ex} $\mathbb{Q}/\mathbb{Z} \otimes_\mathbb{Z} \mathbb{Q}/\mathbb{Z} = 0$. (This an example of \textbf{torsion}.)\\ \\
\textbf{Ex} $\mathbb{Z}/p\mathbb{Z} \otimes \mathbb{Z}/q\mathbb{Z} = 0$. \\ \\
\textbf{Ex} $\mathbb{Q} \otimes_{\mathbb{Z}} \mathbb{Q} = \mathbb{Q}$. \\ \\
So there are well-behaved rules for generating entire number systems.  In a graduate-level textbook or reference book, the category is called $R$-\text{Mod}.

\newpage

\noindent Category theory could let us organize the many different number systems and ``geometric" objects that arise in our computations.  The inner product which is just $x\cdot y = x_1 y_1 + x_2 y_2 + x_3 y_3$.  It turns out to be more correctly written as:
$$ x \cdot y = x_1 y^1 + x_2 y^2 + x_3 y^3 \in \mathbb{R} $$ could be thought us as a map from $\mathbb{R}^3 \times \mathbb{R}_3 \to \mathbb{R}$.   
$$ (\mathbb{R} \oplus \mathbb{R} \oplus \mathbb{R}) \otimes
( \mathbb{R} \oplus \mathbb{R} \oplus \mathbb{R})  = 9 (\mathbb{R} \otimes \mathbb{R}) = \mathbb{R}^9 $$ \\ \\

\noindent $M \otimes_R -$ and $- \otimes_R N$ are \textbf{right-exact functors} so they have well-behaved properties.  \\ \\
Let's do the following typing exercise, a \textbf{triangle}:
\[  
\begin{tikzcd}
M \times N \arrow{r}{\otimes} \arrow[swap]{dr}{f} & M \otimes_R N \arrow{d}{\tilde{f}} \\
 & G \end{tikzcd}
\] 
Here $\tilde{f} \circ \otimes = f$. \\ \\
The danger of ``universality" is that we have to at some point recover the original object.  Yet it's a succinct way of dealing with \textbf{everything}  at one. \\ \\
It would look funny to write the matrix object in terms of matrix objects:
$$
\left[\begin{array}{ccc} 
1 & 0 & 0 \\
0 & 1 & 0 \\
0 & 0 & 1 \end{array} \right] 
= (e_1 \otimes e^1) + (e_2 \otimes e^2) + (e_3 \otimes e^3 )$$
There's confusion about the algebraic objects that we are dealing with.  Here it's called a ``balanced product" or a ``tensor product".
\begin{itemize}
	\item $M \times N$ is called a balance product even though ther are only two factors.
	\item 
\end{itemize}
The category theory way is nice and clean, yet it might not be obvious to interpret.  Maybe we can be suspicious about the role of $\mathbb{Z}$ everywhere.
\footnote{Example if $n \in \mathbb{Z}$ then $n+1 \in \mathbb{Z}$.  Our number line rarely runs to more than $n = 100$ or $n = 1000$ and what if $n = 10^{100}$ or something like that.  We can't use a pocket calculator which is limited to 8 decimal places.  So there's always a notion of \textbf{next}.  Worse than that the operations of $+$ and $\times$ might behave oddly with the spillover with the decimal places.  And ``decimal" here is just the map $T: n \mapsto 10 \times x$. Example $2048 = 2 \times 10^3 + 4 \times 10 + 8 = (2 \times T^3 + 4 \times T + 8 \times I) \times 1$ this is also $2^10 = (T_2)^{10} \times 1$.
$$ 2 \times T^3 + 4 \times T + 8 \times I = (T_2)^{10} $$
Here we try to make decimals look like another operation.} \\ \\
The balanced products are written in terms of the $\text{Hom}$ functor.
$$ \text{Hom}_\mathbb{Z} (M \otimes_R N, G) \simeq \text{Hom}_R(M, \text{Hom}_\mathbb{Z}(N, G)) $$
So that our simple notations of $\sum v_iv^i = v^2$ had basic categorical content where our notions of ``number" had to be revised.
\vfill

\begin{thebibliography}{}

\item 

\end{thebibliography}

\end{document}