\documentclass[12pt]{article}
%Gumm{\color{blue}i}|065|=)
\usepackage{amsmath, amsfonts, amssymb}
\usepackage[margin=0.5in]{geometry}
\usepackage{xcolor}
\usepackage{graphicx}
\usepackage{amsmath}
\usepackage{hyperref}

\usepackage{fontspec}
\usepackage{xcolor}

\usepackage{tikz-cd}

\newcommand{\off}[1]{}
\DeclareMathSizes{20}{30}{20}{18}
\usepackage{tikz}



\title{Lookup: Ring Theory}
\date{}
\begin{document}

\sffamily

\maketitle

{\fontsize{16pt}{16pt}\selectfont 

\noindent We're pretty sure we need the tensor product.  A \textit{tensor} is just a \textbf{box}.  We needed tensors to build the theory of General Relativity and described ``curved" 3- and 4-dimensional spaces.  They are used to describe the \textbf{curvature} of different types of ``shapes" or ``spaces".  By the 1920's academics realized tensors would play a role in topology (the most \textit{qualitative} study of shapes -- invariant under high levels of distortion).  \\ \\
The na\"{i}ve way of constructing the tensor product would be merely to write $x \otimes y$ with $x, y \in R$ two elements of a ring.  We should have:
$$ x \otimes y \neq y \otimes x $$
Then we could have the basic properties of tensor produts:
\begin{itemize}
\item $x \otimes (y_1 + y_2) = (x \otimes y_1) + (x \otimes y_2)$
\item $(x_1 + x_2) \otimes y = x_1 \otimes y + x_2 \otimes  y$
\item $(r\, x ) \otimes y = x \otimes (r \, y)$
\end{itemize}
so we could have $x \in M$ and $y \in N$ elements of two $R$-modules (a generalization of matrices).  Then we continue the inspection of the properties of the ring.  Instead of building the tensor product element by element, there is a \textbf{bilinear map}:
$$ \otimes : M \times N \to M \otimes N $$
In our case, $x = \vec{x} \in M$ and $ y = \vec{y} \in N$ are vector spaces. \\ \\  
\textbf{Ex} if $M = \mathbb{R}$ and $N = \mathbb{R}$ then $M \times N = R \otimes R$. \\ \\
\textbf{Ex} $\mathbb{C} \otimes_{\mathbb{R}} \mathbb{C} = \mathbb{C}^2$. \\ \\
\textbf{Ex} $\mathbb{R} \otimes \mathbb{Z}[i] = \mathbb{R}[i] = \mathbb{C}$. \\ \\
\textbf{Ex} $\mathbb{Q}/\mathbb{Z} \otimes_\mathbb{Z} \mathbb{Q}/\mathbb{Z} = 0$. (This an example of \textbf{torsion}.)\\ \\
\textbf{Ex} $\mathbb{Z}/p\mathbb{Z} \otimes \mathbb{Z}/q\mathbb{Z} = 0$. \\ \\
\textbf{Ex} $\mathbb{Q} \otimes_{\mathbb{Z}} \mathbb{Q} = \mathbb{Q}$. \\ \\
So there are well-behaved rules for generating entire number systems.  In a graduate-level textbook or reference book, the category is called $R$-\text{Mod}.

\newpage

\noindent Category theory could let us organize the many different number systems and ``geometric" objects that arise in our computations.  The inner product which is just $x\cdot y = x_1 y_1 + x_2 y_2 + x_3 y_3$.  It turns out to be more correctly written as:
$$ x \cdot y = x_1 y^1 + x_2 y^2 + x_3 y^3 \in \mathbb{R} $$ could be thought us as a map from $\mathbb{R}^3 \times \mathbb{R}_3 \to \mathbb{R}$.   
$$ (\mathbb{R} \oplus \mathbb{R} \oplus \mathbb{R}) \otimes
( \mathbb{R} \oplus \mathbb{R} \oplus \mathbb{R})  = 9 (\mathbb{R} \otimes \mathbb{R}) = \mathbb{R}^9 $$ \\ \\

\noindent $M \otimes_R -$ and $- \otimes_R N$ are \textbf{right-exact functors} so they have well-behaved properties. 
\vfill

\begin{thebibliography}{}

\item 

\end{thebibliography}

\end{document}