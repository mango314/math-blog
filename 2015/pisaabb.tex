\documentclass[12pt]{article}
%Gummi|065|=)
\usepackage{amsmath, amsfonts, tabto, amssymb}
\title{Numbers and Entropy}
\usepackage{xcolor}
\usepackage[a4paper, total={6.5in, 10in}]{geometry}
\usepackage{framed}
\usepackage{tgadventor}
\colorlet{shadecolor}{red!10}
\author{John Mangual}
\date{}

\definecolor{green}{HTML}{BED46D}
\definecolor{blue}{HTML}{7A6BED}
\usepackage{hyperref}

\begin{document}
{\fontfamily{lmss}\selectfont

\maketitle



\noindent Fermat\footnote{\textbf{Efficiently finding two squares which sum to a prime} \tabto{9cm} http://math.stackexchange.com/q/5877/4997} showed that any $p = a^2 + b^2$ has a solution for $p$ prime and integers $a,b$ iff $p = 4k+1$.  Primes can be arbitrarily large, how do we find $(a,b)$? \newline

\noindent \textbf{Algorithm \#1 } 
\begin{itemize}
\item Let $x$ be quadratic non-residue and set $z=x^{\frac{p-1}{4}} + i $
\item Then $a + ib = \mathrm{gcd}(x^{\frac{p-1}{4}} + i ,p)$
\end{itemize}

\noindent \textbf{Algorithm \#2 } 
\begin{itemize}
\item Let $(\frac{a}{p})$ be the Legendre symbol\footnote{\textbf{Explicit formula for Fermat's $4k+1$ theorem} \tabto{9cm} http://math.stackexchange.com/a/74299/4997}.
\item $a = \sum_{0 \leq x < p} (\frac{x^3 - x}{p})$
\end{itemize}

\noindent \textbf{Algorithm \#3 } 
Let $c = \sqrt{-1} \mod p$ and $\mathrm{gcd}(p, i-c) = a+bi$ then $p = a^2 + b^2$. 
\begin{itemize}
\item Somehow need to compute $\sqrt{-1}$ (maybe theory of Pell's eq)
\item need GCD algorithm in $\mathbb{Z}[i]$.
\end{itemize}

\noindent \textbf{Algorithm \#4 }
\begin{itemize}
\item Find $z^2 = -1 \mod p$.
\item Take Euclidean algorithm of $(z,p) $ until $x,y < \sqrt{p}$
\end{itemize}

\noindent \textbf{Algorithm \#5 }
\begin{itemize}
\item Find $x = \frac{1}{2}\binom{2k}{k} \mod p$ and let  $y = x \cdot (2k)! \mod p$
\item {\color{gray} Why are $x,y < \frac{p}{2}$ ?} Gauss showed $x^2 + y^2 = p$ (that is not a congruence).
\end{itemize}

\section{Prime Number Theorem}

What happens to these algorithms as $p \to \infty$ ? No freaking idea.



\begin{thebibliography}{9}

\bibitem{BK}
Jean Bourgain, Alex Kontorovich. \newline \textit{Beyond Expansion II: Low-Lying Fundamental Geodesics}   \texttt{arXiv:1406.1366} \newline
\textit{Beyond Expansion II: Traces of Thin Semigroups} \texttt{arXiv:1310.7190}

\bibitem{K}
Dubi Kelmer. \textit{Quadratic irrationals and linking numbers of modular knots}  \texttt{arXiv:1205.2230}
 
\bibitem{AS}
Menny Aka, Uri Shapira. \textit{On the evolution of continued fractions in a fixed quadratic field}  \texttt{arXiv:1201.1280}

\bibitem{M}
Curtis McMullen \textit{Uniformly Diophantine numbers in a fixed real quadratic field } \url{http://www.math.harvard.edu/~ctm/papers/home/text/papers/cf/cf.pdf}

\end{thebibliography}
}
\end{document}
