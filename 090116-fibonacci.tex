\documentclass[12pt]{article}
%Gummi|065|=)
\usepackage{amsmath, amsfonts, amssymb}
\usepackage[margin=0.5in]{geometry}
\usepackage{xcolor}
\usepackage{graphicx}
\newcommand{\off}[1]{}
\DeclareMathSizes{20}{30}{21}{18}



\newcommand{\myhrule}{}

\usepackage{tikz}

\title{\textbf{ Fibonacci Numbers }}
\author{John D Mangual}
\date{}
\begin{document}

\fontfamily{qag}\selectfont \fontsize{25}{30}\selectfont

\maketitle

\noindent Every issue of Mathematics Magazine is flooded with proofs of Fibonacci identity.  I myself have solved a few.  So it is surprising to see a discussion by leading dynamicist and Fields Medallalist Curtis McMullen.  \\ \\
Let $\epsilon \in \mathbb{R}$ be an algebraic unit of degree two over $\mathbb{Q}$.  Then $x = \epsilon$ solves a quadratic equation:
$$ x^2 - ax + b = 0 $$
with $a,b \in \mathbb{Q}$.  McMullen writes instead:
$$ \epsilon^2 = t\epsilon - n $$
with $t = \mathrm{tr}_\mathbb{Q}^K(\epsilon)$ and $n = \mathrm{N}_\mathbb{Q}^K(\epsilon)= \pm 1 $. \newpage

\noindent In the number theory jargon:
\begin{itemize}
\item $\mathbb{Z}[\epsilon]$ is called a \textbf{order} in the field $K = \mathbb{Q}(\epsilon)$.
\item The discriminant is $D = t^2 - 4n > 0$.
\item $(1,\epsilon)$ is a basis for $\mathbb{Z}[\epsilon] \subset \mathbb{R}$.
\end{itemize}
We represent algebraic numbers by $2 \times 2$ matrices\footnote{this should be extremely bothersome... and we haven't even done cubic fields\dots}
$$ \epsilon = \left(
\begin{array}{cr} 
0 & -n \\
1 & t
\end{array}\right), 
 1 = \left(
\begin{array}{cr} 
1 & 0 \\
0 & 1
\end{array}
 \right),
  \sqrt{D} = \left(
\begin{array}{rr} 
-t & -2n \\
1 & t
\end{array}\right)
 $$
Here's one place where Curtis gets tricky.  He says:
$$ \mathrm{tr}_\mathbb{Q}^K: \mathrm{M}_2(K) \to \mathrm{M}_2(Q) $$
a $2 \times 2$ matrix in $K = \mathbb{Q}(\epsilon)$ is like a $4 \times 4$ matrix in $\mathbb{Q}$ (with some rules)\footnote{Schemes could be defined as matrices satisfying certain equations.  If Alexander Grothendieck hadn't been around, we could say these equations define a ``variety" but with additional problems.  And spend hours hunting through our commutative algebra textbooks for the properties of these rings.  So there you have it a \textbf{scheme} is a set of \textbf{equations} with certain \textbf{problems}.}.
$$
\left[
\begin{array}{cc|cc}
\;\cdot \; & \;\cdot \; & \;\cdot \; & \;\cdot \; \\
\;\cdot \; & \;\cdot \; & \;\cdot \; & \;\cdot \; \\ \hline
\;\cdot \; & \;\cdot \; & \;\cdot \; & \;\cdot \; \\
\;\cdot \; & \;\cdot \; & \;\cdot \; & \;\cdot \;  
 \end{array}
 \right]
 $$

\newpage

\fontfamily{qag}\selectfont \fontsize{12}{10}\selectfont

\begin{thebibliography}{}

\item Curtis McMullen.  \textbf{Uniformly Diophantine Fixed Numbers in a Real Quadratic Field}

\item Jean Bourgain, Alex Kontorovich.
\textbf{Beyond Expansion II: Traces of Thin Semigroups} \texttt{arXiv:1310.7190v1}

\end{thebibliography}


\end{document}