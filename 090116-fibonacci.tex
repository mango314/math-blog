\documentclass[12pt]{article}
%Gummi|065|=)
\usepackage{amsmath, amsfonts, amssymb}
\usepackage[margin=0.5in]{geometry}
\usepackage{xcolor}
\usepackage{graphicx}
\newcommand{\off}[1]{}
\DeclareMathSizes{20}{30}{21}{18}



\newcommand{\myhrule}{}

\usepackage{tikz}

\title{\textbf{ Fibonacci Numbers }}
\author{John D Mangual}
\date{}
\begin{document}

\fontfamily{qag}\selectfont \fontsize{24}{30}\selectfont

\maketitle

\noindent Every issue of Mathematics Magazine is flooded with proofs of Fibonacci identity.  I myself have solved a few.  So it is surprising to see a discussion by leading dynamicist and Fields Medallalist Curtis McMullen.  \\ \\
Let $\epsilon \in \mathbb{R}$ be an algebraic unit of degree two over $\mathbb{Q}$.  Then $x = \epsilon$ solves a quadratic equation:
$$ x^2 - ax + b = 0 $$
with $a,b \in \mathbb{Q}$.  McMullen writes instead:
$$ \epsilon^2 = t\epsilon - n $$
with $t = \mathrm{tr}_\mathbb{Q}^K(\epsilon)$ and $n = \mathrm{N}_\mathbb{Q}^K(\epsilon)= \pm 1 $. \newpage

\noindent In the number theory jargon:
\begin{itemize}
\item $\mathbb{Z}[\epsilon]$ is called a \textbf{order} in the field $K = \mathbb{Q}(\epsilon)$.
\item The discriminant is $D = t^2 - 4n > 0$.
\item $(1,\epsilon)$ is a basis for $\mathbb{Z}[\epsilon] \subset \mathbb{R}$.
\end{itemize}
We represent algebraic numbers by $2 \times 2$ matrices\footnote{this should be extremely bothersome... and we haven't even done cubic fields\dots}
$$ \epsilon = \left(
\begin{array}{cr} 
0 & -n \\
1 & t
\end{array}\right), 
 1 = \left(
\begin{array}{cr} 
1 & 0 \\
0 & 1
\end{array}
 \right),
  \sqrt{D} = \left(
\begin{array}{rr} 
-t & -2n \\
1 & t
\end{array}\right)
 $$
Here's one place where Curtis gets tricky.  He says:
$$ \mathrm{tr}_\mathbb{Q}^K: \mathrm{M}_2(K) \to \mathrm{M}_2(Q) $$
a $2 \times 2$ matrix in $K = \mathbb{Q}(\epsilon)$ is like a $4 \times 4$ matrix in $\mathbb{Q}$ (with some rules)\footnote{Schemes could be defined as matrices satisfying certain equations.  If Alexander Grothendieck hadn't been around, we could say these equations define a ``variety" but with additional problems.  And spend hours hunting through our commutative algebra textbooks for the properties of these rings.  So there you have it a \textbf{scheme} is a set of \textbf{equations} with certain \textbf{problems}.}.
$$
\left[
\begin{array}{cc|cc}
\;\cdot \; & \;\cdot \; & \;\cdot \; & \;\cdot \; \\
\;\cdot \; & \;\cdot \; & \;\cdot \; & \;\cdot \; \\ \hline
\;\cdot \; & \;\cdot \; & \;\cdot \; & \;\cdot \; \\
\;\cdot \; & \;\cdot \; & \;\cdot \; & \;\cdot \;  
 \end{array}
 \right]
 $$
 
 \newpage
 
\noindent \textbf{2} - some scratchwork \\ \\
How to turn $2 \times 2$ matrix into $3 \times 3$ matrix?
$$ 
\left(
\begin{array}{cc}
a & b \\ c & d
\end{array}
 \right)
$$
There is ``isomorphism" $\mathrm{SL}_2(\mathbb{R})\simeq \mathrm{SO}_{2,1}(\mathbb{R})$:
$$ \large \frac{1}{ad-bc} 
\left(
\begin{array}{ccc}
\frac{1}{2}(a^2 - b^2 - c^2 + d^2)
& ac-bd & \frac{1}{2}(a^2 - b^2 + c^2 - d^2) \\
ab-cd & bc+ad & ab + cd \\
\frac{1}{2}(a^2 + b^2 - c^2 - d^2)
& ac+bd &
\frac{1}{2}(a^2 + b^2 + c^2 + d^2)
\end{array}
 \right)
$$
If I set $a = d = 1$, $c=0$ and $b = t$:
$$  \normalsize
\left(
\begin{array}{rrr}
1 - \frac{1}{2} (a+b)^2  
&-(a+b) &- \frac{1}{2} (a+b)^2   \\
(a+b) \; & 1 & (a+b) \; \\
\frac{1}{2} (a+b)^2  
& (a+b) &
1+\frac{1}{2}(a+b)^2 
\end{array}
 \right)
=
\left(
\begin{array}{rrr}
1 - \frac{1}{2} a^2  
&-a &- \frac{1}{2} a^2   \\
a \; & 1 & a \; \\
\frac{1}{2} a^2  
& a &
1+\frac{1}{2}a^2 
\end{array}
 \right)
\left(
\begin{array}{rrr}
1 - \frac{1}{2} b^2  
&-b &- \frac{1}{2} b^2   \\
b \; & 1 & b \; \\
\frac{1}{2} b^2  
& b &
1+\frac{1}{2}b^2 
\end{array}
 \right)
$$
These help us solve the Pythagorean equation:
$$ x^2 + y^2 = z^2 $$
since we can reverse the equation\footnote{why don't we do this in real life?  $A+B=C$ so we deduce that $B = C - A$ and other deductions of this type?} one has:
$$ x^2 + y^2 - z^2 = 0 $$
This is the quadratic form being preserved by $\mathrm{SO}_{2,1}(\mathbb{R})$ also known as a ``\textbf{\color{blue!50!white}{spinor}}".  \\ \\
This is the metric used in \textbf{Special Relativity} and it is also used in the \textbf{Pythagorean Theorem}.  Nobody talks about this! \newpage

\noindent What about $(3,4,5)$ triangle? 
$$ 3^2 + 4^2 = 5^2 $$
That works.  This matrix equation has:
\begin{eqnarray}
x &=& 3(1 - \frac{1}{2}a^2)+ 4(-a)+5(- \frac{1}{2}a)    \\
y&=& 3a + 4 + 5a \\
z&=&  
3( \frac{1}{2}a^2)+ 4a + 5(1+ \frac{1}{2}a^2)
\end{eqnarray}
and then we always have a Pythagorean triple:
$$ x^2 + y^2 = z^2  $$
I might even finish off the algebra just a bit:
\begin{eqnarray}
x &=& 3-4a -4a^2   \\
y&=& 8a + 4 \\
z&=&  
5 + 4a + 4a^2
\end{eqnarray}
Doesn't it make sense?  This is true for all $a$:
$$ (4 - 1 - 4a - 4a^2)^2 + (8a + 4)^2 
= (4 + 1 +4a + 4a^2)^2 $$
All Pythagorean triples can be written this way for $m, n \in \mathbb{Z}$ -- here $m = 2$ and $n = 1 + 2a$
\begin{eqnarray}
x &=& m^2-n^2   \\
y&=& 2mn \\
z&=&  
m^2+n^2
\end{eqnarray}

\newpage

\fontfamily{qag}\selectfont \fontsize{12}{10}\selectfont

\begin{thebibliography}{}

\item Curtis McMullen.  \textbf{Uniformly Diophantine Fixed Numbers in a Real Quadratic Field}

\item Jean Bourgain, Alex Kontorovich.
\textbf{Beyond Expansion II: Traces of Thin Semigroups} \texttt{arXiv:1310.7190v1}

\end{thebibliography}


\end{document}