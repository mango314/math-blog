\documentclass[12pt]{article}
%Gummi|065|=)
\usepackage{amsmath, amsfonts, amssymb}
\usepackage[landscape, margin=0.5in]{geometry}
\usepackage{xcolor}
\newcommand{\off}[1]{}
\DeclareMathSizes{20}{25}{21}{17}

\title{\textbf{ Theta Functions and Eta Function }}
\author{John D Mangual}
\date{}
\begin{document}

\fontfamily{qag}\selectfont \fontsize{20}{30}\selectfont

\maketitle

\noindent Learn about modular properties of $\theta = \sum q^{n^2}$ and  $\eta =  q^{1/24} \prod (1 - q^n)$

\section{ \fontfamily{qag} \selectfont Theta Functions and $\Gamma_0(4)$}

\noindent Don Zagier's lecture notes on modular forms starts off with a couple of examples.  Number theory is insane, and the motivation for the very difficult theorems can sometimes follow from very simple situations and looking at them closely. \newline

\newpage

\begin{thebibliography}{9}

\bibitem{LT} S. Lang and H. Trotter, \textbf{Continued fractions for some algebraic numbers}, J. Reine Angew. Math. 255 (1972), 112-134. https://eudml.org/doc/151239

\bibitem{FGKP} Philipp Fleig, Henrik P. A. Gustafsson, Axel Kleinschmidt, Daniel Persson
Eisenstein series and automorphic representations arXiv:1511.04265

\bibitem{H}  Thomas C. Hull (April 2011). "Solving Cubics With Creases: The Work of Beloch and Lill" (PDF). American Mathematical Monthly: 307–315. doi:10.4169/amer.math.monthly.118.04.307.

\bibitem{KZ} S. Kharchev, A. Zabrodin \newline 
arXiv:1502.04603 Theta vocabulary I.  \newline
arXiv:1510.02699 Theta vocabulary II. \newline


\end{thebibliography}


\end{document}
