\documentclass[12pt]{article}
%Gummi|065|=)
\usepackage{amsmath, amsfonts, amssymb}
\usepackage[landscape, margin=0.5in]{geometry}
\usepackage{xcolor}
\usepackage{graphicx}
\newcommand{\off}[1]{}
\DeclareMathSizes{20}{30}{21}{18}

\title{\textbf{ Sum of 3 Squares Theorem, Hasse Principle, Banach-Tarski Paradox }}
\author{John D Mangual}
\date{}
\begin{document}

\fontfamily{qag}\selectfont \fontsize{25}{30}\selectfont

\maketitle

\noindent Today we read some difficult papers and do scratchwork. \newline

\noindent $n = a^2 + b^2 + c^2$ iff $n = 4^m(8k+7)$ \newline

\noindent this requires Hasse principle but there is no way to ``multiply" solutions from the various primes into a one solution.  \newline

\noindent Why does solving $10101 = a^2 + b^2 + c^2$ require factoring at all?  We are just adding squares\footnote{Woldfram Alpha says $10101 = 3 \times 7 \times 13 \times 37$}.

\newpage

\noindent The strategy for solving $n  = a^2 + b^2 + c^2$ over $\mathbb{Z}$ is to 
\begin{itemize}
\item solve $n = a^2 + b^2 + c^2$ over $\mathbb{Q}$ 
\item ``reduce" to solution over $\mathbb{Z}$
\end{itemize}
Question\footnote{It's like 50/50 I say ``it work" I say ``produces an integer solution".  Weak langage for some people, strong language for others.}\;: why does this reduction always work? 

\vspace{12pt}
\hrule
\vspace{12pt}

\noindent Integers are a rather clumsy set to work with.  Over $\mathbb{R}$ we can find all solutions:
$$ (x,y,z) = (\cos \theta, \sin \theta \cos \phi, \sin \theta \sin \phi) \in S^2 $$
These are related by rotations of the sphere, which any engineering student can write down:
$$ \left[ \begin{array}{cr|c} 
\cos \theta & -\sin \theta & 0 \\
\sin \theta & \cos \theta & 0 \\ \hline
0 & 0 & 1
\end{array} \right]
, \left[ \begin{array}{c|cr} 
1 & 0 & 0 \\ \hline
0 & \cos \theta & -\sin \theta  \\
0 & \sin \theta & \cos \theta  
\end{array} \right]  \in SO( 3, \mathbb{R} )$$
This brings us to the topic of \textbf{Lie Groups} named after Sophus Lie.
\newpage

\noindent If I have a solution $n = a^2 + b^2 + c^2$ I can do a few things to generate more solutions:
\begin{itemize}
\item $(x,y,z) \mapsto (-x,y,z)$
\item $(x,y,z) \mapsto (x,z,y)$
\end{itemize}
The rotation group over matrices is not very exciting:
$$ \left[ \begin{array}{cr|c} 
0 & -1 & 0 \\
1 & 0 & 0 \\ \hline
0 & 0 & 1
\end{array} \right]
, \left[ \begin{array}{c|cr} 
1 & 0 & 0 \\ \hline
0 & 0 & -1  \\
0 & 1 & 0
\end{array} \right] , 
\left[ \begin{array}{c|c|r} 
\pm 1 & 0 & 0 \\ \hline
0 & \pm 1 & 0  \\ \hline
0 & 0 & \pm 1
\end{array} \right]  \in SO( 3, \mathbb{\mathbb{Z}} )$$
$\theta$ can't be too many values: $0^\circ, 90^\circ, 180^\circ, 270^\circ$ just right angles.
\vspace{12pt}
\hrule
\vspace{12pt}
\noindent We need infinitely many primes in arithmetic progressions to solve $n = a^2 + b^2 + c^2$ over $\mathbb{Q}$ \newline

\noindent The last step in Serre's recipe to solve over $\mathbb{Z}$ is to \textbf{rotate} our solution over the integer fractions into solutions over integers.  \newline

\noindent The mystery is why does it always work?
\newpage

It is not hard to write down elements of $SO(\mathbb{Q})$.  One second thought... OK... Here $\cos \theta = \frac{3}{5}$. 

$$  \left[ \begin{array}{cr|c} 
\cos \theta & -\sin \theta & 0 \\
\sin \theta & \cos \theta & 0 \\ \hline
0 & 0 & 1
\end{array} \right] 
=  \left[ \begin{array}{cr|c} 
\frac{3}{5}& -\frac{4}{5} & 0 \\
\frac{4}{5} & \frac{3}{5} & 0 \\ \hline
0 & 0 & 1
\end{array} \right],
\left[ \begin{array}{c|cr} 
1 & 0 & 0 \\ \hline
0 & \frac{3}{5}& -\frac{4}{5}  \\
0 & \frac{4}{5} & \frac{3}{5}  
\end{array} \right]\in SO(3,\mathbb{Q}) $$
 These two fractions along create infinitely many solutions, which wrap around the sphere... all solving $3^2 + 4^2 = 5^2$ \newline

\noindent  What are our friends, the \textbf{Pythagorean triples} doing here?  Here in a more chaotic guise.. 
$$(x,y,z) \mapsto ( \frac{3}{5}x - \frac{4}{5}y , 
\frac{4}{5}x + \frac{3}{5}y
, z)  \text{ or }
( x, \frac{3}{5}y - \frac{4}{5}z , 
\frac{4}{5}y + \frac{3}{5}z)$$
All pythagorean triples can be generated by two transformstions.  So this is called \textbf{the free group on two elements}.
\newpage

I think we understand $SO(3,\mathbb{R})$ very well but what about the other completions? $SO(3,\mathbb{Q}_2),SO(3,\mathbb{Q}_3),SO(3,\mathbb{Q}_5)$ ? \newline

\noindent The (free group on two elements) generated by two rotations\footnote{http://mathoverflow.net/q/234869/1358 \textbf{Can two rational rotations $F_2 = \langle A, B \rangle \to SO(3)$ efficiently approximate the $3 \times 3$ identity matrix?}}
$$  \overline{ \left \langle {  \left[ \begin{array}{cr|c} 
\frac{3}{5}& -\frac{4}{5} & 0 \\
\frac{4}{5} & \frac{3}{5} & 0 \\ \hline
0 & 0 & 1
\end{array} \right],
\left[ \begin{array}{c|cr} 
1 & 0 & 0 \\ \hline
0 & \frac{3}{5}& -\frac{4}{5}  \\
0 & \frac{4}{5} & \frac{3}{5}  
\end{array} \right] } 
\right \rangle } = SO(3,\mathbb{Q}) $$
These rotations have no where to go but rotation the sphere in various ways. \newline

\noindent So infinitely many of these should cluster around a point, so these rotations are dense in the rotation group $SO(3, \mathbb{R})$.
$$ 
\left[ \begin{array}{cr|c} 
\frac{1}{\sqrt{2}}& -\frac{1}{\sqrt{2}} & 0 \\
\frac{1}{\sqrt{2}} & \frac{1}{\sqrt{2}} & 0 \\ \hline
0 & 0 & 1
\end{array} \right]
$$

How close can we get?  How much work does it take?
\newpage

\fontfamily{qag}\selectfont \fontsize{12}{10}\selectfont

\begin{thebibliography}{}

\item JP Serre \textbf{Course on Arithmetic} Springer-Verlag

\item Jordan S. Ellenberg, Philippe Michel, Akshay Venkatesh \newline
\textbf{Linnik's ergodic method and the distribution of integer points on spheres} \texttt{arXiv:1001.0897
}



\end{thebibliography}


\end{document}
