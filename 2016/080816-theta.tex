\documentclass[12pt]{article}
%Gummi|065|=)
\usepackage{amsmath, amsfonts, amssymb}
\usepackage[margin=0.5in]{geometry}
\usepackage{xcolor}
\usepackage{graphicx}
\newcommand{\off}[1]{}
\DeclareMathSizes{20}{30}{21}{18}

\usepackage{tikz}

\title{\textbf{ Theta Functions }}
\author{John D Mangual}
\date{}
\begin{document}

\fontfamily{qag}\selectfont \fontsize{25}{30}\selectfont

\maketitle

$$\theta(x; p) = (x;p)_\infty (px^{-1}; p)_\infty = \exp \left( - \sum_{ m \neq 0} \frac{x^m}{m(1-p^m)} \right) $$
another one
$$\theta(z; q) := (z;q)_\infty (q/z ;q)_\infty = \frac{1}{(q;q)_\infty} \sum_{k \in \mathbb{Z}} z^k q^{\binom{k}{2}}  $$
the shifted factorials are defined by:
$$ (z;q)_\infty = \prod_{i \geq 0} (1 - z q^i)$$
Let's see if
$$ \binom{k}{2} = \frac{k(k-1)}{2} = \frac{k^2}{2} - \frac{k}{2}$$

\newpage

\noindent Then it could be:

$$ \theta( q^2 ; q) = \frac{1}{(q; q^2) }
\sum_{k \in \mathbb{Z}} q^k q^{2\binom{k}{2}}
 = \frac{1}{(q; q^2) }
\sum_{n \in \mathbb{Z}}  q^{n^2}
$$
Wikipedia has
$$ \sum_{n \in \mathbb{Z}}
a^{\frac{n(n+1)}{2}}
b^{\frac{n(n-1)}{2}}
= (-a; ab)_\infty (-b; ab)_\infty (ab; ab)_\infty
$$
and we can set $a = b = q$:
$$
\sum_{n \in \mathbb{Z}}
q^{n^2}
= (-q; q^2)_\infty (-q; q^2)_\infty (q^2; q^2)_\infty
 $$
 This also seems odd we can try
$$ \theta(q;q^2)
= (q;q^2)_\infty (q;q)_\infty (q^2; q^2)_\infty = \sum_{n \in \mathbb{Z}}q^{n^2}$$

\newpage 

\noindent It might parameterized in terms of two angles:
$$ \theta(z; \tau) = \sum_{n \in \mathbb{Z}} e^{\pi i n^2 \tau + 2\pi i n z}$$
which has another triple product
$$\prod_{m=1}^\infty
(1 - e^{2\i m i \tau})
\left[ 1 + e^{(2m-1)\pi i \tau + 2\pi i z} \right]
\left[ 1 + e^{(2m-1)\pi i \tau - 2\pi i z} \right]$$
Then $q = e^{2\pi i \tau}$ and $x = e^{2\pi i z}$:
$$\theta( 0; q) =  \prod (1 - q^2 )(1 + q^{2m-1})(1 - q^{2m+1}) $$
This is a beautiful triple product but we have to write in terms of rising and falling factorials.
$$
\sum_{n \in \mathbb{Z}}
q^{n^2}
= (-q; q^2)_\infty (-q; q^2)_\infty (q^2; q^2)_\infty
 $$
 \hrule
\vspace{12pt}
\noindent The exponent formula looks like
$$ \log(1-x) = \sum \frac{x^m}{m} $$
and the geometric series formula:
$$ \sum p^{km} = \frac{1}{1-p^m}$$
\newpage

\noindent If we put two of them together it says:
$$ \sum_m \sum_k \frac{1}{m}x^m p^{km} = \sum_m \frac{1}{m} \frac{x^m }{1 - p^m} $$
This is very much the logarithm in the beginning of this article.
% Lambert Series, Bernoulli identity
 \newpage

\noindent \textbf{Part II}

\noindent So one big problem I will have with a lot of elliptic index paper with $\theta$ functions eveywhere is the normalization.  And their endless obsession with modular invariance\footnote{If the object is invariant under $\text{SL}_2(\mathbb{Z})$ or a congruence subgroup $[ \Gamma_0(N) : \text{SL}_2(\mathbb{Z})] = N$ or a \textit{non-congruenc} group.  There are many possibilities that Nekrasov and Shatashvili do not account for ('cuz they're not interested).} \\ \\
Uh... so before I get into that we rewind to 2003 before a lot of this paper and read through Appendix A of Nekrasov-Okounkov:
$$ \gamma_{\hbar}(x; \Lambda)
= \frac{d}{ds}\bigg|_{s=0} \frac{\Lambda^s}{\Gamma(s)} \int_0^\infty \frac{dt}{t} t^s \frac{e^{-tx}}{(e^{\hbar t}-1)(e^{-\hbar t}-1)} $$
This is a mouth-ful but notice right away this is a \textbf{Mellin transform} and also this is \textbf{zeta regularization}. The Nekrasov partition function is \textit{badly} divergent (as most physics formulas are) and here is one way to fix it.\\ \\
However, these gentlement have used a very common idea in number theory.  Here is a baby example:
$$ \sqrt{1} -\sqrt{2} + \sqrt{3} - \sqrt{4} + \dots = (\sqrt{1} -2\sqrt{2} + \sqrt{3}) + (\sqrt{2} -2\sqrt{3} + \sqrt{4}) + \dots $$
Uh... hopefully I remember later\footnote{but \textbf{you} and read \texttt{http://math.stackexchange.com/q/1896464/4997}}
  

\newpage

\noindent Nekrasov and Okounkov state this really is zeta-function regularization so we have
$$ \gamma_{\hbar}(0; \Lambda) = - \frac{1}{12} $$
and even some instance of the volumes of the unitary groups:
$$ \log \big( \text{Vol}\; U(N) \big) = \gamma_1(N; 1)$$ 
and the other functions $\gamma_{\epsilon_1, \epsilon_2}$ are embellishments\footnote{and a nuisiance to read -- painful on the eyes}. \\ \\
These $\gamma_\hbar$ satisfy a second-difference equation:
$$ \gamma_\hbar(x - \hbar, \Lambda  )
- 2\gamma_\hbar(x , \Lambda  )
+ \gamma_\hbar(x + \hbar, \Lambda  )
 = \log \left( \frac{x}{\Lambda} \right)$$
Theres so many logs floating around but I really want to talk about this $\Lambda$:
$$ \sum [\Lambda(n) - 1] \frac{e^{-ny}}{1 - e^{-ny}} \sim - \frac{2\gamma}{y} $$
Then by the Hardy-Littlewood Tauberian theorem (for Lambert series)\footnote{which we will argue is the same kind of regularization as Nikita Nekrasov uses}:
$$ \sum_{n=1}^\infty \frac{\Lambda(n)-1}{n} = - 2\gamma $$
If this thing converges at all the coefficients must have been small:
$$ \sum_{n \leq x} [ \Lambda(n)-1 ] = o(x)$$
and this is very much equivalent to the Prime Number Theorem. \\ \\
Here the $\Lambda$ in question is the Van Mangold function:
$$ \Lambda(n) = \left\{ 
\begin{array}{cl} 
\log p &  \text{if }n = p^k \\
0 & \text{otherwise}
\end{array}
\right. $$
and we regularize the Lambert sum to a normal finite sum
$$ 
y \sum_{n = 1}^\infty \frac{ (\Lambda(n) - 1)e^{-ny}}{1 - e^{-ny}}
\approx 
\sum_{n = 1}^\infty \frac{ (\Lambda(n) - 1)}{n} e^{-ny} \to - 2\gamma
$$
if we let $y \to 0$
\newpage

\noindent \textbf{Part III} -- Review with some more details\footnote{So as you might guess the theme is ambiguity in the literature.  The confusion as an outider to see the same classical formula in 20 different places -- each one with their opinion how it should be developed (or ignored completely).} \\ \\
Let me digress on the values of the Riemann Zeta function.  Here is a formula for $\zeta(2)$:
$$ \zeta(2) = \sum_{k=1}^\infty \frac{1}{k^2}
= \int_{1 > t_1 > t_2 > 0} \frac{dt_1}{1-t_1}\frac{dt_2}{t_2} $$
Then there's the formula by Eugenio Calabi
$$ \frac{3}{8} \zeta(2)=\sum_{k=0}^\infty \frac{1}{(2k+1)^2} = \int_0^1 \int_0^1 \frac{dx \, dy}{1 - (xy)^2} $$
There is another dissimilar looking formula:
$$ \zeta(2)  = \sum_{n=1}^\infty \frac{1}{n^2} = - \int_0^\infty \log(1 - e^{-x})$$
Even before trying our hand at the double-shuffle identities, or relating $\zeta(2)$ to famous constants\footnote{Why stop at $\pi$, there's the Glaisher constant and the Euler-Masceroni number and the Twin Prime Constant etc.} how to transform the first formula into the third? \\ \\
I dug up the fist formula from a paper of Zagier, the second in a paper of Elkies and a third in a paper by Passare\footnote{and there is more... Papers flying everywhere!}

\newpage

\noindent Elkies showed a sum linking L-functions and $\zeta$-functions:
$$ 
S(n) = \left\{ \begin{array}{cl}
(1-2^{-n})\zeta(n) & \text{if }n\text{ is even} \\
L(n, \chi_4) & \text{if }n\text{ is odd}
\end{array} \right.
 $$
Elkies formula also links the Euler numbers and Bernoulli numbers.
$$ \frac{A_n}{n!} = \left( \frac{2}{\pi} \right)^n \left( \frac{4}{\pi}\right) S(n+1) $$
The $A(n)$ numbers count the volume of a certain polytope:
$$ t_1 < t_2 > t_3 < t_4 > \dots < t_n > t_1 $$
this high-dimensional shape splits into a certain number of ``triangular" parts:
$$ 0 < t_1 < t_2 < \dots < t_n < 1$$
The volume integrals are different too:
$$ \int_{\frac{\pi}{2} } 1 du_1, \dots du_n $$
and the integral is:
$$ \int_0^1 \dots \int_0^1 \frac{dx_1 \dots dx_n}{1 \pm (x_1 \dots x_n)^2} $$
\newpage
\noindent The middle integral is an odd duck. 
$$ \frac{3}{8} \zeta(2)=\sum_{k=0}^\infty \frac{1}{(2k+1)^2} = \int_0^1 \int_0^1 \frac{dx \, dy}{1 - (xy)^2} $$
There's no magic change of variables turning into the iterated integral over the triangle $0 < s < t < 1$. \\ \\ I am pushing a square peg into a round hole.
$$ \zeta(2)  = \sum_{n=1}^\infty \frac{1}{n^2} = - \int_0^\infty \log(1 - e^{-x})$$
Passare also shows this is the volume of a polygon just a triangle:
$$ \int_{ x < y < \pi} 1 \; dx \, dy $$
and the reason the triangle and the exponential integral are the same
is they are the real and comlpex parts of the Amoeba. 
$$ x + y + 1 = e^u + e^v + 1 = 0 $$
I've checked already.  This special coincidence doesn't always work.  Just for Harnak curves\footnote{and the variables on this page are screwed up}
 \newpage
\includegraphics{triangle-amoeba.png} \\
Does Elkies polygon decomposition match up with Passare.  We're stuck in a really lame situation\footnote{There's an OK paper by Zurab Silagadze that says yes, but his explanation is a bit disorderly.  It's almost better to try again and I have some stuff he won't think of.} with these two polygon decomposition's don't match up in a fundamental way.
\begin{itemize}
\item Do we have one triangle?
\item or many triangles?
\end{itemize}
There is a paper by Alexander Goncharov\footnote{it's still ``geometry" but it is quite varied.} \\
\textbf{Multiple zeta-values, Galois groups, and geometry of modular varieties}
\texttt{ arXiv:math/0005069}
 \newpage

\noindent At will we can find source that find $\zeta(2n)$ by induction.  First $$\zeta(2) = \frac{\pi^2}{6}, \zeta(4)= \frac{\pi^4}{90}, \zeta(6)=\frac{\pi^6}{945}, \dots $$
skipping $n \mapsto n+2$. The link to even Bernoulli numbers was not that direct anyway:
$$\zeta(2n) = (-1)^{n+1} B_{2n} \frac{(2\pi)^{2n}}{2(2n)!} $$
What about $\zeta(3)$?  We know it is irrational but not transcendental.  Elkies' chain looks more like:
$$ L(1, \chi_4) \to \zeta(2) \to L(3, \chi_4) \to \zeta(4) \to \dots $$
switching between the $\zeta$ and L-functions.  

\fontfamily{qag}\selectfont \fontsize{12}{10}\selectfont

\begin{thebibliography}{}

\item Taro Kimura, Vasily Pestun \textbf{Quiver elliptic W-algebras} \texttt{arXiv:1608.04651}
\item Wikipedia ``Jacobi Triple Product", ``Ramanujan Theta Function"
\item Eric M. Rains, S. Ole Warnaar \textbf{Bounded Littlewood identities} \texttt{arXiv:1506.02755}
\item GH Hardy \textbf{Divergent Series} texttt{https://archive.org/details/DivergentSeries}
\item David Vernon Widder \textbf{The Laplace Transform} \texttt{https://archive.org/details/laplacetransform031816mbp}

\end{thebibliography}


\end{document}