\documentclass[12pt]{article}
%Gummi|065|=)
\usepackage{amsmath, amsfonts, amssymb}
\usepackage[margin=0.5in]{geometry}
\usepackage{xcolor}
\usepackage{graphicx}
\newcommand{\off}[1]{}
\DeclareMathSizes{20}{30}{21}{18}

\newcommand{\myhrule}{}

\usepackage{tikz}

\title{\textbf{ Reflection Positivty }}
\author{John D Mangual}
\date{}
\begin{document}

\fontfamily{qag}\selectfont \fontsize{25}{30}\selectfont

\maketitle

\noindent I don't know if it will lead to a result\footnote{I don't think anything I have ever written has ever qualified as a ``result".} but here is a question on Linear Algebra from MathOverflow.  \\ \\
Somehow we find a matrix
$$ A_i = \left[
\begin{array}{cc}
C_i + E_i & B_i \\
B_i^T & D_i - F_i
 \end{array}
\right] $$
and we are asked to show the determinant is positive:
$$ \det \left( I_{2n} + \prod_i e^{A_i} \right) \geq 0$$
and we are told it can be solved using \textbf{majorana fermions}  and \textbf{reflection positivity}. \\ \\
I am thinking this is a freshman Linear Algebra problem so how can it be on arXiv?

\newpage

\noindent Familiar tools like Hilbert spaces and representation theory are being mixed with physics and topology and a new and exciting way\footnote{
What happens to all the mathmatics that has been generated (and continues to be cranked out) that doesn't use these insights \\ \\
The proofs given on MathOverflow are both very difficult to follow and one might consider trying again.}. \\ \\
I know one result that might help which is simple compared to the formulas I've been seeing.
$$ H = \overline{H}^T \to \lambda \geq 0 $$
All Hermitian matrices have positive (or zero) eigenvalues.  And I definitely learned this one as a freshman and no longer remember the proof\footnote{The only two classes I ever got an A in at Princeton: Honors Linear Algebra and Wavelets.}. \\ \\
Getting very much ahead of myself on blog\footnote{https://terrytao.wordpress.com/2015/05/03/the-standard-branch-of-the-matrix-logarithm/} observes that:
$$ e^{\pi i} + 1 = 0 $$
means we have to have the standard branch of a logarithm.  He shows that $\sqrt{\cdot}$ also has one:
$$ \overline{\sqrt{1+i}} \neq \sqrt{1-i} $$
\newpage

\noindent The ambiguity of the logarithm - $\log 1 = 2\pi i k$ for $k \in \mathbb{Z}$ is tied to the discreteness of electric charge, since $\pi_1[SO(3)] = \mathbb{Z}_2$ or maybe $ H_0[SO(3)] = \mathbb{Z} $ oh this is quite terrible. \\ \\
The branches of the logarithm are very important.  I remember now $\pi_1(S^1) = \mathbb{Z}$ which we phrase as $e^{\pi i } = -1 $.  One can ask for matrix analogy of this equation and get really far. \\ \\
I wonder what is so special about $e$ or $\pi$.  What if we picked some ridiculous constant:
$$ \int_{-1}^1 \sqrt[4]{1 - t^4} $$
and I bet there's some $e$-like constant and an identity waiting to be found.  The equation:
$$ e^{ix} = \cos x + i \sin x $$
has a lot with the differential equation:
$$ \frac{d^2 f}{dx^2} = k\, f$$
The only equation we can really solve.  \newpage 

\noindent Maybe we can be more aggressive and say:
$$ f(x + \Delta x) - 2 f(x) + f(x - \Delta x) \approx k \, f(x) \, (\Delta x)^2 $$
Numerical methods textbooks complain why pick specifically this kind of derivative anyway?  And they show you how to solve all sorts of things. \\ \\
\textbf{Log} or notions of \textbf{size} or \textbf{angle} should play a big role here.  Or \textbf{charge}.  Basic ideas of \textbf{shape}.  \\ \\
Suppose that $f$ solves the following equation:
$$ f(x + \Delta x) - f(x) \approx k \, f(x) \, \Delta x $$
Then observe it also solves the related equation:
$$ f(x + \Delta x) \approx f(x) (1 + k  \, \Delta x) $$
 \newpage

\noindent \textbf{2} - Solving the actual problem... \\ \\
It is shown this bilinar interaction of fermions:
\begin{eqnarray*}
H &=& \sum_{i,j \in A} c_i^\dagger(C_{ij} - B_{1,ij})c_j 
+ \sum_{i,j \in B} c_i^\dagger(D_{ij} + B_{2,ij})c_j \\ 
&+& \sum_{i \in A, j \in B} \left( c_i^\dagger F_{ij} c_j + h.c \right) \end{eqnarray*}
is Majorana reflection positive defined.
\vspace{12pt}
\hrule
\vspace{12pt}
\noindent \textbf{What on earth does this mean???}
\begin{itemize}
\item what do the variables mean?  I think $B,C,D,F$ are matrices but $A$ is a set of some kind.
\item hopefully h.c. stands for ``Hermitian conjugate"
\item How does $c_i$ and $c_i^\dagger$ act? 
\item What is the Majorana reflection?
\item What was the sign problem in Quantum Monte Carlo (QMC)?
\end{itemize}

\newpage

\noindent Which inequality was referred to?
$$ \rho = \det \left( I + \prod_{k=1}^M e^{-\tau h^0} e^{-\eta h^I(\eta_k)} \right)$$
This is equation 20 and it is not derived\footnote{  Instead they refer  you to \texttt{http://journals.aps.org/prb/abstract/10.1103/PhysRevB.31.4403} where it is shown:
$$ \mathrm{tr} e^{-c_i^\dagger A_{ij} c_j}
e^{-c_i^\dagger B_{ij} c_j} = \det (1 + e^{-A}e^{-B}$$
where te summation over indices is ``understood".  I wonder if block diagonal notation might be better for these guys:
$$ \left[ \begin{array}{cc|cc} 
 \cdot & \cdot & \cdot & \cdot \\
 \cdot & \cdot & \cdot & \cdot \\ \hline
 \cdot & \cdot & \cdot & \cdot \\
 \cdot & \cdot & \cdot & \cdot  \end{array}\right] $$}. \\ Otherwise it's written
 $$ \rho = \mathrm{tr}
 \prod_{k=1}^M e^{-\tau H^0} e^{-\eta H^I(\eta_k)} 
  $$
 
\noindent \textbf{Theorem} - $\rho$ is positive semidefinite if all the coefficient matrices are Majorana reflection positive. \\ \\
\textbf{Question} what does it mean for a determinant or trace to be positive-semidefinite?  is $\det (\dots)$ not just a number?  
\newpage

\fontfamily{qag}\selectfont \fontsize{12}{10}\selectfont

\begin{thebibliography}{}

\item TZ. C. Wei, Congjun Wu, Yi Li, Shiwei Zhang, T. Xiang \textbf{Majorana Positivity and the Fermion sign problem of Quantum Monte Carlo Simulations} \texttt{ arXiv:1601.01994v2}

\item MathOverflow \textbf{How to prove this determinant is positive?}  \\ \texttt{http://mathoverflow.net/q/229788/1358} \\ \texttt{http://mathoverflow.net/q/204460/1358}

\item Lei Wang, Ye-Hua Liu, Mauro Iazzi, Matthias Troyer, Gergely Harcos. \textbf{Split orthogonal group: A guiding principle for sign-problem-free fermionic simulations}
\texttt{arXiv:1506.05349}

\item Arthur Jaffe, Zhengwei Liu \textbf{Planar Para Algebras, Reflection Positivity} \texttt{arXiv:1602.02662} 

\item Arthur Cayley \textbf{A Memoir on the Theory of Matrices} \texttt{https://archive.org/details/philtrans05474612}

\end{thebibliography}


\end{document}