\documentclass[12pt]{article}
%Gummi|065|=)
\usepackage{amsmath, amsfonts, amssymb}
\usepackage[landscape, margin=0.5in]{geometry}
\usepackage{xcolor}
\usepackage{graphicx}
\newcommand{\off}[1]{}
\DeclareMathSizes{20}{30}{21}{18}

\title{\textbf{ Getting Good at Inversions }}
\author{John D Mangual}
\date{}
\begin{document}

\fontfamily{qag}\selectfont \fontsize{25}{30}\selectfont

\maketitle

\noindent My apologies in advance that I did not draw lots of pictures!
$$ w = \frac{az+b}{cz+d} $$
If $ad-bc \neq 0$ this map is invertible and maps circles to circles.
$$ \left|\frac{z-p}{z-q}\right| = k $$
represents a circle where $p$ and $q$ are \textbf{inverse} points\footnote{These terms from Eucldean Geometry deserve a lot of reflection! The goal of this note is to jot down one simple hard-to-find formula.}.

\newpage

\noindent Trouble is we don't know the radius and center.  The only book I could find with a formula is Titchmarsh's \textbf{Theory of Functions} which has a nice section on ``conformal" maps. 
$$ \left| \frac{w - \frac{ap+b}{cp+d}}{w - \frac{aq+b}{cq+d}} \right| = k \left| \frac{cq + d}{cp + d} \right| $$
The image of a circle under a fractional linear transformation\footnote{The image of a (very tiny) cirle under any conformal map is another circle.} is another circle. \newline

\noindent A circle of radius $R$ centered at $p$ has $p$ and $\infty$ as inverse points.  Perhaps we can write $p$ and $q \gg 1 $ as the inverse points.
$$ \left|\frac{z-p}{z/q-1}\right| = kq= R $$
so additionally we could have $k \ll 1$.

\newpage

\noindent So what is the limit?  Setting $q = \infty = \frac{1}{0}$ we get:
$$ \left| \frac{w - \frac{ap+b}{cp+d}}{w - \frac{a}{c}} \right| = R\left| \frac{1}{p + \frac{d}{c}} \right| $$
This doesn't look right at all. \newline

\noindent There is a circle for each $k \neq 1$
$$ \left|\frac{z-p}{z-q}\right| = k $$
The radius and center of the circle is:
$$ \left|z - \frac{p - k^2 q}{1 - k^2} \right| = \frac{k}{|1 - k^2|} \; |p - q|$$

\newpage

\fontfamily{qag}\selectfont \fontsize{12}{10}\selectfont

\begin{thebibliography}{}

\item Davide Gaiotto, Peter Koroteev \textbf{On Three Dimensional Quiver Gauge Theories and Integrability} \texttt{arXiv:1304.0779}

\item Titchmarsh \textbf{Theory of Functions} \texttt{https://archive.org/details/TheTheoryOfFunctions}



\end{thebibliography}


\end{document}
