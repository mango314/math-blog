\documentclass[12pt]{article}
%Gummi|065|=)
\usepackage{amsmath, amsfonts, tabto, amssymb}
\title{Numbers and Entropy}
\usepackage{xcolor}
\usepackage[a4paper, total={6.5in, 10in}]{geometry}
\usepackage{framed}
\usepackage{tgadventor}
\usepackage{feynmf}
\usepackage{tikz}
\usetikzlibrary{positioning,arrows}
\usetikzlibrary{decorations.pathmorphing}
\usetikzlibrary{decorations.markings}
\colorlet{shadecolor}{red!10}
\author{John Mangual}
\date{}

\definecolor{green}{HTML}{BED46D}
\definecolor{blue}{HTML}{7A6BED}

\usepackage{hyperref}

\begin{document}
{\fontfamily{lmss}\selectfont

\maketitle

\section{Motive Galois Group}

Goal: To define and study a Galois Theory of Feynman Amplitudes. 

\begin{itemize}
\item Broadhurst-Kreimer (1995) found multi-$\zeta$ values as amplitudes of massless $\phi^4$ theory.
\item Deligne-Ihara-Drinfield (1989) Motivic Galois Group, $MT(\mathbb{Z})$
\item Cartier (1998) Is there a "cosmic" Galois Group.
\item Kontsevich (1998) Counting points over finite fields.
\item Connes-Marcolli (2004) {\color{gray}{Cosmic Galois Group related to renormalization}}.
\item Belkale-Brosner (2003) Graph hypersurfaces of general type.  Physically unrealistic counterexamples.
\item Bloch-Enault-Kreimer (2006) Defined motive on "primitive" graphs.
\item co-workers: Dorin, Panzer, Schentz, Yates.  {\color{blue}{Exist amplitudes which are \textbf{not} multiple-zeta values.  \textbf{not all} multiple zeta values appear as amplitudes.}}
\end{itemize}

\noindent \textbf{NB} motives play no role.  Retrieve as special cases, the ``symbol" of an amplitude.  Renormalization group (Connes-Kreimer Hopf algebra).  Hidden recursive structure on amplitudes; contraints.

\subsection{Feynman Graphs}

Scalar QFT in Euclidean space, $\mathbb{R}^d$.  A \textbf{Feynman graph} is a connected graph.

$$ G = (V, E_G, E_G^{ext}) $$

\noindent Vertices, internal Edges and external half-Edges.  ``kinematic" data.
\begin{itemize}
\item A particle mass $m_e: E_G \to \mathbb{R}$
\item Incoming momentum $q: E_G^{ext} \to \mathbb{R}^d$
\item Conservation of momentum $\displaystyle \sum_{E_G^{ext}} q_i = 0$.  Draw massive edges as \textbf{thickened} lines.
\end{itemize}

\noindent \textbf{Ex} 

\tikzset{
incoming/.style={thick,draw=blue, postaction={decorate},
    decoration={markings,mark=at position .5 with {\arrow[blue]{triangle 45}}}}
 }

\begin{tikzpicture}[node distance=1cm and 1.5cm]

\draw[incoming] (0,0) node [left] {$q_2$} -- (1,0);
\draw[incoming] (4,2.5) node [right] {$q_1$} -- (3, 2) ;
\draw[incoming] (4,-2.5)node [right] {$q_3$} -- (3,-2) ;
\draw[line width=0.1cm, black] (1,0) -- (3,2) node [midway,fill=white,color=blue,text=white] {$1$};
\draw[line width=0.1cm, black] (3,2) -- (3,-2)node [midway,fill=white,color=blue,text=white] {$2$};
\draw (3,-2) -- (1,0) node [midway,fill=white,color=green,text=white] {$3$};

\end{tikzpicture}

\noindent Two massive particles $m_1, m_2 \neq 0$ and one masslesss particle $m_3 = 0$.  Momentum conservation says $q_1 + q_2 + q_3 = 0$.  {\color{orange}Why are mass and momentum identified?}?{\color{orange}?}? \newline

\begin{tikzpicture}[node distance=1cm and 1.5cm]

\draw[incoming] (-1,2) node [left] {$q_1$} -- (0,0);
\draw[incoming] ( 0,2) node [left] {$q_2$} -- (0,0);
\draw[incoming] (2,2) node [right] {$q_n$} -- (0,0);
\node at (1,2) {$...$};

\draw[incoming] ( 5,2) node [right] {$q_1 + q_2 + \dots + q_n$} -- (5,0);


\end{tikzpicture}

\noindent As a short hand usually replace several incoming arrow with a single arrow representing the sum of momenta. \newline

\noindent View Feynman amplitude $I_G(m,q)$ as a function (\textbf{multi-valued} depending on choice of covering space, \textbf{infinite} possibly everywhere) function of affine variety 

$$\mathcal{M}_G^V = \left\{ (m,q) : \begin{array}{lc|lc} m_e \in \mathbb{A}^1 \backslash \{0\} & e \notin V_m & q_i \in \mathbb{A}^1 \backslash \{0\}  &  i \notin V_q \\ m_e = 0 & e \in V_m & q_i=0 & i \in V_q \end{array} \right\} \cap \left\{ \sum q_i = 0 \right\} $$

\noindent The \textbf{Euclidean Region}  is $\mathcal{M}_G^V(\mathbb{R})$.

\subsection{Feynman Amplitudes}

1960's presentation using \textbf{graph polynomials}.  \newline \newline

\noindent A \textbf{spanning $k$-tree} $T = T_1 \cup T_2 \cup \dots \cup T_n \subseteq G$ is a subgraph with $k$ connected components $T_i$ such that $V_G = \dot{\bigcup} V_{T_i}$ to each $e \in E_g$ we associate ``Schwinger parameter"... \newline

\noindent \textbf{Kirkhoff Polynomial} $\psi_G$ "1st Symanzik" polynomial
$$ \psi_G = \sum_{T \subseteq G} \prod_{e \in T} \alpha_e $$
%\noindent \texttt{Over 90 minutes of lecture remain...}


\begin{thebibliography}{9}

\bibitem{B1} Francis Brown.  Iterated Integrals in Quantum Field Theory.  \url{http://www.ihes.fr/~brown/ColombiaNotes7.pdf}
 
\bibitem{B2}
Francis Brown. \textit{Irrationality proofs for zeta values, moduli spaces and dinner parties}  \texttt{ arXiv:1412.6508}

\bibitem{B3}
Francis Brown. \textit{Modular forms in Quantum Field Theory}  \texttt{arXiv:1304.5342}

\bibitem{GR} Chris Godsil, Gordon Royle.  \textbf{Algebraic Graph Theory} (Graduate Texts in Mathematics 207)  Springer, 2001.

\end{thebibliography}
}
\end{document}
