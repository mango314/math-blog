\documentclass[12pt]{article}
%Gummi|065|=)
\usepackage{amsmath, amsfonts, amssymb}
\usepackage[landscape, margin=0.5in]{geometry}
\usepackage{xcolor}
\usepackage{graphicx}
\newcommand{\off}[1]{}
\DeclareMathSizes{20}{30}{21}{18}

\usepackage{tikz}

\title{\textbf{ The Verlinde Formula }}
\author{John D Mangual}
\date{}
\begin{document}

\fontfamily{qag}\selectfont \fontsize{25}{30}\selectfont

\maketitle

% note AVOID THE WORDS 'Ratner Theorem' instead explain what it is

\noindent The Verline Formula\footnote{I have seen Herman Verline or possibly Erik Verline at some point in my life.} is just some term I see in papers to make people sound smart.  What are these terms anyway?
\begin{itemize}
\item $G$ simple and simply connected compact Lie group
\item $M$ moduli space of semi-stable $G^\mathbb{C}$ bundles on $\Sigma$
\item $\Sigma$ smooth algebraic curve over complex numbers
\end{itemize}
I am now totally lost.  

\newpage

\noindent OK.  I used to like this field a lot and somewhere I lost interest.  The starting point is \textbf{orthogonality of characters}. If $\psi, \chi$ are characters of a group then:
$$ \frac{1}{|G|}\sum_{g \in \mathbb{G}} \chi(g) \psi(g) = \left\{ \begin{array}{cl} 1 & \text{if }\chi = \psi \\
0 &\text{if }\chi \neq \psi
\end{array} \right. $$
Nobody was interested when I learned this formula.  Perhap's it's best I leave this topic.

\newpage




\fontfamily{qag}\selectfont \fontsize{12}{10}\selectfont

\begin{thebibliography}{}

\item Jorgen Ellegaard Andersen, Sergei Gukov, Du Pei \textbf{The Verlinde formula for Higgs bundles} \texttt{arXiv:1608.01761
}




\end{thebibliography}


\end{document}

